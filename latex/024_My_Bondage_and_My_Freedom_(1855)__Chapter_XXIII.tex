{}

~

{CHAPTER XXIII.}

INTRODUCED TO THE ABOLITIONISTS.

{FIRST SPEECH AT NANTUCKET---MUCH SENSATION---{EXTRAORDITARY} SPEECH OF
MR. GARRISON---AUTHOR BECOMES A PUBLIC LECTURER---FOURTEEN YEARS'
EXPERIENCE---YOUTHFUL ENTHUSIASM---A BRAND NEW FACT---MATTER OF THE
AUTHOR'S SPEECH---HE COULD NOT FOLLOW THE PROGRAMME---HIS FUGITIVE
SLAVESHIP DOUBTED---TO SETTLE ALL DOUBT HE WRITES HIS EXPERIENCE OF
SLAVERY---DANGER OF RECAPTURE INCREASED.}

\textsc{In} the summer of 1841, a grand anti-slavery convention was held
in Nantucket, under the auspices of Mr. Garrison and his friends. Until
now, I had taken no holiday since my escape from slavery. Having worked
very hard that spring and summer, in Richmond's brass
foundery---sometimes working all night as well as all day---and needing
a day or two of rest, I attended this convention, never supposing that I
should take part in the proceedings. Indeed, I was not aware that any
one connected with the convention even so much as knew my name. I was,
however, quite mistaken. Mr. William C. Coffin, a prominent abolitionist
in those days of trial, had heard me speaking to my colored friends, in
the little school-house on Second street, New Bedford, where we
worshiped. He sought me out in the crowd, and invited me to say a few
words to the convention. Thus sought out, and thus invited, I was
induced to speak {}out the feelings inspired by the occasion, and the
fresh recollection of the scenes through which I had passed as a slave.
My speech on this occasion is about the only one I ever made, of which I
do not remember a single connected sentence. It was with the utmost
difficulty that I could stand erect, or that I could command and
articulate two words without hesitation and stammering. I trembled in
every limb. I am not sure that my embarrassment was not the most
effective part of my speech, if speech it could be called. At any rate,
this is about the only part of my performance that I now distinctly
remember. But excited and convulsed as I was, the audience, though
remarkably quiet before, became as much excited as myself. Mr. Garrison
followed me, taking me as his text; and now, whether I had made an
eloquent speech in behalf of freedom or not, his was one never to be
forgotten by those who heard it. Those who had heard Mr. Garrison
oftenest, and had known him longest, were astonished. It was an effort
of unequaled power, sweeping down, like a very tornado, every opposing
barrier, whether of sentiment or opinion. For a moment, he possessed
that almost fabulous inspiration, often referred to but seldom attained,
in which a public meeting is transformed, as it were, into a single
individuality---the orator wielding a thousand heads and hearts at once,
and by the simple majesty of his all controlling thought, converting his
hearers into the express image of his own soul. That night there were at
least one thousand Garrisonians in Nantucket! At the close of this great
meeting, I was duly waited on by {}Mr. John A. Collins---then the
general agent of the Massachusetts anti-slavery society---and urgently
solicited by him to become an agent of that society, and to publicly
advocate its anti-slavery principles. I was reluctant to take the
proffered position. I had not been quite three years from slavery---was
honestly distrustful of my ability---wished to be excused; publicity
exposed me to discovery and arrest by my master; and other objections
came up, but Mr. Collins was not to be put off, and I finally consented
to go out for three months, for I supposed that I should have got to the
end of my story and my usefulness, in that length of time.

Here opened upon me a new life---a life for which I had had no
preparation. I was a ``graduate from the peculiar institution,'' Mr.
Collins used to say, when introducing me, "\emph{with my diploma written
on my back!}" The three years of my freedom had been spent in the hard
school of adversity. My hands had been furnished by nature with
something like a solid leather coating, and I had bravely marked out for
myself a life of rough labor, suited to the hardness of my hands, as a
means of supporting myself and rearing my children.

Now what shall I say of this fourteen years' experience as a public
advocate of the cause of my enslaved brothers and sisters? The time is
but as a speck, yet large enough to justify a pause for
retrospection---and a pause it must only be.

Young, ardent, and hopeful, I entered upon this new life in the full
gush of unsuspecting enthusiasm. The cause was good; the men engaged in
it were {}good; the means to attain its triumph, good; Heaven's blessing
must attend all, and freedom must soon be given to the pining millions
under a ruthless bondage. My whole heart went with the holy cause, and
my most fervent prayer to the Almighty Disposer of the hearts of men,
were continually offered for its early triumph. ``Who or what,'' thought
I, ``can withstand a cause so good, so holy, so indescribably glorious.
The God of Israel is with us. The might of the Eternal is on our side.
Now let but the truth be spoken, and a nation will start forth at the
sound!'' In this enthusiastic spirit, I dropped into the ranks of
freedom's friends, and went forth to the battle. For a time I was made
to forget that my skin was dark and my hair crisped. For a time I
regretted that I could not have shared the hardships and dangers endured
by the earlier workers for the slave's release. I soon, however, found
that my enthusiasm had been extravagant; that hardships and dangers were
not yet passed; and that the life now before me, had shadows as well as
sunbeams.

Among the first duties assigned me, on entering the ranks, was to
travel, in company with Mr. George Foster, to secure subscribers to the
``Anti-slavery Standard'' and the ``Liberator.'' With him I traveled and
lectured through the eastern counties of Massachusetts. Much interest
was awakened---large meetings assembled. Many came, no doubt, from
curiosity to hear what a negro could say in his own cause. I was
generally introduced as a "\emph{chattel}``---a ''\emph{thing}``---a
piece of southern ''\emph{property}"---the chairman assuring the
audience that \emph{it} could speak. {}Fugitive slaves, at that time,
were not so plentiful as now; and as a fugitive slave lecturer, I had
the advantage of being a "\emph{brand new fact}"---the first one out. Up
to that time, a colored man was deemed a fool who confessed himself a
runaway slave, not only because of the danger to which he exposed
himself of being retaken, but because it was a confession of a very
\emph{low} origin! Some of my colored friends in New Bedford thought
very badly of my wisdom for thus exposing and degrading myself. The only
precaution I took, at the beginning, to prevent Master Thomas from
knowing where I was, and what I was about, was the withholding my former
name, my master's name, and the name of the state and county from which
I came. During the first three or four months, my speeches were almost
exclusively made up of narrations of my own personal experience as a
slave. ``Let us have the facts,'' said the people. So also said Friend
George Foster, who always wished to pin me down to my simple narrative.
``Give us the facts,'' said Collins, ``we will take care of the
philosophy.'' Just here arose some embarrassment. It was impossible for
me to repeat the same old story month after month, and to keep up my
interest in it. It was new to the people, it is true, but it was an old
story to me; and to go through with it night after night, was a task
altogether too mechanical for my nature. ``Tell your story, Frederick,''
would whisper my then revered friend, William Lloyd Garrison, as I
stepped upon the platform. I could not always obey, for I was now
reading and thinking. New views of the subject were presented to my
mind. It {}did not entirely satisfy me to \emph{narrate} wrongs; I felt
like \emph{denouncing} them. I could not always curb my moral
indignation for the perpetrators of slaveholding villainy, long enough
for a circumstantial statement of the facts which I felt almost
everybody must know. Besides, I was growing, and needed room. ``People
won't believe you ever was a slave, Frederick, if you keep on this
way,'' said Friend Foster. ``Be yourself,'' said Collins, ``and tell
your story.'' It was said to me, "Better have a \emph{little} of the
plantation manner of speech than not; 'tis not best that you seem too
learned." These excellent friends were actuated by the best of motives,
and were not altogether wrong in their advice; and still I must speak
just the word that seemed to \emph{me} the word to be spoken \emph{by}
me.

At last the apprehended trouble came. People doubted if I had ever been
a slave. They said I did not talk like a slave, look like a slave, nor
act like a slave, and that they believed I had never been south of Mason
and Dixon's line. ``He don't tell us where he came from---what his
master's name was---how he got away---nor the story of his experience.
Besides, he is educated, and is, in this, a contradiction of all the
facts we have concerning the ignorance of the slaves.'' Thus, I was in a
pretty fair way to be denounced as an impostor. The committee of the
Massachusetts anti-slavery society knew all the facts in my case, and
agreed with me in the prudence of keeping them private. They, therefore,
never doubted my being a genuine fugitive; but going down the aisles of
the churches in which I spoke, and hearing the {}free spoken Yankees
saying, repeatedly, "\emph{He's never been a slave, I'll warrant ye,}" I
resolved to dispel all doubt, at no distant day, by such a revelation of
facts as could not be made by any other than a genuine fugitive.

In a little less than four years, therefore, after becoming a public
lecturer, I was induced to write out the leading facts connected with my
experience in slavery, giving names of persons, places, and dates---thus
putting it in the power of any who doubted, to ascertain the truth or
falsehood of my story of being a fugitive slave. This statement soon
became known in Maryland, and I had reason to believe that an effort
would be made to recapture me.

It is not probable that any open attempt to secure me as a slave could
have succeeded, further than the obtainment, by my master, of the money
value of my bones and sinews. Fortunately for me, in the four years of
my labors in the abolition cause, I had gained many friends, who would
have suffered themselves to be taxed to almost any extent to save me
from slavery. It was felt that I had committed the double offense of
running away, and exposing the secrets and crimes of slavery and
slaveholders. There was a double motive for seeking my
reënslavement---avarice and vengeance; and while, as I have said, there
was little probability of successful recapture, if attempted openly, I
was constantly in danger of being spirited away, at a moment when my
friends could render me no assistance. In traveling about from place to
place---often alone---I was much exposed to this sort of attack. Any one
cherishing the {}design to betray me, could easily do so, by simply
tracing my whereabouts through the anti-slavery journals, for my
meetings and movements were promptly made known in advance. My true
friends, Mr. Garrison and Mr. Phillips, had no faith in the power of
Massachusetts to protect me in my right to liberty. Public sentiment and
the law, in their opinion, would hand me over to the tormentors. Mr.
Phillips, especially, considered me in danger, and said, when I showed
him the manuscript of my story, if in my place, he would throw it into
the fire. Thus, the reader will observe, the settling of one difficulty
only opened the way for another; and that though I had reached a free
state, and had attained a position for public usefulness, I was still
tormented with the liability of losing my liberty. How this liability
was dispelled, will be related, with other incidents, in the next
chapter.
