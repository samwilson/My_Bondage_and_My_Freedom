\chapter{Introduction}

\textsc{When} a man raises himself from the lowest condition in society
to the highest, mankind pay him the tribute of their admiration; when he
accomplishes this elevation by native energy, guided by prudence and
wisdom, their admiration is increased; but when his course, onward and
upward, excellent in itself, furthermore proves a possible, what had
hitherto been regarded as an impossible, reform, then he becomes a
burning and a shining light, on which the aged may look with gladness,
the young with hope, and the down-trodden, as a representative of what
they may themselves become. To such a man, dear reader, it is my
privilege to introduce you.

The life of Frederick Douglass, recorded in the pages which follow, is
not merely an example of self-elevation under the most adverse
circumstances; it is, moreover, a noble vindication of the highest aims
of the American anti-slavery movement. The real object of that movement
is not only to disenthrall, it is, also, to bestow upon the negro the
exercise of all those rights, from the possession of which he has been
so long debarred.

But this full recognition of the colored man to the right, and the
entire admission of the same to the full privileges, political,
religious and social, of manhood, requires powerful effort on the part
of the enthralled, as well as on the part of those who would disenthrall
them. The people at large must feel the conviction, as well as admit the
abstract logic, of human equality; the negro, for the first time in the
world's history, brought in full contact with high civilization, must
prove his title to all that is demanded for him; in the teeth of unequal
chances, he must prove himself equal to the mass of those who oppress
him---therefore, absolutely superior to his apparent fate, and to their
relative ability. And it is most cheering to the friends of freedom,
to-day, that evidence of this equality is {}rapidly accumulating, not
from the ranks of the half-freed colored people of the free states, but
from the very depths of slavery itself; the indestructible equality of
man to man is demonstrated by the ease with which black men, scarce one
remove from barbarism---if slavery can be honored with such a
distinction---vault into the high places of the most advanced and
painfully acquired civilization. Ward and Garnett, Wells Brown and
Pennington, Loguen and Douglass, are banners on the outer wall, under
which abolition is fighting its most successful battles, because they
are living exemplars of the practicability of the most radical
abolitionism; for, they were all of them born to the doom of slavery,
some of them remained slaves until adult age, yet they all have not only
won equality to their white fellow citizens, in civil, religious,
political and social rank, but they have also illustrated and adorned
our common country by their genius, learning and eloquence.

The characteristics whereby Mr. Douglass has won first rank among these
remarkable men, and is still rising toward highest rank among living
Americans, are abundantly laid bare in the book before us. Like the
autobiography of Hugh Miller, it carries us so far back into early
childhood, as to throw light upon the question, ``when positive and
persistent memory begins in the human being.'' And, like
\href{/wiki/Author:Hugh_Miller_(1802-1856)}{Hugh Miller}, he must have
been a shy old fashioned child, occasionally oppressed by what he could
not well account for, peering and poking about among the layers of right
and wrong, of tyrant and thrall, and the wonderfulness of that hopeless
tide of things which brought power to one race, and unrequited toil to
another, until, finally, he stumbled upon his ``first-found Ammonite,''
hidden away down in the depths of his own nature, and which revealed to
him the fact that liberty and right, for all men, were anterior to
slavery and wrong. When his knowledge of the world was bounded by the
visible horizon on Col. Lloyd's plantation, and while every thing around
him bore a fixed, iron stamp, as if it had always been so, this was, for
one so young, a notable discovery.

To his uncommon memory, then, we must add a keen and accurate insight
into men and things; an original breadth of common sense which enabled
him to see, and weigh, and compare whatever passed before him, and which
kindled a desire to search out and define their relations to other
things not so patent, but which never {}succumbed to the marvelous nor
the supernatural; a sacred thirst for liberty and for learning, first as
a means of attaining liberty, then as an end in itself most desirable; a
will; an unfaltering energy and determination to obtain what his soul
pronounced desirable; a majestic self-hood; determined courage; a deep
and agonizing sympathy with his embruted, crushed and bleeding fellow
slaves, and an extraordinary depth of passion, together with that rare
alliance between passion and intellect, which enables the former, when
deeply roused, to excite, develop and sustain the latter.

With these original gifts in view, let us look at his schooling; the
fearful discipline through which it pleased God to prepare him for the
high calling on which he has since entered---the advocacy of
emancipation by the people who are not slaves. And for this special
mission, his plantation education was better than any he could have
acquired in any lettered school. What he needed, was facts and
experiences, welded to acutely wrought up sympathies, and these he could
not elsewhere have obtained, in a manner so peculiarly adapted to his
nature. His physical being was well trained, also, running wild until
advanced into boyhood; hard work and light diet, thereafter, and a skill
in handicraft in youth.

For his special mission, then, this was, considered in connection with
his natural gifts, a good schooling; and, for his special mission, he
doubtless ``left school'' just at the proper moment. Had he remained
longer in slavery---had he fretted under bonds until the ripening of
manhood and its passions, until the drear agony of slave-wife and
slave-children had been piled upon his already bitter
experiences---then, not only would his own history have had another
termination, but the drama of American slavery would have been
essentially varied; for I cannot resist the belief, that the boy who
learned to read and write as he did, who taught his fellow slaves these
precious acquirements as he did, who plotted for their mutual escape as
he did, would, when a man at bay, strike a blow which would make slavery
reel and stagger. Furthermore, blows and insults he bore, at the moment,
without resentment; deep but suppressed emotion rendered him insensible
to their sting; but it was afterward, when the memory of them went
seething through his brain, breeding a fiery indignation at his injured
self-hood, that the resolve came to resist, and the time fixed when to
resist, and {}the plot laid, how to resist; and he always kept his
self-pledged word. In what he undertook, in this line, he looked fate in
the face, and had a cool, keen look at the relation of means to ends.
Henry Bibb, to avoid chastisement, strewed his master's bed with charmed
leaves---and \emph{was whipped.} Frederick Douglass quietly pocketed a
like \emph{fetiche,} compared his muscles with those of Covey---and
\emph{whipped him.}

In the history of his life in bondage, we find, well developed, that
inherent and continuous energy of character which will ever render him
distinguished. What his hand found to do, he did with his might; even
while conscious that he was wronged out of his daily earnings, he
worked, and worked hard. At his daily labor he went with a will; with
keen, well set eye, brawny chest, lithe figure, and fair sweep of arm,
he would have been king among calkers, had that been his mission.

It must not be overlooked, in this glance at his education, that Mr.
Douglass lacked one aid to which so many men of mark have been deeply
indebted he had neither a mother's care, nor a mother's culture, save
that which slavery grudgingly meted out to him. Bitter nurse! may not
even her features relax with human feeling, when she gazes at such
offspring! How susceptible he was to the kindly influences of
mother-culture, may be gathered from his own words, on page 57: ``It has
been a life-long, standing grief to me, that I know so little of my
mother, and that I was so early separated from her. The counsels of her
love must have been beneficial to me. The side view of her face is
imaged on my memory, and I take few steps in life, without feeling her
presence; but the image is mute, and I have no striking words of hers
treasured up.''

From the depths of chattel slavery in Maryland, our author escaped into
the caste-slavery of the north, in New Bedford, Massachusetts. Here he
found oppression assuming another, and hardly less bitter, form; of that
very handicraft which the greed of slavery had taught him, his
half-freedom denied him the exercise for an honest living; he found
himself one of a class---free colored men---whose position he has
described in the following words:

"Aliens are we in our native land. The fundamental principles of the
republic, to which the humblest white man, whether born here or
elsewhere, may appeal with confidence, in the hope of {}awakening a
favorable response, are held to be inapplicable to us. The glorious
doctrines of your revolutionary fathers, and the more glorious teachings
of the Son of God, are construed and applied against us. We are
literally scourged beyond the beneficent range of both authorities,
human and divine. {*~*~*~*} American humanity hates us, scorns us,
disowns and denies, in a thousand ways, our very personality. The
outspread wing of American Christianity, apparently broad enough to give
shelter to a perishing world, refuses to cover us. To us, its bones are
brass, and its features iron. In running thither for shelter and succor,
we have only fled from the hungry blood-hound to the devouring
wolf---from a corrupt and selfish world, to a hollow and hypocritical
church."---\emph{Speech before American and Foreign Anti-Slavery
Society, May}, 1854.

Four years or more, from 1837 to 1841, he struggled on, in New Bedford,
sawing wood, rolling casks, or doing what labor he might, to support
himself and young family; four years he brooded over the scars which
slavery and semi-slavery had inflicted upon his body and soul; and then,
with his wounds yet unhealed, he fell among the Garrisonians---a
glorious waif to those most ardent reformers. It happened one day, at
Nantucket, that he, diffidently and reluctantly, was led to address an
anti-slavery meeting. He was about the age when the younger Pitt entered
the House of Commons; like Pitt, too, he stood up a born orator.

William Lloyd Garrison, who was happily present, writes thus of Mr.
Douglass' maiden effort; "I shall never forget his first speech at the
convention---the extraordinary emotion it excited in my own---mind the
powerful impression it created upon a crowded auditory, completely taken
by surprise. {*~*~*} I think I never hated slavery so intensely as at
that moment; certainly, my perception of the enormous outrage which is
inflicted by it on the godlike nature of its victims, was rendered far
more clear than ever. There stood one in physical proportions and
stature commanding and exact---in intellect richly endowed---in natural
eloquence a
prodigy."\textsuperscript{\protect\hyperlink{cite_note-1}{{[}1{]}}}

It is of interest to compare Mr. Douglass's account of this meeting with
Mr. Garrison's. Of the two, I think the latter the most correct. It must
have been a grand burst of eloquence! The pent {}up agony, indignation
and pathos of an abused and harrowed boyhood and youth, bursting out in
all their freshness and overwhelming earnestness!

This unique introduction to its great leader, led immediately to the
employment of Mr. Douglass as an agent by the American Anti-Slavery
Society. So far as his self-relying and independent character would
permit, he became, after the strictest sect, a
\href{https://en.wikipedia.org/wiki/William_Lloyd_Garrison}{Garrisonian}.
It is not too much to say, that he formed a complement which they
needed, and they were a complement equally necessary to his ``make-up.''
With his deep and keen sensitiveness to wrong, and his wonderful memory,
he came from the land of bondage full of its woes and its evils, and
painting them in characters of living light; and, on his part, he found,
told out in sound Saxon phrase, all those principles of justice and
right and liberty, which had dimly brooded over the dreams of his youth,
seeking definite forms and verbal expression. It must have been an
electric flashing of thought, and a knitting of soul, granted to but few
in this life, and will be a lifelong memory to those who participated in
it. In the society, moreover, of Wendell Phillips, Edmund Quincy,
William Lloyd Garrison, and other men of earnest faith and refined
culture, Mr. Douglass enjoyed the high advantage of their assistance and
counsel in the labor of self-culture, to which he now addressed himself
with wonted energy. Yet, these gentlemen, although proud of Frederick
Douglass, failed to fathom, and bring out to the light of day, the
highest qualities of his mind; the force of their own education stood in
their own way: they did not delve into the mind of a colored man for
capacities which the pride of race led them to believe to be restricted
to their own Saxon blood. Bitter and vindictive sarcasm, irresistible
mimicry, and a pathetic narrative of his own experiences of slavery,
were the intellectual manifestations which they encouraged him to
exhibit on the platform or in the lecture desk.

A visit to England, in 1845, threw Mr. Douglass among men and women of
earnest souls and high culture, and who, moreover, had never drank of
the bitter waters of American caste. For the first time in his life, he
breathed an atmosphere congenial to the longings of his spirit, and felt
his manhood free and unrestricted. The cordial and manly greetings of
the British and Irish audiences in {}public, and the refinement and
elegance of the social circles in which he mingled, not only as an
equal, but as a recognized man of genius, were, doubtless genial and
pleasant resting places in his hitherto thorny and troubled journey
through life. There are joys on the earth, and, to the wayfaring
fugitive from American slavery or American caste, this is one of them.

But his sojourn in England was more than a joy to Mr. Douglass. Like the
platform at Nantucket, it awakened him to the consciousness of new
powers that lay in him. From the pupilage of Garrisonism he rose to the
dignity of a teacher and a thinker; his opinions on the broader aspects
of the great American question were earnestly and incessantly sought,
from various points of view, and he must, perforce, bestir himself to
give suitable answer. With that prompt and truthful perception which has
led their sisters in all ages of the world to gather at the feet and
support the hands of reformers, the gentlewomen of
England\textsuperscript{\protect\hyperlink{cite_note-2}{{[}2{]}}} were
foremost to encourage and strengthen him to carve out for himself a path
fitted to hia powers and energies, in the life-battle against slavery
and caste to which he was pledged. And one stirring thought, inseparable
from the British idea of the evangel of freedom, must have smote his ear
from every side---

{"}Hereditary bondmen! know ye not\\
Who would be free, themselves must strike the blow?"

The result of this visit was, that on his return to the United States,
he established a newspaper. This proceeding was sorely against the
wishes and the advice of the leaders of the American Anti-Slavery
Society, but our author had fully grown up to the conviction of a truth
which they had once promulged, but now forgotten, to wit: that in their
own elevation---self-elevation---colored men have a blow to strike ``on
their own hook,'' against slavery and caste. Differing from his Boston
friends in this matter, {}diffident in his own abilities, reluctant at
their dissuadings, how beautiful is the loyalty with which he still
clung to their principles in all things else, and even in this.

Now came the trial hour. Without cordial support from any large body of
men or party on this side the Atlantic, and too far distant in space and
immediate interest to expect much more, after the much already done, on
the other side, he stood up, almost alone, to the arduous labor and
heavy expenditure of editor and lecturer. The Garrison party, to which
he still adhered, did not want a \emph{colored} newspaper---there was an
odor of \emph{caste} about it; the Liberty party could hardly be
expected to give warm support to a man who smote their principles as
with a hammer; and the wide gulf which separated the free colored people
from the Garrisonians, also separated them from their brother, Frederick
Douglass.

The arduous nature of his labors, from the date of the establishment of
his paper, may be estimated by the fact, that anti-slavery papers in the
United States, even while the organs of, and when supported by,
anti-slavery parties, have, with a single exception, failed to pay
expenses. Mr. Douglass has maintained, and does maintain, his paper
without the support of any party, and even in the teeth of the
opposition of those from whom he had reason to expect counsel and
encouragement. He has been compelled, at one and the same time, and
almost constantly, during the past seven years, to contribute matter to
its columns as editor, and to raise funds for its support as lecturer.
It is within bounds to say, that he has expended twelve thousand dollars
of his own hard earned money, in publishing this paper, a larger sum
than has been contributed by any one individual for the general
advancement of the colored people. There had been many other papers
published and edited by colored men, beginning as far back as 1827, when
the Rev. Samuel E. Cornish and John B. Russworm (a graduate of Bowdoin
college, and afterward Governor of Cape Palmas) published the
\textsc{Freedom's Journal}, in New York city; probably not less than one
hundred newspaper enterprises have been started in. the United States,
by free colored men, born free, and some of them of liberal education
and fair talents for this work; but, one after another, they have fallen
through, although, in several instances, anti-slavery {}friends
contributed to their
support.\textsuperscript{\protect\hyperlink{cite_note-3}{{[}3{]}}} It
had almost been given up, as an impracticable thing, to maintain a
colored newspaper, when Mr. Douglass, with fewest early advantages of
all his competitors, essayed, and has proved, the thing perfectly
practicable, and, moreover, of great public benefit. This paper, in
addition to its power in holding up the hands of those to whom it is
especially devoted, also affords irrefutable evidence of the justice,
safety and practicability of Immediate Emancipation; it further proves
the immense loss which slavery inflicts on the land while it dooms such
energies as his to the hereditary degradation of slavery.

It has been said in this Introduction, that Mr. Douglass had raised
himself by his own efforts to the highest position in society. As a
successful editor, in our land, he occupies this position. Our editors
rule the land, and he is one of them. As an orator and thinker, his
position is equally high, in the opinion of his countrymen. If a
stranger in the United States would seek its most distinguished
men---the movers of public opinion---he will find their names mentioned,
and their movements chronicled, under the head of "\textsc{By Magnetic
Telegraph}," in the daily papers. The keen caterers for the public
attention, set down, in this column, such men only as have won high mark
in the public esteem. During the past winter---1854--5---very frequent
mention of Frederick Douglass was made under this head in the daily
papers; his name glided as often---this week from Chicago, next week
from Boston---over the lightning wires, as the name of any other man, of
whatever note. To no man did the people more widely nor more earnestly
say, "\emph{Tell me thy thought!}" And, somehow or other, revolution
seemed to follow in his wake. His were not the mere words of eloquence
which Kossuth speaks of, that delight the ear and then pass away. No!
They were \emph{work}-able, \emph{do}-able words, that brought forth
fruits in the revolution in Illinois, and in the passage of the
franchise resolutions by the Assembly of New York.

And the secret of his power, what is it? He is a Representative American
man---a type of his countrymen. Naturalists tell us that a full grown
man is a resultant or representative of all animated nature on this
globe; beginning with the early embryo state, then, {}representing the
lowest forms of organic
life,\textsuperscript{\protect\hyperlink{cite_note-4}{{[}4{]}}} and
passing through every subordinate grade or type, until he reaches the
last and highest---manhood. In like manner, and to the fullest extent,
has Frederick Douglass passed through every gradation of rank comprised
in our national make-up, and bears upon his person and upon his soul
every thing that is American. And he has not only full sympathy with
every thing American; his proclivity or bent, to active toil and visible
progress, are in the strictly national direction, delighting to outstrip
``all creation.''

Nor have the natural gifts, already named as his, lost anything by his
severe training. When unexcited, his mental processes are probably slow,
but singularly clear in perception, and wide in vision, the unfailing
memory bringing up all the facts in their every aspect; incongruities he
lays hold of incontinently, and holds up on the edge of his keen and
telling wit. But this wit never descends to frivolity; it is rigidly in
the keeping of his truthful common sense, and always used in
illustration or proof of some point which could not so readily be
reached any other way. ``Beware of a Yankee when he is feeding,'' is a
shaft that strikes home in a matter never so laid bare by satire before.
``The Garrisonian views of disunion, if carried to a successful issue,
would only place the people of the north in the same relation to
American slavery which they now bear to the slavery of Cuba or the
Brazils,'' is a statement, in a few words, which contains the result and
the evidence of an argument which might cover pages, but could not carry
stronger conviction, nor be stated in less pregnable form. In proof of
this, I may say, that having been submitted to the attention of the
Garrisonians in print, in March, it was repeated before them at their
business meeting in May---the platform, \emph{par excellence}, on which
they invite free fight, \emph{a l'outrance}, to all comers. It was given
out in the clear, ringing tones, wherewith the hall of shields was wont
to resound of old, yet neither Garrison, nor Phillips, nor May, nor
Remond, nor Foster, nor Burleigh, with his subtle steel of ``the ice
brook's temper,'' ventured to break a lance upon it! The doctrine of the
dissolution of the Union as a means for the abolition of American
slavery, was silenced upon the lips that gave it birth, and in the
presence {}of an array of defenders who compose the keenest intellects
in the land.

"\emph{The man who is right is a majority}," is an aphorism struck out
by Mr. Douglass in that great gathering of the friends of freedom, at
Pittsburgh, in 1852, where he towered among the highest, because, with
abilities inferior to none, and moved more deeply than any, there was
neither policy nor party to trammel the outpourings of his soul. Thus we
find, opposed to all the disadvantages which a black man in the United
States labors and struggles under, is this one vantage ground---when the
chance comes, and the audience where he may have a say, he stands forth
the freest, most deeply moved and most earnest of all men.

It has been said of Mr. Douglass, that his descriptive and declamatory
powers, admitted to be of the very highest order, take precedence of his
logical force. ``Whilst the schools might have trained him to the
exhibition of the formulas of deductive logic, nature and circumstances
forced him into the exercise of the higher faculties required by
induction. The first ninety pages of this ''Life in Bondage," afford
specimens of observing, comparing, and careful classifying, of such
superior character, that it is difficult to believe them the results of
a child's thinking; he questions the earth, and the children and the
slaves around him again and again, and finally looks to "\emph{God in
the sky}" for the why and the wherefore of the unnatural thing, slavery.
"\emph{Yere, if indeed thou art, wherefore dost thou suffer us to be
slain?}" is the only prayer and worship of the God-forsaken Dodos in the
heart of Africa. Almost the same was his prayer. One of his earliest
observations was that white children should know their ages, while the
colored children were ignorant of theirs; and the songs of the slaves
grated on his inmost soul, because a something told him that harmony in
sound, and music of the spirit, could not consociate with miserable
degradation.

To such a mind, the ordinary processes of logical deduction are like
proving that two and two make four. Mastering the intermediate steps by
an intuitive glance, or recurring to them as Ferguson resorted to
geometry, it goes down to the deeper relation of things, and brings out
what may seem, to some, mere statements, but which are new and brilliant
generalizations, each resting on a broad and stable basis. Thus,
\href{/wiki/Author:John_Marshall_(1755-1835)}{Chief Justice Marshall}
gave his decisions, and {}then told Brother Story to look up the
authorities---and they never differed from him. Thus, also, in his
``Lecture on the Anti-Slavery Movement,'' delivered before the Rochester
Ladies' Anti-Slavery Society, Mr. Douglass presents a mass of thought,
which, without any showy display of logic on his part, requires an
exercise of the reasoning faculties of the reader to keep pace with him.
And his ``Claims of the Negro Ethnologically Considered,'' is full of
new and fresh thoughts on he dawning science of race-history.

If, as has been stated, his intellection is slow, when unexcited, it is
most prompt and rapid when he is thoroughly aroused. Memory, logic, wit,
sarcasm, invective, pathos and bold imagery of rare structural beauty,
well up as from a copious fountain, yet each in its proper place, and
contributing to form a whole, grand in itself, yet complete in the
minutest proportions. It is most difficult to hedge him in a corner, for
his positions are taken so deliberately, that it is rare to find a point
in them undefended aforethought. Professor Reason tells me the
following: ``On a recent visit of a public nature, to Philadelphia, and
in a meeting composed mostly of his colored brethren, Mr. Douglass
proposed a comparison of views in the matters of the relations and
duties of `our people;' he holding that prejudice was the result of
condition, and could be conquered by the efforts of the degraded
themselves. A gentleman present, distinguished for logical acumen and
subtlety, and who had devoted no small portion of the last twenty-five
years to the study and elucidation of this very question, held the
opposite view, that prejudice is innate and unconquerable. He terminated
a series of well dove-tailed, Socratic questions to Mr. Douglass, with
the following: `If the legislature at Harrisburgh should awaken,
to-morrow morning, and find each man's skin turned black and his hair
woolly, what could they do to remove prejudice?' `Immediately pass laws
entitling black men to all civil, political and social privileges,' was
the instant reply---and the questioning ceased.''

The most remarkable mental phenomenon in Mr. Douglass, is his style in
writing and speaking. In March, 1855, he delivered an address in the
assembly chamber before the members of the legislature of the state of
New York. An
eyewitness\textsuperscript{\protect\hyperlink{cite_note-5}{{[}5{]}}}
describes the crowded and most intelligent audience, and their rapt
attention to the speaker {}as the grandest scene he ever witnessed in
the capitol. Among those whose eyes were riveted on the speaker full two
hours and a half, were Thurlow Weed and Lieutenant Governor Raymond; the
latter, at the conclusion of the address, exclaimed to a friend, ``I
would give twenty thousand dollars, if I could deliver that address in
that manner.'' Mr. Raymond is a first class graduate of Dartmouth, a
rising politician, ranking foremost in the legislature; of course, his
ideal of oratory must be of the most polished and finished description.

The style of Mr. Douglass in writing, is to me an intellectual puzzle.
The strength, affluence and terseness may easily be accounted for,
because the style of a man is the man; but how are we to account for
that rare polish in his style of writing, which, most critically
examined, seems the result of careful early culture among the best
classics of our language; it equals if it do not surpass the style of
Hugh Miller, which was the wonder of the British literary public, until
he unraveled the mystery in the most interesting of autobiographies. But
Frederick Douglass was still calking the seams of Baltimore clippers,
and had only written a ``pass,'' at the age when Miller's style was
already formed.

I asked William Whipper, of Pennsylvania, the gentleman alluded to
above, whether he thought Mr. Douglass's power inherited from the
Negroid, or from what is called the Caucasian side of his make-up? After
some reflection, he frankly answered, ``I must admit, although sorry to
do so, that the Caucasian predominates.'' At that time, I almost agreed
with him; but, facts narrated in the first part of this work, throw a
different light on this interesting question.

We are left in the dark as to who was the paternal ancestor of our
author; a fact which generally holds good of the Romuluses and Remuses
who are to inaugurate the new birth of our republic. In the absence of
testimony from the Caucasian side, we must see what evidence is given on
the other side of the house.

"My grandmother, though advanced in years, {*~*~*} was yet a woman of
power and spirit. She was marvelously straight in figure, elastic and
muscular." (p. 46.)

After describing her skill in constructing nets, her perseverance in
using them, and her wide-spread fame in the agricultural way {}he adds,
``It happened to her---as it will happen to any careful and thrifty
person residing in an ignorant and improvident neighborhood---to enjoy
the reputation of being born to good luck.'' And his grandmother was a
black woman.

``My mother was tall, and finely proportioned; of deep black, glossy
complexion; had regular features; and among other slaves was remarkably
sedate in her manners.'' ``Being a field hand, she was obliged to walk
twelve miles and return, between nightfall and daybreak, to see her
children'' (p. 54.) "I shall never forget the indescribable expression
of her countenance when I told her that I had had no food since morning.
{*~*~*} There was pity in her glance at me, and a fiery indignation at
Aunt Katy at the same time; {*~*~*~*} she read Aunt Katy a lecture which
she never forgot." (p. 56.) "I learned, after my mother's death, that
she could read, and that she was the \emph{only} one of all the slaves
and colored people in Tuckahoe who enjoyed that advantage. How she
acquired this knowledge, I know not, for Tuckahoe is the last place in
the world where she would be apt to find facilities for learning." (p.
57.) "There is, in {'}\emph{Prichard's Natural History of Man},' the
head of a figure---on page 157---the features of which so resemble those
of my mother, that I often recur to it with something of the feeling
which I suppose others experience when looking upon the pictures of dear
departed ones." (p. 52.)

The head alluded to is copied from the statue of Ramses the Great, an
Egyptian king of the nineteenth dynasty. The authors of the ``Types of
Mankind'' give a side view of the same on page 148, remarking that the
profile, ``like Napoleon's, is superbly European!'' The nearness of its
resemblance to Mr. Douglass' mother, rests upon the evidence of his
memory, and judging from his almost marvelous feats of recollection of
forms and outlines recorded in this book, this testimony may be
admitted.

These facts show that for his energy, perseverance, eloquence,
invective, sagacity, and wide sympathy, he is indebted to his negro
blood. The very marvel of his style would seem to be a development of
that other marvel,---how his mother learned to read. The versatility of
talent which he wields, in common with Dumas, Ira Aldridge, and Miss
Greenfield, would seem to be the result of the grafting of the
Anglo-Saxon on good, original, negro stock. If the {}friends of
``Caucasus'' choose to claim, for that region, what remains after this
analysis---to wit: combination---they are welcome to it. They will
forgive me for reminding them that the term ``Caucasian'' is dropped by
recent writers on Ethnology; for the people about Mount Caucasus, are,
and have ever been, Mongols. The great ``white race'' now seek
paternity, according to Dr. Pickering, in Arabia---``Arida Nutrix'' of
the best breed of horses \&c. Keep on, gentlemen; you will find
yourselves in Africa, by-and-by. The Egyptians, like the Americans, were
a \emph{mixed race}, with some negro blood circling around the throne,
as well as in the mud hovels.

This is the proper place to remark of our author, that the same strong
self-hood, which led him to measure strength with Mr. Covey, and to
wrench himself from the embrace of the Garrisonians, and which has borne
him through many resistances to the personal indignities offered him as
a colored man, sometimes becomes a hyper-sensitiveness to such assaults
as men of his mark will meet with, on paper. Keen and unscrupulous
opponents have sought, and not unsuccessfully, to pierce him in this
direction; for well they know, that if assailed, he will smite back.

It is not without a feeling of pride, dear reader, that I present you
with this book. The son of a self-emancipated bond-woman, I feel joy in
introducing to you my brother, who has rent his own bonds, and who, in
his every relation---as a public man, as a husband and as a father---is
such as does honor to the land which gave him birth. I shall place this
book in the hands of the only child spared me, bidding him to strive and
emulate its noble example. You may do likewise, it is an American book,
for Americans, in the fullest sense of the idea. It shows that the worst
of our institutions, in its worst aspect, cannot keep down energy,
truthfulness, and earnest struggle for the right. It proves the justice
and practicability of Immediate Emancipation. It shows that any man in
our land, "no matter in what battle his liberty may have been cloven
down, {*~*~*~*} no matter what complexion an Indian or an African sun
may have burned upon him," not only may ``stand forth redeemed and
disenthralled,'' but may also stand up a candidate for the highest
suffrage of a great people the tribute of their honest, hearty
admiration.{}Reader, \emph{Vale!}

{}\emph{New York}, May 23, 1855.{\textsc{James M'Cune Smith.}}

\begin{center}\rule{0.5\linewidth}{\linethickness}\end{center}

\begin{enumerate}
\item
  \hypertarget{cite_note-1}{}

  {\protect\hyperlink{cite_ref-1}{↑}} {Letter, Introduction to
  \href{/wiki/Life_of_Frederick_Douglass}{Life of Frederick Douglass},
  Boston, 1841.}
\item
  \hypertarget{cite_note-2}{}

  {\protect\hyperlink{cite_ref-2}{↑}} {One of these ladies, impelled by
  the same noble spirit which carried Miss Nightingale to Scutari, has
  devoted her time, her untiring energies, to a great extent her means,
  and her high literary abilities, to the advancement and support of
  Frederick Douglass' Paper, the only organ of the downtrodden, edited
  and published by one of themselves, in the United States.}
\item
  \hypertarget{cite_note-3}{}

  {\protect\hyperlink{cite_ref-3}{↑}} {Mr. Stephen Myers, of Albany,
  deserves mention as one of the most persevering among the colored
  editorial fraternity.}
\item
  \hypertarget{cite_note-4}{}

  {\protect\hyperlink{cite_ref-4}{↑}} {The German physiologists have
  even discovered vegetable matter---starch---in the human body. See
  Med. Chirurgical Rev., Oct., 1864, p. 889.}
\item
  \hypertarget{cite_note-5}{}

  {\protect\hyperlink{cite_ref-5}{↑}} {Mr. Wm. H. Topp, of Albany.}
\end{enumerate}
