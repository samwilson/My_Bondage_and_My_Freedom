{}

\href{/wiki/File:Frederickdouglassengraving.jpg}{\includegraphics[width=4.37500in,height=5.11458in]{//upload.wikimedia.org/wikipedia/commons/thumb/b/b2/Frederickdouglassengraving.jpg/420px-Frederickdouglassengraving.jpg}}

\href{/wiki/File:Frederickdouglasssignature.png}{\includegraphics[width=3.33333in,height=0.77083in]{//upload.wikimedia.org/wikipedia/commons/thumb/9/95/Frederickdouglasssignature.png/320px-Frederickdouglasssignature.png}}

{\protect\hypertarget{ux5cux7bux5cux7bux5cux7b1ux5cux7dux5cux7dux5cux7d}{}{}}

{}

{{MY BONDAGE}}

~

AND

~

{{MY FREEDOM.}}

~

{\textgerman{Part} I.---\textgerman{Life as a Slave. Part}
II.---\textgerman{Life as a Freeman.}}

~

{{\textsc{By} FREDERICK DOUGLASS.}}

~

{WITH}

{{AN INTRODUCTION.}}

~

\textsc{By} \href{/wiki/Author:James_McCune_Smith}{DR. JAMES M'CUNE
SMITH}.

{By a principle essential to christianity, a \textsc{person} is
eternally differenced from a\\
\textsc{thing}; so that the idea of a \textsc{human being}, necessarily
excludes the idea of \textsc{property}\\
\textsc{in that being}.}

\textsc{Coleridge.}

\begin{longtable}[]{@{}lll@{}}
\toprule
\includegraphics[width=0.52083in,height=0.01042in]{//upload.wikimedia.org/wikipedia/commons/thumb/1/1c/Rule_Segment_-_Span_-_50px.svg/50px-Rule_Segment_-_Span_-_50px.svg.png}
&
\includegraphics[width=0.06250in,height=0.07292in]{//upload.wikimedia.org/wikipedia/commons/thumb/2/28/Rule_Segment_-_Circle_-_6px.svg/6px-Rule_Segment_-_Circle_-_6px.svg.png}
&
\includegraphics[width=0.52083in,height=0.01042in]{//upload.wikimedia.org/wikipedia/commons/thumb/1/1c/Rule_Segment_-_Span_-_50px.svg/50px-Rule_Segment_-_Span_-_50px.svg.png}\tabularnewline
\bottomrule
\end{longtable}

{NEW YORK AND AUBURN:}

{{MILLER, ORTON \& MULLIGAN.}}

{New York: 25 Park Row.---Auburn: 107 Genesee-st.}

1855.

{\protect\hypertarget{ux5cux7bux5cux7bux5cux7b1ux5cux7dux5cux7dux5cux7d}{}{}}

{}

\begin{center}\rule{0.5\linewidth}{\linethickness}\end{center}

{Entered according to Act of Congress, in the year one thousand eight
hundred and fifty-five,}\\
{BY FREDERICK DOUGLASS,}\\
{In the Clerk's Office of the District Court of the Northern District of
New York.}

\begin{center}\rule{0.5\linewidth}{\linethickness}\end{center}

~

{AUBURN:}\\
{{MILLER, ORTON \& MULLIGAN,}}\\
{STEREOTYPERS AND PRINTERS.}

{\protect\hypertarget{ux5cux7bux5cux7bux5cux7b1ux5cux7dux5cux7dux5cux7d}{}{}}

{}

TO

{{HONORABLE GERRIT SMITH,}}

{AS A SLIGHT TOKEN OF}

{ESTEEM FOR HIS CHARACTER},

{ADMIRATION FOR HIS GENIUS AND BENEVOLENCE,}

AFFECTION FOR HIS PERSON, AND

{{GRATITUDE FOR HIS FRIENDSHIP},}

{AND AS}

{\textgerman{A Small but most Sincere Acknowledgment of}}

{HIS PRE-EMINENT SERVICES IN BEHALF OF THE RIGHTS AND LIBERTIES}

{OF AN}

AFFILICTED, DESPISED AND DEEPLY OUTRAGED PEOPLE,

{BY RANKING SLAVERY WITH PIRACY AND MURDER,}

{AND BY}

DENYING IT EITHER A LEGAL OR CONSTITUTIONAL EXISTENCE,

{\textgerman{{This Volume is Respectfully Dedicated,}}}

{BY HIS FAITHFUL AND FIRMLY ATTACHED FRIEND},

{{FREDERICK DOUGLASS.}}

{}{ROCHESTER, N. Y.}

{\protect\hypertarget{ux5cux7bux5cux7bux5cux7b1ux5cux7dux5cux7dux5cux7d}{}{}}

{}

~

{EDITOR'S PREFACE.}

\begin{longtable}[]{@{}lll@{}}
\toprule
\includegraphics[width=0.52083in,height=0.01042in]{//upload.wikimedia.org/wikipedia/commons/thumb/1/1c/Rule_Segment_-_Span_-_50px.svg/50px-Rule_Segment_-_Span_-_50px.svg.png}
&
\includegraphics[width=0.06250in,height=0.07292in]{//upload.wikimedia.org/wikipedia/commons/thumb/2/28/Rule_Segment_-_Circle_-_6px.svg/6px-Rule_Segment_-_Circle_-_6px.svg.png}
&
\includegraphics[width=0.52083in,height=0.01042in]{//upload.wikimedia.org/wikipedia/commons/thumb/1/1c/Rule_Segment_-_Span_-_50px.svg/50px-Rule_Segment_-_Span_-_50px.svg.png}\tabularnewline
\bottomrule
\end{longtable}

\textsc{If} the volume now presented to the public were a mere work of
\textsc{Art}, the history of its misfortune might be written in two very
simple words---\textsc{too late}. The nature and character of slavery
have been subjects of an almost endless variety of artistic
representation; and after the brilliant achievements in that field, and
while those achievements are yet fresh in the memory of the million, he
who would add another to the legion, must possess the charm of
transcendent excellence, or apologize for something worse than rashness.
The reader is, therefore, assured, with all due promptitude, that his
attention is not invited to a work of \textsc{Art}, but to a work of
\textsc{Facts}---Facts, terrible and almost incredible, it may be---yet
\textsc{Facts}, nevertheless.

I am authorized to say that there is not a fictitious name nor place in
the whole volume; but that names and places are literally given, and
that every transaction therein described actually transpired.

Perhaps the best Preface to this volume is furnished {}in the following
letter of Mr. Douglass, written in answer to my urgent solicitation for
such a work:

\textsc{Rochester}, N. Y. \emph{July} 2, 1855.

\textsc{Dear Friend}: I have long entertained, as you very well know, a
somewhat positive repugnance to writing or speaking anything for the
public, which could, with any degree of plausibility, make me liable to
the imputation of seeking personal notoriety, for its own sake.
Entertaining that feeling very sincerely, and permitting its control,
perhaps, quite unreasonably, I have often refused to narrate my personal
experience in public anti-slavery meetings, and in sympathizing circles,
when urged to do so by friends, with whose views and wishes, ordinarily,
it were a pleasure to comply. In my letters and speeches, I have
generally aimed to discuss the question of Slavery in the light of
fundamental principles, and upon facts, notorious and open to all;
making, I trust, no more of the fact of my own former enslavement, than
circumstances seemed absolutely to require. I have never placed my
opposition to slavery on a basis so narrow as my own enslavement, but
rather upon the indestructible and unchangeable laws of human nature,
every one of which is perpetually and flagrantly violated by the slave
system. I have also felt that it was best for those having histories
worth the writing---or supposed to be so---to commit such work to hands
other than their own. To write of one's self, in such a manner as not to
incur the imputation of weakness, vanity, and egotism, is a work within
the ability of but few; and I have little reason to believe that I
belong to that fortunate few.

These considerations caused me to hesitate, when first you {}kindly
urged me to prepare for publication a full account of my life as a
slave, and my life as a freeman.

Nevertheless, I see, with you, many reasons for regarding my
autobiography as exceptional in its character, and as being, in some
sense, naturally beyond the reach of those reproaches which honorable
and sensitive minds dislike to incur. It is not to illustrate any heroic
achievements of a man, but to vindicate a just and beneficent principle,
in its application to the whole human family, by letting in the light of
truth upon a system, esteemed by some as a blessing, and by others as a
curse and a crime. I agree with you, that this system is now at the bar
of public opinion---not only of this country, but of the whole civilized
world---for judgment. Its friends have made for it the usual
plea---``not guilty;'' the case must, therefore, proceed. Any facts,
either from slaves, slaveholders, or by-standers, calculated to
enlighten the public mind, by revealing the true nature, character, and
tendency of the slave system, are in order, and can scarcely be
innocently withheld.

I see, too, that there are special reasons why I should write my own
biography, in preference to employing another to do it. Not only is
slavery on trial, but unfortunately, the enslaved people are also on
trial. It is alleged, that they are, naturally, inferior; that they are
\emph{so low} in the scale of humanity, and so utterly stupid, that they
are unconscious of their wrongs, and do not apprehend their rights.
Looking, then, at your request, from this stand-point, and wishing
everything of which you think me capable to go to the benefit of my
afflicted people, I part with my doubts and hesitation, and proceed to
furnish you the desired manuscript; hoping that you may be able to make
such arrangements for its publication as shall be best adapted {}to
accomplish that good which you so enthusiastically anticipate.

\textsc{Frederick Douglass}.

There was little necessity for doubt and hesitation on the part of Mr.
Douglass, as to the propriety of his giving to the world a full account
of himself. A man who was born and brought up in slavery, a living
witness of its horrors; who often himself experienced its cruelties; and
who, despite the depressing influences surrounding his birth, youth and
manhood, has risen, from a dark and almost absolute obscurity, to the
distinguished position which he now occupies, might very well assume the
existence of a commendable curiosity, on the part of the public, to know
the facts of his remarkable history.

\textsc{Editor}.

~

{\protect\hypertarget{ux5cux7bux5cux7bux5cux7b1ux5cux7dux5cux7dux5cux7d}{}{}}

{}

~

{CONTENTS.}

\begin{longtable}[]{@{}lll@{}}
\toprule
\includegraphics[width=0.52083in,height=0.01042in]{//upload.wikimedia.org/wikipedia/commons/thumb/1/1c/Rule_Segment_-_Span_-_50px.svg/50px-Rule_Segment_-_Span_-_50px.svg.png}
&
\includegraphics[width=0.07292in,height=0.09375in]{//upload.wikimedia.org/wikipedia/commons/thumb/2/21/Rule_Segment_-_Diamond_-_6px.svg/7px-Rule_Segment_-_Diamond_-_6px.svg.png}
&
\includegraphics[width=0.52083in,height=0.01042in]{//upload.wikimedia.org/wikipedia/commons/thumb/1/1c/Rule_Segment_-_Span_-_50px.svg/50px-Rule_Segment_-_Span_-_50px.svg.png}\tabularnewline
\bottomrule
\end{longtable}

{}

{}

{}

{}

{}

{}

{}

PAGE.

\href{/wiki/My_Bondage_and_My_Freedom_(1855)/Introduction}{Introduction},

17

\href{/wiki/My_Bondage_and_My_Freedom_(1855)/Chapter_I}{Chapter I}.

THE AUTHOR'S CHILDHOOD.

Place of Birth,

33

Character of the District,

34

Time of Birth---My Grandparents,

35

Character of my Grandmother,

36

The Log Cabin---Its Charms,

37

First Knowledge of being a Slave,

38

Old Master---Griefs and Joys of Childhood,

39

Comparative Happiness of the Slave-Boy and his White Brother,

40

\href{/wiki/My_Bondage_and_My_Freedom_(1855)/Chapter_II}{Chapter II}.

THE AUTHOR REMOVED FROM HIS FIRST HOME.

The name ``Old Master'' a Terror,

43

Home Attractions---Dread of being removed from Tuckahoe,

44

The Journey to Col. Lloyd's Plantation,

46

Scene on reaching Old Master's,

47

First Meeting with my Brothers and Sisters,

48

Departure of Grandmother---Author's Grief,

49

\href{/wiki/My_Bondage_and_My_Freedom_(1855)/Chapter_III}{Chapter III}.

THE AUTHOR'S PARENTAGE.

Author's Father shrouded in Mystery,

51

My Mother---Her Personal Appearance,

52

Her Situation---Visits to her Boy,

53

Cruelty of ``Aunt Katy''---Threatened Starvation,

55

My Mother's Interference,

56

Her Death,

57

Her Love of Knowledge,

58

Penalty for having a White Father,

59

\href{/wiki/My_Bondage_and_My_Freedom_(1855)/Chapter_IV}{Chapter IV}.

A GENERAL SURVEY OF THE SLAVE PLANTATION.

Slaveholding Cruelty restrained by Public Opinion,

61

Isolation of Lloyd's Plantation,

62

Beyond the reach of Public Opinion,

{53}

Religion and Politics alike Excluded,

64

Natural and Artificial Charms of the Place,

65

The ``Great House,''

67

Etiquette among Slaves,

69

The Comic Slave-Doctor,

70

Praying and Flogging,

71

Business of Old Master,

73

Sufferings from Hunger,

75

Jargon of the Plantation,

76

Family of Col. Lloyd---Mas' Daniel,

77

Family of Old Master---Social Position,

78

\href{/wiki/My_Bondage_and_My_Freedom_(1855)/Chapter_V}{Chapter V}.

GRADUAL INITIATION INTO THE MYSTERIES OF SLAVERY.

Growing Acquaintance with Old Master---His Character,

79

Evils of Unrestrained Passion---A Man of Trouble,

80

Supposed Obtuseness of Slave-Children,

81

Brutal Outrage on my Aunt Milly by a drunken Overseer,

82

Slaveholders' Impatience at Appeals against Cruelty,

83

Wisdom of appealing to Superiors,

84

Attempt to break up a Courtship,

85

Slavery destroys all Incentives to a Virtuous Life,

86

A Harrowing Scene,

87

\href{/wiki/My_Bondage_and_My_Freedom_(1855)/Chapter_VI}{Chapter VI}.

TREATMENT OF SLAVES ON LLOYD'S PLANTATION.

The Author's Early Reflections on Slavery,

89

Conclusions at which he Arrived,

90

Presentiment of one day being a Freeman,

91

Combat between an Overseer and \& Slave-Woman,

92

Nelly's noble Resistance,

94

Advantages of Resistance,

95

Mr. Sevier, the brutal Overseer, and his Successors,

96

Allowance-day on the Home Plantation,

97

The Singing of the Slaves no Proof of Contentment,

98

Food and Clothing of the Slaves,

100

Naked Children,

101

Nursing Children carried to the Field,

102

Description of the Cowskin,

103

Manner of making the Ash Cake---The Dinner Hour,

104

Contrast at the Great House,

105

\href{/wiki/My_Bondage_and_My_Freedom_(1855)/Chapter_VII}{Chapter VII}.

LIFE IN THE GREAT HOUSE.

Comfort And Luxuries---Elaborate Expenditure,

107

Men and Maid Servants---Black Aristocracy,

109

Stable and Carriage House,

110

Deceptive Character of Slavery,

111

Slaves and Slaveholders alike Unhappy,

112

Fretfulness and Capriciousness of Slaveholders,

113

Whipping of Old Barney by Col. Lloyd,

114

William Wilks, a supposed son of Col. Lloyd,

115

Curious Incident---Penalty of telling the Truth,

116

Preference of Slaves for Rich Masters,

118

\href{/wiki/My_Bondage_and_My_Freedom_(1855)/Chapter_VIII}{Chapter
VIII}.

A CHAPTER OF HORRORS.

Austin Gore---Sketch of his Character,

119

Absolute Power of Overseers,

121

Murder of Denby---How it Occurred,

122

How Gore made Peace with Col. Lloyd,

123

Murder of a Slave-girl by Mrs. Hicks,

125

No Laws for the Protection of Slaves can be Enforced,

127

\href{/wiki/My_Bondage_and_My_Freedom_(1855)/Chapter_IX}{Chapter IX}.

PERSONAL TREATMENT OF THE AUTHOR.

Miss Lucretia Auld---Her Kindness,

129

A Battle with ``Ike,'' and its Consequences,

130

Beams of Sunlight,

131

Suffering from Cold---How we took our Meals,

132

Orders to prepare to go to Baltimore---Extraordinary Cleansing,

134

Cousin Tom's Description of Baltimore,

135

The Journey,

136

Arrival at Baltimore,

137

Kindness of my new Mistress---Little Tommy,

138

A Turning Point in my History,

139

\href{/wiki/My_Bondage_and_My_Freedom_(1855)/Chapter_X}{Chapter X}.

LIFE IN BALTIMORE.

City Annoyances---Plantation Regrets,

141

My Improved Condition,

142

Character of my new Master, Hugh Auld,

143

My Occupation---Increased Sensitiveness,

144

Commencement of Learning to Read---Why Discontinued,

145

Master Hugh's Exposition of the true Philosophy of Slavery,

146

Increased Determination to Learn,

147

Contrast between City and Plantation Slaves,

148

Mrs. Hamilton's Brutal Treatment of her Slaves,

149

\href{/wiki/My_Bondage_and_My_Freedom_(1855)/Chapter_XI}{Chapter XI}.

``A CHANGE CAME O'ER THE SPIRIT OF MY DREAM.''

Knowledge Acquired by Stealth,

151

My Mistress---Her Slaveholding Duties,

152

Deplorable Effects on her Character,

153

How I pursued my Education---My Tutors,

155

My Deliberations on the Character of Slavery,

156

The Columbian Orator and its Lessons,

157

Speeches of Chatham, Sheridan, Pitt, and Fox,

158

Knowledge ever Increasing---My Eyes Opened,

159

How I pined for Liberty,

160

Dissatisfaction of my poor Mistress,

161

\href{/wiki/My_Bondage_and_My_Freedom_(1855)/Chapter_XII}{Chapter XII}.

RELIGIOUS NATURE AWAKENED.

Abolitionists spoken of,

163

Eagerness to know what the word meant,

164

The Enigma solved---Turner's Insurrection,

165

First Awakened on the subject of Religion,

166

My Friend Lawson---His Character and Occupation,

167

Comfort Derived from his Teaching,

168

New Hopes and Aspirations,

169

The Irishmen on the Wharf---Their Sympathy,

170

How I learned to Write,

171

\href{/wiki/My_Bondage_and_My_Freedom_(1855)/Chapter_XIII}{Chapter
XIII}.

THE VICISSITUDES OF SLAVE LIFE.

Death of Young Master Richard,

173

Author's Presence required at the Division of Old Master's Property,

174

Attachment of Slaves to their Homes,

176

Sad Prospects and Grief,

177

General Dread of Master Andrew---His Cruelty,

178

Return to Baltimore---Death of Mistress Lucretia,

179

My poor old Grandmother---Her sad Fate,

180

Second Marriage of Master Thomas,

181

Again Removed from Master Hugh's,

182

Regrets at Leaving Baltimore,

183

A Plan of Escape Entertained,

184

\href{/wiki/My_Bondage_and_My_Freedom_(1855)/Chapter_XIV}{Chapter XIV}.

EXPERIENCE IN ST. MICHAEL'S.

The Village and its Inhabitants,

185

Meteoric Phenomena---Author's Impressions,

186

Character of my new Master and Mistress,

187

Allowance of Food---Sufferings from Hunger,

188

Stealing and its Vindication,

189

A new Profession of Faith,

190

Morality of Free Society has no Application to Slave Society,

191

Southern Camp-Meeting---Master Thomas professes Conversion,

193

Hopes and Suspicions,

194

The Result---Faith and Works entirely at Variance,

195

No more Meal brought from the Mill---Methodist Preachers,

197

Their utter Disregard of the Slaves---An Exception,

198

A Sabbath School Instituted,

199

How broken up and by whom,

200

Cruel Treatment of Cousin Henny by Master Thomas,

201

Differences with Master Thomas, and the Consequences,

202

Edward Covey---His Character,

203

\href{/wiki/My_Bondage_and_My_Freedom_(1855)/Chapter_XV}{Chapter XV}.

COVEY, THE NEGRO-BREAKER.

Journey to my new Master's,

205

Meditations by the way,

206

View of Covey's Residence---The Family,

207

Awkwardness as a Field Hand,

208

First Adventure at Ox Driving,

209

Unruly Animals---Hair-breadth Escapes,

211

Oxen and Men---Points of Similarity,

212

Sent back to the Woods,

213

Covey's Manner of proceeding to Whip,

214

His Cunning and Trickery---Severe Labor,

215

Family Worship,

217

Shocking Contempt for Chastity---An Illustration,

218

Author Broken Down---His only Leisure Time,

219

Freedom of the Ships and his own Slavery Contrasted,

220

Anguish beyond Description,

221

\href{/wiki/My_Bondage_and_My_Freedom_(1855)/Chapter_XVI}{Chapter XVI}.

ANOTHER PRESSURE OF THE TYRANT'S VICE.

Experience at Covey's summed up,

222

Scene in the Treading Yard,

223

Author taken Ill,

224

Unusual Brutality of Covey,

225

Escape to St Michael's---Suffering in the Woods,

227

Circumstances Narrated to Master Thomas---His Bearing,

229

The Case Prejudged---Driven back to Covey's,

231

\href{/wiki/My_Bondage_and_My_Freedom_(1855)/Chapter_XVII}{Chapter
XVII}.

THE LAST FLOGGING.

A Sleepless Night---Return to Covey's,

233

His Conduct---Again Escape to the Woods,

234

Deplorable Spectacle---Night in the Woods,

235

An Alarm---A Friend, not an Enemy,

236

Sandy's Hospitality---The Ash Cake Supper,

237

A Conjuror---His Advice---The Magic Root,

238

Want of Faith---The Talisman Accepted,

239

Meeting with Covey---His Sunday Face,

240

His Manner on Monday---A Defensive Resolve,

241

A Rough and Tumble Fight,

242

Unexpected Resistance,

243

Covey's Ineffectual Commands for Assistance,

244

The Victory and its Results,

246

Effects upon my own Character,

247

\href{/wiki/My_Bondage_and_My_Freedom_(1855)/Chapter_XVIII}{Chapter
XVIII}.

NEW RELATIONS AND DUTIES.

Change of Masters---Resolve to Fight my Way,

250

Ability to Read a cause of Prejudice,

251

Manner of Spending the Holidays,

252

The Effects---Sharp hit at Slavery,

253

A Device of Slavery,

255

Difference between Master Freeland and Covey,

257

An Irreligious Master Preferred---The Reasons Why,

258

The Reverend Rigby Hopkins,

259

Catalogue of Floggable Offenses,

260

Rivalry among Slaves Encouraged,

261

Improved Condition at Freeland's,

262

Reasons for continued Discontent,

263

Congenial Society---The Sabbath School,

264

Its Members---Necessity for Secrecy,

265

Affectionate Relations of Master and Pupils,

267

Confidence and Friendship among Slaves,

268

Slavery the Inviter of Vengeance,

269

\href{/wiki/My_Bondage_and_My_Freedom_(1855)/Chapter_XIX}{Chapter XIX}.

THE RUNAWAY PLOT.

New Year's Thoughts and Reflections,

271

Again hired by Freeland,

272

Still Devising Plans for gaining Freedom,

273

A Solemn Vow---Plan Divulged to the Slaves,

274

Arguments in its Support---The Scheme gains Favor,

275

Danger of Discovery---Difficulty of Concealment,

276

Skill of Slaveholders---Suspicion and Coercion,

277

Hymns with a Double Meaning,

278

Author's Confederates---His Influence over them,

279

Preliminary Consultations---Pass-Words,

280

Conflict of Hopes and Fears---Ignorance of Geography,

281

Survey of Imaginary Difficulties,

282

Effect upon our Minds,

283

Sandy becomes a Dreamer,

284

Route to the North laid out---Objections Considered,

285

Frauds Practiced on Freemen---Passes Written,

286

Anxieties as the Time drew near,

287

Appeals to Comrades---A Presentiment,

289

The Betrayal Discovered,

290

Manner of Arresting us,

291

Resistance made by Henry Harris---Its Effects,

292

Unique Speech of Mrs. Freeland,

294

Our Sad Procession to Easton,

295

Passes Eaten---The Examination at St. Michael's,

296

No Evidence Produced---Who was the Betrayer?

297

Dragged behind Horses---The Jail a Relief,

298

A New set of Tormentors,

299

Release of my Companions,

300

Author taken out of Prison and sent to Baltimore,

302

\href{/wiki/My_Bondage_and_My_Freedom_(1855)/Chapter_XX}{Chapter XX}.

APPRENTICESHIP LIFE.

Nothing Lost by the Attempt to Run Away,

304

Reasons for sending the Author Away,

305

Unlooked for Clemency in Master Thomas,

306

Return to Baltimore---Change in Little Tommy,

307

Trials in Gardiner's Ship Yard,

308

Desperate Fight with the White Apprentices,

309

Conflict between White and Black Labor,

310

Description of the Outrage,

313

Conduct of Master Hugh,

315

Testimony of a Colored Man Nothing,

316

Spirit of Slavery in Baltimore,

317

Author's Condition Improves,

318

New Associates---Benefits derived therefrom,

319

How to make a Contented Slave,

320

\href{/wiki/My_Bondage_and_My_Freedom_(1855)/Chapter_XXI}{Chapter XXI}.

MY ESCAPE FROM SLAVERY.

Manner of Escape not given---Reasons why,

321

Craftiness and Malice of Slaveholders,

322

Want of Wisdom in Publishing Details of Escape,

324

Suspicions Implied by Master Hugh's Manner,

325

Difficulty of Escape---Discontent,

326

Author allowed to Hire his Time,

327

A Gleam of Hope---Hard Terms,

328

Author attends Camp Meeting without Permission,

329

Anger of Master Hugh thereat,

330

Plans of Escape Accelerated thereby,

332

Painful Thoughts of Separation from Friends,

333

The Attempt made---Its Success,

334

\href{/wiki/My_Bondage_and_My_Freedom_(1855)/Chapter_XXII}{Chapter
XXII}.

LIBERTY ATTAINED.

Author a Wanderer in New York---Feelings on Reaching that City,

{336}

An Old Acquaintance met,

337

Unfavorable Impressions---Loneliness and Insecurity,

338

Apology for Slaves who Return to their Masters,

339

Make known my Condition---David Ruggles,

340

Author's Marriage---Removal to New Bedford,

341

Kindness of Nathan Johnson---Change of Name,

342

Dark Notions of Northern Civilization enlightened,

344

Contrast between the North and the South,

345

Colored People in New Bedford,

346

An Incident Illustrating their Spirit,

347

The Author finds Employment,

348

Denied Work at his Trade,

349

The first Winter at the North,

350

Proscription in the Church,

351

An Incident at the Communion Table,

353

First Acquaintance with the Liberator,

354

Character of its Editor,

355

Prompt Attendance at Anti-Slavery Meetings,

356

\href{/wiki/My_Bondage_and_My_Freedom_(1855)/Chapter_XXIII}{Chapter
XXIII}.

INTRODUCED TO THE ABOLITIONISTS.

Anti-Slavery Convention at Nantucket,

357

Author's First Speech,

358

Becomes a Public Lecturer,

359

Youthful Enthusiasm,

360

Difficulties in his Position,

361

His Fugitive Slaveship Doubted,

362

Publishes his Narrative---Danger of Recapture,

363

Advised not to Publish his Story,

364

\href{/wiki/My_Bondage_and_My_Freedom_(1855)/Chapter_XXIV}{Chapter
XXIV}.

TWENTY-ONE MONTHS IN GREAT BRITAIN.

Good arising out of Unpropitious Events,

365

Embarks for England---Denied Cabin Passage,

366

Mob on board the Cambria---Happy Introduction to the British Public,

367

Letter to Mr. Garrison,

368

``We dont allow Niggers in here,''

371

Time and Labors Abroad,

373

Freedom Purchased---Free Papers,

374

Abolitionists Displeased with the Ransom,

375

How the Author's Energies were Directed in Great Britain,

376

Reception Speech in Finsbury Chapel, London,

377

Character of the Speech Defended,

378

Causes Contributing to my Success,

380

The Free Church of Scotland---Its Position,

381

Agitation of the Slavery Question,

382

Debates in the General Assembly---``Send back the Money,''

383

Dr. Cunningham's Speech---A Striking Incident,

385

The World's Temperance Convention---Collision with Dr. Cox,

387

Proposed Testimonial to the Author,

388

Project of Establishing a Newspaper,

389

Return to America---Again Denied Cabin Passage,

390

\href{/wiki/My_Bondage_and_My_Freedom_(1855)/Chapter_XXV}{Chapter XXV}.

VARIOUS INCIDENTS.

Unexpected Opposition to my Newspaper Enterprise,

392

The Objections to it---Their Plausibility Admitted,

393

Motives for going to Rochester,

395

A Change of Opinions---Causes leading to it,

396

Prejudice against Color---The ``Jim Crow Car,''

399

An Amusing Domestic Scene,

401

The Author in High Company,

403

Elevation of the Free People of Color---Pledge for the Future,

405

\href{/wiki/My_Bondage_and_My_Freedom_(1855)/Appendix}{Appendix}.

EXTRACTS FROM SPEECHES, ETC.

Reception Speech at Finsbury Chapel, Moorfields, England,

407

Letter to his Old Master,

421

The Nature of Slavery,

429

Inhumanity of Slavery,

435

What to the Slave is the Fourth of July?

441

The Internal Slave Trade,

446

The Slavery Party,

451

The Anti-Slavery Movement,

457

\protect\hypertarget{imageLeft}{}{\includegraphics[width=0.50000in,height=0.50000in]{//upload.wikimedia.org/wikipedia/commons/thumb/6/62/PD-icon.svg/48px-PD-icon.svg.png}}

\hypertarget{licFrame-centertext}{}
This work was published before January 1, 1923, and is in the
\textbf{\href{https://en.wikipedia.org/wiki/Public_domain}{public
domain}} worldwide because the author died at least 100 years ago.

\protect\hypertarget{noimageRight}{}{~}

{Public domain}{Public domain}{false}{false}

\begin{longtable}[]{@{}l@{}}
\toprule
\begin{minipage}[t]{0.97\columnwidth}\raggedright\strut
\begin{longtable}[]{@{}ll@{}}
\toprule
\begin{minipage}[t]{0.48\columnwidth}\raggedright\strut
\href{/wiki/Wikisource:Authority_control}{Authority control}\strut
\end{minipage} & \begin{minipage}[t]{0.48\columnwidth}\raggedright\strut
\begin{itemize}
\tightlist
\item
  \href{https://en.wikipedia.org/wiki/OCLC}{OCLC}:~{\href{http://www.worldcat.org/oclc/80238595?lang=en}{80238595}}
\item
  \href{https://en.wikipedia.org/wiki/Internet_Archive}{Internet
  Archive}:~{\href{http://www.archive.org/details/mybondagemyfreed00douguoft}{mybondagemyfreed00douguoft}}
\item
  ~{\href{/wiki/Wikisource:Authority_control}{English Wikisource}:
  \href{//en.wikisource.org/w/index.php?curid=2335323}{2335323}}
\item
  ~{\href{http://www.worldcat.org/oclc/80238595}{WorldCat}}
\end{itemize}\strut
\end{minipage}\tabularnewline
\bottomrule
\end{longtable}\strut
\end{minipage}\tabularnewline
\bottomrule
\end{longtable}
