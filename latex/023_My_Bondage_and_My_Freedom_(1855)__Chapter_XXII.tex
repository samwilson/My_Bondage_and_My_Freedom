{}

\href{/wiki/File:Frederickdouglaspg346.png}{\includegraphics[width=4.37500in,height=6.93750in]{//upload.wikimedia.org/wikipedia/commons/thumb/3/3a/Frederickdouglaspg346.png/420px-Frederickdouglaspg346.png}}

{\protect\hypertarget{ux5cux7bux5cux7bux5cux7b1ux5cux7dux5cux7dux5cux7d}{}{}}

{}

~

{LIFE AS A FREEMAN.}

\begin{longtable}[]{@{}lll@{}}
\toprule
\includegraphics[width=0.41667in,height=0.01042in]{//upload.wikimedia.org/wikipedia/commons/thumb/b/b2/Rule_Segment_-_Span_-_40px.svg/40px-Rule_Segment_-_Span_-_40px.svg.png}
&
\includegraphics[width=0.06250in,height=0.07292in]{//upload.wikimedia.org/wikipedia/commons/thumb/2/28/Rule_Segment_-_Circle_-_6px.svg/6px-Rule_Segment_-_Circle_-_6px.svg.png}
&
\includegraphics[width=0.41667in,height=0.01042in]{//upload.wikimedia.org/wikipedia/commons/thumb/b/b2/Rule_Segment_-_Span_-_40px.svg/40px-Rule_Segment_-_Span_-_40px.svg.png}\tabularnewline
\bottomrule
\end{longtable}

{CHAPTER XXII.}

LIBERTY ATTAINED.

{TRANSITION FROM SLAVERY TO FREEDOM---A WANDERER IN NEW YORK---FEELINGS
ON REACHING THAT CITY---AN OLD ACQUAINTANCE MET---UNFAVORABLE
IMPRESSIONS---LONELINESS AND INSECURITY---APOLOGY FOR SLAVES WHO RETURN
TO THEIR MASTERS---COMPELLED TO TELL MY CONDITION---SUCCORED BY A
SAILOR---DAVID RUGGLES---THE UNDER-GROUND RAILROAD---MARRIAGE---BAGGAGE
TAKEN FROM ME---KINDNESS OF NATHAN JOHNSON---THE AUTHOR'S CHANGE OF
NAME---DARK NOTIONS OF NORTHERN CIVILIZATION---THE CONTRAST---COLORED
PEOPLE IN NEW BEDFORD---AN INCIDENT ILLUSTRATING THEIR SPIRIT---THE
AUTHOR AS A COMMON LABORER---DENIED WORK AT HIS TRADE---THE FIRST WINTER
AT THE NORTH---REPULSE AT THE DOORS OF THE CHURCH---SANCTIFIED
HATE---THE LIBERATOR AND ITS EDITOR.}

\textsc{There} is no necessity for any extended notice of the incidents
of this part of my life. There is nothing very striking or peculiar
about my career as a freeman, when viewed apart from my life as a slave.
The relation subsisting between my early experience and that which I am
now about to narrate, is, perhaps, my best apology for adding another
chapter to this book.

Disappearing from the kind reader, in a flying cloud or balloon, (pardon
the figure,) driven by the {}wind, and knowing not where I should
land---whether in slavery or in freedom---it is proper that I should
remove, at once, all anxiety, by frankly making known where I alighted.
The flight was a bold and perilous one; but here I am, in the great city
of New York, safe and sound, without loss of blood or bone. In less than
a week after leaving Baltimore, I was walking amid the hurrying throng,
and gazing upon the dazzling wonders of Broadway. The dreams of my
childhood and the purposes of my manhood were now fulfilled. A free
state around me, and a free earth under my feet! What a moment was this
to me! A whole year was pressed into a single day. A new world burst
upon my agitated vision. I have often been asked, by kind friends to
whom I have told my story, how I felt when first I found myself beyond
the limits of slavery; and I must say here, as I have often said to
them, there is scarcely anything about which I could not give a more
satisfactory answer. It was a moment of joyous excitement, which no
words can describe. In a letter to a friend, written soon after reaching
New York, I said I felt as one might be supposed to feel, on escaping
from a den of hungry lions. But, in a moment like that, sensations are
too intense and too rapid for words. Anguish and grief, like darkness
and rain, may be described, but joy and gladness, like the rainbow of
promise, defy alike the pen and pencil.

For ten or fifteen years I had been dragging a heavy chain, with a huge
block attached to it, cumbering my every motion. I had felt myself
doomed {}to drag this chain and this block through life. All efforts,
before, to separate myself from the hateful encumbrance, had only seemed
to rivet me the more firmly to it. Baffled and discouraged at times, I
had asked myself the question, May not this, after all, be God's work?
May He not, for wise ends, have doomed me to this lot? A contest had
been going on in my mind for years, between the clear consciousness of
right and the plausible errors of superstition; between the wisdom of
manly courage, and the foolish weakness of timidity. The contest was now
ended; the chain was severed; God and right stood vindicated. \textsc{I
was a freeman}, and the voice of peace and joy thrilled my heart.

Free and joyous, however, as I was, joy was not the only sensation I
experienced. It was like the quick blaze, beautiful at the first, but
which subsiding, leaves the building charred and desolate. I was soon
taught that I was still in an enemy's land. A sense of loneliness and
insecurity oppressed me sadly. I had been but a few hours in New York,
before I was met in the streets by a fugitive slave, well known to me,
and the information I got from him respecting New York, did nothing to
lessen my apprehension of danger. The fugitive in question was
``Allender's Jake,'' in Baltimore; but, said he, I am "\textsc{William
Dixon}," in New York! I knew Jake well, and knew when Tolly Allender and
Mr. Price (for the latter employed Master Hugh as his foreman, in his
shipyard on Fell's Point) made an attempt to recapture Jake, and failed.
Jake told me all about his circumstances, and how narrowly he {}escaped
being taken back to slavery; that the city was now full of southerners,
returning from the springs; that the black people in New York were not
to be trusted; that there were hired men on the lookout for fugitives
from slavery, and who, for a few dollars, would betray me into the hands
of the slave-catchers; that I must trust no man with my secret; that I
must not think of going either on the wharves to work, or to a
boarding-house to board; and, worse still, this same Jake told me it was
not in his power to help me. He seemed, even while cautioning me, to be
fearing lest, after all, I might be a party to a second attempt to
recapture him. Under the inspiration of this thought, I must suppose it
was, he gave signs of a wish to get rid of me, and soon left me---his
whitewash brush in hand---as he said, for his work. He was soon lost to
sight among the throng, and I was alone again, an easy prey to the
kidnappers, if any should happen to be on my track.

New York, seventeen years ago, was less a place of safety for a runaway
slave than now, and all know how unsafe it now is, under the new
fugitive slave bill. I was much troubled. I had very little
money---enough to buy me a few loaves of bread, but not enough to pay
board, outside a lumber yard. I saw the wisdom of keeping away from the
ship yards, for if Master Hugh pursued me, he would naturally expect to
find me looking for work among the calkers. For a time, every door
seemed closed against me. A sense of my loneliness and helplessness
crept over me, and covered me with something bordering on despair. In
the midst of thousands of my {}fellowmen, and yet a perfect stranger! In
the midst of human brothers, and yet more fearful of them than of hungry
wolves! I was without home, without friends, without work, without
money, and without any definite knowledge of which way to go, or where
to look for succor.

Some apology can easily be made for the few slaves who have, after
making good their escape, turned back to slavery, preferring the actual
rule of their masters, to the life of loneliness, apprehension, hunger,
and anxiety, which meets them on their first arrival in a free state. It
is difficult for a freeman to enter into the feelings of such fugitives.
He cannot see things in the same light with the slave, because he does
not, and cannot, look from the same point from which the slave does.
``Why do you tremble,'' he says to the slave---``you are in a free
state;'' but the difficulty is, in realizing that he is in a free state,
the slave might reply. A freeman cannot understand why the
slave-master's shadow is bigger, to the slave, than the might and
majesty of a free state; but when he reflects that the slave knows more
about the slavery of his master than he does of the might and majesty of
the free state, he has the explanation. The slave has been all his life
learning the power of his master---being trained to dread his
approach---and only a few hours learning the power of the state. The
master is to him a stern and flinty reality, but the state is little
more than a dream. He has been accustomed to regard every white man as
the friend of his master, and every colored man as more or less under
the control of his {}master's friends---the white people. It takes stout
nerves to stand up, in such circumstances. A man, homeless, shelterless,
breadless, friendless, and moneyless, is not in a condition to assume a
very proud or joyous tone; and in just this condition was I, while
wandering about the streets of New York city, and lodging, at least one
night, among the barrels on one of its wharves. I was not only free from
slavery, but I was free from home, as well. The reader will easily see
that I had something more than the simple fact of being free to think
of, in this extremity.

I kept my secret as long as I could, and at last was forced to go in
search of an honest man---a man sufficiently \emph{human} not to betray
me into the hands of slave-catchers. I was not a bad reader of the human
face, nor long in selecting the right man, when once compelled to
disclose the facts of my condition to some one.

I found my man in the person of one who said his name was Stewart. He
was a sailor, warm-hearted and generous, and he listened to my story
with a brother's interest. I told him I was running for my
freedom---knew not where to go---money almost gone---was
hungry---thought it unsafe to go the shipyards for work, and needed a
friend. Stewart promptly put me in the way of getting out of my trouble.
He took me to his house, and went in search of the late David Ruggles,
who was then the secretary of the New York Vigilance Committee, and a
very active man in all anti-slavery works. Once in the hands of Mr.
Ruggles, I was comparatively safe. I was {}hidden with Mr. Ruggles
several days. In the meantime, my intended wife, Anna, came on from
Baltimore---to whom I had written, informing her of my safe arrival at
New York---and, in the presence of Mrs. Mitchell and Mr. Ruggles, we
were married, by Rev. James W. C. Pennington.

Mr. Ruggles\textsuperscript{\protect\hyperlink{cite_note-1}{{[}1{]}}}
was the first officer on the under-ground railroad with whom I met after
reaching the north, and, indeed, the first of whom I ever heard
anything. Learning that I was a calker by trade, he promptly decided
that New Bedford was the proper place to send me. ``Many ships,'' said
he, ``are there fitted out for the whaling business, and you may there
find work at your trade, and make a good living.'' Thus, in one
fortnight after my night from Maryland, I was safe in New Bedford,
regularly entered upon the exercise of the rights, responsibilities, and
duties of a freeman.

I may mention a little circumstance which annoyed me on reaching New
Bedford. I had not a cent of money, and lacked two dollars toward paying
our fare from Newport, and our baggage---not very {}costly---was taken
by the stage driver, and held until I could raise the money to redeem
it. This difficulty was soon surmounted. Mr. Nathan Johnson, to whom we
had a line from Mr. Ruggles, not only received us kindly and hospitably,
but, on being informed about our baggage, promptly loaned me two dollars
with which to redeem my little property. I shall ever be deeply
grateful, both to Mr. and Mrs. Nathan Johnson, for the lively interest
they were pleased to take in me, in this the hour of my extremest need.
They not only gave myself and wife bread and shelter, but taught us how
to begin to secure those benefits for ourselves. Long may they live, and
may blessings attend them in this life and in that which is to come!

Once initiated into the new life of freedom, and assured by Mr. Johnson
that New Bedford was a safe place, the comparatively unimportant matter,
as to what should be my name, came up for consideration. It was
necessary to have a name in my new relations. The name given me by my
beloved mother was no less pretentious than ``Frederick Augustus
Washington Bailey.'' I had, however, before leaving Maryland, dispensed
with the \emph{Augustus Washington}, and retained the name
\emph{Frederick Bailey}. Between Baltimore and New Bedford, however, I
had several different names, the better to avoid being overhauled by the
hunters, which I had good reason to believe would be put on my track.
Among honest men an honest man may well be content with one name, and to
acknowledge it at all times and in all places; but toward fugitives,
Americans are not {}honest. When I arrived at New Bedford, my name was
Johnson; and finding that the Johnson family in New Bedford were already
quite numerous---sufficiently so to produce some confusion in attempts
to distinguish one from another---there was the more reason for making
another change in my name. In fact, ``Johnson'' had been assumed by
nearly every slave who had arrived in New Bedford from Maryland, and
this, much to the annoyance of the original ``Johnsons'' (of whom there
were many) in that place. Mine host, unwilling to have another of his
own name added to the community in this unauthorized way, after I spent
a night and a day at his house, gave me my present name. He had been
reading the "\href{/wiki/The_Lady_of_the_Lake}{Lady of the Lake}," and
was pleased to regard me as a suitable person to wear this, one of
Scotland's many famous names. Considering the noble hospitality and
manly character of Nathan Johnson, I have felt that he, better than I,
illustrated the virtues of the great Scottish chief. Sure I am, that had
any slave-catcher entered his domicile, with a view to molest any one of
his household, he would have shown himself like him of the ``stalwart
hand.''

The reader will be amused at my ignorance, when I tell the notions I had
of the state of northern wealth, enterprise, and civilization. Of wealth
and refinement, I supposed the north had none. My Columbian Orator,
which was almost my only book, had not done much to enlighten me
concerning northern society. The impressions I had received were all
wide of the truth. New Bedford, especially, took me by surprise, in the
solid wealth and grandeur there {}exhibited. I had formed my notions
respecting the social condition of the free states, by what I had seen
and known of free, white, non-slaveholding people in the slave states.
Regarding slavery as the basis of wealth, I fancied that no people could
become very wealthy without slavery. A free white man, holding no
slaves, in the country, I had known to be the most ignorant and
poverty-stricken of men, and the laughing stock even of slaves
themselves---called generally by them, in derision, "\emph{poor white
trash}." Like the non-slaveholders at the south, in holding no slaves, I
supposed the northern people like them, also, in poverty and
degradation. Judge, then, of my amazement and joy, when I found---as I
did find---the very laboring population of New Bedford living in better
houses, more elegantly furnished---surrounded by more comfort and
refinement---than a majority of the slaveholders on the Eastern Shore of
Maryland. There was my friend, Mr. Johnson, himself a colored man, (who
at the south would have been regarded as a proper marketable commodity,)
who lived in a better house---dined at a richer board---was the owner of
more books---the reader of more newspapers---was more conversant with
the political and social condition of this nation and the world---than
nine-tenths of all the slaveholders of Talbot county, Maryland. Yet Mr.
Johnson was a working man, and his hands were hardened by honest toil.
Here, then, was something for observation and study. Whence the
difference? The explanation was soon furnished, in the superiority of
mind over simple brute force. Many pages might be given to the contrast,
and in {}explanation of its causes. But an incident or two will suffice
to show the reader as to how the mystery gradually vanished before me.

My first afternoon, on reaching New Bedford, was spent in visiting the
wharves and viewing the shipping. The sight of the broad brim and the
plain, Quaker dress, which met me at every turn, greatly increased my
sense of freedom and security. ``I am among the Quakers,'' thought I,
``and am safe.'' Lying at the wharves and riding in the stream, were
full-rigged ships of finest model, ready to start on whaling voyages.
Upon the right and the left, I was walled in by large granite-fronted
warehouses, crowded with the good things of this world. On the wharves,
I saw industry without bustle, labor without noise, and heavy toil
without the whip. There was no loud singing, as in southern ports, where
ships are loading or unloading---no loud cursing or swearing---but
everything went on as smoothly as the works of a well adjusted machine.
How different was all this from the noisily fierce and clumsily absurd
manner of labor-life in Baltimore and St. Michael's! One of the first
incidents which illustrated the superior mental character of northern
labor over that of the south, was the manner of unloading a ship's cargo
of oil. In a southern port, twenty or thirty hands would have been
employed to do what five or six did here, with the aid of a single ox
attached to the end of a fall. Main strength, unassisted by skill, is
slavery's method of labor. An old ox, worth eighty dollars, was doing,
in New Bedford, what would have required fifteen thousand dollars
{}worth of human bones and muscles to have performed in a southern port.
I found that everything was done here with a scrupulous regard to
economy, both in regard to men and things, time and strength. The maid
servant, instead of spending at least a tenth part of her time in
bringing and carrying water, as in Baltimore, had the pump at her elbow.
The wood was dry, and snugly piled away for winter. Wood-houses, in-door
pumps, sinks, drains, self-shutting gates, washing machines, pounding
barrels, were all new things, and told me that I was among a thoughtful
and sensible people. To the ship-repairing dock I went, and saw the same
wise prudence. The carpenters struck where they aimed, and the calkers
wasted no blows in idle nourishes of the mallet. I learned that men went
from New Bedford to Baltimore, and bought old ships, and brought them
here to repair, and made them better and more valuable than they ever
were before. Men talked here of going whaling on a four \emph{years{'}}
voyage with more coolness than sailors where I came from talked of going
a four \emph{months{'}} voyage.

I now find that I could have landed in no part of the United States,
where I should have found a more striking and gratifying contrast to the
condition of the free people of color in Baltimore, than I found here in
New Bedford. No colored man is really free in a slaveholding state. He
wears the badge of bondage while nominally free, and is often subjected
to hardships to which the slave is a stranger; but here in New Bedford,
it was my good fortune to see a pretty near approach to freedom on the
part of the {}colored people. I was taken all aback when Mr.
Johnson---who lost no time in making me acquainted with the fact---told
me that there was nothing in the constitution of Massachusetts to
prevent a colored man from holding any office in the state. There, in
New Bedford, the black man's children---although anti-slavery was then
far from popular---went to school side by side with the white children,
and apparently without objection from any quarter. To make me at home,
Mr. Johnson assured me that no slaveholder could take a slave from New
Bedford; that there were men there who would lay down their lives,
before such an outrage could be perpetrated. The colored people
themselves were of the best metal, and would fight for liberty to the
death.

Soon after my arrival in New Bedford, I was told the following story,
which was said to illustrate the spirit of the colored people in that
goodly town: A colored man and a fugitive slave happened to have a
little quarrel, and the former was heard to threaten the latter with
informing his master of his whereabouts. As soon as this threat became
known, a notice was read from the desk of what was then the only colored
church in the place, stating that business of importance was to be then
and there transacted. Special measures had been taken to secure the
attendance of the would-be Judas, and had proved successful.
Accordingly, at the hour appointed, the people came, and the betrayer
also. All the usual formalities of public meetings were scrupulously
gone through, even to the offering prayer for Divine direction in the
duties of the occasion. The president {}himself performed this part of
the ceremony, and I was told that he was unusually fervent. Yet, at the
close of his prayer, the old man (one of the numerous family of
Johnsons) rose from his knees, deliberately surveyed his audience, and
then said, in a tone of solemn resolution, "\emph{Well, friends, we have
got him here, and I would now recommend that you young men should just
take him outside the door and kill him}." With this, a large body of the
congregation, who well understood the business they had come there to
transact, made a rush at the villain, and doubtless would have killed
him, had he not availed himself of an open sash, and made good his
escape. He has never shown his head in New Bedford since that time. This
little incident is perfectly characteristic of the spirit of the colored
people in New Bedford. A slave could not be taken from that town
seventeen years ago, any more than he could be so taken away now. The
reason is, that the colored people in that city are educated up to the
point of fighting for their freedom, as well as speaking for it.

Once assured of my safety in New Bedford, I put on the habiliments of a
common laborer, and went on the wharf in search of work. I had no notion
of living on the honest and generous sympathy of my colored brother,
Johnson, or that of the abolitionists. My cry was like that of
\href{/wiki/Author:Thomas_Hood}{Hood's} laborer, ``Oh! only give me
work.'' Happily for me, I was not long in searching. I found employment,
the third day after my arrival in New Bedford, in stowing a sloop with a
load of oil for the New York market. It was new, hard, and dirty work,
even for a calker, but I went {}at it with a glad heart and a willing
hand. I was now my own master---a tremendous fact---and the rapturous
excitement with which I seized the job, may not easily be understood,
except by some one with an experience something like mine. The
thoughts---``I can work! I can work for a living; I am not afraid of
work; I have no Master Hugh to rob me of my earnings''---placed me in a
state of independence, beyond seeking friendship or support of any man.
That day's work I considered the real starting point of something like a
new existence. Having finished this job and got my pay for the same, I
went next in pursuit of a job at calking. It so happened that Mr. Rodney
French, late mayor of the city of New Bedford, had a ship fitting out
for sea, and to which there was a large job of calking and coppering to
be done. I applied to that noble-hearted man for employment, and he
promptly told me to go to work; but going on the float-stage for the
purpose, I was informed that every white man would leave the ship if I
struck a blow upon her. ``Well, well,'' thought I, ``this is a hardship,
but yet not a very serious one for me.'' The difference between the
wages of a calker and that of a common day laborer, was an hundred per
cent. in favor of the former; but then I was free, and free to work,
though not at my trade. I now prepared myself to do anything which came
to hand in the way of turning an honest penny; sawed wood---dug
cellars---shoveled coal---swept chimneys with Uncle Lucas
Debuty---rolled oil casks on the wharves---helped to load and unload
vessels---worked in Ricketson's candle {}works---in Richmond's brass
foundery, and elsewhere; and thus supported myself and family for three
years.

The first winter was unusually severe, in consequence of the high prices
of food; but even during that winter we probably suffered less than many
who had been free all their lives. During the hardest of the winter, I
hired out for nine dollars a month; and out of this rented two rooms for
nine dollars per quarter, and supplied my wife---who was unable to
work---with food and some necessary articles of furniture. We were
closely pinched to bring our wants within our means; but the jail stood
over the way, and I had a wholesome dread of the consequences of running
in debt. This winter past, and I was up with the times---got plenty of
work---got well paid for it---and felt that I had not done a foolish
thing to leave Master Hugh and Master Thomas. I was now living in a new
world, and was wide awake to its advantages. I early began to attend the
meetings of the colored people of New Bedford, and to take part in them.
I was somewhat amazed to see colored men drawing up resolutions and
offering them for consideration. Several colored young men of New
Bedford, at that period, gave promise of great usefulness. They were
educated, and possessed what seemed to me, at that time, very superior
talents. Some of them have been cut down by death, and others have
removed to different parts of the world, and some remain there now, and
justify, in their present activities, my early impressions of them.

Among my first concerns on reaching New Bedford, was to become united
with the church, for I had {}never given up, in reality, my religious
faith. I had become lukewarm and in a backslidden state, but I was still
convinced that it was my duty to join the Methodist church. I was not
then aware of the powerful influence of that religious body in favor of
the enslavement of my race, nor did I see how the northern churches
could be responsible for the conduct of southern churches; neither did I
fully understand how it could be my duty to remain separate from the
church, because bad men were connected with it. The slaveholding church,
with its Coveys, Weedens, Aulds, and Hopkins, I could see through at
once, but I could not see how Elm Street church, in New Bedford, could
be regarded as sanctioning the christianity of these characters in the
church at St. Michael's. I therefore resolved to join the Methodist
church in New Bedford, and to enjoy the spiritual advantage of public
worship. The minister of the Elm Street Methodist church, was the Rev.
Mr. Bonney; and although I was not allowed a seat in the body of the
house, and was proscribed on account of my color, regarding this
proscription simply as an accommodation of the unconverted congregation
who had not yet been won to Christ and his brotherhood, I was willing
thus to be proscribed, lest sinners should be driven away from the
saving power of the gospel. Once converted, I thought they would be sure
to treat me as a man and a brother. ``Surely,'' thought I, ``these
christian people have none of this feeling against color. They, at
least, have renounced this unholy feeling.'' Judge, then, dear reader,
of my astonishment and mortification, when I found, as {}soon I did
find, all my charitable assumptions at fault.

An opportunity was soon afforded me for ascertaining the exact position
of Elm Street church on that subject. I had a chance of seeing the
religious part of the congregation by themselves; and although they
disowned, in effect, their black brothers and sisters, before the world,
I did think that where none but the saints were assembled, and no
offense could be given to the wicked, and the gospel could not be
``blamed,'' they would certainly recognize us as children of the same
Father, and heirs of the same salvation, on equal terms with themselves.

The occasion to which I refer, was the sacrament of the Lord's Supper,
that most sacred and most solemn of all the ordinances of the christian
church. Mr. Bonney had preached a very solemn and searching discourse,
which really proved him to be acquainted with the inmost secrets of the
human heart. At the close of his discourse, the congregation was
dismissed, and the church remained to partake of the sacrament. I
remained to see, as I thought, this holy sacrament celebrated in the
spirit of its great Founder.

There were only about a half dozen colored members attached to the Elm
Street church, at this time. After the congregation was dismissed, these
descended from the gallery, and took a seat against the wall most
distant from the altar. Brother Bonney was very animated, and sung very
sweetly, ``Salvation 'tis a joyful sound,'' and soon began to administer
the sacrament. I was anxious to observe the {}bearing of the colored
members, and the result was most humiliating. During the whole ceremony,
they looked like sheep without a shepherd. The white members went
forward to the altar by the bench full; and when it was evident that all
the whites had been served with the bread and wine, Brother
Bonney---pious Brother Bonney---after a long pause, as if inquiring
whether all the white members had been served, and fully assuring
himself on that important point, then raised his voice to an unnatural
pitch, and looking to the corner where his black sheep seemed penned,
beckoned with his hand, exclaiming, ``Come forward, colored
friends!---come forward! You, too, have an interest in the blood of
Christ. God is no respecter of persons. Come forward, and take this holy
sacrament to your comfort.'' The colored members---poor, slavish
souls---went forward, as invited. I went \emph{out}, and have never been
in that church since, although I honestly went there with a view to
joining that body. I found it impossible to respect the religious
profession of any who were under the dominion of this wicked prejudice,
and I could not, therefore, feel that in joining them, I was joining a
christian church, at all. I tried other churches in New Bedford, with
the same result, and, finally, I attached myself to a small body of
colored Methodists, known as the Zion Methodists. Favored with the
affection and confidence of the members of this humble communion, I was
soon made a class-leader and a local preacher among them. Many seasons
of peace and joy I experienced among them, the remembrance of which is
still precious, although {}I could not see it to be my duty to remain
with that body, when I found that it consented to the same spirit which
held my brethren in chains.

In four or five months after reaching New Bedford, there came a young
man to me, with a copy of the ``Liberator,'' the paper edited by
\href{/wiki/Author:William_Lloyd_Garrison}{\textsc{William Lloyd
Garrison}}, and published by \textsc{Isaac Knapp}, and asked me to
subscribe for it. I told him I had but just escaped from slavery, and
was of course very poor, and remarked further, that I was unable to pay
for it then; the agent, however, very willingly took me as a subscriber,
and appeared to be much pleased with securing my name to his list. From
this time I was brought in contact with the mind of William Lloyd
Garrison. His paper took its place with me next to the bible.

The Liberator was a paper after my own heart. It detested
slavery---exposed hypocrisy and wickedness in high places---made no
truce with the traffickers in the bodies and souls of men; it preached
human brotherhood, denounced oppression, and, with all the solemnity of
God's word, demanded the complete emancipation of my race. I not only
liked---I \emph{loved} this paper, and its editor. He seemed a match for
all the opponents of emancipation, whether they spoke in the name of the
law, or the gospel. His words were few, full of holy fire, and straight
to the point. Learning to love him, through his paper, I was prepared to
be pleased with his presence. Something of a hero worshiper, by nature,
here was one, on first sight, to excite my love and reverence.

Seventeen years ago, few men possessed a more {}heavenly countenance
than William Lloyd Garrison, and few men evinced a more genuine or a
more exalted piety. The bible was his text book---held sacred, as the
word of the Eternal Father---sinless perfection---complete submission to
insults and injuries---literal obedience to the injunction, if smitten
on one side to turn the other also. Not only was Sunday a Sabbath, but
all days were Sabbaths, and to be kept holy. All sectarism false and
mischievous---the regenerated, throughout the world, members of one
body, and the \textsc{Head} Christ Jesus. Prejudice against color was
rebellion against God. Of all men beneath the sky, the slaves, because
most neglected and despised, were nearest and dearest to his great
heart. Those ministers who defended slavery from the bible, were of
their ``father the devil;'' and those churches which fellowshiped
slaveholders as christians, were synagogues of Satan, and our nation was
a nation of liars. Never loud or noisy---calm and serene as a summer
sky, and as pure. ``You are the man, the Moses, raised up by God, to
deliver his modern Israel from bondage,'' was the spontaneous feeling of
my heart, as I sat away back in the hall and listened to his mighty
words; mighty in truth---mighty in their simple earnestness.

I had not long been a reader of the Liberator, and listener to its
editor, before I got a clear apprehension of the principles of the
anti-slavery movement. I had already the spirit of the movement, and
only needed to understand its principles and measures. These I got from
the Liberator, and from those who believed in that paper. My
acquaintance with the {}movement increased my hope for the ultimate
freedom of my race, and I united with it from a sense of delight, as
well as duty.

Every week the Liberator came, and every week I made myself master of
its contents. All the anti-slavery meetings held in New Bedford I
promptly attended, my heart burning at every true utterance against the
slave system, and every rebuke of its friends and supporters. Thus
passed the first three years of my residence in New Bedford. I had not
then dreamed of the possibility of my becoming a public advocate of the
cause so deeply imbedded in my heart. It was enough for me to
listen---to receive and applaud the great words of others, and only
whisper in private, among the white laborers on the wharves, and
elsewhere, the truths which burned in my breast.

~

\begin{center}\rule{0.5\linewidth}{\linethickness}\end{center}

\begin{enumerate}
\item
  \hypertarget{cite_note-1}{}

  {\protect\hyperlink{cite_ref-1}{↑}} {He was a whole-souled man, fully
  imbued with a love of his afflicted and hunted people, and took
  pleasure in being to me, as was his wont, ``Eyes to the blind, and
  legs to the lame.'' This brave and devoted man suffered much from the
  persecutions common to all who have been prominent benefactors. He at
  last became blind, and needed a friend to guide him, even as he had
  been a guide to others. Even in his blindness, he exhibited his manly
  character. In search of health, he became a physician. When hope of
  gaining his own was gone, he had hope for others. Believing in
  hydropathy, he established, at Northampton, Massachusetts, a large
  "\emph{Water Cure}," and became one of the most successful of all
  engaged in that mode of treatment.}
\end{enumerate}
