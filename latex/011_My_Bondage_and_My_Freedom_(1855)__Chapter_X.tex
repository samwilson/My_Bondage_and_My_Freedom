{}

~

{CHAPTER X.}

LIFE IN BALTIMORE.

{CITY ANNOYANCES---PLANTATION REGRETS---MY MISTRESS, MISS SOPHA---HER
HISTORY---HER KINDNESS TO ME---MY MASTER, HUGH AULD---HIS SOURNESS---MY
INCREASED SENSITIVENESS---MY COMFORTS---MY OCCUPATION---THE BANEFUL
EFFECTS OF SLAVEHOLDING ON MY DEAR AND GOOD MISTRESS---HOW SHE COMMENCED
TEACHING ME TO READ---WHY SHE CEASED TEACHING ME---CLOUDS GATHERING OVER
MY BRIGHT PROSPECTS---MASTER AULD'S EXPOSITION OF THE TRUE PHILOSOPHY OF
SLAVERY---CITY SLAVES---PLANTATION SLAVES---THE
CONTRAST---EXCEPTIONS---MR. HAMILTON'S TWO SLAVES, HENRIETTA AND
MARY---MRS. HAMILTON'S CRUEL TREATMENT OF THEM---THE PITEOUS ASPECT THEY
PRESENTED---NO POWER MUST COME BETWEEN THE SLAVE AND THE SLAVEHOLDER.}

\textsc{Once} in Baltimore, with hard brick pavements under my feet,
which almost raised blisters, by their very heat, for it was in the
height of summer; walled in on all sides by towering brick buildings;
with troops of hostile boys ready to pounce upon me at every street
corner; with new and strange objects glaring upon me at every step, and
with startling sounds reaching my ears from all directions, I for a time
thought that, after all, the home plantation was a more desirable place
of residence than my home on Alliciana street, in Baltimore. My country
eyes and ears were confused and bewildered here; but the boys were my
chief trouble. They chased me, and called me "\emph{Eastern Shore man},"
till really I almost wished myself back on the Eastern Shore. I had to
undergo a sort {}of moral acclimation, and when that was over, I did
much better. My new mistress happily proved to be all she \emph{seemed}
to be, when, with her husband, she met me at the door, with a most
beaming, benignant countenance. She was, naturally, of an excellent
disposition, kind, gentle and cheerful. The supercilious contempt for
the rights and feelings of the slave, and the petulance and bad humor
which generally characterize slaveholding ladies, were all quite absent
from kind ``Miss'' Sophia's manner and bearing toward me. She had, in
truth, never been a slaveholder, but had---a thing quite unusual in the
south---depended almost entirely upon her own industry for a living. To
this fact the dear lady, no doubt, owed the excellent preservation of
her natural goodness of heart, for slavery can change a saint into a
sinner, and an angel into a demon. I hardly knew how to behave toward
``Miss Sopha,'' as I used to call Mrs. Hugh Auld. I had been treated as
a \emph{pig} on the plantation; I was treated as a \emph{child} now. I
could not even approach her as I had formerly approached Mrs. Thomas
Auld. How could I hang down my head, and speak with bated breath, when
there was no pride to scorn me, no coldness to repel me, and no hatred
to inspire me with fear? I therefore soon learned to regard her as
something more akin to a mother, than a slaveholding mistress. The
crouching servility of a slave, usually so acceptable a quality to the
haughty slaveholder, was not understood nor desired by this gentle
woman. So far from deeming it impudent in a slave to look her straight
in the face, as some slaveholding ladies do, she seemed ever to say,
"look up, {}child; don't be afraid; see, I am full of kindness and good
will toward you." The hands belonging to Col. Lloyd's sloop, esteemed it
a great privilege to be the bearers of parcels or messages to my new
mistress; for whenever they came, they were sure of a most kind and
pleasant reception. If little Thomas was her son, and her most dearly
beloved child, she, for a time, at least, made me something like his
half-brother in her affections. If dear Tommy was exalted to a place on
his mother's knee, ``Feddy'' was honored by a place at his mother's
side. Nor did he lack the caressing strokes of her gentle hand, to
convince him that, though \emph{motherless,} he was not
\emph{friendless.} Mrs. Auld was not only a kind-hearted woman, but she
was remarkably pious; frequent in her attendance of public worship, much
given to reading the bible, and to chanting hymns of praise, when alone.
Mr. Hugh Auld was altogether a different character. He cared very little
about religion, knew more of the world, and was more of the world, than
his wife. He set out, doubtless, to be---as the world goes---a
respectable man, and to get on by becoming a successful ship builder, in
that city of ship building. This was his ambition, and it fully occupied
him. I was, of course, of very little consequence to him, compared with
what I was to good Mrs. Auld; and, when he smiled upon me, as he
sometimes did, the smile was borrowed from his lovely wife, and, like
all borrowed light, was transient, and vanished with the source whence
it was derived. While I must characterize Master Hugh as being a very
sour man, and of forbidding appearance, it is due to him to acknowledge,
that he was never {}very cruel to me, according to the notion of cruelty
in Maryland. The first year or two which I spent in his house, he left
me almost exclusively to the management of his wife. She was my
law-giver. In hands so tender as hers, and in the absence of the
cruelties of the plantation, I became, both physically and mentally,
much more sensitive to good and ill treatment; and, perhaps, suffered
more from a frown from my mistress, than I formerly did from a cuff at
the hands of Aunt Katy. Instead of the cold, damp floor of my old
master's kitchen, I found myself on carpets; for the corn bag in winter,
I now had a good straw bed, well furnished with covers; for the coarse
corn-meal in the morning, I now had good bread, and mush occasionally;
for my poor tow-linen shirt, reaching to my knees, I had good, clean
clothes. I was really well off. My employment was to run of errands, and
to take care of Tommy; to prevent his getting in the way of carriages,
and to keep him out of harm's way generally. Tommy, and I, and his
mother, got on swimmingly together, for a time. I say \emph{for a time,}
because the fatal poison of irresponsible power, and the natural
influence of slavery customs, were not long in making a suitable
impression on the gentle and loving disposition of my excellent
mistress. At first, Mrs. Auld evidently regarded me simply as a child,
like any other child; she had not come to regard me as \emph{property.}
This latter thought was a thing of conventional growth. The first was
natural and spontaneous. A noble nature, like hers, could not,
instantly, be wholly perverted; and it took several years to change the
natural sweetness of her {}temper into fretful bitterness. In her worst
estate, however, there were, during the first seven years I lived with
her, occasional returns of her former kindly disposition.

The frequent hearing of my mistress reading the bible---for she often
read aloud when her husband was absent---soon awakened my curiosity in
respect to this \emph{mystery} of reading, and roused in me the desire
to learn. Having no fear of my kind mistress before my eyes, (she had
then given me no reason to fear,) I frankly asked her to teach me to
read; and, without hesitation, the dear woman began the task, and very
soon, by her assistance, I was master of the alphabet, and could spell
words of three or four letters. My mistress seemed almost as proud of my
progress, as if I had been her own child; and, supposing that her
husband would be as well pleased, she made no secret of what she was
doing for me. Indeed, she exultingly told him of the aptness of her
pupil, of her intention to persevere in teaching me, and of the duty
which she felt it to teach me, at least to read \emph{the bible.} Here
arose the first cloud over my Baltimore prospects, the precursor of
drenching rains and chilling blasts.

Master Hugh was amazed at the simplicity of his spouse, and, probably
for the first time, he unfolded to her the true philosophy of slavery,
and the peculiar rules necessary to be observed by masters and
mistresses, in the management of their human chattels. Mr. Auld promptly
forbade the continuance of her instruction; telling her, in the first
place, that the thing itself was unlawful; that it was also unsafe, and
could only lead to mischief. To use his own words, {}further, he said,
``if you give a nigger an inch, he will take an ell;'' ``he should know
nothing but the will of his master, and learn to obey it.'' ``Learning
would spoil the best nigger in the world;'' ``if you teach that
nigger---speaking of myself---how to read the bible, there will be no
keeping him;'' ``it would forever unfit him for the duties of a slave;''
and ``as to himself, learning would do him no good, but probably, a
great deal of harm---making him disconsolate and unhappy.'' ``If you
learn him now to read, he'll want to know how to write; and, this
accomplished, he'll be running away with himself.'' Such was the tenor
of Master Hugh's oracular exposition of the true philosophy of training
a human chattel; and it must be confessed that he very clearly
comprehended the nature and the requirements of the relation of master
and slave. His discourse was the first decidedly antislavery lecture to
which it had been my lot to listen. Mrs. Auld evidently felt the force
of his remarks; and, like an obedient wife, began to shape her course in
the direction indicated by her husband. The effect of his words,
\emph{on me}, was neither slight nor transitory. His iron
sentences---cold and harsh---sunk deep into my heart, and stirred up not
only my feelings into a sort of rebellion, but awakened within me a
slumbering train of vital thought. It was a new and special revelation,
dispelling a painful mystery, against which my youthful understanding
had struggled, and struggled in vain, to wit: the \emph{white} man's
power to perpetuate the enslavement of the \emph{black} man. ``Very
well,'' thought I; ``knowledge unfits a child to be a slave.'' I
instinctively assented to the proposition; {}and from that moment I
understood the direct pathway from slavery to freedom. This was just
what I needed; and I got it at a time, and from a source whence I least
expected it. I was saddened at the thought of losing the assistance of
my kind mistress; but the information, so instantly derived, to some
extent compensated me for the loss I had sustained in this direction.
Wise as Mr. Auld was, he evidently underrated my comprehension, and had
little idea of the use to which I was capable of putting the impressive
lesson he was giving to his wife. \emph{He} wanted me to be \emph{a
slave;} I had already voted against that on the home plantation of Col.
Lloyd. That which he most loved I most hated; and the very determination
which he expressed to keep me in ignorance, only rendered me the more
resolute in seeking intelligence. In learning to read, therefore, I am
not sure that I do not owe quite as much to the opposition of my master,
as to the kindly assistance of my amiable mistress. I acknowledge the
benefit rendered me by the one, and by the other; believing, that but
for my mistress, I might have grown up in ignorance.

I had resided but a short time in Baltimore, before I observed a marked
difference in the manner of treating slaves, generally, from that which
I had witnessed in that isolated and out-of-the-way part of the country
where I began life. A city slave is almost a free citizen, in Baltimore,
compared with a slave on Col. Lloyd's plantation. He is much better fed
and clothed, is less dejected in his appearance, and enjoys privileges
altogether unknown to the whip-driven slave on the plantation. Slavery
dislikes a dense {}population, in which there is a majority of
non-slaveholders. The general sense of decency that must pervade such a
population, does much to check and prevent those outbreaks of atrocious
cruelty, and those dark crimes without a name, almost openly perpetrated
on the plantation. He is a desperate slaveholder who will shock the
humanity of his non-slaveholding neighbors, by the cries of the
lacerated slaves; and very few in the city are willing to incur the
odium of being cruel masters. I found, in Baltimore, that no man was
more odious to the white, as well as to the colored people, than he, who
had the reputation of starving his slaves. Work them, flog them, if need
be, but don't starve them. There are, however, some painful exceptions
to this rule. While it is quite true that most of the slaveholders in
Baltimore feed and clothe their slaves well, there are others who keep
up their country cruelties in the city.

An instance of this sort is furnished in the case of a family who lived
directly opposite to our house, and were named Hamilton. Mrs. Hamilton
owned two slaves. Their names were Henrietta and Mary. They had always
been house slaves. One was aged about twenty-two, and the other about
fourteen. They were a fragile couple by nature, and the treatment they
received was enough to break down the constitution of a horse. Of all
the dejected, emaciated, mangled and excoriated creatures I ever saw,
those two girls---in the refined, church going and Christian city of
Baltimore---were the most deplorable. Of stone must that heart be made,
that could look upon Henrietta and Mary, without being {}sickened to the
core with sadness. Especially was Mary a heart-sickening object. Her
head, neck and shoulders, were literally cut to pieces. I have
frequently felt her head, and found it nearly covered over with
festering sores, caused by the lash of her cruel mistress. I do not know
that her master ever whipped her, but I have often been an eye witness
of the revolting and brutal inflictions by Mrs. Hamilton; and what lends
a deeper shade to this woman's conduct, is the fact, that, almost in the
very moments of her shocking outrages of humanity and decency, she would
charm you by the sweetness of her voice and her seeming piety. She used
to sit in a large rocking chair, near the middle of the room, with a
heavy cowskin, such as I have elsewhere described; and I speak within
the truth when I say, that those girls seldom passed that chair, during
the day, without a blow from that cowskin, either upon their bare arms,
or upon their shoulders. As they passed her, she would draw her cowskin
and give them a blow, saying, "\emph{move faster, you black jip!}" and,
again, "\emph{take that, you black jip!}" continuing, "\emph{if you
don't move faster, I will give you more.}" Then the lady would go on,
singing her sweet hymns, as though her \emph{righteous} soul were
sighing for the holy realms of paradise.

Added to the cruel lashings to which these poor slave-girls were
subjected---enough in themselves to crush the spirit of men---they were,
really, kept nearly half starved; they seldom knew what it was to eat a
full meal, except when they got it in the kitchens of neighbors, less
mean and stingy than the {}psalm-singing Mrs. Hamilton. I have seen poor
Mary contending for the offal, with the pigs in the street. So much was
the poor girl pinched, kicked, cut and pecked to pieces, that the boys
in the street knew her only by the name of "\emph{pecked}," a name
derived from the scars and blotches on her neck, head and shoulders.

It is some relief to this picture of slavery in Baltimore, to say---what
is but the simple truth---that Mrs. Hamilton's treatment of her slaves
was generally condemned, as disgraceful and shocking; but while I say
this, it must also be remembered, that the very parties who censured the
cruelty of Mrs. Hamilton, would have condemned and promptly punished any
attempt to interfere with Mrs. Hamilton's \emph{right} to cut and slash
her slaves to pieces. There must be no force between the slave and the
slaveholder, to restrain the power of the one, and protect the weakness
of the other; and the cruelty of Mrs. Hamilton is as justly chargeable
to the upholders of the slave system, as drunkenness is chargeable on
those who, by precept and example, or by indifference, uphold the
drinking system.
