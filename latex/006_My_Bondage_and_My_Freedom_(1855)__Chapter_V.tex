\hypertarget{headerContainer}{}
\hypertarget{navigationHeader}{}
\protect\hypertarget{headerprevious}{}{←\href{/wiki/My_Bondage_and_My_Freedom_(1855)/Chapter_IV}{Chapter
IV}}

\textbf{\protect\hypertarget{header_title_text}{}{\href{/wiki/My_Bondage_and_My_Freedom_(1855)}{My
Bondage and My Freedom}}} ~(1855)~ \emph{by
\href{/wiki/Author:Frederick_Douglass}{\protect\hypertarget{header_author_text}{}{{Frederick
Douglass}}}}\\
\protect\hypertarget{header_section_text}{}{Chapter V}

\protect\hypertarget{headernext}{}{\href{/wiki/My_Bondage_and_My_Freedom_(1855)/Chapter_VI}{Chapter
VI}→}

\hypertarget{navigationNotes}{}

\hypertarget{ws-data}{}
\protect\hypertarget{ws-article-id}{}{2336061}\protect\hypertarget{ws-title}{}{\href{/wiki/My_Bondage_and_My_Freedom_(1855)}{My
Bondage and My Freedom} --- \emph{Chapter
V}}\protect\hypertarget{ws-author}{}{Frederick
Douglass}\protect\hypertarget{ws-year}{}{1855}

{\protect\hypertarget{79}{}{}}

~

{CHAPTER V.}

GRADUAL INITIATION INTO THE MYSTERIES OF SLAVERY.

{GROWING ACQUAINTANCE WITH OLD MASTER---HIS CHARACTER---EVILS OF
UNRESTRAINED PASSION---APPARENT TENDERNESS---OLD MASTER A MAN OF
TROUBLE---CUSTOM OF MUTTERING TO HIMSELF---NECESSITY OF BEING AWARE OF
HIS WORDS---THE SUPPOSED OBTUSENESS OF SLAVE-CHILDREN---BRUTAL
OUTRAGE---DRUNKEN OVERSEER---SLAVEHOLDERS' IMPATIENCE---WISDOM OF
APPEALING TO SUPERIORS---THE SLAVEHOLDER'S WRATH BAD AS THAT OF THE
OVERSEER---A BASE AND SELFISH ATTEMPT TO BREAK UP A COURTSHIP---A
HARROWING SCENE.}

\textsc{Although} my old master---Capt. Anthony---gave me at first, (as
the reader will have already seen,) very little attention, and although
that little was of a remarkably mild and gentle description, a few
months only were sufficient to convince me that mildness and gentleness
were not the prevailing or governing traits of his character. These
excellent qualities were displayed only occasionally. He could, when it
suited him, appear to be literally insensible to the claims of humanity,
when appealed to by the helpless against an aggressor, and he could
himself commit outrages, deep, dark and nameless. Yet he was not by
nature worse than other men. Had he been brought up in a free state,
surrounded by the just restraints of free society---restraints which are
necessary to the freedom of all its members, alike and equally---Capt.
Anthony might have been as {\protect\hypertarget{80}{}{}}humane a man,
and every way as respectable, as many who now oppose the slave system;
certainly as humane and respectable as are members of society generally.
The slaveholder, as well as the slave, is the victim of the slave
system. A man's character greatly takes its hue and shape from the form
and color of things about him. Under the whole heavens there is no
relation more unfavorable to the development of honorable character,
than that sustained by the slaveholder to the slave. Reason is
imprisoned here, and passions run wild. Like the fires of the prairie,
once lighted, they are at the mercy of every wind, and must burn, till
they have consumed all that is combustible within their remorseless
grasp. Capt. Anthony could be kind, and, at times, he even showed an
affectionate disposition. Could the reader have seen him gently leading
me by the hand---as he sometimes did---patting me on the head, speaking
to me in soft, caressing tones and calling me his ``little Indian boy,''
he would have deemed him a kind old man, and, really, almost fatherly.
But the pleasant moods of a slaveholder are remarkably brittle; they are
easily snapped; they neither come often, nor remain long. His temper is
subjected to perpetual trials; but, since these trials are never borne
patiently, they add nothing to his natural stock of patience.

Old master very early impressed me with the idea that he was an unhappy
man. Even to my child's eye, he wore a troubled, and at times, a haggard
aspect. His strange movements excited my curiosity, and awakened my
compassion. He seldom walked alone {\protect\hypertarget{81}{}{}}without
muttering to himself; and he occasionally stormed about, as if defying
an army of invisible foes. ``He would do this, that, and the other; he'd
be d---d if he did not,''---was the usual form of his threats. Most of
his leisure was spent in walking, cursing and gesticulating, like one
possessed by a demon. Most evidently, he was a wretched man, at war with
his own soul, and with all the world around him. To be overheard by the
children, disturbed him very little. He made no more of \emph{our}
presence, than of that of the ducks and geese which he met on the green.
He little thought that the little black urchins around him, could see,
through those vocal crevices, the very secrets of his heart.
Slaveholders ever underrate the intelligence with which they have to
grapple. I really understood the old man's mutterings, attitudes and
gestures, about as well as he did himself. But slaveholders never
encourage that kind of communication, with the slaves, by which they
might learn to measure the depths of his knowledge. Ignorance is a high
virtue in a human chattel; and as the master studies to keep the slave
ignorant, the slave is cunning enough to make the master think he
succeeds. The slave fully appreciates the saying, ``where ignorance is
bliss, 'tis folly to be wise.'' When old master's gestures were violent,
ending with a threatening shake of the head, and a sharp snap of his
middle finger and thumb, I deemed it wise to keep at a respectable
distance from him; for, at such times, trifling faults stood, in his
eyes, as momentous offenses; and, having both the power and the
disposition, the victim had only to {\protect\hypertarget{82}{}{}}be
near him to catch the punishment, deserved or undeserved.

One of the first circumstances that opened my eyes to the cruelty and
wickedness of slavery, and the heartlessness of my old master, was the
refusal of the latter to interpose his authority, to protect and shield
a young woman, who had been most cruelly abused and beaten by his
overseer in Tuckahoe. This overseer---a Mr. Plummer---was a man like
most of his class, little better than a human brute; and, in addition to
his general profligacy and repulsive coarseness, the creature was a
miserable drunkard. He was, probably, employed by my old master, less on
account of the excellence of his services, than for the cheap rate at
which they could be obtained. He was not fit to have the management of a
drove of mules. In a fit of drunken madness, he committed the outrage
which brought the young woman in question down to my old master's for
protection. This young woman was the daughter of Milly, an own aunt of
mine. The poor girl, on arriving at our house, presented a pitiable
appearance. She had left in haste, and without preparation; and,
probably, without the knowledge of Mr. Plummer. She had traveled twelve
miles, bare-footed, bare-necked and bare-headed. Her neck and shoulders
were covered with scars, newly made; and, not content with marring her
neck and shoulders, with the cowhide, the cowardly brute had dealt her a
blow on the head with a hickory club, which cut a horrible gash, and
left her face literally covered with blood. In this condition, the poor
young woman came down, to implore protection at
{\protect\hypertarget{83}{}{}}the hands of my old master. I expected to
see him boil over with rage at the revolting deed, and to hear him fill
the air with curses upon the brutal Plummer; but I was disappointed. He
sternly told her, in an angry tone, he ``believed she deserved every bit
of it,'' and, if she did not go home instantly, he would himself take
the remaining skin from her neck and back. Thus was the poor girl
compelled to return, without redress, and perhaps to receive an
additional flogging for daring to appeal to old master against the
overseer.

Old master seemed furious at the thought of being troubled by such
complaints. I did not, at that time, understand the philosophy of his
treatment of my cousin. It was stern, unnatural, violent. Had the man no
bowels of compassion? Was he dead to all sense of humanity? No. I think
I now understand it. This treatment is a part of the system, rather than
a part of the man. Were slaveholders to listen to complaints of this
sort against the overseers, the luxury of owning large numbers of
slaves, would be impossible. It would do away with the office of
overseer, entirely; or, in other words, it would convert the master
himself into an overseer. It would occasion great loss of time and
labor, leaving the overseer in fetters, and without the necessary power
to secure obedience to his orders. A privilege so dangerous as that of
appeal, is, therefore, strictly prohibited; and any one exercising it,
runs a fearful hazard. Nevertheless, when a slave has nerve enough to
exercise it, and boldly approaches his master, with a well-founded
{\protect\hypertarget{84}{}{}}complaint against an overseer, though he
may be repulsed, and may even have that of which he complains repeated
at the time, and, though he may be beaten by his master, as well as by
the overseer, for his temerity, in the end the policy of complaining is,
generally, vindicated by the relaxed rigor of the overseer's treatment.
The latter becomes more careful, and less disposed to use the lash upon
such slaves thereafter. It is with this final result in view, rather
than with any expectation of immediate good, that the outraged slave is
induced to meet his master with a complaint. The overseer very naturally
dislikes to have the ear of the master disturbed by complaints; and,
either upon this consideration, or upon advice and warning privately
given him by his employers, he generally modifies the rigor of his rule,
after an outbreak of the kind to which I have been referring.

Howsoever the slaveholder may allow himself to act toward his slave,
and, whatever cruelty he may deem it wise, for example's sake, or for
the gratification of his humor, to inflict, he cannot, in the absence of
all provocation, look with pleasure upon the bleeding wounds of a
defenseless slave-woman. When he drives her from his presence without
redress, or the hope of redress, he acts, generally, from motives of
policy, rather than from a hardened nature, or from innate brutality.
Yet, let but his own temper be stirred, his own passions get loose, and
the slave-owner will go \emph{far beyond} the overseer in cruelty. He
will convince the slave that his wrath is far more terrible and
boundless, and vastly more to be dreaded, than that of the underling
overseer. What may have been {\protect\hypertarget{85}{}{}}mechanically
and heartlessly done by the overseer, is now done with a will. The man
who now wields the lash is irresponsible. He may, if he pleases, cripple
or kill, without fear of consequences; except in so far as it may
concern profit or loss. To a man of violent temper---as my old master
was---this was but a very slender and inefficient restraint. I have seen
him in a tempest of passion, such as I have just described---a passion
into which entered all the bitter ingredients of pride, hatred, envy,
jealousy, and the thirst for revenge.

The circumstances which I am about to narrate, and which gave rise to
this fearful tempest of passion, are not singular nor isolated in slave
life, but are common in every slaveholding community in which I have
lived. They are incidental to the relation of master and slave, and
exist in all sections of slaveholding countries.

The reader will have noticed that, in enumerating the names of the
slaves who lived with my old master, \emph{Esther} is mentioned. This
was a young woman who possessed that which is ever a curse to the
slave-girl; namely,---personal beauty. She was tall, well formed, and
made a fine appearance. The daughters of Col. Lloyd could scarcely
surpass her in personal charms. Esther was courted by Ned Roberts, and
he was as fine looking a young man, as she was a woman. He was the son
of a favorite slave of Col. Lloyd. Some slaveholders would have been
glad to promote the marriage of two such persons; but, for some reason
or other, my old master took it upon him to break up the growing
intimacy between Esther and Edward. {\protect\hypertarget{86}{}{}}He
strictly ordered her to quit the company of said Roberts, telling her
that he would punish her severely if he ever found her again in Edward's
company. This unnatural and heartless order was, of course, broken. A
woman's love is not to be annihilated by the peremptory command of any
one, whose breath is in his nostrils. It was impossible to keep Edward
and Esther apart. Meet they would, and meet they did. Had old master
been a man of honor and purity, his motives, in this matter, might have
been viewed more favorably. As it was, his motives were as abhorrent, as
his methods were foolish and contemptible. It was too evident that he
was not concerned for the girl's welfare. It is one of the damning
characteristics of the slave system, that it robs its victims of every
earthly incentive to a holy life. The fear of God, and the hope of
heaven, are found sufficient to sustain many slave-women, amidst the
snares and dangers of their strange lot; but, this side of God and
heaven, a slave-woman is at the mercy of the power, caprice and passion
of her owner. Slavery provides no means for the honorable continuance of
the race. Marriage---as imposing obligations on the parties to it---has
no existence here, except in such hearts as are purer and higher than
the standard morality around them. It is one of the consolations of my
life, that I know of many honorable instances of persons who maintained
their honor, where all around was corrupt.

Esther was evidently much attached to Edward, and abhorred---as she had
reason to do---the tyrannical and base behavior of old master. Edward
was young, and fine looking, and he loved and courted
{\protect\hypertarget{87}{}{}}her. He might have been her husband, in
the high sense just alluded to; but \textsc{who} and \emph{what} was
this old master? His attentions were plainly brutal and selfish, and it
was as natural that Esther should loathe him, as that she should love
Edward. Abhorred and circumvented as he was, old master, having the
power, very easily took revenge. I happened to see this exhibition of
his rage and cruelty toward Esther. The time selected was singular. It
was early in the morning, when all besides was still, and before any of
the family, in the house or kitchen, had left their beds. I saw but few
of the shocking preliminaries, for the cruel work had begun before I
awoke. I was probably awakened by the shrieks and piteous cries of poor
Esther. My sleeping place was on the floor of a little, rough closet,
which opened into the kitchen; and through the cracks of its unplaned
boards, I could dictinctly see and hear what was going on, without being
seen by old master. Esther's wrists were firmly tied, and the twisted
rope was fastened to a strong staple in a heavy wooden joist above, near
the fire-place. Here she stood, on a bench, her arms tightly drawn over
her breast. Her back and shoulders were bare to the waist. Behind her
stood old master, with cowskin in hand, preparing his barbarous work
with all manner of harsh, coarse, and tantalizing epithets. The screams
of his victim were most piercing. He was cruelly deliberate, and
protracted the torture, as one who was delighted with the scene. Again
and again he drew the hateful whip through his hand, adjusting it with a
view of dealing the most pain-giving blow. Poor Esther had never yet
been {\protect\hypertarget{88}{}{}}severely whipped, and her shoulders
were plump and tender. Each blow, vigorously laid on, brought screams as
well as blood. "\emph{Have mercy; Oh! have mercy}" she cried; "\emph{I
won't do so no more;}" but her piercing cries seemed only to increase
his fury. His answers to them are too coarse and blasphemous to be
produced here. The whole scene, with all its attendants, was revolting
and shocking, to the last degree; and when the motives of this brutal
castigation are considered, language has no power to convey a just sense
of its awful criminality. After laying on some thirty or forty stripes,
old master untied his suffering victim, and let her get down. She could
scarcely stand, when untied. From my heart I pitied her, and---child
though I was---the outrage kindled in me a feeling far from peaceful;
but I was hushed, terrified, stunned, and could do nothing, and the fate
of Esther might be mine next. The scene here described was often
repeated in the case of poor Esther, and her life, as I knew it, was one
of wretchedness.
