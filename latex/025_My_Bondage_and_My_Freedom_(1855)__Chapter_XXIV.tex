\hypertarget{headerContainer}{}
\hypertarget{navigationHeader}{}
\protect\hypertarget{headerprevious}{}{←\href{/wiki/My_Bondage_and_My_Freedom_(1855)/Chapter_XXIII}{Chapter
XXIII}}

\textbf{\protect\hypertarget{header_title_text}{}{\href{/wiki/My_Bondage_and_My_Freedom_(1855)}{My
Bondage and My Freedom}}} ~(1855)~ \emph{by
\href{/wiki/Author:Frederick_Douglass}{\protect\hypertarget{header_author_text}{}{{Frederick
Douglass}}}}\\
\protect\hypertarget{header_section_text}{}{Chapter XXIV}

\protect\hypertarget{headernext}{}{\href{/wiki/My_Bondage_and_My_Freedom_(1855)/Chapter_XXV}{Chapter
XXV}→}

\hypertarget{navigationNotes}{}

\hypertarget{ws-data}{}
\protect\hypertarget{ws-article-id}{}{2339052}\protect\hypertarget{ws-title}{}{\href{/wiki/My_Bondage_and_My_Freedom_(1855)}{My
Bondage and My Freedom} --- \emph{Chapter
XXIV}}\protect\hypertarget{ws-author}{}{Frederick
Douglass}\protect\hypertarget{ws-year}{}{1855}

{\protect\hypertarget{365}{}{}}

~

{CHAPTER XXIV.}

TWENTY-ONE MONTHS IN GREAT BRITAIN.

{GOOD ARISING OUT OF UNPROPITIOUS EVENTS---DENIED CABIN
PASSAGE---PROSCRIPTION TURNED TO GOOD ACCOUNT---THE HUTCHINSON
FAMILY---THE MOB ON BOARD THE CAMBRIA---HAPPY INTRODUCTION TO THE
BRITISH PUBLIC---LETTER ADDRESSED TO WILLIAM LLOYD GARRISON---TIME AND
LABORS WHILE ABROAD---FREEDOM PURCHASED---MRS. HENRY RICHARDSON---FREE
PAPERS---ABOLITIONISTS DISPLEASED WITH THE RANSOM---HOW THE AUTHOR'S
ENERGIES WERE DIRECTED---RECEPTION SPEECH IN LONDON---CHARACTER OF THE
SPEECH DEFENDED---CIRCUMSTANCES EXPLAINED---CAUSES CONTRIBUTING TO THE
SUCCESS OF HIS MISSION---FREE CHURCH OF SCOTLAND---TESTIMONIAL.}

\textsc{The} allotments of Providence, when coupled with trouble and
anxiety, often conceal from finite vision the wisdom and goodness in
which they are sent; and, frequently, what seemed a harsh and invidious
dispensation, is converted by after experience into a happy and
beneficial arrangement. Thus, the painful liability to be returned again
to slavery, which haunted me by day, and troubled my dreams by night,
proved to be a necessary step in the path of knowledge and usefulness.
The writing of my pamphlet, in the spring of 1845, endangered my
liberty, and led me to seek a refuge from republican slavery in
monarchical England. A rude, uncultivated fugitive slave was driven, by
stern necessity, to that country to which young American gentlemen go to
increase {\protect\hypertarget{366}{}{}}their stock of knowledge, to
seek pleasure, to have their rough, democratic manners softened by
contact with English aristocratic refinement. On applying for a passage
to England, on board the Cambria, of the Canard line, my friend, James
N. Buffum, of Lynn, Massachusetts, was informed that I could not be
received on board as a cabin passenger. American prejudice against color
triumphed over British liberality and civilization, and erected a color
test and condition for crossing the sea in the cabin of a British
vessel. The insult was keenly felt by my white friends, but to me, it
was common, expected, and therefore, a thing of no great consequence,
whether I went in the cabin or in the steerage. Moreover, I felt that if
I could not go into the first cabin, first-cabin passengers could come
into the second cabin, and the result justified my anticipations to the
fullest extent. Indeed, I soon found myself an object of more general
interest than I wished to be; and so far from being degraded by being
placed in the second cabin, that part of the ship became the scene of as
much pleasure and refinement, during the voyage, as the cabin itself.
The Hutchinson Family, celebrated vocalists---fellow-passengers---often
came to my rude forecastle deck, and sung their sweetest songs,
enlivening the place with eloquent music, as well as spirited
conversation, during the voyage. In two days after leaving Boston, one
part of the ship was about as free to me as another. My
fellow-passengers not only visited me, but invited me to visit them, on
the saloon deck. My visits there, however, were but seldom. I preferred
to live within my {\protect\hypertarget{367}{}{}}privileges, and keep
upon my own premises. I found this quite as much in accordance with good
policy, as with my own feelings. The effect was, that with the majority
of the passengers, all color distinctions were flung to the winds, and I
found myself treated with every mark of respect, from the beginning to
the end of the voyage, except in a single instance; and in that, I came
near being mobbed, for complying with an invitation given me by the
passengers, and the captain of the ``Cambria,'' to deliver a lecture on
slavery. Our New Orleans and Georgia passengers were pleased to regard
my lecture as an insult offered to them; and swore I should not speak.
They went so far as to threaten to throw me overboard, and but for the
firmness of Captain Judkins, probably would have (under the inspiration
of \emph{slavery} and \emph{brandy}) attempted to put their threats into
execution. I have no space to describe this scene, although its tragic
and comic peculiarities are well worth describing. An end was put to the
\emph{melee}, by the captain's calling the ship's company to put the
salt water mobocrats in irons. At this determined order, the gentlemen
of the lash scampered, and for the rest of the voyage conducted
themselves very decorously.

This incident of the voyage, in two days after landing at Liverpool,
brought me at once before the British public, and that by no act of my
own. The gentlemen so promptly snubbed in their meditated violence, flew
to the press to justify their conduct, and to denounce me as a worthless
and insolent negro. This course was even less wise than the conduct it
{\protect\hypertarget{368}{}{}}was intended to sustain; for, besides
awakening something like a national interest in me, and securing me an
audience, it brought out counter statements, and threw the blame upon
themselves, which they had sought to fasten upon me and the gallant
captain of the ship.

Some notion may be formed of the difference in my feelings and
circumstances, while abroad, from the following extract from one of a
series of letters addressed by me to Mr. Garrison, and published in the
Liberator. It was written on the first day of January, 1846:

"\textsc{My Dear Friend Garrison}: Up to this time, I have given no
direct expression of the views, feelings, and opinions which I have
formed, respecting the character and condition of the people of this
land. I have refrained thus, purposely. I wish to speak advisedly, and
in order to do this, I have waited till, I trust, experience has brought
my opinions to an intelligent maturity. I have been thus careful, not
because I think what I say will have much effect in shaping the opinions
of the world, but because whatever of influence I may possess, whether
little or much, I wish it to go in the right direction, and according to
truth. I hardly need say that, in speaking of Ireland, I shall be
influenced by no prejudices in favor of America. I think my
circumstances all forbid that. I have no end to serve, no creed to
uphold, no government to defend; and as to nation, I belong to none. I
have no protection at home, or resting-place abroad. The land of my
birth welcomes me to her shores only as a slave, and spurns with
contempt the idea of treating me differently; so that I am an outcast
from the society of my childhood, and an outlaw in the land of my birth.
'I am a stranger with thee, and a sojourner, as all my
{\protect\hypertarget{369}{}{}}fathers were.' That men should be
patriotic, is to me perfectly natural; and as a philosophical fact, I am
able to give it an \emph{intellectual} recognition. But no further can I
go. If ever I had any patriotism, or any capacity for the feeling, it
was whipped out of me long since, by the lash of the American
soul-drivers.

"In thinking of America, I sometimes find myself admiring her bright
blue sky, her grand old woods, her fertile fields, her beautiful rivers,
her mighty lakes, and star-crowned mountains. But my rapture is soon
checked, my joy is soon turned to mourning. When I remember that all is
cursed with the infernal spirit of slaveholding, robbery, and wrong;
when I remember that with the waters of her noblest rivers, the tears of
my brethren are borne to the ocean, disregarded and forgotten, and that
her most fertile fields drink daily of the warm blood of my outraged
sisters; I am filled with unutterable loathing, and led to reproach
myself that anything could fall from my lips in praise of such a land.
America will not allow her children to love her. She seems bent on
compelling those who would be her warmest friends, to be her worst
enemies. May God give her repentance, before it is too late, is the
ardent prayer of my heart. I will continue to pray, labor, and wait,
believing that she cannot always be insensible to the dictates of
justice, or deaf to the voice of humanity.

"My opportunities for learning the character and condition of the people
of this land have been very great. I have traveled almost from the Hill
of Howth to the Giant's Causeway, and from the Giant's Causeway to Cape
Clear. During these travels, I have met with much in the character and
condition of the people to approve, and much to condemn; much that has
thrilled me with pleasure, and very much that has filled me with pain. I
will not, in this letter, attempt to give any description of those
scenes which have given me pain. This I will do hereafter. I have
enough, and more than your {\protect\hypertarget{370}{}{}}subscribers
will be disposed to read at one time, of the bright side of the picture.
I can truly say, I have spent some of the happiest moments of my life
since landing in this country. I seem to have undergone a
transformation. I live a new life. The warm and generous cooperation
extended to me by the friends of my despised race; the prompt and
liberal manner with which the press has rendered me its aid; the
glorious enthusiasm with which thousands have flocked to hear the cruel
wrongs of my down-trodden and long-enslaved fellow-countrymen portrayed;
the deep sympathy for the slave, and the strong abhorrence of the
slaveholder, everywhere evinced; the cordiality with which members and
ministers of various religious bodies, and of various shades of
religious opinion, have embraced me, and lent me their aid; the kind
hospitality constantly proffered to me by persons of the highest rank in
society; the spirit of freedom that seems to animate all with whom I
come in contact, and the entire absence of everything that looked like
prejudice against me, on account of the color of my skin---contrasted so
strongly with my long and bitter experience in the United States, that I
look with wonder and amazement on the transition. In the southern part
of the United States, I was a slave, thought of and spoken of as
property; in the language of the \textsc{law}, {'}\emph{held, taken,
reputed, and adjudged to be a chattel in the hands of my owners and
possessors, and their executors, administrators, and assigns, to all
intents, constructions, and purposes whatsoever.}{'} (Brev. Digest,
224.) In the northern states, a fugitive slave, liable to be hunted at
any moment, like a felon, and to be hurled into the terrible jaws of
slavery---doomed by an inveterate prejudice against color to insult and
outrage on every hand, (Massachusetts out of the question)---denied the
privileges and courtesies common to others in the use of the most humble
means of conveyance---shut out from the cabins on steamboats---refused
admission to respectable hotels---caricatured, scorned,
{\protect\hypertarget{371}{}{}}scoffed, mocked, and maltreated with
impunity by any one, (no matter how black his heart,) so he has a white
skin. But now behold the change! Eleven days and a half gone, and I have
crossed three thousand miles of the perilous deep. Instead of a
democratic government, I am under a monarchical government. Instead of
the bright, blue sky of America, I am covered with the soft, grey fog of
the Emerald Isle. I breathe, and lo! the chattel becomes a man. I gaze
around in vain for one who will question my equal humanity, claim me as
his slave, or offer me an insult. I employ a cab---I am seated beside
white people---I reach the hotel---I enter the same door---I am shown
into the same parlor---I dine at the same table---and no one is
offended. No delicate nose grows deformed in my presence. I find no
difficulty here in obtaining admission into any place of worship,
instruction, or amusement, on equal terms with people as white as any I
ever saw in the United States. I meet nothing to remind me of my
complexion. I find myself regarded and treated at every turn with the
kindness and deference paid to white people. When I go to church, I am
met by no upturned nose and scornful lip to tell me, {'}\emph{We don't
allow niggers in here!}{'}

"I remember, about two years ago, there was in Boston, near the
south-west corner of Boston Common, a menagerie. I had long desired to
see such a collection as I understood was being exhibited there. Never
having had an opportunity while a slave, I resolved to seize this, my
first, since my escape. I went, and as I approached the entrance to gain
admission, I was met and told by the door-keeper, in a harsh and
contemptuous tone, {'}\emph{We don't allow niggers in here.}{'} I also
remember attending a revival meeting in the Rev. Henry Jackson's
meeting-house, at New Bedford, and going up the broad aisle to find a
seat, I was met by a good deacon, who told me, in a pious tone,
{'}\emph{We don't allow niggers in here!}{'} Soon after my arrival in
New Bedford, from the south, I had a strong
{\protect\hypertarget{372}{}{}}desire to attend the Lyceum, but was
told, {'}\emph{They don't allow niggers in here!}{'} While passing from
New York to Boston, on the steamer Massachusetts, on the night of the
9th of December, 1843, when chilled almost through with the cold, I went
into the cabin to get a little warm. I was soon touched upon the
shoulder, and told, {'}\emph{We don't allow niggers in here!}{'} On
arriving in Boston, from an anti-slavery tour, hungry and tired, I went
into an eating-house, near my friend, Mr. Campbell's, to get some
refreshments. I was met by a lad. in a white apron, {'}\emph{We don't
allow niggers in here!}{'} A week or two before leaving the United
States, I had a meeting appointed at Weymouth, the home of that glorious
band of true abolitionists, the Weston family, and others. On attempting
to take a seat in the omnibus to that place, I was told by the driver,
(and I never shall forget his fiendish hate,) {'}\emph{I don't allow
niggers in here!}{'} Thank heaven for the respite I now enjoy! I had
been in Dublin but a few days, when a gentleman of great respectability
kindly offered to conduct me through all the public buildings of that
beautiful city; and a little afterward, I found myself dining with the
lord mayor of Dublin. What a pity there was not some American democratic
Christian at the door of his splendid mansion, to bark out at my
approach, {'}\emph{They don't allow niggers in here!}{'} The truth is,
the people here know nothing of the republican negro hate prevalent in
our glorious land. They measure and esteem men according to their moral
and intellectual worth, and not according to the color of their skin.
Whatever may be said of the aristocracies here, there is none based on
the color of a man's skin. This species of aristocracy belongs
preëminently to `the land of the free, and the home of the brave.' I
have never found it abroad, in any but Americans. It sticks to them
wherever they go. They find it almost as hard to get rid of, as to get
rid of their skins.

"The second day after my arrival at Liverpool, in company
{\protect\hypertarget{373}{}{}}with my friend, Buffum, and several other
friends, I went to Eaton Hall, the residence of the Marquis of
Westminster, one of the most splendid buildings in England. On
approaching the door, I found several of our American passengers, who
came out with us in the Cambria, waiting for admission, as but one party
was allowed in the house at a time. We all had to wait till the company
within came out. And of all the faces, expressive of chagrin, those of
the Americans were preeminent. They looked as sour as vinegar, and as
bitter as gall, when they found I was to be admitted on equal terms with
themselves. When the door was opened, I walked in, on an equal footing
with my white fellow-citizens, and from all I could see, I had as much
attention paid me by the servants that showed us through the house, as
any with a paler skin. As I walked through the building, the statuary
did not fall down, the pictures did not leap from their places, the
doors did not refuse to open, and the servants did not say, {'}\emph{We
don't allow niggers in here!}{'}

``A happy new-year to you, and all the friends of freedom.''

My time and labors, while abroad, were divided between England, Ireland,
Scotland, and Wales. Upon tins experience alone, I might write a book
twice the size of this, "\emph{My Bondage and my Freedom.}" I visited
and lectured in nearly all the large towns and cities in the United
Kingdom, and enjoyed, many favorable opportunities for observation and
information. But books on England are abundant, and the public may,
therefore, dismiss any fear that I am meditating another infliction in
that line; though, in truth, I should like much to write a book on those
countries, if for nothing else, to make grateful mention of the many
dear friends, whose benevolent {\protect\hypertarget{374}{}{}}actions
toward me are ineffaceably stamped upon my memory, and warmly treasured
in my heart. To these friends I owe my freedom in the United States. On
their own motion, without any solicitation from me, (Mrs. Henry
Richardson, a clever lady, remarkable for her devotion to every good
work, taking the lead,) they raised a fund sufficient to purchase my
freedom, and actually paid it over, and placed the
papers\textsuperscript{\protect\hyperlink{cite_note-p374-1}{{[}1{]}}} of
{\protect\hypertarget{375}{}{}}my manumission in my hands, before they
would tolerate the idea of my returning to this, my native country. To
this commercial transaction I owe my exemption from the democratic
operation of the fugitive slave bill of 1850. But for this, I might at
any time become a victim of this most cruel and scandalous enactment,
and be doomed to end my life, as I began it, a slave. The sum paid for
my freedom was one hundred and fifty pounds sterling.

Some of my uncompromising anti-slavery friends in this country failed to
see the wisdom of this arrangement, and were not pleased that I
consented to it, even by my silence. They thought it a violation of
anti-slavery principles---conceding a right of property in man---and a
wasteful expenditure of money. On the other hand, viewing it simply in
the light of a ransom, or as money extorted by a robber, and my liberty
of more value than one hundred and fifty pounds sterling, I could not
see either a violation of the laws of morality, or those of economy, in
the transaction.

{\protect\hypertarget{376}{}{}}It is true, I was not In the possession
of my claimants, and could have easily remained in England, for the same
friends who had so generously purchased my freedom, would have assisted
me in establishing myself in that country. To this, however, I could not
consent. I felt that I had a duty to perform---and that was, to labor
and suffer with the oppressed in my native land. Considering, therefore,
all the circumstances---the fugitive slave bill included---I think the
very best thing was done in letting Master Hugh have the hundred and
fifty pounds sterling, and leaving me free to return to my appropriate
field of labor. Had I been a private person, having no other relations
or duties than those of a personal and family nature, I should never
have consented to the payment of so large a sum for the privilege of
living securely under our glorious republican form of government. I
could have remained in England, or have gone to some other country; and
perhaps I could even have lived unobserved in this. But to this I could
not consent. I had already become somewhat notorious, and withal quite
as unpopular as notorious; and I was, therefore, much exposed to arrest
and re-capture.

The main object to which my labors in Great Britain were directed, was
the concentration of the moral and religious sentiment of its people
against American slavery, England is often charged with having
{\protect\hypertarget{377}{}{}}established slavery in the United States,
and if there were no other justification than this, for appealing to her
people to lend their moral aid for the abolition of slavery, I should be
justified. My speeches in Great Britain were wholly extemporaneous, and
I may not always have been so guarded in my expressions, as I otherwise
should have been. I was ten years younger then than now, and only seven
years from slavery. I cannot give the reader a better idea of the nature
of my discourses, than by republishing one of them, delivered in
Finsbury chapel, London, to an audience of about two thousand persons,
and which was published in the ``London Universe,'' at the
time.\textsuperscript{\protect\hyperlink{cite_note-2}{{[}2{]}}}

Those in the United States who may regard this speech as being harsh in
its spirit and unjust in its statements, because delivered before an
audience supposed to be anti-republican in their principles and
feelings, may view the matter differently, when they learn that the case
supposed did not exist. It so happened that the great mass of the people
in England who attended and patronized my anti-slavery meetings, were,
in truth, about as good republicans as the mass of Americans, and with
this decided advantage over the latter---they are lovers of
republicanism for all men, for black men as well as for white men. They
are the people who sympathize with Louis Kossuth and Mazzini, and with
the oppressed and enslaved, of every color and nation, the world over.
They constitute the democratic element in British politics, and are as
much opposed to the {\protect\hypertarget{378}{}{}}union of church and
state as we, in America, are to such an union. At the meeting where this
speech was delivered, Joseph Sturge---a world-wide philanthropist, and a
member of the society of Friends---presided, and addressed the meeting.
George William Alexander, another Friend, who has spent more than an
American fortune in promoting the anti-slavery cause in different
sections of the world, was on the platform; and also Dr. Campbell, (now
of the ``British Banner,'') who combines all the humane tenderness of
Melancthon, with the directness and boldness of Luther. He is in the
very front ranks of non-conformists, and looks with no unfriendly eye
upon America. George Thompson, too, was there; and America will yet own
that he did a true man's work in relighting the rapidly dying-out fire
of true republicanism in the American heart, and be ashamed of the
treatment he met at her hands. Coming generations in this country will
applaud the spirit of this much abused republican friend of freedom.
There were others of note seated on the platform, who would gladly
ingraft upon English institutions all that is purely republican in the
institutions of America. Nothing, therefore, must be set down against
this speech on the score that it was delivered in the presence of those
who cannot appreciate the many excellent things belonging to our system
of government, and with a view to stir up prejudice against republican
institutions.

Again, let it also be remembered---for it is the simple truth---that
neither in this speech, nor in any other which I delivered in England,
did I ever allow {\protect\hypertarget{379}{}{}}myself to address
Englishmen as against Americans. I took my stand on the high ground of
human brotherhood, and spoke to Englishmen as men, in behalf of men.
Slavery is a crime, not against Englishmen, but against God, and all the
members of the human family; and it belongs to the whole human family to
seek its suppression. In a letter to Mr. Greeley, of the New York
Tribune, written while abroad, I said:

{``I am, nevertheless, aware that the wisdom of exposing the sins of one
nation in the ear of another, has been seriously questioned by good and
clear-sighted people, both on this and on your side of the Atlantic. And
the thought is not without weight on my own mind. I am satisfied that
there are many evils which can be best removed by confining our efforts
to the immediate locality where such evils exist. This, however, is by
no means the case with the system of slavery. It is such a giant
sin---such a monstrous aggregation of iniquity---so hardening to the
human heart---so destructive to the moral sense, and so well calculated
to beget a character, in every one around it, favorable to its own
continuance,---that I feel not only at liberty, but abundantly
justified, in appealing to the whole world to aid in its removal.''}

But, even if I had---as has been often charged---labored to bring
American institutions generally into disrepute, and had not confined my
labors strictly within the limits of humanity and morality, I should not
have been without illustrious examples to support me. Driven into
semi-exile by civil and barbarous laws, and by a system which cannot be
thought of without a shudder, I was fully justified in turning, if
possible, the tide of the moral universe against the heaven-daring,
outrage.

{\protect\hypertarget{380}{}{}}Four circumstances greatly assisted rue
in getting the question of American slavery before the British public.
First, the mob on board the Cambria, already referred to, which was a
sort of national announcement of my arrival in England. Secondly, the
highly reprehensible course pursued by the Free Church of Scotland, in
soliciting, receiving, and retaining money in its sustentation fund for
supporting the gospel in Scotland, which was evidently the ill-gotten
gain of slaveholders and slave-traders. Third, the great Evangelical
Alliance---or rather the attempt to form such an alliance, which should
include slaveholders of a certain description---added immensely to the
interest felt in the slavery question. About the same time, there was
the World's Temperance Convention, where I had the misfortune to come in
collision with sundry American doctors of divinity---Dr. Cox among the
number---with whom I had a small controversy.

It has happened to me---as it has happened to most other men engaged in
a good cause---often to be more indebted to my enemies than to my own
skill or to the assistance of my friends, for whatever success has
attended my labors. Great surprise was expressed by American newspapers,
north and south, during my stay in Great Britain, that a person so
illiterate and insignificant as myself could awaken an interest so
marked in England. These papers were not the only parties surprised. I
was myself not far behind them in surprise. But the very contempt and
scorn, the systematic and extravagant disparagement of which I was the
object, served, perhaps, to magnify {\protect\hypertarget{381}{}{}}my
few merits, and to render me of some account, whether deserving or not.
A man is sometimes made great, by the greatness of the abuse a portion
of mankind may think proper to heap upon him. Whether I was of as much
consequence as the English papers made me out to be, or not, it was
easily seen, in England, that I could not be the ignorant and worthless
creature, some of the American papers would have them believe I was.
Men, in their senses, do not take bowie-knives to kill mosquitoes, nor
pistols to shoot flies; and the American passengers who thought proper
to get up a mob to silence me, on board the Cambria, took the most
effective method of telling the British public that I had something to
say.

But to the second circumstance, namely, the position of the Free Church
of Scotland, with the great Doctors Chalmers, Cunningham, and Candlish
at its head. That church, with its leaders, put it out of the power of
the Scotch people to ask the old question, which we in the north have
often most wickedly asked---"\emph{What have we to do with slavery?}"
That church had taken the price of blood into its treasury, with which
to build \emph{free} churches, and to pay \emph{free} church ministers
for preaching the gospel; and, worse still, when honest John Murray, of
Bowlien Bay---now gone to his reward in heaven---with "William Smeal,
Andrew Paton, Frederick Card, and other sterling anti-slavery men in
Glasgow, denounced the transaction as disgraceful and shocking to the
religious sentiment of Scotland, this church, through its leading
divines, instead of repenting and seeking to mend the mistake into which
it had fallen, made {\protect\hypertarget{382}{}{}}it a flagrant sin, by
undertaking to defend, in the name of God and the bible, the principle
not only of taking the money of slave-dealers to build churches, but of
holding fellowship with the holders and traffickers in human flesh.
This, the reader will see, brought up the whole question of slavery, and
opened the way to its full discussion, without any agency of mine. I
have never seen a people more deeply moved than were the people of
Scotland, on this very question. Public meeting succeeded public
meeting. Speech after speech, pamphlet after pamphlet, editorial after
editorial, sermon after sermon, soon lashed the conscientious Scotch
people into a perfect \emph{furore}. "\textsc{Send back the money}!" was
indignantly cried out, from Greenock to Edinburgh, and from. Edinburgh
to Aberdeen. George Thompson, of London, Henry C. Wright, of the United
States, James N. Buffum, of Lynn, Massachusetts, and myself were on the
anti-slavery side; and Doctors Chalmers, Cunningham, and Candlish on the
other. In a conflict where the latter could have had even the show of
right, the truth, in our hands as against them, must have been driven to
the wall; and while I believe we were able to carry the conscience of
the country against the action of the Free Church, the battle, it must
be confessed, was a hard-fought one. Abler defenders of the doctrine of
fellowshiping slaveholders as christians, have not been met with. In
defending this doctrine, it was necessary to deny that slavery is a sin.
If driven from this position, they were compelled to deny that
slaveholders were responsible for the sin; and if driven from both these
{\protect\hypertarget{383}{}{}}positions, they must deny that it is a
sin in such a sense, and that slaveholders are sinners in such a sense,
as to make it wrong, in the circumstances in which they were placed, to
recognize them as christians. Dr. Cunningham was the most powerful
debater on the slavery side of the question; Mr. Thompson was the ablest
on the anti-slavery side. A scene occurred between these two men, a
parallel to which I think I never witnessed before, and I know I never
have since. The scene was caused by a single exclamation on the part of
Mr. Thompson.

The general assembly of the Free Church was in progress at Cannon Mills,
Edinburgh. The building would hold about twenty-five hundred persons;
and on this occasion it was densely packed, notice having been given
that Doctors Cunningham and Candlish would speak, that day, in defense
of the relations of the Free Church of Scotland to slavery in America.
Messrs. Thompson, Buffum, myself, and a few anti-slavery friends,
attended, but sat at such a distance, and in such a position, that,
perhaps, we were not observed from the platform. The excitement was
intense, having been greatly increased by a series of meetings held by
Messrs. Thompson, Wright, Buffum, and myself, in the most splendid hall
in that most beautiful city, just previous to the meetings of the
general assembly. "\textsc{Send back the money}!" stared at us from
every street corner; "\textsc{Send back the money}!" in large capitals,
adorned the broad flags of the pavement; "\textsc{Send back the money}!"
was the chorus of the popular street songs; "\textsc{Send back the
money}!" was the heading of leading editorials in the
{\protect\hypertarget{384}{}{}}daily newspapers. This clay, at Cannon
Mills, the great doctors of the church were to give an answer to this
loud and stern demand. Men of all parties and all sects were most eager
to hear. Something great was expected. The occasion was great, the men
great, and great speeches were expected from them.

In addition to the outside pressure upon Doctors Cunningham and
Candlish, there was wavering in their own ranks. The conscience of the
church itself was not ease. A dissatisfaction with the position of the
church touching slavery, was sensibly manifest among the members, and
something must be done to counteract this untoward influence. The great
Dr. Chalmers was in feeble health, at the time. His most potent
eloquence could not now be summoned to Cannon Mills, as formerly. He
whose voice was able to rend asunder and dash down the granite walls of
the established church of Scotland, and to lead a host in solemn
procession from it, as from a doomed city, was now old and enfeebled.
Besides, he had said his word on this very question; and his word had
not silenced the clamor without, nor stilled the anxious heavings
within. The occasion was momentous, and felt to be so. The church was in
a perilous condition. A change of some sort must take place in her
condition, or she must go to pieces. To stand where she did, was
impossible. The whole weight of the matter fell on Cunningham and
Candlish. No shoulders in the church were broader than theirs; and I
must say, badly as I detest the principles laid down and defended by
them, I was compelled to acknowledge the vast mental endowments of the
men. {\protect\hypertarget{385}{}{}}Cunningham rose; and his rising was
the signal for almost tumultous applause. You will say this was scarcely
in keeping with the solemnity of the occasion, but to me it served to
increase its grandeur and gravity. The applause, though tumultuous, was
not joyous. It seemed to me, as it thundered up from the vast audience,
like the fall of an immense shaft, flung from shoulders already galled
by its crushing weight. It was like saying, "Doctor, we have borne this
burden long enough, and willingly fling it upon you. Since it was you
who brought it upon us, take it now, and do what you will with it, for
we are too weary to bear it.

Doctor Cunningham proceeded with his speech, abounding in logic,
learning, and eloquence, and apparently bearing down all opposition; but
at the moment---the fatal moment---when he was just bringing all his
arguments to a point, and that point being, that neither Jesus Christ
nor his holy apostles regarded slaveholding as a sin, George Thompson,
in a clear, sonorous, but rebuking voice, broke the deep stillness of
the audience, exclaiming, "\textsc{Hear! hear! hear}!" The effect of
this simple and common exclamation is almost incredible. It was as if a
granite wall had been suddenly flung up against the advancing current of
a mighty river. For a moment, speaker and audience were brought to a
dead silence. Both the doctor and his hearers seemed appalled by the
audacity, as well as the fitness of the rebuke. At length a shout went
up to the cry of "\emph{Put him out!}" Happily, no one attempted to
execute this cowardly order, and the doctor proceeded with his
discourse. {\protect\hypertarget{386}{}{}}Not, however, as before, did
the learned doctor proceed. The exclamation of Thompson must have
reëchoed itself a thousand times in his memory, during the remainder of
his speech, for the doctor never recovered from the blow.

The deed was done, however; the pillars of the church---\emph{the proud,
Free Church of Scotland}---were committed, and the humility of
repentance was absent. The Free Church held on to the blood-stained
money, and continued to justify itself in its position---and of course
to apologize for slavery---and does so till this day. She lost a
glorious opportunity for giving her voice, her vote, and her example to
the cause of humanity; and to-day she is staggering under the curse of
the enslaved, whose blood is in her skirts. The people of Scotland are,
to this day, deeply grieved at the course pursued by the Free Church,
and would hail, as a relief from a deep and blighting shame, the
``sending back the money'' to the slaveholders from whom it was
gathered.

One good result followed the conduct of the Free Church; it furnished an
occasion for making the people of Scotland thoroughly acquainted with
the character of slavery, and for arraying against the system the moral
and religious sentiment of that country. Therefore, while we did not
succeed in accomplishing the specific object of our mission,
namely---procure the sending back of the money---we were amply justified
by the good which really did result from our labors.

Next comes the Evangelical Alliance. This was an attempt to form a union
of all evangelical {\protect\hypertarget{387}{}{}}christians throughout
the world. Sixty or seventy American divines attended, and some of them
went there merely to weave a world-wide garment with which to clothe
evangelical slaveholders. Foremost among these divines, was the Rev.
Samuel Hanson Cox, moderator of the New School Presbyterian General
Assembly. He and his friends spared no pains to secure a platform broad
enough to hold American slaveholders, and in this they partly succeeded.
But the question of slavery is too large a question to be finally
disposed of, even by the Evangelical Alliance. We appealed from the
judgment of the Alliance, to the judgment of the people of Great
Britain, and with the happiest effect. This controversy with the
Alliance might be made the subject of extended remark, but I must
forbear, except to say, that this effort to shield the christian
character of slaveholders greatly served to open a way to the British
ear for anti-slavery discussion, and that it was well improved.

The fourth and last circumstance that assisted me in getting before the
British public, was an attempt on the part of certain doctors of
divinity to silence me on the platform of the World's Temperance
Convention. Here I was brought into point blank collision with Rev. Dr.
Cox, who made me the subject not only of bitter remark in the
convention, but also of a long denunciatory letter published in the New
York Evangelist and other American papers. I replied to the doctor as
well as I could, and was successful in getting a respectful hearing
before the British public, who are by nature and practice ardent
{\protect\hypertarget{388}{}{}}lovers of fair play, especially in a
conflict between the weak and the strong.

Thus did circumstances favor me, and favor the cause of which I strove
to be the advocate. After such distinguished notice, the public in both
countries was compelled to attach some importance to my labors. By the
very ill usage I received at the hands of Dr. Cox and his party, by the
mob on board the Cambria, by the attacks made upon me in the American
newspapers, and by the aspersions cast upon me through the organs of the
Free Church of Scotland, I became one of that class of men, who, for the
moment, at least, ``have greatness forced upon them.'' People became the
more anxious to hear for themselves, and to judge for themselves, of the
truth which I had to unfold. While, therefore, it is by no means easy
for a stranger to get fairly before the British public, it was my lot to
accomplish it in the easiest manner possible.

Having continued in Great Britain and Ireland nearly two years, and
being about to return to America---not as I left it, a slave, but a
freeman---leading friends of the cause of emancipation in that country
intimated their intention to make me a testimonial, not only on grounds
of personal regard to myself, but also to the cause to which they were
so ardently devoted. How far any such thing could have succeeded, I do
not know; but many reasons led me to prefer that my friends should
simply give me the means of obtaining a printing press and printing
materials, to enable me to start a paper, devoted to the interests of my
enslaved and oppressed people. I {\protect\hypertarget{389}{}{}}told
them that perhaps the greatest hinderance to the adoption of abolition
principles by the people of the United States, was the low estimate,
everywhere in that country, placed upon the negro, as a man; that
because of his assumed natural inferiority, people reconciled themselves
to his enslavement and oppression, as things inevitable, if not
desirable. The grand thing to be done, therefore, was to change the
estimation in which the colored people of the United States were held;
to remove the prejudice which depreciated and depressed them; to prove
them worthy of a higher consideration; to disprove their alleged
inferiority, and demonstrate their capacity for a more exalted
civilization than slavery and prejudice had assigned to them. I further
stated, that, in my judgment, a tolerably well conducted press, in the
hands of persons of the despised race, by calling out the mental
energies of the race itself; by making them acquainted with their own
latent powers; by enkindling among them the hope that for them there is
a future; by developing their moral power; by combining and reflecting
their talents---would prove a most powerful means of removing prejudice,
and of awakening an interest in them. I further informed them---and at
that time the statement was true---that there was not, in the United
States, a single newspaper regularly published by the colored people;
that many attempts had been made to establish such papers; but that, up
to that time, they had all failed. These views I laid before my friends.
The result was, nearly two thousand five hundred dollars were speedily
raised toward starting my paper. For this
{\protect\hypertarget{390}{}{}}prompt and generous assistance, rendered
upon my bare suggestion, without any personal efforts on my part, I
shall never cease to feel deeply grateful; and the thought of fulfilling
the noble expectations of the dear friends who gave me this evidence of
their confidence, will never cease to be a motive for persevering
exertion.

Proposing to leave England, and turning my face toward America, in the
spring of 1847, I was met, on the threshold, with something which
painfully reminded me of the kind of life which awaited me in my native
land. For the first time in the many months spent abroad, I was met with
proscription on account of my color. A few weeks before departing from
England, while in London, I was careful to purchase a ticket, and secure
a berth for returning home, in the Cambria---the steamer in which I left
the United States---paying therefor the round sum of forty pounds and
nineteen shillings sterling. This was first cabin fare. But on going
aboard the Cambria, I found that the Liverpool agent had ordered my
berth to be given to another, and had forbidden my entering the saloon!
This contemptible conduct met with stern rebuke from the British press.
For, upon the point of leaving England, I took occasion to expose the
disgusting tyranny, in the columns of the London Times. That journal,
and other leading journals throughout the United Kingdom, held up the
outrage to unmitigated condemnation. So good an opportunity for calling
out a full expression of British sentiment on the subject, had not
before occurred, and it was most fully embraced. The result
{\protect\hypertarget{391}{}{}}was, that Mr. Cunard came out in a letter
to the public journals, assuring them of his regret at the outrage, and
promising that the like should never occur again on board his steamers;
and the like, we believe, has never since occurred on board the
steamships of the Cunard line.

It is not very pleasant to be made the subject of such insults; but if
all such necessarily resulted as this one did, I should be very happy to
bear, patiently, many more than I have borne, of the same sort. Albeit,
the lash of proscription, to a man accustomed to equal social position,
even for a time, as I was, has a sting for the soul hardly less severe
than that which bites the flesh and draws the blood from the back of the
plantation slave. It was rather hard, after having enjoyed nearly two
years of equal social privileges in England, often dining with gentlemen
of great literary, social, political, and religious eminence---never,
during the whole time, having met with a single word, look, or gesture,
which gave me the slightest reason to think my color was an offense to
anybody---now to be cooped up in the stern of the Cambria, and denied
the right to enter the saloon, lest my dark presence should be deemed an
offense to some of my democratic fellow-passengers. The reader will
easily imagine what must have been my feelings.

\begin{center}\rule{0.5\linewidth}{\linethickness}\end{center}

\begin{enumerate}
\item
  \hypertarget{cite_note-p374-1}{}

  {\protect\hyperlink{cite_ref-p374_1-0}{↑}} {The following is a copy of
  these curious papers, both of my transfer from Thomas to Hugh Auld,
  and from Hugh to myself:\\[2\baselineskip]"Know all men by these
  Presents, That I, Thomas Auld, of Talbot county, and state of
  Maryland, for and in consideration of the sum of one hundred dollars,
  current money, to me paid by Hugh Auld, of the city of Baltimore, in
  the said state, at and before the sealing and delivery of these
  presents, the receipt whereof, I, the said Thomas Auld, do hereby
  acknowledge, have granted, bargained, and sold, and by these presents
  do grant, bargain, and sell unto the said Hugh Auld, his executors,
  administrators, and assigns, {ONE NEGRO MAN}, by the name of
  \textsc{Frederick Baily}, or \textsc{Douglass}, as he calls himself he
  is now about twenty-eight years of age to have and to hold the said
  negro man for life. And I, the said Thomas Auld, for myself, my heirs,
  executors, and administrators, all and singular, the said
  \textsc{Frederick Baily}, alias \textsc{Douglass}, unto the said Hugh
  Auld, his executors, administrators, and assigns, against me, the said
  Thomas Auld, my executors, and administrators, and against all and
  every other person or persons whatsoever, shall and will warrant and
  forever defend by these presents. In witness whereof, I set my hand
  and seal, this thirteenth day of November, eighteen hundred and
  forty-six.}

  Thomas Auld. 

   "Signed, sealed, and delivered in presence of Wrightson Jones.

    ``John C. Leas.''

  The authenticity of this bill of sale is attested by N. Harrington, a
  justice of the peace of the state of Maryland, and for the county of
  Talbot, dated same day as above.

  \begin{center}\rule{0.5\linewidth}{\linethickness}\end{center}

  "To all whom it may concern: Be it known, that I, Hugh Auld, of the
  city of Baltimore, in Baltimore county, in the state of Maryland, for
  divers good causes and considerations, me thereunto moving, have
  released from slavery, liberated, manumitted, and set free, and by
  these presents do hereby release from slavery, liberate, manumit, and
  set free, \textsc{my negro man}, named \textsc{Frederick Baily},
  otherwise called \textsc{Douglass}, being of the age of twenty-eight
  years, or thereabouts, and able to work and gain a sufficient
  livelihood and maintenance; and him the said negro man, named
  \textsc{Frederick Baily}, otherwise called \textsc{Frederick
  Douglass}, I do declare to be henceforth free, manumitted, and
  discharged from all manner of servitude to me, my executors, and
  administrators forever.

  "In witness whereof, I, the said Hugh Auld, have hereunto set my hand
  and seal, the fifth of December, in the year one thousand eight
  hundred and forty-six.

  \textsc{Hugh Auld}. 

   "Sealed and delivered in presence of T. Hanson Belt,

  {}"\textsc{James N. S. T. Wright}."
\item
  \hypertarget{cite_note-2}{}

  {\protect\hyperlink{cite_ref-2}{↑}} {See Appendix to this volume, page
  411.}
\end{enumerate}
