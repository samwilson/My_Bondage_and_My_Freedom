\hypertarget{headerContainer}{}
\hypertarget{navigationHeader}{}
\protect\hypertarget{headerprevious}{}{←\href{/wiki/My_Bondage_and_My_Freedom_(1855)/Chapter_XIV}{Chapter
XIV}}

\textbf{\protect\hypertarget{header_title_text}{}{\href{/wiki/My_Bondage_and_My_Freedom_(1855)}{My
Bondage and My Freedom}}} ~(1855)~ \emph{by
\href{/wiki/Author:Frederick_Douglass}{\protect\hypertarget{header_author_text}{}{{Frederick
Douglass}}}}\\
\protect\hypertarget{header_section_text}{}{Chapter XV}

\protect\hypertarget{headernext}{}{\href{/wiki/My_Bondage_and_My_Freedom_(1855)/Chapter_XVI}{Chapter
XVI}→}

\hypertarget{navigationNotes}{}

\hypertarget{ws-data}{}
\protect\hypertarget{ws-article-id}{}{2339058}\protect\hypertarget{ws-title}{}{\href{/wiki/My_Bondage_and_My_Freedom_(1855)}{My
Bondage and My Freedom} --- \emph{Chapter
XV}}\protect\hypertarget{ws-author}{}{Frederick
Douglass}\protect\hypertarget{ws-year}{}{1855}

{\protect\hypertarget{205}{}{}}

~

{CHAPTER XV.}

COVEY, THE NEGRO BREAKER.

{JOURNEY TO MY NEW MASTER'S---MEDITATIONS BY THE WAY---VIEW OF COVEY'S
RESIDENCE---THE FAMILY---THE AUTHOR'S AWKWARDNESS AS A FIELD HAND---A
CRUEL BEATING---WHY IT WAS GIVEN---DESCRIPTION OF COVEY---FIRST
ADVENTURE AT OX DRIVING---HAIR BREADTH ESCAPES---OX AND MAN ALIKE
PROPERTY---COVEY'S MANNER OF PROCEEDING TO WHIP---HARD LABOR BETTER THAN
THE WHIP FOR BREAKING DOWN THE SPIRIT---CUNNING AND TRICKERY OF
COVEY---FAMILY WORSHIP---SHOCKING CONTEMPT FOR CHASTITY---THE AUTHOR
BROKEN DOWN---GREAT MENTAL AGITATION IN CONTRASTING THE FREEDOM OF THE
SHIPS WITH HIS OWN SLAVERY---ANGUISH BEYOND DESCRIPTION.}

\textsc{The} morning of the first of January, 1834, with its chilling
wind and pinching frost, quite in harmony with the winter in my own
mind, found me, with my little bundle of clothing on the end of a stick,
swung across my shoulder, on the main road, bending my way toward
Covey's, whither I had been imperiously ordered by Master Thomas. The
latter had been as good as his word, and had committed me, without
reserve, to the mastery of Mr. Edward Covey. Eight or ten years had now
passed since I had been taken from my grandmother's cabin, in Tuckahoe;
and these years, for the most part, I had spent in Baltimore, where---as
the reader has already seen---I was treated with comparative tenderness.
I was now about to sound profounder depths in slave life. The
{\protect\hypertarget{206}{}{}}rigors of a field, less tolerable than
the field of battle, awaited me. My new master was notorious for his
fierce and savage disposition, and my only consolation in going to live
with him was, the certainty of finding him precisely as represented by
common fame. There was neither joy in my heart, nor elasticity in my
step, as I started in search of the tyrant's home. Starvation made me
glad to leave Thomas Auld's, and the cruel lash made me dread to go to
Covey's. Escape was impossible; so, heavy and sad, I paced the seven
miles, which separated Covey's house from St. Michael's---thinking much
by the solitary way---averse to my condition; but \emph{thinking} was
all I could do. Like a fish in a net, allowed to play for a time, I was
now drawn rapidly to the shore, secured at all points. ``I am,'' thought
I, "but the sport of a power which makes no account, either of my
welfare or of my happiness. By a law which I can clearly comprehend, but
cannot evade nor resist, I am ruthlessly snatched from the hearth of a
fond grandmother, and hurried away to the home of a mysterious `old
master;' again I am removed from there, to a master in Baltimore; thence
am I snatched away to the Eastern Shore, to be valued with the beasts of
the field, and, with them, divided and set apart for a possessor; then I
am sent back to Baltimore; and by the time I have formed new
attachments, and have begun to hope that no more rude shocks shall touch
me, a difference arises between brothers, and I am again broken up, and
sent to St. Michael's; and now, from the latter place, I am footing my
way to the home of a new master, where, I am given to understand, that,
{\protect\hypertarget{207}{}{}}like a wild young working animal, I am to
be broken to the yoke of a bitter and life-long bondage."

With thoughts and reflections like these, I came in sight of a small
wood-colored building, about a mile from the main road, which, from the
description I had received, at starting, I easily recognized as my new
home. The Chesapeake bay---upon the jutting banks of which the little
wood-colored house was standing---white with foam, raised by the heavy
north-west wind; Poplar Island, covered with a thick, black pine forest,
standing out amid this half ocean; and Kent Point, stretching its sandy,
desert-like shores out into the foam-crested bay,---were all in sight,
and deepened the wild and desolate aspect of my new home.

The good clothes I had brought with me from Baltimore were now worn
thin, and had not been replaced; for Master Thomas was as little careful
to provide us against cold, as against hunger. Met here by a north wind,
sweeping through an open space of forty miles, I was glad to make any
port; and, therefore, I speedily pressed on to the little wood-colored
house. The family consisted of Mr. and Mrs. Covey; Miss Kemp, (a
broken-backed woman,) a sister of Mrs. Covey; William Hughes, cousin to
Edward Covey; Caroline, the cook; Bill Smith, a hired man; and myself.
Bill Smith, Bill Hughes, and myself, were the working force of the farm,
which consisted of three or four hundred acres. I was now, for the first
time in my life, to be a field hand; and in my new employment I found
myself even more awkward than a green country boy may be supposed to be,
{\protect\hypertarget{208}{}{}}upon his first entrance into the
bewildering scenes of city life; and my awkwardness gave me much
trouble. Strange and unnatural as it may seem, I had been at my new home
but three days, before Mr. Covey, (my brother in the Methodist church,)
gave me a bitter foretaste of what was in reserve for me. I presume he
thought, that since he had but a single year in which to complete his
work, the sooner he began, the better. Perhaps he thought that, by
coming to blows at once, we should mutually better understand our
relations. But to whatever motive, direct or indirect, the cause may be
referred, I had not been in his possession three whole days, before he
subjected me to a most brutal chastisement. Under his heavy blows, blood
flowed freely, and wales were left on my back as large as my little
finger. The sores on my back, from this flogging, continued for weeks,
for they were kept open by the rough and coarse cloth which I wore for
shirting. The occasion and details of this first chapter of my
experience as a field hand, must be told, that the reader may see how
unreasonable, as well as how cruel, my new master, Covey, was. The whole
thing I found to be characteristic of the man; and I was probably
treated no worse by him than scores of lads who had previously been
committed to him, for reasons similar to those which induced my master
to place me with him. But, here are the facts connected with the affair,
precisely as they occurred.

On one of the coldest days of the whole month of January, 1834, I was
ordered, at day break, to get a load of wood, from a forest about two
miles from the {\protect\hypertarget{209}{}{}}house. In order to perform
this work, Mr. Covey gave me a pair of unbroken oxen, for, it seems, his
breaking abilities had not been turned in this direction; and I may
remark, in passing, that working animals in the south, are seldom so
well trained as in the north. In due form, and with all proper ceremony,
I was introduced to this huge yoke of unbroken oxen, and was carefully
told which was ``Buck,'' and which was ``Darby''---which was the ``in
hand,'' and which was the ``off hand'' ox. The master of this important
ceremony was no less a person than Mr. Covey, himself; and the
introduction, was the first of the kind I had ever had. My life,
hitherto, had led me away from horned cattle, and I had no knowledge of
the art of managing them. What was meant by the ``in ox,'' as against
the ``off ox,'' when both were equally fastened to one cart, and under
one yoke, I could not very easily divine; and the difference, implied by
the names, and the peculiar duties of each, were alike \emph{Greek} to
me. Why was not the ``off ox'' called the ``in ox?'' Where and what is
the reason for this distinction in names, when there is none in the
things themselves? After initiating me into the "\emph{woa},"
"\emph{back}" "\emph{gee,}" "\emph{hither}``---the entire spoken
language between oxen and driver---Mr. Covey took a rope, about ten feet
long and one inch thick, and placed one end of it around the horns of
the ''in hand ox," and gave the other end to me, telling me that if the
oxen started to run away, as the scamp knew they would, I must hold on
to the rope and stop them. I need not tell any one who is acquainted
with either the strength or the disposition of an untamed ox, that
{\protect\hypertarget{210}{}{}}this order was about as unreasonable, as
a command to shoulder a mad bull! I had never driven oxen before, and I
was as awkward, as a driver, as it is possible to conceive. It did not
answer for me to plead ignorance, to Mr. Covey; there was something in
his manner that quite forbade that. He was a man to whom a slave seldom
felt any disposition to speak. Cold, distant, morose, with a face
wearing all the marks of captious pride and malicious sternness, he
repelled all advances. Covey was not a large man; he was only about five
feet ten inches in height, I should think; short necked, round
shoulders; of quick and wiry motion, of thin and wolfish visage; with a
pair of small, greenish-gray eyes, set well back under a forehead
without dignity, and constantly in motion, and floating his passions,
rather than his thoughts, in sight, but denying them utterance in words.
The creature presented an appearance altogether ferocious and sinister,
disagreeable and forbidding, in the extreme. When he spoke, it was from
the corner of his mouth, and in a sort of light growl, like a dog, when
an attempt is made to take a bone from him. The fellow had already made
me believe him even \emph{worse} than he had been represented. With his
directions, and without stopping to question, I started for the woods,
quite anxious to perform my first exploit in driving, in a creditable
manner. The distance from the house to the woods gate---a full mile, I
should think---was passed over with very little difficulty; for although
the animals ran, I was fleet enough, in the open field, to keep pace
with them; especially as they pulled me along at the end of the
{\protect\hypertarget{211}{}{}}rope; but, on reaching the woods, I was
speedily thrown into a distressing plight. The animals took fright, and
started off ferociously into the woods, carrying the cart, full tilt,
against trees, over stumps, and dashing from side to side, in a manner
altogether frightful. As I held the rope, I expected every moment to be
crushed between the cart and the huge trees, among which they were so
furiously dashing. After running thus for several minutes, my oxen were,
finally, brought to a stand, by a tree, against which they dashed
themselves with great violence, upsetting the cart, and entangling
themselves among sundry young saplings. By the shock, the body of the
cart was flung in one direction, and the wheels and tongue in another,
and all in the greatest confusion. There I was, all alone, in a thick
wood, to which I was a stranger; my cart upset and shattered; my oxen
entangled, wild, and enraged; and I, poor soul! but a green hand, to set
all this disorder right. I knew no more of oxen, than the ox driver is
supposed to know of wisdom. After standing a few moments surveying the
damage and disorder, and not without a pre-sentiment that this trouble
would draw after it others, even more distressing, I took one end of the
cart body, and, by an extra outlay of strength, I lifted it toward the
axle-tree, from which it had been violently flung; and after much
pulling and straining, I succeeded in getting the body of the cart in
its place. This was an important step out of the difficulty, and its
performance increased my courage for the work which remained to be done.
The cart was provided with an ax, a tool with which I had become pretty
{\protect\hypertarget{212}{}{}}well acquainted in the ship yard at
Baltimore. With this, I cut down the saplings by which my oxen were
entangled, and again pursued my journey, with my heart in my mouth, lest
the oxen should again take it into their senseless heads to cut up a
caper. My fears were groundless. Their spree was over for the present,
and the rascals now moved off as soberly as though their behavior had
been natural and exemplary. On reaching the part of the forest where I
had been, the day before, chopping wood, I filled the cart with a heavy
load, as a security against another running away. But, the neck of an ox
is equal in strength to iron. It defies all ordinary burdens, when
excited. Tame and docile to a proverb, when \emph{well} trained, the ox
is the most sullen and and intractable of animals when but half broken
to the yoke.

I now saw, in my situation, several points of similarity with that of
the oxen. They were property, so was I; they were to be broken, so was
I. Covey was to break me, I was to break them; break and be
broken---such is life.

Half the day already gone, and my face not yet homeward! It required
only two day's experience and observation to teach me, that such
apparent waste of time would not be lightly overlooked by Covey. I
therefore hurried toward home; but, on reaching the lane gate, I met
with the crowning disaster for the day. This gate was a fair specimen of
southern handicraft. There were two huge posts, eighteen inches in
diameter, rough hewed and square, and the heavy gate was so hung on one
of these, that it opened only about half the proper distance. On
{\protect\hypertarget{213}{}{}}arriving here, it was necessary for me to
let go the end of the rope on the horns of the ``in hand ox;'' and now
as soon as the gate was open, and I let go of it to get the rope, again,
off went my oxen---making nothing of their load---full tilt; and in
doing so they caught the huge gate between the wheel and the cart body,
literally crushing it to splinters, and coming only within a few inches
of subjecting me to a similar crushing, for I was just in advance of the
wheel when it struck the left gate post. With these two hair-breadth
escapes, I thought I could successfully explain to Mr. Covey the delay,
and avert apprehended punishment. I was not without a faint hope of
being commended for the stern resolution which I had displayed in
accomplishing the difficult task---a task which, I afterwards learned,
even Covey himself would not have undertaken, without first driving the
oxen for some time in the open field, preparatory to their going into
the woods. But, in this I was disappointed. On coming to him, his
countenance assumed an aspect of rigid displeasure, and, as I gave him a
history of the casualties of my trip, his wolfish face, with his
greenish eyes, became intensely ferocious. ``Go back to the woods
again,'' he said, muttering something else about wasting time. I hastily
obeyed; but I had not gone far on my way, when I saw him coming after
me. My oxen now behaved themselves with singular propriety, opposing
their present conduct to my representation of their former antics. I
almost wished, now that Covey was coming, they would do something in
keeping with the character I had given them; but no, they had already
{\protect\hypertarget{214}{}{}}had their spree, and they could afford
now to be extra good, readily obeying my orders, and seeming to
understand them quite as well as I did myself. On reaching the woods, my
tormentor---who seemed all the way to be remarking upon the good
behavior of his oxen---came up to me, and ordered me to stop the cart,
accompanying the same with the threat that he would now teach me how to
break gates, and idle away my time, when he sent me to the woods.
Suiting the action to the word, Covey paced off, in his own wiry
fashion, to a large, black-gum tree, the young shoots of which are
generally used for \emph{ox goads}, they being exceedingly tough. Three
of these \emph{goads}, from four to six feet long, he cut off, and
trimmed up, with his large jack-knife. This done, he ordered me to take
off my clothes. To this unreasonable order I made no reply, but sternly
refused to take off my clothing. ``If you will beat me,'' thought I,
``you shall do so over my clothes.'' After many threats, which made no
impression on me, he rushed at me with something of the savage
fierceness of a wolf, tore off the few and thinly worn clothes I had on,
and proceeded to wear out, on my back, the heavy goads which he had cut
from the gum tree. This flogging was the first of a series of floggings;
and though very severe, it was less so than many which came after it,
and these, for offenses far lighter than the gate breaking.

I remained with Mr. Covey one year, (I cannot say I \emph{lived} with
him,) and during the first six months that I was there, I was whipped,
either with sticks or cowskins, every week. Aching bones and a sore back
{\protect\hypertarget{215}{}{}}were my constant companions. Frequent as
the lash was used, Mr. Covey thought less of it, as a means of breaking
down my spirit, than that of hard and long continued labor. He worked me
steadily, up to the point of my powers of endurance. From the dawn of
day in the morning, till the darkness was complete in the evening, I was
kept at hard work, in the field or the woods. At certain seasons of the
year, we were all kept in the field till eleven and twelve o'clock at
night. At these times, Covey would attend us in the field, and urge us
on with words or blows, as it seemed best to him. He had, in his life,
been an overseer, and he well understood the business of slave driving.
There was no deceiving him. He knew just what a man or boy could do, and
he held both to strict account. When he pleased, he would work himself,
like a very Turk, making everything fly before him. It was, however,
scarcely necessary for Mr. Covey to be really present in the field, to
have his work go on industriously. He had the faculty of making us feel
that he was always present. By a series of adroitly managed surprises,
which he practiced, I was prepared to expect him at any moment. His plan
was, never to approach the spot where his hands were at work, in an
open, manly and direct manner. No thief was ever more artful in his
devices than this man Covey. He would creep and crawl, in ditches and
gullies; hide behind stumps and bushes, and practice so much of the
cunning of the serpent, that Bill Smith and I---between
ourselves---never called him by any other name than "\emph{the snake}."
We fancied that in his eyes and his gait we could see a
{\protect\hypertarget{216}{}{}}snakish resemblance. One half of his
proficiency in the art of negro breaking, consisted, I should think, in
this species of cunning. We were never secure. He could see or hear us
nearly all the time. He was, to us, behind every stump, tree, bush and
fence on the plantation. He carried this kind of trickery so far, that
he would sometimes mount his horse, and make believe he was going to St.
Michael's; and, in thirty minutes afterward, you might find his horse
tied in the woods, and the snake-like Covey lying flat in the ditch,
with his head lifted above its edge, or in a fence corner, watching
every movement of the slaves! I have known him walk up to us and give us
special orders, as to our work, in advance, as if he were leaving home
with a view to being absent several days; and before he got half way to
the house, he would avail himself of our inattention to his movements,
to turn short on his heels, conceal himself behind a fence corner or a
tree, and watch us until the going down of the sun. Mean and
contemptible as is all this, it is in keeping with the character which
the life of a slaveholder is calculated to produce. There is no earthly
inducement, in the slave's condition, to incite him to labor faithfully.
The fear of punishment is the sole motive for any sort of industry, with
him. Knowing this fact, as the slaveholder does, and judging the slave
by himself, he naturally concludes the slave will be idle whenever the
cause for this fear is absent. Hence, all sorts of petty deceptions are
practiced, to inspire this fear.

But, with Mr. Covey, trickery was natural. Everything in the shape of
learning or religion, which {\protect\hypertarget{217}{}{}}he possessed,
was made to conform to this semi-lying propensity. He did not seem
conscious that the practice had anything unmanly, base or contemptible
about it. It was a part of an important system, with him, essential to
the relation of master and slave. I thought I saw, in his very religious
devotions, this controlling element of his character. A long prayer at
night made up for the short prayer in the morning; and few men could
seem more devotional than he, when he had nothing else to do.

Mr. Covey was not content with the cold style of family worship, adopted
in these cold latitudes, which begin and end with a simple prayer. No!
the voice of praise, as well as of prayer, must be heard in his house,
night and morning. At first, I was called upon to bear some part in
these exercises; but the repeated flogging given me by Covey, turned the
whole thing into mockery. He was a poor singer, and mainly relied on me
for raising the hymn for the family, and when I failed to do so, he was
thrown into much confusion. I do not think that he ever abused me on
account of these vexations. His religion was a thing altogether apart
from his worldly concerns. He knew nothing of it as a holy principle,
directing and controlling his daily life, making the latter conform to
the requirements of the gospel. One or two facts will illustrate his
character better than a volume of generalities.

I have already said, or implied, that Mr. Edward Covey was a poor man.
He was, in fact, just commencing to lay the foundation of his fortune,
as fortune is regarded in a slave state. The first condition of wealth
{\protect\hypertarget{218}{}{}}and respectability there, being the
ownership of human property, every nerve is strained, by the poor man,
to obtain it, and very little regard is had to the manner of obtaining
it. In pursuit of this object, pious as Mr. Covey was, he proved himself
to be as unscrupulous and base as the worst of his neighbors. In the
beginning, he was only able---as he said---``to buy one slave;'' and,
scandalous and shocking as is the fact, he boasted that he bought her
simply "\emph{as a breeder.}" But the worst is not told in this naked
statement. This young woman (Caroline was her name) was virtually
compelled by Mr. Covey to abandon herself to the object for which he had
purchased her; and the result was, the birth of twins at the end of the
year. At this addition to his human stock, both Edward Covey and his
wife, Susan, were extatic with joy. No one dreamed of reproaching the
woman, or of finding fault with the hired man---Bill Smith---the father
of the children, for Mr. Covey himself had locked the two up together
every night, thus inviting the result.

But I will pursue this revolting subject no further. No better
illustration of the unchaste and demoralizing character of slavery can
be found, than is furnished in the fact that this professedly christian
slaveholder, amidst all his prayers and hymns, was shamelessly and
boastfully encouraging, and actually compelling, in his own house,
undisguised and unmitigated fornication, as a means of increasing his
human stock. I may remark here, that, while this fact will be read with
disgust and shame at the north, it will be \emph{laughed at}, as smart
and praiseworthy in Mr. Covey, at the
{\protect\hypertarget{219}{}{}}south; for a man is no more condemned
there for buying a woman and devoting her to this life of dishonor, than
for buying a cow, and raising stock from her. The same rules are
observed, with a view to increasing the number and quality of the
former, as of the latter.

I will here reproduce what I said of my own experience in this wretched
place, more than ten years ago:

~

"If at any one time of my life, more than another, I was made to drink
the bitterest dregs of slavery, that time was during the first six
months of my stay with Mr. Covey. We were worked all weathers. It was
never too hot or too cold; it could never rain, blow, snow, or hail too
hard for us to work in the field. Work, work, work, was scarcely more
the order of the day than of the night. The longest days were too short
for him, and the shortest nights were too long for him. I was somewhat
unmanageable when I first went there; but a few months of this
discipline tamed me. Mr. Covey succeeded in breaking me. I was broken in
body, soul and spirit. My natural elasticity was crushed; my intellect
languished; the disposition to read departed; the cheerful spark that
lingered about my eye died; the dark night of slavery closed in upon me;
and behold a man transformed into a brute!

"Sunday was my only leisure time. I spent this in a sort of beast-like
stupor, between sleep and wake, under some large tree. At times, I would
rise up, a flash of energetic freedom would dart through my soul,
accompanied with a faint beam of hope, that flickered for a moment, and
then vanished. I sank down again, mourning over my wretched condition. I
was sometimes prompted to take my life, and that of Covey, but was
prevented by a combination of hope and fear. My sufferings on this
plantation seem now like a dream rather than a stern reality.

{\protect\hypertarget{220}{}{}}"Our house stood within a few rods of the
Chesapeake bay, whose broad bosom was ever white with sails from every
quarter of the habitable globe. Those beautiful vessels, robed in purest
white, so delightful to the eye of freemen, were to me so many shrouded
ghosts, to terrify and torment me with thoughts of my wretched
condition. I have often, in the deep stillness of a summer's Sabbath,
stood all alone upon the banks of that noble bay, and traced, with
saddened heart and tearful eye, the countless number of sails moving off
to the mighty ocean. The sight of these always affected me powerfully.
My thoughts would compel utterance; and there, with no audience but the
Almighty, I would pour out my soul's complaint in my rude way, with an
apostrophe to the moving multitude of ships:

{"}'You are loosed from your moorings, and free; I am fast in my chains,
and am a slave! You move merrily before the gentle gale, and I sadly
before the bloody whip! You are freedom's swift-winged angels, that fly
around the world; I am confined in bands of iron! O, that I were free!
O, that I were on one of your gallant decks, and under your protecting
wing! Alas! betwixt me and you the turbid waters roll. Go on, go on. O
that I could also go! Could I but swim! If I could fly! O, why was I
born a man, of whom to make a brute! The glad ship is gone; she hides in
the dim distance. I am left in the hottest hell of unending slavery. O
God, save me! God, deliver me! Let me be free! Is there any God? Why am
I a slave? I will run away. I will not stand it. Get caught, or get
clear, I'll try it. I had as well die with ague as with fever. I have
only one life to lose. I had as well be killed running as die standing.
Only think of it; one hundred miles straight north, and I am free! Try
it? Yes! God helping me, I will. It cannot be that I shall live and die
a slave. I will take to the water. This very bay shall yet bear me into
freedom. The steamboats steered in a north-east coast from North
{\protect\hypertarget{221}{}{}}Point. I will do the same; and when I get
to the head of the bay, I will turn my canoe adrift, and walk straight
through Delaware into Pennsylvania. When I get there, I shall not be
required to have a pass; I will travel without being disturbed. Let but
the first opportunity offer, and, come what will, I am off. Meanwhile, I
will try to bear up under the yoke. I am not the only slave in the
world. Why should I fret? I can bear as much as any of them. Besides, I
am but a boy, and all boys are bound to some one. It may be that my
misery in slavery will only increase my happiness when I get free. There
is a better day coming.{'}"

~

I shall never be able to narrate the mental experience through which it
was my lot to pass during my stay at Covey's. I was completely wrecked,
changed and bewildered; goaded almost to madness at one time, and at
another reconciling myself to my wretched condition. Everything in the
way of kindness, which I had experienced at Baltimore; all my former
hopes and aspirations for usefulness in the world, and the happy moments
spent in the exercises of religion, contrasted with my then present lot,
but increased my anguish.

I suffered bodily as well as mentally. I had neither sufficient time in
which to eat or to sleep, except on Sundays. The over work, and the
brutal chastisements of which I was the victim, combined with that
ever-gnawing and soul-devouring thought---"\emph{I am a slave---a slave
for life---a slave with no rational ground to hope for
freedom}"---rendered me a living embodiment of mental and physical
wretchedness.
