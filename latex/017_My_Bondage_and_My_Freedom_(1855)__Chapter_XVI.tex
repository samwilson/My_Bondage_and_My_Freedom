\hypertarget{headerContainer}{}
\hypertarget{navigationHeader}{}
\protect\hypertarget{headerprevious}{}{←\href{/wiki/My_Bondage_and_My_Freedom_(1855)/Chapter_XV}{Chapter
XV}}

\textbf{\protect\hypertarget{header_title_text}{}{\href{/wiki/My_Bondage_and_My_Freedom_(1855)}{My
Bondage and My Freedom}}} ~(1855)~ \emph{by
\href{/wiki/Author:Frederick_Douglass}{\protect\hypertarget{header_author_text}{}{{Frederick
Douglass}}}}\\
\protect\hypertarget{header_section_text}{}{Chapter XVI}

\protect\hypertarget{headernext}{}{\href{/wiki/My_Bondage_and_My_Freedom_(1855)/Chapter_XVII}{Chapter
XVII}→}

\hypertarget{navigationNotes}{}

\hypertarget{ws-data}{}
\protect\hypertarget{ws-article-id}{}{2339059}\protect\hypertarget{ws-title}{}{\href{/wiki/My_Bondage_and_My_Freedom_(1855)}{My
Bondage and My Freedom} --- \emph{Chapter
XVI}}\protect\hypertarget{ws-author}{}{Frederick
Douglass}\protect\hypertarget{ws-year}{}{1855}

{\protect\hypertarget{222}{}{}}

~

{CHAPTER XVI.}

ANOTHER PRESSURE OF THE TYRANT'S VICE.

{EXPERIENCE AT COVEY'S SUMMED UP---FIRST SIX MONTHS SEVERER THAN THE
SECOND---PRELIMINARIES TO THE CHANGE---REASONS FOR NARRATING THE
CIRCUMSTANCES---SCENE IN THE TREADING YARD---AUTHOR TAKEN ILL---UNUSUAL
BRUTALITY OF COVEY---AUTHOR ESCAPES TO ST. MICHAEL'S---THE
PURSUIT---SUFFERING IN THE WOODS---DRIVEN BACK AGAIN TO
COVEY'S---BEARING OF ``MASTER THOMAS''---THE SLAVE IS NEVER
SICK---NATURAL TO EXPECT SLAVES TO FEIGN SICKNESS---LAZINESS OF
SLAVEHOLDERS.}

\textsc{The} foregoing chapter, with all its horrid incidents and
shocking features, may be taken as a fair representation of the first
six months of my life at Covey's. The reader has but to repeat, in his
own mind, once a week, the scene in the woods, where Covey subjected me
to his merciless lash, to have a true idea of my bitter experience
there, during the first period of the breaking process through which Mr.
Covey carried me. I have no heart to repeat each separate transaction,
in which I was a victim of his violence and brutality. Such a narration
would fill a volume much larger than the present one. I aim only to give
the reader a truthful impression of my slave life, without unnecessarily
affecting him with harrowing details.

As I have elsewhere intimated that my hardships were much greater during
the first six months of my stay at Covey's, than during the remainder of
the year, {\protect\hypertarget{223}{}{}}and as the change in my
condition was owing to causes which may help the reader to a better
understanding of human nature, when subjected to the terrible
extremities of slavery, I will narrate the circumstances of this change,
although I may seem thereby to applaud my own courage.

You have, dear reader, seen me humbled, degraded, broken down, enslaved,
and brutalized, and you understand how it was done; now let us see the
converse of all this, and how it was brought about; and this will take
us through the year 1834.

On one of the hottest days of the month of August, of the year just
mentioned, had the reader been passing through Covey's farm, he might
have seen me at work, in what is there called the ``treading yard''---a
yard upon which wheat is trodden out from the straw, by the horses'
feet. I was there, at work, feeding the ``fan,'' or rather bringing
wheat to the fan, while Bill Smith was feeding. Our force consisted of
Bill Hughes, Bill Smith, and a slave by the name of Eli; the latter
having been hired for this occasion. The work was simple, and required
strength and activity, rather than any skill or intelligence, and yet,
to one entirely unused to such work, it came very hard. The heat was
intense and overpowering, and there was much hurry to get the wheat,
trodden out that day, through the fan; since, if that work was done an
hour before sundown, the hands would have, according to a promise of
Covey, that hour added to their night's rest. I was not behind any of
them in the wish to complete the day's work before sundown, and, hence,
I struggled with all my might to get the work forward. The
{\protect\hypertarget{224}{}{}}promise of one hour's repose on a week
day, was sufficient to quicken my pace, and to spur me on to extra
endeavor. Besides, we had all planned to go fishing, and I certainly
wished to have a hand in that. But I was disappointed, and the day
turned out to be one of the bitterest I ever experienced. About three
o'clock, while the sun was pouring down his burning rays, and not a
breeze was stirring, I broke down; my strength failed me; I was seized
with a violent aching of the head, attended with extreme dizziness, and
trembling in every limb. Finding what was coming, and feeling it would
never do to stop work, I nerved myself up, and staggered on until I fell
by the side of the wheat fan, feeling that the earth had fallen upon me.
This brought the entire work to a dead stand. There was work for four;
each one had his part to perform, and each part depended on the other,
so that when one stopped, all were compelled to stop. Covey, who had now
become my dread, as well as my tormentor, was at the house, about a
hundred yards from where I was fanning, and instantly, upon hearing the
fan stop, he came down to the treading yard, to inquire into the cause
of our stopping. Bill Smith told him I was sick, and that I was unable
longer to bring wheat to the fan.

I had, by this time, crawled away, under the side of a post-and-rail
fence, in the shade, and was exceedingly ill. The intense heat of the
sun, the heavy dust rising from the fan, the stooping, to take up the
wheat from the yard, together with the hurrying, to get through, had
caused a rush of blood to my head. In this condition, Covey finding out
where I was, came {\protect\hypertarget{225}{}{}}to me; and, after
standing over me a while, he asked me what the matter was. I told him as
well as I could, for it was with difficulty that I could speak. He then
gave me a savage kick in the side, which jarred my whole frame, and
commanded me to get up. The man had obtained complete control over me;
and if he had commanded me to do any possible thing, I should, in my
then state of mind, have endeavored to comply. I made an effort to rise,
but fell back in the attempt, before gaining my feet. The brute now gave
me another heavy kick, and again told me to rise. I again tried to rise,
and succeeded in gaining my feet; but, upon stooping to get the tub with
which I was feeding the fan, I again staggered and fell to the ground;
and I must have so fallen, had I been sure that a hundred bullets would
have pierced me, as the consequence. While down, in this sad condition,
and perfectly helpless, the merciless negro breaker took up the hickory
slab, with which Hughes had been striking off the wheat to a level with
the sides of the half bushel measure, (a very hard weapon,) and with the
sharp edge of it, he dealt me a heavy blow on my head which made a large
gash, and caused the blood to run freely, saying, at the same time,
"\emph{If you have got the headache, I'll cure you}." This done, he
ordered me again to rise, but I made no effort to do so; for I had made
up my mind that it was useless, and that the heartless monster might
\emph{now} do his worst; he could but kill me, and that might put me out
of my misery. Finding me unable to rise, or rather despairing of my
doing so, Covey left me, with a view to getting on with the work without
me. I {\protect\hypertarget{226}{}{}}was bleeding very freely, and my
face was soon covered with my warm blood. Cruel and merciless as was the
motive that dealt that blow, dear reader, the wound was fortunate for
me. Bleeding was never more efficacious. The pain in my head speedily
abated, and I was soon able to rise. Covey had, as I have said, now left
me to my fate; and the question was, shall I return to my work, or shall
I find my way to St. Michael's, and make Capt. Auld acquainted with the
atrocious cruelty of his brother Covey, and beseech him to get me
another master? Remembering the object he had in view, in placing me
under the management of Covey, and further, his cruel treatment of my
poor crippled cousin, Henny, and his meanness in the matter of feeding
and clothing his slaves, there was little ground to hope for a favorable
reception at the hands of Capt. Thomas Auld. Nevertheless, I resolved to
go straight to Capt. Auld, thinking that, if not animated by motives of
humanity, he might be induced to interfere on my behalf from selfish
considerations. ``He cannot,'' thought I, ``allow his property to be
thus bruised and battered, marred and defaced; and I will go to him, and
tell him the simple truth about the matter.'' In order to get to St.
Michael's, by the most favorable and direct road, I must walk seven
miles; and this, in my sad condition, was no easy performance. I had
already lost much blood; I was exhausted by over exertion; my sides were
sore from the heavy blows planted there by the stout boots of Mr. Covey;
and I was, in every way, in an unfavorable plight for the journey. I
however watched my chance, while the cruel and
{\protect\hypertarget{227}{}{}}cunning Covey was looking in an opposite
direction, and started off, across the field, for St. Michael's. This
was a daring step; if it failed, it would only exasperate Covey, and
increase the rigors of my bondage, during the remainder of my term of
service under him; but the step was taken, and I must go forward. I
succeeded in getting nearly half way across the broad field, toward the
woods, before Mr. Covey observed me. I was still bleeding, and the
exertion of running had started the blood afresh. "\emph{Come back! Come
back!}" vociferated Covey, with threats of what he would do if I did not
return instantly. But, disregarding his calls and his threats, I pressed
on toward the woods as fast as my feeble state would allow. Seeing no
signs of my stopping, Covey caused his horse to be brought out and
saddled, as if he intended to pursue me. The race was now to be an
unequal one; and, thinking I might be overhauled by him, if I kept the
main road, I walked nearly the whole distance in the woods, keeping far
enough from the road to avoid detection and pursuit. But, I had not gone
far, before my little strength again failed me, and I laid down. The
blood was still oozing from the wound in my head; and, for a time, I
suffered more than I can describe. There I was, in the deep woods, sick
and emaciated, pursued by a wretch whose character for revolting cruelty
beggars all opprobrious speech---bleeding, and almost bloodless. I was
not without the fear of bleeding to death. The thought of dying in the
woods, all alone, and of being torn to pieces by the buzzards, had not
yet been rendered tolerable by my many troubles and hardships,
{\protect\hypertarget{228}{}{}}and I was glad when the shade of the
trees, and the cool evening breeze, combined with my matted hair to stop
the flow of blood. After lying there about three quarters of an hour,
brooding over the singular and mournful lot to which I was doomed, my
mind passing over the whole scale or circle of belief and unbelief, from
faith in the overruling providence of God, to the blackest atheism, I
again took up my journey toward St. Michael's, more weary and sad than
in the morning when I left. Thomas Auld's for the home of Mr. Covey. I
was bare-footed and bareheaded, and in my shirt sleeves. The way was
through bogs and briers, and I tore my feet often during the journey. I
was full five hours in going the seven or eight miles; partly, because
of the difficulties of the way, and partly, because of the feebleness
induced by my illness, bruises and loss of blood. On gaining my master's
store, I presented an appearance of wretchedness and woe, fitted to move
any but a heart of stone. From the crown of my head to the sole of my
feet, there were marks of blood. My hair was all clotted with dust and
blood, and the back of my shirt was literally stiff with the same.
Briers and thorns had scarred and torn my feet and legs, leaving blood
marks there. Had I escaped from a den of tigers, I could not have looked
worse than I did on reaching St. Michael's. In this unhappy plight, I
appeared before my professedly \emph{christian} master, humbly to invoke
the interposition of his power and authority, to protect me from further
abuse and violence. I had begun to hope, during the latter part of my
tedious journey toward St. Michael's, that Capt. Auld would
{\protect\hypertarget{229}{}{}} now show himself in a nobler light than
I had ever before seen him. I was disappointed. I had jumped from a
sinking ship into the sea; I had fled from the tiger to something worse.
I told him all the circumstances, as well as I could; how I was
endeavoring to please Covey; how hard I was at work in the present
instance; how unwillingly I sunk down under the heat, toil and pain; the
brutal manner in which Covey had kicked me in the side; the gash cut in
my head; my hesitation about troubling him (Capt. Auld) with complaints;
but, that now I felt it would not be best longer to conceal from him the
outrages committed on me from time to time by Covey. At first, master
Thomas seemed somewhat affected by the story of my wrongs, but he soon
repressed his feelings and became cold as iron. It was impossible---as I
stood before him at the first---for him to seem indifferent. I
distinctly saw his human nature asserting its conviction against the
slave system, which made cases like mine \emph{possible;} but, as I have
said, humanity fell before the systematic tyranny of slavery. He first
walked the floor, apparently much agitated by my story, and the sad
spectacle I presented; but, presently, it was \emph{his} turn to talk.
He began moderately, by finding excuses for Covey, and ending with a
full justification of him, and a passionate condemnation of me. ``He had
no doubt I deserved the flogging. He did not believe I was sick; I was
only endeavoring to get rid of work. My dizziness was laziness, and
Covey did right to flog me, as he had done.'' After thus fairly
annihilating me, and rousing himself by his own eloquence,
{\protect\hypertarget{230}{}{}}he fiercely demanded what I wished
\emph{him} to do in the case!

With such a complete knock-down to all my hopes, as he had given me, and
feeling, as I did, my entire subjection to his power, I had very little
heart to reply. I must not affirm my innocence of the allegations which
he had piled up against me; for that would be impudence, and would
probably call down fresh violence as well as wrath upon me. The guilt of
a slave is always, and everywhere, presumed; and the innocence of the
slaveholder or the slave employer, is always asserted. The word of the
slave, against this presumption, is generally treated as impudence,
worthy of punishment. ``Do you contradict me, you rascal?'' is a final
silencer of counter statements from the lips of a slave.

Calming down a little in view of my silence and hesitation, and,
perhaps, from a rapid glance at the picture of misery I presented, he
inquired again, ``what I would have him do?'' Thus invited a second
time, I told Master Thomas I wished him to allow me to get a new home
and to find a new master; that, as sure as I went back to live with Mr.
Covey again, I should be killed by him; that he would never forgive my
coming to him (Capt Auld) with a complaint against him (Covey;) that,
since I had lived with him, he had almost crushed my spirit, and I
believed that he would ruin me for future service; that my life was not
safe in his hands. This, Master Thomas (\emph{my brother in the church})
regarded as ``nonsense.'' "There was no danger of Mr. Covey's killing
me; he was a good man, industrious and religious,
{\protect\hypertarget{231}{}{}}and he would not think of removing me
from that home; ``besides,'' said he,---and this I found was the most
distressing thought of all to him---"if you should leave Covey now, that
your year has but half expired, I should lose your wages for the entire
year. You belong to Mr. Covey for one year, and you \emph{must go back}
to him, come what will. You must not trouble me with any more stories
about Mr. Covey; and if you do not go immediately home, I will get hold
of you myself." This was just what I expected, when I found he had
\emph{prejudged} the case against me. ``But, Sir,'' I said, ``I am sick
and tired, and I cannot get home to-night.'' At this, he again relented,
and finally he allowed me to remain all night at St. Michael's; but said
I must be off early in the morning, and concluded his directions by
making me swallow a huge dose of \emph{epsom salts}---about the only
medicine ever administered to slaves.

It was quite natural for Master Thomas to presume I was feigning
sickness to escape work, for he probably thought that were \emph{he} in
the place of a slave---with no wages for his work, no praise for well
doing, no motive for toil but the lash---he would try every possible
scheme by which to escape labor. I say I have no doubt of this; the
reason is, that there are not, under the whole heavens, a set of men who
cultivate such an intense dread of labor as do the slaveholders. The
charge of laziness against the slaves is ever on their lips, and is the
standing apology for every species of cruelty and brutality. These men
literally "bind heavy burdens, grievous to be borne, and lay them
{\protect\hypertarget{232}{}{}}on men's shoulders; but they, themselves,
will not move them with one of their fingers."

My kind readers shall have, in the next chapter---what they were led,
perhaps, to expect to find in this---namely: an account of my partial
disenthrallment from the tyranny of Covey, and the marked change which
it brought about.
