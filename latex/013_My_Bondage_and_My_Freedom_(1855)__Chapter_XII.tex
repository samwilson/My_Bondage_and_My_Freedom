{\protect\hypertarget{163}{}{}}

~

{CHAPTER XII.}

RELIGIOUS NATURE AWAKENED.

{ABOLITIONISTS SPOKEN OF---MY EAGERNESS TO KNOW WHAT THIS WORD
MEANT---MY CONSULTATION OF THE DICTIONARY---INCENDIARY INFORMATION---HOW
AND WHERE DERIVED---THE ENIGMA SOLVED---NATHANIEL TURNER'S
INSURRECTION---THE CHOLERA---RELIGION---FIRST AWAKENED BY A METHODIST
MINISTER, NAMED HANSON---MY DEAR AND GOOD OLD COLORED FRIEND,
LAWSON---HIS CHARACTER AND OCCUPATION---HIS INFLUENCE OVER ME---OUR
MUTUAL ATTACHMENT---THE COMFORT I DERIVED FROM HIS TEACHING---NEW HOPES
AND ASPIRATIONS---HEAVENLY LIGHT AMIDST EARTHLY DARKNESS---THE TWO
IRISHMEN ON THE WHARF---THEIR CONVERSATION---HOW I LEARNED TO
WRITE---WHAT WERE MY AIMS.}

\textsc{Whilst} in the painful state of mind described in the foregoing
chapter, almost regretting my very existence, because doomed to a life
of bondage, so goaded and so wretched, at times, that I was even tempted
to destroy my own life, I was yet keenly sensitive and eager to know
any, and every thing that transpired, having any relation to the subject
of slavery. I was all ears, all eyes, whenever the words \emph{slave,
slavery}, dropped from the lips of any white person, and the occasions
were not unfrequent when these words became leading ones, in high,
social debate, at our house. Every little while, I could overhear Master
Hugh, or some of his company, speaking with much warmth and excitement
about "\emph{abolitionists}." Of \emph{who} or \emph{what} these were, I
was totally ignorant. I found, however, that whatever they might be,
they were most {\protect\hypertarget{164}{}{}}cordially hated and
soundly abused by slaveholders, of every grade. I very soon discovered,
too, that slavery was, in some sort, under consideration, whenever the
abolitionists were alluded to. This made the term a very interesting one
to me. If a slave, for instance, had made good his escape from slavery,
it was generally alleged, that he had been persuaded and assisted by the
abolitionists. If, also, a slave killed his master---as was sometimes
the case---or struck down his overseer, or set fire to his master's
dwelling, or committed any violence or crime, out of the common way, it
was certain to be said, that such a crime was the legitimate fruits of
the abolition movement. Hearing such charges often repeated, I,
naturally enough, received the impression that abolition---whatever else
it might be---could not be unfriendly to the slave, nor very friendly to
the slaveholder. I therefore set about finding out, if possible,
\emph{who} and \emph{what} the abolitionists were, and \emph{why} they
were so obnoxious to the slaveholders. The dictionary afforded me very
little help. It taught me that abolition was the ``act of abolishing;''
but it left me in ignorance at the very point where I most wanted
information---and that was, as to the \emph{thing} to be abolished. A
city newspaper, the ``Baltimore American,'' gave me the incendiary
information denied me by the dictionary. In its columns I found, that,
on a certain day, a vast number of petitions and memorials had been
presented to congress, praying for the abolition of slavery in the
District of Columbia, and for the abolition of the slave trade between
the states of the Union. This was enough. The vindictive bitterness,
{\protect\hypertarget{165}{}{}}the marked caution, the studied reserve,
and the cumbrous ambiguity, practiced by our white folks, when allluding
to this subject, was now fully explained. Ever, after that, when I heard
the words ``abolition,'' or ``abolition movement,'' mentioned, I felt
the matter one of a personal concern; and I drew near to listen, when I
could do so, without seeming too solicitous and prying. There was
\textsc{Hope} in those words. Ever and anon, too, I could see some
terrible denunciation of slavery, in our papers---copied from abolition
papers at the north,---and the injustice of such denunciation commented
on. These I read with avidity. I had a deep satisfaction in the thought,
that the rascality of slaveholders was not concealed from the eyes of
the world, and that I was not alone in abhorring the cruelty and
brutality of slavery. A still deeper train of thought was stirred. I saw
that there was \emph{fear}, as well as \emph{rage}, in the manner of
speaking of the abolitionists. The latter, therefore, I was compelled to
regard as having some power in the country; and I felt that they might,
possibly, succeed in their designs. When I met with a slave to whom I
deemed it safe to talk on the subject, I would impart to him so much of
the mystery as I had been able to penetrate. Thus, the light of this
grand movement broke in upon my mind, by degrees; and I must say, that,
ignorant as I then was of the philosophy of that movement, I believed in
it from the first---and I believed in it, partly, because I saw that it
alarmed the consciences of slaveholders. The insurrection of Nathaniel
Turner had been quelled, but the alarm and terror had not subsided. The
cholera was on its way, {\protect\hypertarget{166}{}{}}and the thought
was present, that God was angry with the white people because of their
slaveholding wickedness, and, therefore, his judgments were abroad in
the land. It was impossible for me not to hope much from the abolition
movement, when I saw it supported by the Almighty, and armed with
\textsc{Death}!

Previous to my contemplation of the anti-slavery movement, and its
probable results, my mind had been seriously awakened to the subject of
religion. I was not more than thirteen years old, when I felt the need
of God, as a father and protector. My religious nature was awakened by
the preaching of a white Methodist minister, named Hanson. He thought
that all men, great and small, bond and free, were sinners in the sight
of God; that they were, by nature, rebels against His government; and
that they must repent of their sins, and be reconciled to God, through
Christ. I cannot say that I had a very distinct notion of what was
required of me; but one thing I knew very well---I was wretched, and had
no means of making myself otherwise. Moreover, I knew that I could pray
for light. I consulted a good colored man, named Charles Johnson; and,
in tones of holy affection, he told me to pray, and what to pray for. I
was, for weeks, a poor, broken-hearted mourner, traveling through the
darkness and misery of doubts and fears. I finally found that change of
heart which comes by ``casting all one's care'' upon God, and by having
faith in Jesus Christ, as the Redeemer, Friend, and Savior of those who
diligently seek Him.

After this, I saw the world in a new light. I
{\protect\hypertarget{167}{}{}}seemed to live in a new world, surrounded
by new objects, and to be animated by new hopes and desires. I loved all
mankind---slaveholders not excepted; though I abhorred slavery more than
ever. My great concern was, now, to have the world converted. The desire
for knowledge increased, and especially did I want a thorough
acquaintance with the contents of the bible. I have gathered scattered
pages from this holy book, from the filthy street gutters of Baltimore,
and washed and dried them, that in the moments of my leisure, I might
get a word or two of wisdom from them. While thus religiously seeking
knowledge, I became acquainted with a good old colored man, named
Lawson. A more devout man than he, I never saw. He drove a dray for Mr.
James Ramsey, the owner of a rope-walk on Fell's Point, Baltimore. This
man not only prayed three times a day, but he prayed as he walked
through the streets, at his work---on his dray---everywhere. His life
was a life of prayer, and his words, (when he spoke to his friends,)
were about a better world. Uncle Lawson lived near Master Hugh's house;
and, becoming deeply attached to the old man, I went often with him to
prayer-meeting, and spent much of my leisure time with him on Sunday.
The old man could read a little, and I was a great help to him, in
making out the hard words, for I was a better reader than he. I could
teach him "\emph{the letter,}" but he could teach me "\emph{the
spirit;}" and high, refreshing times we had together, in singing,
praying and glorifying God. These meetings with Uncle Lawson went on for
a long time, without the knowledge of Master Hugh or
{\protect\hypertarget{168}{}{}}my mistress. Both knew, however, that I
had become religious, and they seemed to respect my conscientious piety.
My mistress was still a professor of religion, and belonged to class.
Her leader was no less a person than the Rev. Beverly Waugh, the
presiding elder, and now one of the bishops of the Methodist Episcopal
church. Mr. Waugh was then stationed over Wilk street church. I am
careful to state these facts, that the reader may be able to form an
idea of the precise influences which had to do with shaping and
directing my mind.

In view of the cares and anxieties incident to the life she was then
leading, and, especially, in view of the separation from religious
associations to which she was subjected, my mistress had, as I have
before stated, become lukewarm, and needed to be looked up by her
leader. This brought Mr. Waugh to our house, and gave me an opportunity
to hear him exhort and pray. But my chief instructor, in matters of
religion, was Uncle Lawson. He was my spiritual father; and I loved him
intensely, and was at his house every chance I got.

This pleasure was not long allowed me. Master Hugh became averse to my
going to Father Lawson's, and threatened to whip me if I ever went there
again. I now felt myself persecuted by a wicked man; and I \emph{would}
go to Father Lawson's, notwithstanding the threat. The good old man had
told me, that the ``Lord had a great work for me to do;'' and I must
prepare to do it; and that he had been shown that I must preach the
gospel. His words made a deep impression on my mind, and I verily felt
that some such {\protect\hypertarget{169}{}{}}work was before me, though
I could not see \emph{how} I should ever engage in its performance.
``The good Lord,'' he said, ``would bring it to pass in his own good
time,'' and that I must go on reading and studying the scriptures. The
advice and the suggestions of Uncle Lawson, were not without their
influence upon my character and destiny. He threw my thoughts into a
channel from which they have never entirely diverged. He fanned my
already intense love of knowledge into a flame, by assuring me that I
was to be a useful man in the world. When I would say to him, "How can
these things be---and what can \emph{I} do?" his simple reply was,
"\emph{Trust in the Lord}." When I told him that "I was a slave, and a
slave \textsc{for life}," he said, "the Lord can make you free, my dear.
All things are possible with him, only \emph{have faith in God}." ``Ask,
and it shall be given.'' ``If you want liberty,'' said the good old man,
"ask the Lord for it, \emph{in faith}, \textsc{and he will give it to
you}."

Thus assured, and cheered on, under the inspiration of hope, I worked
and prayed with a light heart, believing that my life was under the
guidance of a wisdom higher than my own. With all other blessings sought
at the mercy seat, I always prayed that God would, of His great mercy,
and in His own good time, deliver me from my bondage.

I went, one day, on the wharf of Mr. Waters; and seeing two Irishmen
unloading a large scow of stone, or ballast, I went on board, unasked,
and helped them. When we had finished the work, one of the men came to
me, aside, and asked me a number of questions, and among them, if I were
a slave. I told him {\protect\hypertarget{170}{}{}}``I was a slave, and
a slave for life.'' The good Irishman gave his shoulders a shrug, and
seemed deeply affected by the statement. He said, ``it was a pity so
fine a little fellow as myself should be a slave for life.'' They both
had much to say about the matter, and expressed the deepest sympathy
with me, and the most decided hatred of slavery. They went so far as to
tell me that I ought to run away, and go to the north; that I should
find friends there, and that I would be as free as anybody. I, however,
pretended not to be interested in what they said, for I feared they
might be treacherous. White men have been known to encourage slaves to
escape, and then---to get the reward---they have kidnapped them, and
returned them to their masters. And while I mainly inclined to the
notion that these men were honest and meant me no ill, I feared it might
be otherwise. I nevertheless remembered their words and their advice,
and looked forward to an escape to the north, as a possible means of
gaining the liberty for which my heart panted. It was not my
enslavement, at the then present time, that most affected me; the being
a slave \emph{for life}, was the saddest thought. I was too young to
think of running away immediately; besides, I wished to learn how to
write, before going, as I might have occasion to write my own pass. I
now not only had the hope of freedom, but a foreshadowing of the means
by which I might, some day, gain that inestimable boon. Meanwhile, I
resolved to add to my educational attainments the art of writing.

After this manner I began to learn to write: I was much in the ship
yard---Master Hugh's, and that of {\protect\hypertarget{171}{}{}}Durgan
\& Bailey---and I observed that the carpenters, after hewing and getting
a piece of timber ready for use, wrote on it the initials of the name of
that part of the ship for which it was intended. When, for instance, a
piece of timber was ready for the starboard side, it was marked with a
capital ``S.'' A piece for the larboard side was marked ``L;'' larboard
forward, ``L. F.;'' larboard aft, was marked ``L. A.;'' starboard aft,
``S. A.;'' and starboard forward ``S. F.'' I soon learned these letters,
and for what they were placed on the timbers.

My work was now, to keep fire under the steam box, and to watch the ship
yard while the carpenters had gone to dinner. This interval gave me a
fine opportunity for copying the letters named. I soon astonished myself
with the ease with which I made the letters; and the thought was soon
present, ``if I can make four, I can make more.'' But having made these
easily, when I met boys about Bethel church, or any of our play-grounds,
I entered the lists with them in the art of writing, and would make the
letters which I had been so fortunate as to learn, and ask them to
``beat that if they could.'' With playmates for my teachers, fences and
pavements for my copy books, and chalk for my pen and ink, I learned the
art of writing. I, however, afterward adopted various methods of
improving my hand. The most successful, was copying the \emph{italics}
in Webster's spelling book, until I could make them all without looking
on the book. By this time, my little ``Master Tommy'' had grown to be a
big boy, and had written over a number of copy books, and brought them
{\protect\hypertarget{172}{}{}}home. They had been shown to the
neighbors, had elicited due praise, and were now laid carefully away.
Spending my time between the ship yard and house, I was as often the
lone keeper of the latter as of the former. When my mistress left me in
charge of the house, I had a grand time; I got Master Tommy's copy books
and a pen and ink, and, in the ample spaces between the lines, I wrote
other lines, as nearly like his as possible. The process was a tedious
one, and I ran the risk of getting a flogging for marring the highly
prized copy books of the oldest son. In addition to these opportunities,
sleeping, as I did, in the kitchen loft---a room seldom visited by any
of the family,---I got a flour barrel up there, and a chair; and upon
the head of that barrel I have written, (or endeavored to write,)
copying from the bible and the Methodist hymn book, and other books
which had accumulated on my hands, till late at night, and when all the
family were in bed and asleep. I was supported in my endeavors by
renewed advice, and by holy promises from the good Father Lawson, with
whom I continued to meet, and pray, and read the scriptures. Although
Master Hugh was aware of my going there, I must say, for his credit,
that he never executed his threat to whip me, for having thus,
innocently, employed my leisure time.
