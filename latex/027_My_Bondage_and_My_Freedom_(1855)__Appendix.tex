{\protect\hypertarget{407}{}{}}

~

{APPENDIX,}

CONTAINING EXTRACTS FROM SPEECHES,
ETC\textsuperscript{\protect\hyperlink{cite_note-1}{{[}1{]}}}

\begin{longtable}[]{@{}lll@{}}
\toprule
\includegraphics[width=0.41667in,height=0.01042in]{//upload.wikimedia.org/wikipedia/commons/thumb/b/b2/Rule_Segment_-_Span_-_40px.svg/40px-Rule_Segment_-_Span_-_40px.svg.png}
&
\includegraphics[width=0.05208in,height=0.07292in]{//upload.wikimedia.org/wikipedia/commons/thumb/d/db/Rule_Segment_-_Diamond_-_4px.svg/5px-Rule_Segment_-_Diamond_-_4px.svg.png}
&
\includegraphics[width=0.41667in,height=0.01042in]{//upload.wikimedia.org/wikipedia/commons/thumb/b/b2/Rule_Segment_-_Span_-_40px.svg/40px-Rule_Segment_-_Span_-_40px.svg.png}\tabularnewline
\bottomrule
\end{longtable}

{RECEPTION SPEECH}

{AT FINSBURY CHAPEL, MOORFIELDS, ENGLAND, MAY 12, 1846.}

\textsc{Mr. Douglass} rose amid loud cheers, and said: I feel
exceedingly glad of the opportunity now afforded me of presenting the
claims of my brethren in bonds in the United States, to so many in
London and from various parts of Britain, who have assembled here on the
present occasion. I have nothing to commend me to your consideration in
the way of learning, nothing in the way of education, to entitle me to
your attention; and you are aware that slavery is a very bad school for
rearing teachers of morality and religion. Twenty-one years of my life
have been spent in slavery---personal slavery---surrounded by degrading
influences, such as can exist nowhere beyond the pale of slavery; and it
will not be strange, if under such circumstances, I should betray, in
what I have to say to you, a deficiency of that refinement which is
seldom or ever found, except among persons that have experienced
superior advantages to those which I have enjoyed. But I will take it
for granted that you know something about the degrading influences of
slavery, and that you will not expect great things from me this evening,
but simply such facts as I may be able to advance immediately in
connection with my own experience of slavery.

{\protect\hypertarget{408}{}{}}Now, what is this system of slavery? This
is the subject of my lecture this evening---what is the character of
this institution? I am about to answer the inquiry, what is American
slavery? I do this the more readily, since I have found persons in this
country who have identified the term slavery with that which I think it
is not, and in some instances, I have feared, in so doing, have rather
(unwittingly, I know,) detracted much from the horror with which the
term slavery is contemplated. It is common in this country to
distinguish every bad thing by the name of slavery. Intemperance is
slavery; to be deprived of the right to vote is slavery, says one; to
have to work hard is slavery, says another; and I do not know but that
if we should let them go on, they would say that to eat when we are
hungry, to walk when we desire to have exercise, or to minister to our
necessities, or have necessities at all, is slavery. I do not wish for a
moment to detract from the horror with which the evil of intemperance is
contemplated---not at all; nor do I wish to throw the slightest
obstruction in the way of any political freedom that any class of
persons in this country may desire to obtain. But I am here to say that
I think the term slavery is sometimes abused by identifying it with that
which it is not. Slavery in the United States is the granting of that
power by which one man exercises and enforces a right of property in the
body and soul of another. The condition of a slave is simply that of the
brute beast. He is a piece of property---a marketable commodity, in the
language of the law, to be bought or sold at the will and caprice of the
master who claims him to be his property; he is spoken of, thought of,
and treated as property. His own good, his conscience, his intellect,
his affections, are all set aside by the master. The will and the wishes
of the master are the law of the slave. He is as much a piece of
property as a horse. If he is fed, he is fed because he is property. If
he is clothed, it is with a view to the increase of his value as
property. Whatever of comfort is necessary to him for his body or soul
that is inconsistent with his being property, is carefully wrested from
him, not only by public opinion, but by the law of the country. He is
carefully deprived of everything that tends in the slightest degree to
detract from his value as property. He is deprived of education. God has
given him an intellect; the slaveholder declares it shall not be
cultivated. If his moral perception leads him in a course contrary to
his value as {\protect\hypertarget{409}{}{}}property, the slaveholder
declares he shall not exercise it. The marriage institution cannot exist
among slaves, and one-sixth of the population of democratic America is
denied its privileges by the law of the land. What is to be thought of a
nation boasting of its liberty, boasting of its humanity, boasting of
its christianity, boasting of its love of justice and purity, and yet
having within its own borders three millions of persons denied by law
the right of marriage?---what must be the condition of that people? I
need not lift up the veil by giving you any experience of my own. Every
one that can put two ideas together, must see the most fearful results
from such a state of things as I have just mentioned. If any of these
three millions find for themselves companions, and prove themselves
honest, upright, virtuous persons to each other, yet in these
cases---few as I am bound to confess they are---the virtuous live in
constant apprehension of being torn asunder by the merciless
men-stealers that claim them as their property. This is American
slavery; no marriage---no education---the light of the gospel shut out
from the dark mind of the bondman---and he forbidden by law to learn to
read. If a mother shall teach her children to read, the law in Louisiana
proclaims that she may be hanged by the neck. If the father attempt to
give his son a knowledge of letters, he may be punished by the whip in
one instance, and in another be killed, at the discretion of the court.
Three millions of people shut out from the light of knowledge! It is
easy for you to conceive the evil that must result from such a state of
things.

I now come to the physical evils of slavery. I do not wish to dwell at
length upon these, but it seems right to speak of them, not so much to
influence your minds on this question, as to let the slaveholders of
America know that the curtain which conceals their crimes is being
lifted abroad; that we are opening the dark cell, and leading the people
into the horrible recesses of what they are pleased to call their
domestic institution. We want them to know that a knowledge of their
whippings, their scourgings, their brandings, their chainings, is not
confined to their plantations, but that some negro of theirs has broken
loose from his chains---has burst through the dark incrustation of
slavery, and is now exposing their deeds of deep damnation to the gaze
of the christian people of England.

The slaveholders resort to all kinds of cruelty. If I were disposed, I
have matter enough to interest you on this question for
{\protect\hypertarget{410}{}{}}five or six evenings, but I will not
dwell at length upon these cruelties. Suffice it to say, that all the
peculiar modes of torture that were resorted to in the West India
islands, are resorted to, I believe, even more frequently, in the United
States of America. Starvation, the bloody whip, the chain, the gag, the
thumb-screw, cat-hauling, the cat-o'-nine-tails, the dungeon, the
blood-hound, are all in requisition to keep the slave in his condition
as a slave in the United States. If any one has a doubt upon this point,
I would ask him to read the chapter on slavery in
\href{/wiki/Author:Charles_Dickens}{Dickens's} \emph{Notes on America}.
If any man has a doubt upon it, I have here the ``testimony of a
thousand witnesses,'' which I can give at any length, all going to prove
the truth of my statement. The blood-hound is regularly trained in the
United States, and advertisements are to be found in the southern papers
of the Union, from persons advertising themselves as blood-hound
trainers, and offering to hunt down slaves at fifteen dollars a piece,
recommending their hounds as the fleetest in the neighborhood, never
known to fail. Advertisements are from time to time inserted, stating
that slaves have escaped with iron collars about their necks, with bands
of iron about their feet, marked with the lash, branded with red-hot
irons, the initials of their master's name burned into their flesh; and
the masters advertise the fact of their being thus branded with their
own signature, thereby proving to the world, that, however damning it
may appear to non-slaveholders, such practices are not regarded
discreditable among the slaveholders themselves. Why, I believe if a man
should brand his horse in this country---burn the initials of his name
into any of his cattle, and publish the ferocious deed here---that the
united execrations of christians in Britain would descend upon him. Yet,
in the United States, human beings are thus branded As Whittier says---

{"} {.~.~.} Our countrymen in chains,\\
{}The whip on woman's shrinking flesh,\\
Our soil yet reddening with the stains\\
{}Caught from her scourgings warm and fresh."

The slave-dealer boldly publishes his infamous acts to the world. Of all
things that have been said of slavery to which exception has been taken
by slaveholders, this, the charge of cruelty, stands foremost, and yet
there is no charge capable of clearer demonstration, than that of the
most barbarous inhumanity on the part of the
{\protect\hypertarget{411}{}{}}slaveholders toward their slaves. And all
this is necessary; it is necessary to resort to these cruelties, in
order to \emph{make the slave a slave}, and to \emph{keep him a slave}.
Why, my experience all goes to prove the truth of what you will call a
marvelous proposition, that the better you treat a slave, the more you
destroy his value \emph{as a slave}, and enhance the probability of his
eluding the grasp of the slaveholder; the more kindly you treat him, the
more wretched you make him, while you keep him in the condition of a
slave. My experience, I say, confirms the truth of this propostion. When
I was treated exceedingly ill; when my back was being scourged daily;
when I was whipped within an inch of my life---\emph{life} was all I
cared for. ``Spare my life,'' was my continual prayer. When I was
looking for the blow about to be inflicted upon my head, I was not
thinking of my liberty; it was my life. But, as soon as the blow was not
to be feared, then came the longing for liberty. If a slave has a bad
master, his ambition is to get a better; when he gets a better, he
aspires to have the best; and when he gets the best, he aspires to be
his own master. But the slave must be brutalized to keep him as a slave.
The slaveholder feels this necessity. I admit this necessity. If it be
right to hold slaves at all, it is right to hold them in the only way in
which they can be held; and this can be done only by shutting out the
light of education from their minds, and brutalizing their persons. The
whip, the chain, the gag, the thumb-screw, the blood-hound, the stocks,
and all the other bloody paraphernalia of the slave system, are
indispensably necessary to the relation of master and slave. The slave
must be subjected to these, or he ceases to be a slave. Let him know
that the whip is burned; that the fetters have been turned to some
useful and profitable employment; that the chain is no longer for his
limbs; that the blood-hound is no longer to be put upon his track; that
his master's authority over him is no longer to be enforced by taking
his life---and immediately he walks out from the house of bondage and
asserts his freedom as a man. The slaveholder finds it necessary to have
these implements to keep the slave in bondage; finds it necessary to be
able to say, ``Unless you do so and so; unless you do as I bid you---I
will take away your life!''

Some of the most awful scenes of cruelty are constantly taking place in
the middle states of the Union. We have in those states what are called
the slave-breeding states. Allow me to speak plainly. Although it is
harrowing to your feelings, it is necessary
{\protect\hypertarget{412}{}{}}that the facts of the case should be
stated. We have in the United States slave-breeding states. The very
state from which the minister from our court to yours comes, is one of
these states---Maryland, where men, women, and children are reared for
the market, just as horses, sheep, and swine are raised for the market.
Slave-rearing is there looked upon as a legitimate trade; the law
sanctions it, public opinion upholds it, the church does not condemn it.
It goes on in all its bloody horrors, sustained by the auctioneer's
block. If you would see the cruelties of this system, hear the following
narrative. Not long since the following scene occurred. A slave-woman
and a slave-man had united themselves as man and wife in the absence of
any law to protect them as man and wife. They had lived together by the
permission, not by right, of their master, and they had reared a family.
The master found it expedient, and for his interest, to sell them. He
did not ask them their wishes in regard to the matter at all; they were
not consulted. The man and woman were brought to the auctioneer's block,
under the sound of the hammer. The cry was raised, ``Here goes; who bids
cash?'' Think of it---a man and wife to be sold! The woman was placed on
the auctioneer's block; her limbs, as is customary, were brutally
exposed to the purchasers, who examined her with all the freedom with
which they would examine a horse. There stood the husband, powerless; no
right to his wife; the master's right preëminent. She was sold. He was
next brought to the auctioneer's block. His eyes followed his wife in
the distance; and he looked beseechingly, imploringly, to the man that
had bought his wife, to buy him also. But he was at length bid off to
another person. He was about to be separated forever from her he loved.
No word of his, no work of his, could save him from this separation. He
asked permission of his new master to go and take the hand of his wife
at parting. It was denied him. In the agony of his soul he rushed from
the man who had just bought him, that he might take a farewell of his
wife; but his way was obstructed, he was struck over the head with a
loaded whip, and was held for a moment; but his agony was too great.
When he was let go, he fell a corpse at the feet of his master. His
heart was broken. Such scenes are the every-day fruits of American
slavery. Some two years since, the Hon. Seth M. Gates, an anti-slavery
gentleman of the state of New York, a representative in the congress of
the United States, told me he saw with his own eyes the following
circumstance. In the national {\protect\hypertarget{413}{}{}}District of
Columbia, over which the star-spangled emblem is constantly waving,
where orators are ever holding forth on the subject of American liberty,
American democracy, American republicanism, there are two slave prisons.
When going across a bridge, leading to one of these prisons, he saw a
young woman run out, bare-footed and bare-headed, and with very little
clothing on. She was running with all speed to the bridge he was
approaching. His eye was fixed upon her, and he stopped to see what was
the matter. He had not paused long before he saw three men run out after
her. He now knew what the nature of the case was; a slave escaping from
her chains---a young woman, a sister---escaping from the bondage in
which she had been held. She made her way to the bridge, but had not
reached it, ere from the Virginia side there came two slaveholders. As
soon as they saw them, her pursuers called out, ``Stop her!'' True to
their Virginian instincts, they came to the rescue of their brother
kidnappers, across the bridge. The poor girl now saw that there was no
chance for her. It was a trying time. She knew if she went back, she
must be a slave forever---she must be dragged down to the scenes of
pollution which the slaveholders continually provide for most of the
poor, sinking, retched young women, whom they call their property. She
formed her resolution; and just as those who were about to take her,
were going to put hands upon her, to drag her back, she leaped over the
balustrades of the bridge, and down she went to rise no more. She chose
death, rather than to go back into the hands of those christian
slaveholders from whom she had escaped.

Can it be possible that such things as these exist in the United States?
Are not these the exceptions? Are any such scenes as this general? Are
not such deeds condemned by the law and denounced by public opinion? Let
me read to you a few of the laws of the slaveholding states of America.
I think no better exposure of slavery can be made than is made by the
laws of the states in which slavery exists. I prefer reading the laws to
making any statement in confirmation of what I have said myself; for the
slaveholders cannot object to this testimony, since it is the calm, the
cool, the deliberate enactment of their wisest heads, of their most
clear-sighted, their own constituted representatives. "If more than
seven slaves together are found in any road without a white person,
twenty lashes a piece; for visiting a plantation without a written pass,
ten lashes; for letting loose a boat from where it is made
{\protect\hypertarget{414}{}{}}fast, thirty-nine lashes for the first
offense; and for the second, shall have cut off from his head one ear;
for keeping or carrying a club, thirty-nine lashes; for having any
article for sale, without a ticket from his master, ten lashes; for
traveling in any other than the most usual and accustomed road, when
going alone to any place, forty lashes; for traveling in the night
without a pass, forty lashes." I am afraid you do not understand the
awful character of these lashes. You must bring it before your mind. A
human being in a perfect state of nudity, tied hand and foot to a stake,
and a strong man standing behind with a heavy whip, knotted at the end,
each blow cutting into the flesh, and leaving the warm blood dripping to
the feet; and for these trifles. ``For being found in another person's
negro-quarters, forty lashes; for hunting with dogs in the the woods,
thirty lashes; for being on horseback without the written permission of
his master, twenty-five lashes; for riding or going abroad in the night,
or riding horses in the day time, without leave, a slave may be whipped,
cropped, or branded in the cheek with the letter R, or otherwise
punished, such punishment not extending to life, or so as to render him
unfit for labor.'' The laws referred to, may be found by consulting
Brevard's Digest; Haywood's Manual; Virginia Revised Code; Prince's
Digest; Missouri Laws; Mississippi Revised Code. A man, or going to
visit his brethren, without the permission of his master---and in many
instances he may not have that permission; his master, from caprice or
other reasons, may not be willing to allow it---may be caught on his
way, dragged to a post, the branding-iron heated, and the name of his
master or the letter R branded into his cheek or on his forehead. They
treat slaves thus, on the principle that they must punish for light
offenses, in order to prevent the commission of larger ones. I wish you
to mark that in the single state of Virginia there are seventy-one
crimes for which a colored man may be executed; while there are only
three of these crimes, which, when committed by a white man, will
subject him to that punishment. There are many of these crimes which if
the white man did not commit, he would be regarded as a scoundrel and a
coward. In the state of Maryland, there is a law to this effect: that if
a slave shall strike his master, he may be hanged, his head severed from
his body, his body quartered, and his head and quarters set up in the
most prominent places in the neighborhood. If a colored woman, in the
defense of her own virtue, in defense of her own person, should shield
{\protect\hypertarget{415}{}{}}herself from the brutal attacks of her
tyrannical master, or make the slightest resistance, she may be killed
on the spot. No law whatever will bring the guilty man to justice for
the crime.

But you will ask me, can these things be possible in a land professing
christianity? Yes, they are so; and this is not the worst. No; a darker
feature is yet to be presented than the mere existence of these facts. I
have to inform you that the religion of the southern states, at this
time, is the great supporter, the great sanctioner of the bloody
atrocities to which I have referred. While America is printing tracts
and bibles; sending missionaries abroad to convert the heathen;
expending her money in various ways for the promotion of the gospel in
foreign lands---the slave not only lies forgotten, uncared for, but is
trampled under foot by the very churches of the land. What have we in
America? Why, we have slavery made part of the religion of the land.
Yes, the pulpit there stands up as the great defender of this cursed
\emph{institution}, as it is called. Ministers of religion come forward
and torture the hallowed pages of inspired wisdom to sanction the bloody
deed. They stand forth as the foremost, the strongest defenders of this
``institution.'' As a proof of this, I need not do more than state the
general fact, that slavery has existed under the droppings of the
sanctuary of the south for the last two hundred years, and there has not
been any war between the \emph{religion} and the \emph{slavery} of the
south. Whips, chains, gags, and thumb-screws have all lain under the
droppings of the sanctuary, and instead of rusting from off the limbs of
the bondman, those droppings have served to preserve them in all their
strength. Instead of preaching the gospel against this tyranny, rebuke,
and wrong, ministers of religion have sought, by all and every means, to
throw in the back-ground whatever in the bible could be construed into
opposition to slavery, and to bring forward that which they could
torture into its support. This I conceive to be the darkest feature of
slavery, and the most difficult to attack, because it is identified with
religion, and exposes those who denounce it to the charge of infidelity.
Yes, those with whom I have been laboring, namely, the old organization
anti-slavery society of America, have been again and again stigmatized
as infidels, and for what reason\emph{?} Why, solely in consequence of
the faithfulness of their attacks upon the slaveholding religion of the
southern states, and the northern religion that sympathizes with it. I
have found it difficult to speak on this matter without persons coming
forward and {\protect\hypertarget{416}{}{}}saying, ``Douglass, are you
not afraid of injuring the cause of Christ? You do not desire to do so,
we know; but are you not undermining religion?'' This has been said to
me again and again, even since I came to this country, but I cannot be
induced to leave off these exposures. I love the religion of our blessed
Savior. I love that religion that comes from above, in the "wisdom of
God, which is first pure, then peaceable, gentle, and easy to be
entreated, full of mercy and good fruits, without partiality and without
hypocrisy. I love that religion that sends its votaries to bind up the
wounds of him that has fallen among thieves. I love that religion that
makes it the duty of its disciples to visit the fatherless and the widow
in their affliction. I love that religion that is based upon the
glorious principle, of love to God and love to man; which makes its
followers do unto others as they themselves would be done by. If you
demand liberty to yourself, it says, grant it to your neighbors. If you
claim a right to think for yourself, it says, allow your neighbors the
same right. If you claim to act for yourself, it says, allow your
neighbors the same right. It is because I love this religion that I hate
the slaveholding, the woman-whipping, the mind-darkening, the
soul-destroying religion that exists in the southern states of America.
It is because I regard the one as good, and pure, and holy, that I
cannot but regard the other as bad, corrupt, and wicked. Loving the one
I must hate the other; holding to the one I must reject the other.

I may be asked, why I am so anxious to bring this subject before the
British public---why I do not confine my efforts to the United States?
My answer is, first, that slavery is the common enemy of mankind, and
all mankind should be made acquainted with its abominable character. My
next answer is, that the slave is a man, and, as such, is entitled to
your sympathy as a brother. All the feelings, all the susceptibilities,
all the capacities, which you have, he has. He is a part of the human
family. He has been the prey---the common prey---of christendom for the
last three hundred years, and it is but right, it is but just, it is but
proper, that his wrongs should be known throughout the world. I have
another reason for bringing this matter before the British public, and
it is this: slavery is a system of wrong, so blinding to all around, so
hardening to the heart, so corrupting to the morals, so deleterious to
religion, so sapping to all the principles of justice in its immediate
vicinity, that the community surrounding it lack the moral
{\protect\hypertarget{417}{}{}}stamina necessary to its removal. It is a
system of such gigantic evil, so strong, so overwhelming in its power,
that no one nation is equal to its removal. It requires the humanity of
christianity, the morality of the world to remove it. Hence, I call upon
the people of Britain to look at this matter, and to exert the influence
I am about to show they possess, for the removal of slavery from
America. I can appeal to them, as strongly by their regard for the
slaveholder as for the slave, to labor in this cause. I am here, because
you have an influence on America that no other nation can have. You have
been drawn together by the power of steam to a marvelous extent; the
distance between London and Boston is now reduced to some twelve or
fourteen days, so that the denunciations against slavery, uttered in
London this week, may be heard in a fortnight in the streets of Boston,
and reverberating amidst the hills of Massachusetts. There is nothing
said here against slavery that will not be recorded in the United
States. I am here, also, because the slaveholders do not want me to be
here; they would rather that I were not here. I have adopted a maxim
laid down by Napoleon, never to occupy ground which the enemy would like
me to occupy. The slaveholders would much rather have me, if I will
denounce slavery, denounce it in the northern states, where their
friends and supporters are, who will stand by and mob me for denouncing
it. They feel something as the man felt, when he uttered his prayer, in
which he made out a most horrible case for himself, and one of his
neighbors touched him and said, ``My friend, I always had the opinion of
you that you have now expressed for yourself---that you are a very great
sinner.'' Coming from himself, it was all very well, but coming from a
stranger it was rather cutting. The slaveholders felt that when slavery
was denounced among themselves, it was not so bad; but let one of the
slaves get loose, let him summon the people of Britain, and make known
to them the conduct of the slaveholders toward their slaves, and it cuts
them to the quick, and produces a sensation such as would be produced by
nothing else. The power I exert now is something like the power that is
exerted by the man at the end of the lever; my influence now is just in
proportion to the distance that I am from the United States. My exposure
of slavery abroad will tell more upon the hearts and consciences of
slaveholders, than if I was attacking them in America; for almost every
paper that I now receive from the United States, comes teeming with
statements about this fugitive negro, calling
{\protect\hypertarget{418}{}{}}him a ``glib-tongued scoundrel,'' and
saying that he is running out against the institutions and people of
America. I deny the charge that I am saying a word against the
institutions of America, or the people, as such. What I have to say is
against slavery and slaveholders. I feel at liberty to speak on this
subject. I have on my back the marks of the lash; I have four sisters
and one brother now under the galling chain. I feel it my duty to cry
aloud and spare not. I am not averse to having the good opinion of my
fellow creatures. I am not averse to being kindly regarded by all men;
but I am bound, even at the hazard of making a large class of
religionists in this country hate me, oppose me, and malign me as they
have done---I am bound by the prayers, and tears, and entreaties of
three millions of kneeling bondsmen, to have no compromise with men who
are in any shape or form connected with the slaveholders of America. I
expose slavery in this country, because to expose it is to kill it.
Slavery is one of those monsters of darkness to whom the light of truth
is death. Expose slavery, and it dies. Light is to slavery what the heat
of the sun is to the root of a tree; it must die under it. All the
slaveholder asks of me is silence. He does not ask me to go abroad and
preach \emph{in favor} of slavery; he does not ask any one to do that.
He would not say that slavery is a good thing, but the best under the
circumstances. The slaveholders want total darkness on the subject. They
want the hatchway shut down, that the monster may crawl in his den of
darkness, crushing human hopes and happiness, destroying the bondman at
will, and having no one to reprove or rebuke him. Slavery shrinks from
the light; it hateth the light, neither cometh to the light, lest its
deeds should be reproved. To tear off the mask from this abominable
system, to expose it to the light of heaven, aye, to the heat of the
sun, that it may burn and wither it out of existence, is my object in
coming to this country. I want the slaveholder surrounded, as by a wall
of anti-slavery fire, so that he may see the condemnation of himself and
his system glaring down in letters of light. I want him to feel that he
has no sympathy in England, Scotland, or Ireland; that he has none in
Canada, none in Mexico, none among the poor wild Indians; that the voice
of the civilized, aye, and savage world is against him. I would have
condemnation blaze down upon him in every direction, till, stunned and
overwhelmed with shame and confusion, he is compelled to let go the
grasp he holds upon the persons of his victims, and restore them to
their long-lost rights.

{\protect\hypertarget{419}{}{}}

DR. CAMPBELL'S REPLY.

From Rev. Dr. Campbell's brilliant reply we extract the following:

\textsc{Frederick Douglass}, the ``beast of burden,'' the portion of
``goods and chattels,'' the representative of three millions of men, has
been raised up! Shall I say the \emph{man?} If there is a man on earth,
he is a man. My blood boiled within me when I heard his address
to-night, and thought that he had left behind him three millions of such
men.

``We must see more of this man; we must have more of this man. One would
have taken a voyage round the globe some forty years back---especially
since the introduction of steam---to have heard such an exposure of
slavery from the lips of a slave. It will be an era in the individual
history of the present assembly. Our children---our boys and girls---I
have to-night seen the delightful sympathy of their hearts evinced by
their heaving breasts, while their eyes sparkled with wonder and
admiration, that this black man---this slave---had so much logic, so
much wit, so much fancy, so much eloquence. He was something more than a
man, according to their little notions. Then, I say, we must hear him
again. We have got a purpose to accomplish. He has appealed to the
pulpit of England. The English pulpit is with him. He has appealed to
the press of England; the press of England is conducted by English
hearts, and that press will do him justice. About ten days hence, and
his second master, who may well prize ''such a piece of goods," will
have the pleasure of reading his burning words, and his first master
will bless himself that he has got quit of him. We have to create public
opinion, or rather, not to create it, for it is created already; but we
have to foster it; and when to-night I heard those magnificent
words---the words of Curran, by which my heart, from boyhood, has
ofttimes been deeply moved---I rejoice to think that they embody an
instinct of an Englishman's nature. I heard, with inexpressible delight,
how they told on this mighty mass of the citizens of the metropolis.

Britain has now no slaves; we can therefore talk to the other nations
now, as we could not have talked a dozen years ago. I want the whole of
the London ministry to meet Douglass. For as his appeal is to England,
and throughout England, I should rejoice in the
{\protect\hypertarget{420}{}{}}idea of churchmen and dissenters merging
all sectional distinctions in this cause. Let us have a public
breakfast. Let the ministers meet him; let them hear him; let them grasp
his hand; and let him enlist their sympathies on behalf of the slave.
Let him inspire them with abhorrence of the man-stealer---the
slaveholder. No slaveholding American shall ever my cross my door. No
slaveholding or slavery-supporting minister shall ever pollute my
pulpit. While I have a tongue to speak, or a hand to write, I will, to
the utmost of my power, oppose these slaveholding men. We must have
Douglass amongst us to aid in fostering public opinion.

The great conflict with slavery must now take place in America; and
while they are adding other slave states to the Union, our business is
to step forward and help the abolitionists there. It is a pleasing
circumstance that such a body of men has risen in America, and whilst we
hurl our thunders against her slavers, let us make a distinction between
those who advocate slavery and those who oppose it. George Thompson has
been there. This man, Frederick Douglass, has been there, and has been
compelled to flee. I wish, when he first set foot on our shores, he had
made a solemn vow, and said, ``Now that I am free, and in the sanctuary
of freedom, I will never return till I have seen the emancipation of my
country completed.'' He wants to surround these men, the slaveholders,
as by a wall of fire; and he himself may do much toward kindling it. Let
him travel over the island---east, west, north, and south---everywhere
diffusing knowledge and awakening principle, till the whole nation
become a body of petitioners to America. He will, he must, do it. He
must for a season make England his home. He must send for his wife. He
must send for his children. I want to see the sons and daughters of such
a sire. We, too, must do something for him and them worthy of the
English name. I do not like the idea of a man of such mental dimensions,
such moral courage, and all but incomparable talent, having his own
small wants, and the wants of a distant wife and children, supplied by
the poor profits of his publication, the sketch of his life. Let the
pamphlet be bought by tens of thousands. But we will do something more
for him, shall we not?

It only remains that we pass a resolution of thanks to Frederick
Douglass, the slave that was, the man that is! He that was covered with
chains, and that is now being covered with glory, and whom we will send
back a gentleman.

{\protect\hypertarget{ux5cux7bux5cux7bux5cux7b1ux5cux7dux5cux7dux5cux7d}{}{}}

{\protect\hypertarget{421}{}{}}

{LETTER TO HIS OLD
MASTER.\textsuperscript{\protect\hyperlink{cite_note-2}{{[}2{]}}}}

\emph{To My Old Master, Thomas Auld.}

\textsc{Sir}---The long and intimate, though by no means friendly,
relation which unhappily subsisted between you and myself, leads me to
hope that you will easily account for the great liberty which I now take
in addressing you in this open and public manner. The same fact may
possibly remove any disagreeable surprise which you may experience on
again finding your name coupled with mine, in any other way than in an
advertisement, accurately describing my person, and offering a large sum
for my arrest. In thus dragging you again before the public, I am aware
that I shall subject myself to no inconsiderable amount of censure. I
shall probably be charged with an unwarrantable, if not a wanton and
reckless disregard of the rights and proprieties of private life. There
are those north as well as south who entertain a much higher respect for
rights which are merely conventional, than they do for rights which are
personal and essential. Not a few there are in our country, who, while
they have no scruples against robbing the laborer of the hard earned
results of his patient industry, will be shocked by the extremely
indelicate manner of bringing your name before the public. Believing
this to be the case, and wishing to meet every reasonable or plausible
objection to my conduct, I will frankly state the ground upon which I
justify myself in this instance, as well as on former occasions when I
have thought proper to mention your name in public. All will agree that
a man guilty of theft, robbery, or murder, has forfeited the right to
concealment and private life; that the community have a right to subject
such persons to the most complete exposure. However much they may desire
{\protect\hypertarget{422}{}{}}retirement, and aim to conceal themselves
and their movements from the popular gaze, the public have a right to
ferret them out, and bring their conduct before the proper tribunals of
the country for investigation. Sir, you will undoubtedly make the proper
application of these generally admitted principles, and will easily see
the light in which you are regarded by me; I will not therefore manifest
ill temper, by calling you hard names. I know you to be a man of some
intelligence, and can readily determine the precise estimate which I
entertain of your character. I may therefore indulge in language which
may seem to others indirect and ambiguous, and yet be quite well
understood by yourself.

I have selected this day on which to address you, because it is the
anniversary of my emancipation; and knowing no better way, I am led to
this as the best mode of celebrating that truly important event. Just
ten years ago this beautiful September morning, yon bright sun beheld me
a slave---a poor degraded chattel---trembling at the sound of your
voice, lamenting that I was a man, and wishing myself a brute. The hopes
which I had treasured up for weeks of a safe and successful escape from
your grasp, were powerfully confronted at this last hour by dark clouds
of doubt and fear, making my person shake and my bosom to heave with the
heavy contest between hope and fear. I have no words to describe to you
the deep agony of soul which I experienced on that never-to-be-forgotten
morning---for I left by daylight. I was making a leap in the dark. The
probabilities, so far as I could by reason determine them, were stoutly
against the undertaking. The preliminaries and precautions I had adopted
previously, all worked badly. I was like one going to war without
weapons---ten chances of defeat to one of victory. One in whom I had
confided, and one who had promised me assistance, appalled by fear at
the trial hour, deserted me, thus leaving the responsibility of success
or failure solely with myself. You, sir, can never know my feelings. As
I look back to them, I can scarcely realize that I have passed through a
scene so trying. Trying, however, as they were, and gloomy as was the
prospect, thanks be to the Most High, who is ever the God of the
oppressed, at the moment which was to determine my whole earthly career,
His grace was sufficient; my mind was made up. I embraced the golden
opportunity, took the morning tide at the flood, and a free man, young,
active, and strong, is the result.

I have often thought I should like to explain to you the grounds
{\protect\hypertarget{423}{}{}}upon which I have justified myself in
running away from you. I am almost ashamed to do so now, for by this
time you may have discovered them yourself. I will, however, glance at
them. When yet but a child about six years old, I imbibed the
determination to run away. The very first mental effort that I now
remember on my part, was an attempt to solve the mystery---why am I a
slave? and with this question my youthful mind was troubled for many
days, pressing upon me more heavily at times than others. When I saw the
slave-driver whip a slave-woman, cut the blood out of her neck, and
heard her piteous cries, I went away into the corner of the fence, wept
and pondered over the mystery. I had, through some medium, I know not
what, got some idea of God, the Creator of all mankind, the black and
the white, and that he had made the blacks to serve the whites as
slaves. How he could do this and be \emph{good}, I could not tell. I was
not satisfied with this theory, which made God responsible for slavery,
for it pained me greatly, and I have wept over it long and often. At one
time, your first wife, Mrs. Lucretia, heard me sighing and saw me
shedding tears, and asked of me the matter, but I was afraid to tell
her. I was puzzled with this question, till one night while sitting in
the kitchen, I heard some of the old slaves talking of their parents
having been stolen from Africa by white men, and were sold here as
slaves. The whole mystery was solved at once. Very soon after this, my
Aunt Jinny and Uncle Noah ran away, and the great noise made about it by
your father-in-law, made me for the first time acquainted with the fact,
that there were free states as well as slave states. From that time, I
resolved that I would some day run away. The morality of the act I
dispose of as follows: I am myself; you are yourself; we are two
distinct persons, equal persons. What you are, I am. You are a man, and
so am I. God created both, and made us separate beings. I am not by
nature bond to you, or you to me. Nature does not make your existence
depend upon me, or mine to depend upon yours. I cannot walk upon your
legs, or you upon mine. I cannot breathe for you, or you for me; I must
breathe for myself, and you for yourself. We are distinct persons, and
are each equally provided with faculties necessary to our individual
existence. In leaving you, I took nothing but what belonged to me, and
in no way lessened your means for obtaining an \emph{honest} living.
Your faculties remained yours, and mine became useful to their rightful
owner. I therefore see no wrong in any part of the transaction. It is
true, I went off {\protect\hypertarget{424}{}{}}secretly; but that was
more your fault than mine. Had I let you into the secret, you would have
defeated the enterprise entirely; but for this, I should have been
really glad to have made you acquainted with my intentions to leave.

You may perhaps want to know how I like my present condition. I am free
to say, I greatly prefer it to that which I occupied in Maryland. I am,
however, by no means prejudiced against the state as such. Its
geography, climate, fertility, and products, are such as to make it a
very desirable abode for any man; and but for the existence of slavery
there, it is not impossible that I might again take up my abode in that
state. It is not that I love Maryland less, but freedom more. You will
be surprised to learn that people at the north labor under the strange
delusion that if the slaves were emancipated at the south, they would
flock to the north. So far from this being the case, in that event, you
would see many old and familiar faces back again to the south. The fact
is, there are few here who would not return to the south in the event of
emancipation. We want to live in the land of our birth, and to lay our
bones by the side of our fathers; and nothing short of an intense love
of personal freedom keeps us from the south. For the sake of this, most
of us would live on a crust of bread and a cup of cold water.

Since I left you, I have had a rich experience. I have occupied stations
which I never dreamed of when a slave. Three out of the ten years since
I left you, I spent as a common laborer on the wharves of New Bedford,
Massachusetts. It was there I earned my first free dollar. It was mine.
I could spend it as I pleased. I could buy hams or herring with it,
without asking any odds of anybody. That was a precious dollar to me.
You remember when I used to make seven, or eight, or even nine dollars a
week in Baltimore, you would take every cent of it from me every
Saturday night, saying that I belonged to you, and my earnings also. I
never liked this conduct on your part---to say the best, I thought it a
little mean. I would not have served you so. But let that pass. I was a
little awkward about counting money in New England fashion when I first
landed in New Bedford. I came near betraying myself several times. I
caught myself saying phip, for fourpence; and at one time a man actually
charged me with being a runaway, whereupon I was silly enough to become
one by running away from him, for I was greatly afraid he might adopt
{\protect\hypertarget{425}{}{}}measures to get me again into slavery, a
condition I then dreaded more than death.

I soon learned, however, to count money, as well as to make it, and got
on swimmingly. I married soon after leaving you; in fact, I was engaged
to be married before I left you; and instead of finding my companion a
burden, she was truly a helpmate. She went to live at service, and I to
work on the wharf, and though we toiled hard the first winter, we never
lived more happily. After remaining in New Bedford for three years, I
met with William Lloyd Garrison, a person of whom you have
\emph{possibly} heard, as he is pretty generally known among
slaveholders. He put it into my head that I might make myself
serviceable to the cause of the slave, by devoting a portion of my time
to telling my own sorrows, and those of other slaves, which had come
under my observation. This was the commencement of a higher state of
existence than any to which I had ever aspired. I was thrown into
society the most pure, enlightened, and benevolent, that the country
affords. Among these I have never forgotten you, but have invariably
made you the topic of conversation---thus giving you all the notoriety I
could do. I need not tell you that the opinion formed of you in these
circles is far from being favorable. They have little respect for your
honesty, and less for your religion.

But I was going on to relate to you something of my interesting
experience. I had not long enjoyed the excellent society to which I have
referred, before the light of its excellence exerted a beneficial
influence on my mind and heart. Much of my early dislike of white
persons was removed, and their manners, habits, and customs, so entirely
unlike what I had been used to in the kitchen-quarters on the
plantations of the south, fairly charmed me, and gave me a strong
disrelish for the coarse and degrading customs of my former condition. I
therefore made an effort so to improve my mind and deportment, as to be
somewhat fitted to the station to which I seemed almost providentially
called. The transition from degradation to respectability was indeed
great, and to get from one to the other without carrying some marks of
one's former condition, is truly a difficult matter. I would not have
you think that I am now entirely clear of all plantation peculiarities,
but my friends here, while they entertain the strongest dislike to them,
regard me with that charity to which my past life somewhat entitles me,
so that my condition in this respect is exceedingly pleasant. So far
{\protect\hypertarget{426}{}{}}as my domestic affairs are concerned, I
can boast of as comfortable a dwelling as your own. I have an
industrious and neat companion, and four dear children---the oldest a
girl of nine years, and three fine boys, the oldest eight, the next six,
and the youngest four years old. The three oldest are now going
regularly to school---two can read and write, and the other can spell,
with tolerable correctness, words of two syllables. Dear fellows! they
are all in comfortable beds, and are sound asleep, perfectly secure
under my own roof. There are no slaveholders here to rend my heart by
snatching them from my arms, or blast a mother's dearest hopes by
tearing them from her bosom. These dear children are ours---not to work
up into rice, sugar, and tobacco, but to watch over, regard, and
protect, and to rear them up in the nurture and admonition of the
gospel---to train them up in the paths of wisdom and virtue, and, as far
as we can, to make them useful to the world and to themselves. Oh! sir,
a slaveholder never appears to me so completely an agent of hell, as
when I think of and look upon my dear children. It is then that my
feelings rise above my control. I meant to have said more with respect
to my own prosperity and happiness, but thoughts and feelings which this
recital has quickened, unfits me to proceed further in that direction.
The grim horrors of slavery rise in all their ghastly terror before me;
the wails of millions pierce my heart and chill my blood. I remember the
chain, the gag, the bloody whip; the death-like gloom overshadowing the
broken spirit of the fettered bondman; the appalling liability of his
being torn away from wife and children, and sold like a beast in the
market. Say not that this is a picture of fancy. You well know that I
wear stripes on my back, inflicted by your direction; and that you,
while we were brothers in the same church, caused this right hand, with
which I am now penning this letter, to be closely tied to my left, and
my person dragged, at the pistol's mouth, fifteen miles, from the Bay
Side to Easton, to be sold like a beast in the market, for the alleged
crime of intending to escape from your possession. All this, and more,
you remember, and know to be perfectly true, not only of yourself, but
of nearly all of the slaveholders around you.

At this moment, you are probably the guilty holder of at least three of
my own dear sisters, and my only brother, in bondage. These you regard
as your property. They are recorded on your ledger, or perhaps have been
sold to human flesh-mongers, with a {\protect\hypertarget{427}{}{}}view
to filling your own ever-hungry purse. Sir, I desire to know how and
where these dear sisters are. Have you sold them? or are they still in
your possession? What has become of them? are they living or dead? And
my dear old grandmother, whom you turned out like an old horse to die in
the woods---is she still alive? Write and let me know all about them. If
my grandmother be still alive, she is of no service to you, for by this
time she must be nearly eighty years old---too old to be cared for by
one to whom she has ceased to be of service; send her to me at
Rochester, or bring her to Philadelphia, and it shall be the crowning
happiness of my life to take care of her in her old age. Oh! she was to
me a mother and a father, so far as hard toil for my comfort could make
her such. Send me my grandmother! that I may watch over and take care of
her in her old age. And my sisters---let me know all about them. I would
write to them, and learn all I want to know of them, without disturbing
you in any way, but that, through your unrighteous conduct, they have
been entirely deprived of the power to read and write. You have kept
them in utter ignorance, and have therefore robbed them of the sweet
enjoyments of writing or receiving letters from absent friends and
relatives. Your wickedness and cruelty, committed in this respect on
your fellow-creatures, are greater than all the stripes you have laid
upon my back or theirs. It is an outrage upon the soul, a war upon the
immortal spirit, and one for which you must give account at the bar of
our common Father and Creator.

The responsibility which you have assumed in this regard is truly awful,
and how you could stagger under it these many years is marvelous. Your
mind must have become darkened, your heart hardened, your conscience
seared and petrified, or you would have long since thrown off the
accursed load, and sought relief at the hands of a sin-forgiving God.
How, let me ask, would you look upon me, were I, some dark night, in
company with a band of hardened villains, to enter the precincts of your
elegant dwelling, and seize the person of your own lovely daughter,
Amanda, and carry her off from your family, friends, and all the loved
ones of her youth---make her my slave---compel her to work, and I take
her wages---place her name on my ledger as property---disregard her
personal rights---fetter the powers of her immortal soul by denying her
the right and privilege of learning to read and write---feed her
coarsely---clothe her scantily, and whip her on the naked back
occasionally; {\protect\hypertarget{428}{}{}}more, and still more
horrible, leave her unprotected---a degraded victim to the brutal lust
of fiendish overseers, who would pollute, blight, and blast her fair
soul---rob her of all dignity---destroy her virtue, and annihilate in
her person all the graces that adorn the character of virtuous
womanhood? I ask, how would you regard me, if such were my conduct? Oh!
the vocabulary of the damned would not afford a word sufficiently
infernal to express your idea of my God-provoking wickedness. Yet, sir,
your treatment of my beloved sisters is in all essential points
precisely like the case I have now supposed. Damning as would be such a
deed on my part, it would be no more so than that which you have
committed against me and my sisters.

I will now bring this letter to a close; you shall hear from me again
unless you let me hear from you. I intend to make use of you as a weapon
with which to assail the system of slavery---as a means of concentrating
public attention on the system, and deepening the horror of trafficking
in the souls and bodies of men. I shall make use of you as a means of
exposing the character of the American church and clergy---and as a
means of bringing this guilty nation, with yourself, to repentance. In
doing this, I entertain no malice toward you personally. There is no
roof under which you would be more safe than mine, and there is nothing
in my house which you might need for your comfort, which I would not
readily grant. Indeed, I should esteem it a privilege to set you an
example as to how mankind ought to treat each other.

I am your fellow-man, but not your slave.

{\protect\hypertarget{ux5cux7bux5cux7bux5cux7b1ux5cux7dux5cux7dux5cux7d}{}{}}

{\protect\hypertarget{429}{}{}}

{THE NATURE OF SLAVERY.}

{EXTRACT FROM A LECTURE ON SLAVERY, AT ROCHESTER, DECEMBER 1, 1850.}

\textsc{More} than twenty years of my life were consumed in a state of
slavery. My childhood was environed by the baneful peculiarities of the
slave system. I grew up to manhood in the presence of this hydra-headed
monster---not as a master---not as an idle spectator---not as the guest
of the slaveholder---but as \textsc{a slave}, eating the bread and
drinking the cup of slavery with the most degraded of my
brother-bondmen, and sharing with them all the painful conditions of
their wretched lot. In consideration of these facts, I feel that I have
a right to speak, and to speak \emph{strongly}. Yet, my friends, I feel
bound to speak truly.

Goading as have been the cruelties to which I have been
subjected---bitter as have been the trials through which I have
passed---exasperating as have been, and still are, the indignities
offered to my manhood I find in them no excuse for the slightest
departure from truth in dealing with any branch of this subject.

First of all, I will state, as well as I can, the legal and social
relation of master and slave. A master is one---to speak in the
vocabulary of the southern states---who claims and exercises a right of
property in the person of a fellow-man. This he does with the force of
the law and the sanction of southern religion. The law gives the master
absolute power over the slave. He may work him, flog him, hire him out,
sell him, and, in certain contingencies, \emph{kill} him, with perfect
impunity. The slave is a human being, divested of all rights reduced to
the level of a brute---a mere ``chattel'' in the eye of the law---placed
beyond the circle of human brotherhood---cut off from his kind---his
name, which the ``recording angel'' may have enrolled in heaven, among
the blest, is impiously inserted in a \emph{master's ledger}, with
horses, sheep, and swine. In law, the slave has no wife, no children, no
country, and no home. He {\protect\hypertarget{430}{}{}}can own nothing,
possess nothing, acquire nothing, but what must belong to another. To
eat the fruit of his own toil, to clothe his person with the work of his
own hands, is considered stealing. He toils that another may reap the
fruit; he is industrious that another may live in idleness; he eats
unbolted meal that another may eat the bread of fine flour; he labors in
chains at home, under a burning sun and biting lash, that another may
ride in ease and splendor abroad; he lives in ignorance that another may
be educated; he is abused that another may be exalted; he rests his
toil-worn limbs on the cold, damp ground that another may repose on the
softest pillow; he is clad in coarse and tattered raiment that another
may be arrayed in purple and fine linen; he is sheltered only by the
wretched hovel that a master may dwell in a magnificent mansion; and to
this condition he is bound down as by an arm of iron.

From this monstrous relation there springs an unceasing stream of most
revolting cruelties. The very accompaniments of the slave system stamp
it as the offspring of hell itself. To ensure good behavior, the
slaveholder relies on the whip; to induce proper humility, he relies on
the whip; to rebuke what he is pleased to term insolence, he relies on
the whip; to supply the place of wages as an incentive to toil, he
relies on the whip; to bind down the spirit of the slave, to imbrute and
destroy his manhood, he relies on the whip, the chain, the gag, the
thumb-screw, the pillory, the bowie-knife, the pistol, and the
blood-hound. These are the necessary and unvarying accompaniments of the
system. Wherever slavery is found, these horrid instruments are also
found. Whether on the coast of Africa, among the savage tribes, or in
South Carolina, among the refined and civilized, slavery is the same,
and its accompaniments one and the same. It makes no difference whether
the slaveholder worships the God of the christians, or is a follower of
Mahomet, he is the minister of the same cruelty, and the author of the
same misery. \emph{Slavery} is always \emph{slavery;} always the same
foul, haggard, and damning scourge, whether found in the eastern or in
the western hemisphere.

There is a still deeper shade to be given to this picture. The physical
cruelties are indeed sufficiently harassing and revolting; but they are
as a few grains of sand on the sea shore, or a few drops of water in the
great ocean, compared with the stupendous wrongs which it inflicts upon
the mental, moral, and religious {\protect\hypertarget{431}{}{}}nature
of its hapless victims. It is only when we contemplate the slave as a
moral and intellectual being, that we can adequately comprehend the
unparalleled enormity of slavery, and the intense criminality of the
slaveholder. I have said that the slave was a man. ``What a piece of
work is man! How noble in reason! How infinite in faculties! In form and
moving how express and admirable! In action how like an angel! In
apprehension how like a God! the beauty of the world! the paragon of
animals!''

The slave is a man, ``the image of God,'' but ``a little lower than the
angels;'' possessing a soul, eternal and indestructible; capable of
endless happiness, or immeasurable woe; a creature of hopes and fears,
of affections and passions, of joys and sorrows, and he is endowed with
those mysterious powers by which man soars above the things of time and
sense, and grasps, with undying tenacity, the elevating and sublimely
glorious idea of a God. It is \emph{such} a being that is smitten and
blasted. The first work of slavery is to mar and deface those
characteristics of its victims which distinguish \emph{men} from
\emph{things}, and \emph{persons} from \emph{property}. Its first aim is
to destroy all sense of high moral and religious responsibility. It
reduces man to a mere machine. It cuts him off from his Maker, it hides
from him the laws of God, and leaves him to grope his way from time to
eternity in the dark, under the arbitrary and despotic control of a
frail, depraved, and sinful fellow-man. As the serpent-charmer of India
is compelled to extract the deadly teeth of his venomous prey before he
is able to handle him with impunity, so the slaveholder must strike down
the conscience of the slave before he can obtain the entire mastery over
his victim.

It is, then, the first business of the enslaver of men to blunt, deaden,
and destroy the central principle of human responsibility. Conscience
is, to the individual soul, and to society, what the law of gravitation
is to the universe. It holds society together; it is the basis of all
trust and confidence; it is the pillar of all moral rectitude. Without
it, suspicion would take the place of trust; vice would be more than a
match for virtue; men would prey upon each other, like the wild beasts
of the desert; and earth would become a \emph{hell}.

Nor is slavery more adverse to the conscience than it is to the mind.
This is shown by the fact, that in every state of the American Union,
where slavery exists, except the state of Kentucky, there are laws
absolutely prohibitory of education among the slaves.
{\protect\hypertarget{432}{}{}}The crime of teaching a slave to read is
punishable with severe fines and imprisonment, and, in some instances,
with \emph{death itself}.

Nor are the laws respecting this matter a dead letter. Cases may occur
in which they are disregarded, and a few instances may be found where
slaves may have learned to read; but such are isolated cases, and only
prove the rule. The great mass of slaveholders look upon education among
the slaves as utterly subversive of the slave system. I well remember
when my mistress first announced to my master that she had discovered
that I could read. His face colored at once with surprise and chagrin.
He said that ``I was ruined, and my value as a slave destroyed; that a
slave should know nothing but to obey his master; that to give a negro
an inch would lead him to take an ell; that having learned how to read,
I would soon want to know how to write; and that by-and-by I would be
running away.'' I think my audience will bear witness to the correctness
of this philosophy, and to the literal fulfillment of this prophecy.

It is perfectly well understood at the south, that to educate a slave is
to make him discontented with slavery, and to invest him with a power
which shall open to him the treasures of freedom; and since the object
of the slaveholder is to maintain complete authority over his slave, his
constant vigilance is exercised to prevent everything which militates
against, or endangers, the stability of his authority. Education being
among the menacing influences, and, perhaps, the most dangerous, is,
therefore, the most cautiously guarded against.

It is true that we do not often hear of the enforcement of the law,
punishing as a crime the teaching of slaves to read, but this is not
because of a want of disposition to enforce it. The true reason or
explanation of the matter is this: there is the greatest unanimity of
opinion among the white population in the south in favor of the policy
of keeping the slave in ignorance. There is, perhaps, another reason why
the law against education is so seldom violated. The slave is too poor
to be able to offer a temptation sufficiently strong to induce a white
man to violate it; and it is not to be supposed that in a community
where the moral and religious sentiment is in favor of slavery, many
martyrs will be found sacrificing their liberty and lives by violating
those prohibitory enactments.

As a general rule, then, darkness reigns over the abodes of the
enslaved, and ``how great is that darkness!''

"We are sometimes told of the contentment of the slaves, and are
{\protect\hypertarget{433}{}{}}entertained with vivid pictures of their
happiness. We are told that they often dance and sing; that their
masters frequently give them wherewith to make merry; in fine, that they
have little of which to complain. I admit that the slave does sometimes
sing, dance, and appear to be merry. But what does this prove? It only
proves to my mind, that though slavery is armed with a thousand stings,
it is not able entirely to kill the elastic spirit of the bondman. That
spirit will rise and walk abroad, despite of whips and chains, and
extract from the cup of nature occasional drops of joy and gladness. No
thanks to the slaveholder, nor to slavery, that the vivacious captive
may sometimes dance in his chains; his very mirth in such circumstances
stands before God as an accusing angel against his enslaver.

It is often said, by the opponents of the anti-slavery cause, that the
condition of the people of Ireland is more deplorable than that of the
American slaves. Far be it from me to underrate the sufferings of the
Irish people. They have been long oppressed; and the same heart that
prompts me to plead the cause of the American bondman, makes it
impossible for me not to sympathize with the oppressed of all lands. Yet
I must say that there is no analogy between the two cases. The Irishman
is poor, but he is not a slave. He may be in rags, but he is not a
slave. He is still the master of his own body, and can say with the
poet, ``The hand of Douglass is his own.'' ``The world is all before
him, where to choose;'' and poor as may be my opinion of the British
parliament, I cannot believe that it will ever sink to such a depth of
infamy as to pass a law for the recapture of fugitive Irishmen! The
shame and scandal of kidnapping will long remain wholly monopolized by
the American congress. The Irishman has not only the liberty to emigrate
from his country, but he has liberty at home. He can write, and speak,
and cooperate for the attainment of his rights and the redress of his
wrongs.

The multitude can assemble upon all the green hills and fertile plains
of the Emerald Isle; they can pour out their grievances, and proclaim
their wants without molestation; and the press, that ``swift-winged
messenger,'' can bear the tidings of their doings to the extreme bounds
of the civilized world. They have their ``Conciliation Hall,'' on the
banks of the Liffey, their reform clubs, and their newspapers; they pass
resolutions, send forth addresses, and enjoy the right of petition. But
how is it with the American {\protect\hypertarget{434}{}{}}slave? "Where
may he assemble? Where is his Conciliation Hall? Where are his
newspapers? Where is his right of petition? Where is his freedom of
speech? his liberty of the press? and his right of locomotion? He is
said to be happy; happy men can speak. But ask the slave what is his
condition---what his state of mind what he thinks of enslavement? and
you had as well address your inquiries to the \emph{silent dead}. There
comes no \emph{voice} from the enslaved. We are left to gather his
feelings by imagining what ours would be, were our souls in his soul's
stead.

If there were no other fact descriptive of slavery, than that the slave
is dumb, this alone would be sufficient to mark the slave system as a
grand aggregation of human horrors.

Most who are present, will have observed that leading men in this
country have been putting forth their skill to secure quiet to the
nation. A system of measures to promote this object was adopted a few
months ago in congress. The result of those measures is known. Instead
of quiet, they have produced alarm; instead of peace, they have brought
us war; and so it must ever be.

While this nation is guilty of the enslavement of three millions of
innocent men and women, it is as idle to think of having a sound and
lasting peace, as it is to think there is no God to take cognizance of
the affairs of men. There can be no peace to the wicked while slavery
continues in the land. It will be condemned; and while it is condemned
there will be agitation. Nature must cease to be nature; men must become
monsters; humanity must be transformed; christianity must be
exterminated; all ideas of justice and the laws of eternal goodness must
be utterly blotted out from the human soul,---ere a system so foul and
infernal can escape condemnation, or this guilty republic can have a
sound, enduring peace.

{\protect\hypertarget{ux5cux7bux5cux7bux5cux7b1ux5cux7dux5cux7dux5cux7d}{}{}}

{\protect\hypertarget{435}{}{}}

{INHUMANITY OF SLAVERY.}

{EXTRACT FROM A LECTURE ON SLAVERY, AT ROCHESTER, DECEMBER 8, 1850.}

\textsc{The} relation of master and slave has been called patriarchal,
and only second in benignity and tenderness to that of the parent and
child. This representation is doubtless believed by many northern
people; and this may account, in part, for the lack of interest which we
find among persons whom we are bound to believe to be honest and humane.
What, then, are the facts? Here I will not quote my own experience in
slavery; for this you might call one-sided testimony. I will not cite
the declarations of abolitionists; for these you might pronounce
exaggerations. I will not rely upon advertisements cut from newspapers;
for these you might call isolated cases. But I will refer you to the
laws adopted by the legislatures of the slave states. I give you such
evidence, because it cannot be invalidated nor denied. I hold in my hand
sundry extracts from the slave codes of our country, from which I will
quote.{*~*~*}

Now, if the foregoing be an indication of kindness, \emph{what is
cruelty?} If this be parental affection, \emph{what is bitter
malignity?} A more atrocious and blood-thirsty string of laws could not
well be conceived of. And yet I am bound to say that they fall short of
indicating the horrible cruelties constantly practiced in the slave
states.

I admit that there are individual slaveholders less cruel and barbarous
than is allowed by law; but these form the exception. The majority of
slaveholders find it necessary, to insure obedience, at times, to avail
themselves of the utmost extent of the law, and many go beyond it. If
kindness were the rule, we should' not see advertisements filling the
columns of almost every southern newspaper, offering large rewards for
fugitive slaves, and describing them as being branded with irons, loaded
with chains, and scarred by the whip. One of the most telling
testimonies against the pretended kindness of slaveholders, is the fact
that uncounted numbers of fugitives are now inhabiting the Dismal Swamp,
preferring the {\protect\hypertarget{436}{}{}}untamed wilderness to
their cultivated homes---choosing rather to encounter hunger and thirst,
and to roam with the wild beasts of the forest, running the hazard of
being hunted and shot down, than to submit to the authority of
\emph{kind} masters.

I tell you, my friends, humanity is never driven to such an unnatural
course of life, without great wrong. The slave finds more of the milk of
human kindness in the bosom of the savage Indian, than in the heart of
his \emph{christian} master. He leaves the man of the \emph{bible}, and
takes refuge with the man of the \emph{tomahawk}. He rushes from the
praying slaveholder into the paws of the bear. He quits the homes of men
for the haunts of wolves. He prefers to encounter a life of trial,
however bitter, or death, however terrible, to dragging out his
existence under the dominion of these kind masters.

The apologists for slavery often speak of the abuses of slavery; and
they tell us that they are as much opposed to those abuses as we are;
and that they would go as far to correct those abuses and to ameliorate
the condition of the slave as anybody. The answer to that view is, that
slavery is \emph{itself} an abuse; that it lives by abuse; and dies by
the absence of abuse. Grant that slavery is right; grant that the
relation of master and slave may innocently exist; and there is not a
single outrage which was ever committed against the slave but what finds
an apology in the very necessity of the case. As was said by a
slaveholder, (the Rev. A. G. Few,) to the Methodist conference, ``If the
relation be right, the means to maintain it are also right;'' for
without those means slavery could not exist. Remove the dreadful
scourge---the plaited thong---the galling fetter---the accursed
chain---and let the slaveholder rely solely upon moral and religious
power, by which to secure obedience to his orders, and how long do you
suppose a slave would remain on his plantation? The case only needs to
be stated; it carries its own refutation with it.

Absolute and arbitrary power can never be maintained by one man over the
body and soul of another man, without brutal chastisment and enormous
cruelty.

To talk of \emph{kindness} entering into a relation in which one party
is robbed of wife, of children, of his hard earnings, of home, of
friends, of society, of knowledge, and of all that makes this life
desirable, is most absurd, wicked, and preposterous.

I have shown that slavery is wicked---wicked, in that it violates
{\protect\hypertarget{437}{}{}}the great law of liberty, written on
every human heart---wicked, in that it violates the first command of the
decalogue---wicked, in that it fosters the most disgusting
licentiousness---wicked, in that it mars and defaces the image of God by
cruel and barbarous inflictions wicked, in that it contravenes the laws
of eternal justice, and tramples in the dust all the humane and heavenly
precepts of the New Testament.

The evils resulting from this huge system of iniquity are not confined
to the states south of Mason and Dixon's line. Its noxious influence can
easily be traced throughout our northern borders. It comes even as far
north as the state of New York. Traces of it may be seen even in
Rochester; and travelers have told me it casts its gloomy shadows across
the lake, approaching the very shores of Queen Victoria's dominions.

The presence of slavery may be explained by as it is the explanation
of---the mobocratic violence which lately disgraced New York, and which
still more recently disgraced the city of Boston. These violent
demonstrations, these outrageous invasions of human rights, faintly
indicate the presence and power of slavery here. It is a significant
fact, that while meetings for almost any purpose under heaven may be
held unmolested in the city of Boston, that in the same city, a meeting
cannot be peaceably held for the purpose of preaching the doctrine of
the American Declaration of Independence, ``that all men are created
equal.'' The pestiferous breath of slavery taints the whole moral
atmosphere of the north, and enervates the moral energies of the whole
people.

The moment a foreigner ventures upon our soil, and utters a natural
repugnance to oppression, that moment he is made to feel that there is
little sympathy in this land for him. If he were greeted with smiles
before, he meets with frowns now; and it shall go well with him if he be
not subjected to that peculiarly fitting method of showing fealty to
slavery, the assaults of a mob.

Now, will any man tell me that such a state of things is natural, and
that such conduct on the part of the people of the north, springs from a
consciousness of rectitude? No! every fibre of the human heart unites in
detestation of tyranny, and it is only when the human mind has become
familiarized with slavery, is accustomed to its injustice, and corrupted
by its selfishness, that it fails to record its abhorrence of slavery,
and does not exult in the triumphs of liberty.

{\protect\hypertarget{438}{}{}}The northern people have been long
connected with slavery; they have been linked to a decaying corpse,
which has destroyed the moral health. The union of the government; the
union of the north and south, in the political parties; the union in the
religious organizations of the land, have all served to deaden the moral
sense of the northern people, and to impregnate them with sentiments and
ideas forever in conflict with what as a nation we call \emph{genius of
American institutions}. Rightly viewed, this is an alarming fact, and
ought to rally all that is pure, just, and holy in one determined effort
to crush the monster of corruption, and to scatter ``its guilty
profits'' to the winds. In a high moral sense, as well as in a national
sense, the whole American people are responsible for slavery, and must
share, in its guilt and shame, with the most obdurate men-stealers of
the south.

While slavery exists, and the union of these states endures, every
American citizen must bear the chagrin of hearing his country branded
before the world as a nation of liars and hypocrites; and behold his
cherished national flag pointed at with the utmost scorn and derision.
Even now an American \emph{abroad} is pointed out in the crowd, as
coming from a land where men gain their fortunes by ``the blood of
souls,'' from a land of slave markets, of blood-hounds, and
slave-hunters; and, in some circles, such a man is shunned altogether,
as a moral pest. Is it not time, then, for every American to awake, and
inquire into his duty with respect to this subject?

{\href{/wiki/Author:Wendell_Phillips}{Wendell Phillips}}---the eloquent
New England orator---on his return from Europe, in 1842, said, "As I
stood upon the shores of Genoa, and saw floating on the placid waters of
the Mediterranean, the beautiful American war ship Ohio, with her masts
tapering proportionately aloft, and an eastern sun reflecting her noble
form upon the sparkling waters, attracting the gaze of the multitude, my
first impulse was of pride, to think myself an American; but when I
thought that the first time that gallant ship would gird on her gorgeous
apparel, and wake from beneath her sides her dormant thunders, it would
be in defense of the African slave trade, I blushed in utter
\emph{shame} for my country."

Let me say again, \emph{slavery is alike the sin and the shame of the
American people;} it is a blot upon the American name, and the only
national reproach which need make an American hang his head in shame, in
the presence of monarchical governments.

With this gigantic evil in the land, we are constantly told to look
{\protect\hypertarget{439}{}{}}\emph{at home;} if we say ought against
crowned heads, we are pointed to our enslaved millions; if we talk of
sending missionaries and bibles abroad, we are pointed to three millions
now lying in worse than heathen darkness; if we express a word of
sympathy for Kossuth and his Hungarian fugitive brethren, we are pointed
to that horrible and hell-black enactment, ``the fugitive slave bill.''

Slavery blunts the edge of all our rebukes of tyranny abroad---the
criticisms that we make upon other nations, only call forth ridicule,
contempt, and scorn. In a word, we are made a reproach and a by-word to
a mocking earth, and we must continue to be so made, so long as slavery
continues to pollute our soil.

We have heard much of late of the virtue of patriotism, the love of
country, \&c., and this sentiment, so natural and so strong, has been
impiously appealed to, by all the powers of human selfishness, to
cherish the viper which is stinging our national life away. In In its
name, we have been called upon to deepen our infamy before the world, to
rivet the fetter more firmly on the limbs of the enslaved, and to become
utterly insensible to the voice of human woe that is wafted to us on
every southern gale. "We have been called upon, in its name, to
desecrate our whole land by the footprints of slave-hunters, and even to
engage ourselves in the horrible business of kidnapping.

I, too, would invoke the spirit of patriotism; not in a narrow and
restricted sense, but, I trust, with a broad and manly signification;
not to cover up our national sins, but to inspire us with sincere
repentance; not to hide our shame from the world's gaze, but utterly to
abolish the cause of that shame; not to explain away our gross
inconsistencies as a nation, but to remove the hateful, jarring, and
incongruous elements from the land; not to sustain an egregious wrong,
but to unite all our energies in the grand effort to remedy that wrong.

I would invoke the spirit of patriotism, in the name of the law of the
living God, natural and revealed, and in the full belief that
``righteousness exalteth a nation, while sin is a reproach to any
people.'' ``He that walketh righteously, and speaketh uprightly; he that
despiseth the gain of oppressions, that shaketh his hands from the
holding of bribes, he shall dwell on high, his place of defense shall be
the munitions of rocks, bread shall be given him, his water shall be
sure.''

We have not only heard much lately of patriotism, and of its aid
{\protect\hypertarget{440}{}{}}being invoked on the side of slavery and
injustice, but the very prosperity of this people has been called in to
deafen them to the voice of duty, and to lead them onward in the pathway
of sin. Thus has the blessing of God been converted into a curse. In the
spirit of genuine patriotism, I warn the American people, by all that is
just and honorable, to \textsc{beware}!

I warn them that, strong, proud, and prosperous though we be, there is a
power above us that can ``bring down high looks; at the breath of whose
mouth our wealth may take wings; and before whom every knee shall bow;''
and who can tell how soon the avenging angel may pass over our land, and
the sable bondmen now in chains, may become the instruments of our
nation's chastisement! ``Without appealing to any higher feeling, I
would warn the American people, and the American government, to be wise
in their day and generation. I exhort them to remember the history of
other nations; and I remind them that America cannot always sit ''as a
queen," in peace and repose; that prouder and stronger governments than
this have been shattered by the bolts of a just God; that the time
\emph{may} come when those they now despise and hate, may be needed;
when those whom they now compel by oppression to be enemies, may be
wanted as friends. What has been, may be again. There is a point beyond
which human endurance cannot go. The crushed worm may yet turn under the
heel of the oppressor. I warn them, then, with all solemnity, and in the
name of retributive justice, \emph{to look to their ways;} for in an
evil hour, those sable arms that have, for the last two centuries, been
engaged in cultivating and adorning the fair fields of our country, may
yet become the instruments of terror, desolation, and death, throughout
our borders.

It was the sage of the Old Dominion that said---while speaking of the
possibility of a conflict between the slaves and the slaveholders---"God
has no attribute that could take sides with the oppressor in such a
contest. I tremble for my country when I reflect that God \emph{is
just}, and that his justice cannot sleep forever." Such is the warning
voice of Thomas Jefferson; and every day's experience since its
utterance until now, confirms its wisdom, and commends its truth.

{\protect\hypertarget{ux5cux7bux5cux7bux5cux7b1ux5cux7dux5cux7dux5cux7d}{}{}}

{\protect\hypertarget{441}{}{}}

{WHAT TO THE SLAVE IS THE FOURTH OF JULY?}

{EXTRACT FROM AN ORATION, AT ROCHESTER, JULY 6, 1852.}

\textsc{Fellow-Citizens}---Pardon me, and allow me to ask, why am I
called upon to speak here to-day? What have I, or those I represent, to
do with your national independence? Are the great principles of
political freedom and of natural justice, embodied in that Declaration
of Independence, extended to us? and am I, therefore, called upon to
bring our humble offering to the national altar, and to confess the
benefits, and express devout gratitude for the blessings, resulting from
your independence to us?

Would to God, both for your sakes and ours, that an affirmative answer
could be truthfully returned to these questions! Then would my task be
light, and my burden easy and delightful. For who is there so cold that
a nation's sympathy could not warm him? Who so obdurate and dead to the
claims of gratitude, that would not thankfully acknowledge such
priceless benefits? Who so stolid and selfish, that would not give his
voice to swell the hallelujahs of a nation's jubilee, when the chains of
servitude had been torn from his limbs? I am not that man. In a case
like that, the dumb might eloquently speak, and the ``lame man leap as
an hart.''

But, such is not the state of the case. I say it with a sad sense of the
disparity between us. I am not included within the pale of this glorious
anniversary! Your high independence only reveals the immeasurable
distance between us. The blessings in which you this day rejoice, are
not enjoyed in common. The rich inheritance of justice, liberty,
prosperity, and independence, bequeathed by your fathers, is shared by
you, not by me. The sunlight that brought life and healing to you, has
brought stripes and death to me. This Fourth of July is \emph{yours},
not \emph{mine}. \emph{You} may rejoice, \emph{I} must mourn. To drag a
man in fetters into the grand illuminated temple of liberty, and call
upon him to join you in joyous anthems, were inhuman mockery and
sacrilegious irony. Do you mean,
{\protect\hypertarget{442}{}{}}citizens, to mock me, by asking me to
speak to-day? If so, there is a parallel to your conduct. And let me
warn you that it is dangerous to copy the example of a nation whose
crimes, towering up to heaven, were thrown down by the breath of the
Almighty, burying that nation in irrecoverable ruin! I can to-day take
up the plaintive lament of a peeled and woe-smitten people.

``By the rivers of Babylon, there we sat down. Yea! we wept when we
remembered Zion. We hanged our harps upon the willows in the midst
thereof. For there, they that carried us away captive, required of us a
song; and they who wasted us required of us mirth, saying, Sing us one
of the songs of Zion. How can we sing the Lord's song in a strange land?
If I forget thee, O Jerusalem, let my right hand forget her cunning. If
I do not remember thee, let my tongue cleave to the roof of my mouth.''

Fellow-citizens, above your national, tumultuous joy, I hear the
mournful wail of millions, whose chains, heavy and grievous yesterday,
are to-day rendered more intolerable by the jubilant shouts that reach
them. If I do forget, if I do not faithfully remember those bleeding
children of sorrow this day, ``may my right hand forget her cunning, and
may my tongue cleave to the roof of my mouth!'' To forget them, to pass
lightly over their wrongs, and to chime in with the popular theme, would
be treason most scandalous and shocking, and would make me a reproach
before God and the world. My subject, then, fellow-citizens, is
\textsc{American Slavery}. I shall see this day and its popular
characteristics from the slave's point of view. Standing there,
identified with the American bondman, making his wrongs mine, I do not
hesitate to declare, with all my soul, that the character and conduct of
this nation never looked blacker to me than on this Fourth of July.
Whether we turn to the declarations of the past, or to the professions
of the present, the conduct of the nation seems equally hideous and
revolting. America is false to the past, false to the present, and
solemnly binds herself to be false to the future. Standing with God and
the crushed and bleeding slave on this occasion, I will, in the name of
humanity which is outraged, in the name of liberty which is fettered, in
the name of the constitution and the bible, which are disregarded and
trampled upon, dare to call in question and to denounce, with all the
emphasis I can command, everything that serves to perpetuate
slavery---the great sin and shame of America! ``I will not equivocate; I
will not excuse;'' I will use the severest language I can
{\protect\hypertarget{443}{}{}}command; and yet not one word shall
escape me that any man, whose judgment is not blinded by prejudice, or
who is not at heart a slaveholder, shall not confess to be right and
just.

But I fancy I hear some one of my audience say, it is just in this
circumstance that you and your brother abolitionists fail to make a
favorable impression on the public mind. "Would you argue more, and
denounce less, would you persuade more and rebuke less, your cause would
be much more likely to succeed. But, I submit, where all is plain there
is nothing to be argued. What point in the anti-slavery creed would you
have me argue? On what branch of the subject do the people of this
country need light? Must I undertake to prove that the slave is a man?
That point is conceded already. Nobody doubts it. The slaveholders
themselves acknowledge it in the enactment of laws for their government.
They acknowledge it when they punish disobedience on the part of the
slave. There are seventy-two crimes in the state of Virginia, which, if
committed by a black man, (no matter how ignorant he be,) subject him to
the punishment of death; while only two of these same crimes will
subject a white man to the like punishment. What is this but the
acknowledgment that the slave is a moral, intellectual, and responsible
being. The manhood of the slave is conceded. It is admitted in the fact
that southern statute books are covered with enactments forbidding,
under severe fines and penalties, the teaching of the slave to read or
write. When you can point to any such laws, in reference to the beasts
of the field, then I may consent to argue the manhood of the slave. When
the dogs in your streets, when the fowls of the air, when the cattle on
your hills, when the fish of the sea, and the reptiles that crawl, shall
be unable to distinguish the slave from a brute, then will I argue with
you that the slave is a man!

For the present, it is enough to affirm the equal manhood of the negro
race. Is it not astonishing that, while we are plowing, planting, and
reaping, using all kinds of mechanical tools, erecting houses,
constructing bridges, building ships, working in metals of brass, iron,
copper, silver, and gold; that, while we are reading, writing, and
cyphering, acting as clerks, merchants, and secretaries, having among us
lawyers, doctors, ministers, poets, authors, editors, orators, and
teachers; that, while we are engaged in all manner of enterprises common
to other men---digging gold in California, capturing the whale in the
Pacific, feeding sheep and cattle on the
{\protect\hypertarget{444}{}{}}hillside, living, moving, acting,
thinking, planning, living in families as husbands, wives, and children,
and, above all, confessing and worshiping the christian's God, and
looking hopefully for life and immortality beyond the grave,---we are
called upon to prove that we are men!

Would you have me argue that man is entitled to liberty? that he is the
rightful owner of his own body? You have already declared it. Must I
argue the wrongfulness of slavery? Is that a question for republicans?
Is it to be settled by the rules of logic and argumentation, as a matter
beset with great difficulty, involving a doubtful application of the
principle of justice, hard to be understood? How should I look to-day in
the presence of Americans, dividing and subdividing a discourse, to show
that men have a natural right to freedom, speaking of it relatively and
positively, negatively and affirmatively? To do so, would be to make
myself ridiculous, and to offer an insult to your understanding. There
is not a man beneath the canopy of heaven that does not know that
slavery is wrong \emph{for him}.

What! am I to argue that it is wrong to make men brutes, to rob them of
their liberty, to work them without wages, to keep them ignorant of
their relations to their fellow-men, to beat them with sticks, to flay
their flesh with the lash, to load their limbs with irons, to hunt them
with dogs, to sell them at auction, to sunder their families, to knock
out their teeth, to burn their flesh, to starve them into obedience and
submission to their masters? Must I argue that a system, thus marked
with blood and stained with pollution, is wrong? No; I will not. I have
better employment for my time and strength than such arguments would
imply.

What, then, remains to be argued? Is it that slavery is not divine; that
God did not establish it; that our doctors of divinity are mistaken?
There is blasphemy in the thought. That which is inhuman cannot be
divine. Who can reason on such a proposition! They that can, may; I
cannot. The time for such argument is past.

At a time like this, scorching irony, not convincing argument, is
needed. Oh! had I the ability, and could I reach the nation's ear, I
would to-day pour out a fiery stream of biting ridicule, blasting
reproach, withering sarcasm, and stern rebuke. For it is not light that
is needed, but fire; it is not the gentle shower, but thunder. We need
the storm, the whirlwind, and the earthquake.
{\protect\hypertarget{445}{}{}}The feeling of the nation must be
quickened; the conscience of the nation must be roused; the propriety of
the nation must be startled; the hypocrisy of the nation must be
exposed; and its crimes against God and man must be proclaimed and
denounced.

What to the American slave is your Fourth of July? I answer, a day that
reveals to him, more than all other days in the year, the gross
injustice and cruelty to which he is the constant victim. To him, your
celebration is a sham; your boasted liberty, an unholy license; your
national greatness, swelling vanity; your sounds of rejoicing are empty
and heartless; your denunciations of tyrants, brass-fronted impudence;
your shouts of liberty and equality, hollow mockery; your prayers and
hymns, your sermons and thanksgivings, with all your religious parade
and solemnity, are to him mere bombast, fraud, deception, impiety, and
hypocrisy---a thin veil to cover up crimes which would disgrace a nation
of savages. There is not a nation on the earth guilty of practices more
shocking and bloody, than are the people of these United States, at this
very hour.

Go where you may, search where you will, roam through all the monarchies
and despotisms of the old world, travel through South America, search
out every abuse, and when you have found the last, lay your facts by the
side of the every-day practices of this nation, and you will say with
me, that, for revolting barbarity and shameless hypocrisy, America,
reigns without a rival.

{\protect\hypertarget{ux5cux7bux5cux7bux5cux7b1ux5cux7dux5cux7dux5cux7d}{}{}}

{\protect\hypertarget{446}{}{}}

{THE INTERNAL SLAVE TRADE.}

{EXTRACT FROM AN ORATION, AT ROCHESTER, JULY 5, 1852.}

\textsc{Take} the American slave trade, which, we are told by the
papers, is especially prosperous just now. Ex-senator Benton tells us
that the price of men was never higher than now. He mentions the fact to
show that slavery is in no danger. This trade is one of the
peculiarities of American institutions. It is carried on in all the
large towns and cities in one-half of this confederacy; and millions are
pocketed every year by dealers in this horrid traffic. In several states
this trade is a chief source of wealth. It is called (in
contradistinction to the foreign slave trade) "\emph{the internal slave
trade.}" It is, probably, called so, too, in order to divert from it the
horror with which the foreign slave trade is contemplated. That trade
has long since been denounced by this government as piracy. It has been
denounced with burning words, from the high places of the nation, as an
execrable traffic. To arrest it, to put an end to it, this nation keeps
a squadron, at immense cost, on the coast of Africa. Everywhere in this
country, it is safe to speak of this foreign slave trade as a most
inhuman traffic, opposed alike to the laws of God and of man. The duty
to extirpate and destroy it is admitted even by our \emph{doctors of
divinity}. In order to put an end to it, some of these last have
consented that their colored brethren (nominally free) should leave this
country, and establish themselves on the western coast of Africa. It is,
however, a notable fact, that, while so much execration is poured out by
Americans, upon those engaged in the foreign slave trade, the men
engaged in the slave trade between the states pass without condemnation,
and their business is deemed honorable.

Behold the practical operation of this internal slave trade---the
American slave trade sustained by American politics and American
religion! Here you will see men and women reared like swine for the
market. You know what is a swine-drover? I will show you
{\protect\hypertarget{447}{}{}}a man-drover. They inhabit all our
southern states. They perambulate the country, and crowd the highways of
the nation with droves of human stock. You will see one of these
human-flesh-jobbers, armed with pistol, whip, and bowie-knife, driving a
company of a hundred men, women, and children, from the Potomac to the
slave market at New Orleans. These wretched people are to be sold
singly, or in lots, to suit purchasers. They are food for the
cotton-field and the deadly sugar-mill. Mark the sad procession as it
moves wearily along, and the inhuman wretch who drives them. Hear his
savage yells and his blood-chilling oaths, as he hurries on his
affrighted captives. There, see the old man, with locks thinned and
gray. Cast one glance, if you please, upon that young mother, whose
shoulders are bare to the scorching sun, her briny tears falling on the
brow of the babe in her arms. See, too, that girl of thirteen, weeping,
yes, weeping, as she thinks of the mother from whom she has been torn.
The drove moves tardily. Heat and sorrow have nearly consumed their
strength. Suddenly you hear a quick snap, like the discharge of a rifle;
the fetters clank, and the chain rattles simultaneously; your ears are
saluted with a scream that seems to have torn its way to the center of
your soul. The crack you heard was the sound of the slave whip; the
scream you heard was from the woman you saw with the babe. Her speed had
faltered under the weight of her child and her chains; that gash on her
shoulder tells her to move on. Follow this drove to New Orleans, Attend
the auction; see men examined like horses; gee the forms of women rudely
and brutally exposed to the shocking gaze of American slave-buyers. See
this drove sold and separated forever; and never forget the deep, sad
sobs that arose from that scattered multitude. Tell me, citizens, where,
under the sun, can you witness a spectacle more fiendish and shocking.
Yet this is but a glance at the American slave trade, as it exists at
this moment, in the ruling part of the United States.

I was born amid such sights and scenes. To me the American slave trade
is a terrible reality. "When a child, my soul was often pierced with a
sense of its horrors. I lived on Philpot street, Fell's Point,
Baltimore, and have watched from the wharves the slave (ships in the
basin, anchored from the shore, with their cargoes of human flesh,
waiting for favorable winds to waft them down the Chesapeake. There was,
at that time, a grand slave mart kept at the head of Pratt street, by
Austin Woldfolk. His agents were {\protect\hypertarget{448}{}{}}sent
into every town and county in Maryland, announcing their arrival through
the papers, and on flaming hand-bills, headed, ``cash for negroes.''
These men were generally well dressed, and very captivating in their
manners; ever ready to drink, to treat, and to gamble. The fate of many
a slave has depended upon the turn of a single card; and many a child
has been snatched from the arms of its mother by bargains arranged in a
state of brutal drunkenness.

The flesh-mongers gather up their victims by dozens, and drive them,
chained, to the general depot at Baltimore. When a sufficient number
have been collected here, a ship is chartered, for the purpose of
conveying the forlorn crew to Mobile or to New Orleans. From the
slave-prison to the ship, they are usually driven in the darkness of
night; for since the anti-slavery agitation a certain caution is
observed.

In the deep, still darkness of midnight, I have been often aroused by
the dead, heavy footsteps and the piteous cries of the chained gangs
that passed our door. The anguish of my boyish heart was intense; and I
was often consoled, when speaking to my mistress in the morning, to hear
her say that the custom was very wicked; that she hated to hear the
rattle of the chains, and the heart-rending cries. I was glad to find
one who sympathized with me in my horror.

Fellow-citizens, this murderous traffic is to-day in active operation in
this boasted republic. In the solitude of my spirit, I see clouds of
dust raised on the highways of the south; I see the bleeding footsteps;
I hear the doleful wail of fettered humanity, on the way to the slave
markets, where the victims are to be sold like horses, sheep, and swine,
knocked off to the highest bidder. There I see the tenderest ties
ruthlessly broken, to gratify the lust, caprice, and rapacity of the
buyers and sellers of men. My soul sickens at the sight.

{"}Is this the land your fathers loved?\\
{}The freedom which they toiled to win?\\
Is this the earth whereon they moved?\\
{}Are these the graves they slumber in?"

But a still more inhuman, disgraceful, and scandalous state of things
remains to be presented. By an act of the American congress, not yet two
years old, slavery has been nationalized in its
{\protect\hypertarget{449}{}{}}most horrible and revolting form. By that
act, Mason and Dixon's line has been obliterated; New York has become as
Virginia; and the power to hold, hunt, and sell men, women, and children
as slaves, remains no longer a mere state institution, but is now an
institution of the whole United States. The power is co-extensive with
the star-spangled banner and American christianity. Where these go, may
also go the merciless slave-hunter. Where these are, man is not sacred.
He is a bird for the sportsman's gun. By that most foul and fiendish of
all human decrees, the liberty and person of every man are put in peril.
Your broad republican domain is a hunting-ground for \emph{men}. Not for
thieves and robbers, enemies of society, merely, but for men guilty of
no crime. Your law-makers have commanded all good citizens to engage in
this hellish sport. Your president, your secretary of state, your lords,
nobles, and ecclesiastics, enforce as a duty you owe to your free and
glorious country and to your God, that you do this accursed thing. Not
fewer than forty Americans have within the past two years been hunted
down, and without a moment's warning, hurried away in chains, and
consigned to slavery and excruciating torture. Some of these have had
wives and children dependent on them for bread; but of this no account
was made. The right of the hunter to his prey, stands superior to the
right of marriage, and to \emph{all} rights in this republic, the rights
of God included! For black men there are neither law, justice, humanity,
nor religion. The fugitive slave law makes \textsc{mercy to them a
crime}; and bribes the judge who tries them. An American judge
\textsc{gets ten dollars for every victim he consigns} to slavery, and
five, when he fails to do so. The oath of {and} two villians is
sufficient, under this hell-black enactment, to send the most pious and
exemplary black man into the remorseless jaws of slavery! His own
testimony is nothing. He can bring no witnesses for himself. The
minister of American justice is bound by the law to hear but \emph{one
side;} and that side is the side of the oppressor. Let this damning fact
be perpetually told. Let it be thundered around the world, that, in
tyrant-killing, king-hating, people-loving, democratic, christian
America, the seats of justice are filled with judges, who hold their
office under an open and palpable \emph{bribe}, and are bound, in
deciding in the case of a man's liberty, \emph{to hear only his
accusers!}

In glaring violation of justice, in shamelesss disregard of the forms of
administering law, in cunning arrangement to entrap the
{\protect\hypertarget{450}{}{}}defenseless, and in diabolical intent,
this fugitive slave law stands alone in the annals of tyrannical
legislation. I doubt if there be another nation on the globe having the
brass and the baseness to put such a law on the statute-book. If any man
in this assembly thinks differently from me in this matter, and feels
able to disprove my statements, I will gladly confront him at any
suitable time and place he may select.

{\protect\hypertarget{ux5cux7bux5cux7bux5cux7b1ux5cux7dux5cux7dux5cux7d}{}{}}

{\protect\hypertarget{451}{}{}}

{THE SLAVERY PARTY.}

{EXTRACT FROM A SPEECH DELIVERED BEFORE THE A. A. S. SOCIETY, IN NEW
YORK, MAY, 1853.}

\textsc{Sir}, it is evident that there is in this country a purely
slavery party---a party which exists for no other earthly purpose but to
promote the interests of slavery. The presence of this party is felt
everywhere in the republic. It is known by no particular name, and has
assumed no definite shape; but its branches reach far and wide in the
church and in the state. This shapeless and nameless party is not
intangible in other and more important respects. That party, sir, has
determined upon a fixed, definite, and comprehensive policy toward the
whole colored population of the United States. ``What that policy is, it
becomes us as abolitionists, and especially does it become the colored
people themselves, to consider and to understand fully. We ought to know
who our enemies are, where they are, and what are their objects and
measures. ''Well, sir, here is my version of it---not original with
me---but mine because I hold it to be true.

I understand this policy to comprehend five cardinal objects. They are
these: 1st, The complete suppression of all anti-slavery discussion. 2d.
The expatriation of the entire free people of color from the United
States. 3d. The unending perpetuation of slavery in this republic. 4th.
The nationalization of slavery to the extent of making slavery respected
in every state of the Union. 5th. The extension of slavery over Mexico
and the entire South American states.

Sir, these objects are forcibly presented to us in the stern logic of
passing events; in the facts which are and have been passing around us
during the last three years. The country has been and is now dividing on
these grand issues. In their magnitude, these issues cast all others
into the shade, depriving them of all life and vitality. Old party ties
are broken. Like is finding its like on
{\protect\hypertarget{452}{}{}}either side of these great issues, and
the great battle is at hand. For the present, the best representative of
the slavery party in politics is the democratic party. Its great head
for the present is President Pierce, whose boast it was, before his
election, that his whole life had been consistent with the interests of
slavery, that he is above reproach on that score. In his inaugural
address, he reassures the south on this point. ``Well, the head of the
slave power being in power, it is natural that the pro-slavery elements
should cluster around the administration, and this is rapidly being
done. A fraternization is going on. The stringent protectionists and the
free-traders strike hands. The supporters of Fillmore are becoming the
supporters of Pierce. The silver-gray whig shakes hands with the hunker
democrat; the former only differing from the latter in name. They are of
one heart, one mind, and the union is natural and perhaps inevitable.
Both hate negroes; both hate progress; both hate the ''higher law;" both
hate William H. Seward; both hate the free democratic party; and upon
this hateful basis they are forming a union of hatred. ``Pilate and
Herod are thus made friends.'' Even the central organ of the whig party
is extending its beggar hand for a morsel from the table of slavery
democracy, and when spurned from the feast by the more deserving, it
pockets the insult; when kicked on one side it turns the other, and
perseveres in its importunities. The fact is, that paper comprehends the
demands of the times; it understands the age and its issues; it wisely
sees that slavery and freedom are the great antagonistic forces in the
country, and it goes to its own side. Silver grays and hunkers all
understand this. They are, therefore, rapidly sinking all other
questions to nothing, compared with the increasing demands of slavery.
They are collecting, arranging, and consolidating their forces for the
accomplishment of their appointed work.

The keystone to the arch of this grand union of the slavery party of the
United States, is the compromise of 1850. In that compromise we have all
the objects of our slaveholding policy specified. It is, sir, favorable
to this view of the designs of the slave power, that both the whig and
the democratic party bent lower, sunk deeper, and strained harder, in
their conventions, preparatory to the late presidential election, to
meet the demands of the slavery party than at any previous time in their
history. Never did parties come before the northern people with
propositions of such {\protect\hypertarget{453}{}{}}undisguised contempt
for the moral sentiment and the religious ideas of that people. They
virtually asked them to unite in a war upon free speech, and upon
conscience, and to drive the Almighty presence from the councils of the
nation. Resting their platforms upon the fugitive slave bill, they
boldly asked the people for political power to execute the horrible and
hell-black provisions of that bill. The history of that election
reveals, with great clearness, the extent to which slavery has shot its
leprous distillment through the life-blood of the nation. The party most
thoroughly opposed to the cause of justice and humanity, triumphed;
while the party suspected of a leaning toward liberty, was
overwhelmingly defeated, some say annihilated.

But here is a still more important fact, illustrating the designs of the
slave power. It is a fact full of meaning, that no sooner did the
democratic slavery party come into power, than a system of legislation
was presented to the legislatures of the northern states, designed to
put the states in harmony with the fugitive slave law, and the malignant
bearing of the national government toward the colored inhabitants of the
country. This whole movement on the part of the states, bears the
evidence of having one origin, emanating from one head, and urged
forward by one power. It was simultaneous, uniform, and general, and
looked to one end. It was intended to put thorns under feet already
bleeding; to crush a people already bowed down; to enslave a people
already but half free; in a word, it was intended to discourage,
dishearten, and drive the free colored people out of the country. In
looking at the recent black law of Illinois, one is struck dumb with its
enormity. It would seem that the men who enacted that law, had not only
banished from their minds all sense of justice, but all sense of shame.
It coolly proposes to sell the bodies and souls of the black to increase
the intelligence and refinement of the whites; to rob every black
stranger who ventures among them, to increase their literary fund.

While this is going on in the states, a pro-slavery, political board of
health is established at Washington. Senators Hale, Chase, and Sumner
are robbed of a part of their senatorial dignity and consequence as
representing sovereign states, because they have refused to be
inoculated with the slavery virus. Among the services which a senator is
expected by his state to perform, are many that can only be done
efficiently on committees; and, in saying to these honorable senators,
you shall not serve on the committees of this body,
{\protect\hypertarget{454}{}{}}the slavery party took the responsibility
of robbing and insulting the states that sent them. It is an attempt at
Washington to decide for the states who shall be sent to the senate.
Sir, it strikes me that this aggression on the part of the slave power
did not meet at the hands of the proscribed senators the rebuke which we
had a right to expect would be administered. It seems to me that an
opportunity was lost, that the great principle of senatorial equality
was left undefended, at a time when its vindication was sternly
demanded. But it is not to the purpose of my present statement to
criticise the conduct of our friends. I am persuaded that much ought to
be left to the discretion of anti-slavery men in congress, and charges
of recreancy should never be made but on the most sufficient grounds.
For, of all the places in the world where an anti-slavery man needs the
confidence and encouragement of friends, I take Washington to be that
place.

Let me now call attention to the social influences which are operating
and coöperating with the slavery party of the country, designed to
contribute to one or all of the grand objects aimed at by that party. We
see here the black man attacked in his vital interests; prejudice and
hate are excited against him; enmity is stirred up between him and other
laborers. The Irish people, warm-hearted, generous, and sympathizing
with the oppressed everywhere, when they stand upon their own green
island, are instantly taught, on arriving in this christian country, to
hate and despise the colored people. They are taught to believe that we
eat the bread which of right belongs to them. The cruel lie is told the
Irish, that our adversity is essential to their prosperity. Sir, the
Irish-American will find out his mistake one day. He will find that in
assuming our avocation he also has assumed our degradation. But for the
present we are sufferers. The old employments by which we have
heretofore gained our livelihood, are gradually, and it may be
inevitably, passing into other hands. Every hour sees us elbowed out of
some employment to make room perhaps for some newly-arrived emigrants,
whose hunger and color are thought to give them a title to especial
favor. White men are becoming house-servants, cooks, and stewards,
common laborers, and flunkeys to our gentry, and, for aught I see, they
adjust themselves to their stations with all becoming obsequiousness.
This fact proves that if we cannot rise to the whites, the whites can
fall to us. Now, sir, look once more. While the colored people are thus
{\protect\hypertarget{455}{}{}}elbowed out of employment; while the
enmity of emigrants is being excited against us; while state after state
enacts laws against us; while we are hunted down, like wild game, and
oppressed with a general feeling of insecurity,---the American
colonization society---that old offender against the best interests and
slanderer of the colored people---awakens to new life, and vigorously
presses its scheme upon the consideration of the people and the
government. New papers are started---some for the north and some for the
south---and each in its tone adapting itself to its latitude.
Government, state and national, is called upon for appropriations to
enable the society to send us out of the country by steam! They want
steamers to carry letters and negroes to Africa. Evidently, this society
looks upon our ``extremity as its opportunity,'' and we may expect that
it will use the occasion well. They do not deplore, but glory, in our
misfortunes.

But, sir, I must hasten. I have thus briefly given my view of one aspect
of the present condition and future prospects of the colored people of
the United States. And what I have said is far from encouraging to my
afflicted people. I have seen the cloud gather upon the sable brows of
some who hear me. I confess the case looks black enough. Sir, I am not a
hopeful man. I think I am apt even to undercalculate the benefits of the
future. Yet, sir, in this seemingly desperate case, I do not despair for
my people. There is a bright side to almost every picture of this kind;
and ours is no exception to the general rule. If the influences against
us are strong, those for us are also strong. To the inquiry, will our
enemies prevail in the execution of their designs. In my God and in my
soul, I believe they \emph{will not}. Let us look at the first object
sought for by the slavery party of the country, viz: the suppression of
anti-slavery discussion. They desire to suppress discussion on this
subject, with a view to the peace of the slaveholder and the security of
slavery. Now, sir, neither the principle nor the subordinate objects
here declared, can be at all gained by the slave power, and for this
reason: It involves the proposition to padlock the lips of the whites,
in order to secure the fetters on the limbs of the blacks. The right of
speech, precious and priceless, \emph{cannot, will not}, be surrendered
to slavery. Its suppression is asked for, as I have said, to give peace
and security to slaveholders. Sir, that thing cannot be done. God has
interposed an insuperable obstacle to any such result. "There can be
\emph{no peace}, saith my God, to the wicked,"
{\protect\hypertarget{456}{}{}}Suppose it were possible to put down this
discussion, what would it avail the guilty slaveholder, pillowed as he
is upon the heaving bosoms of ruined souls? He could not have a peaceful
spirit. If every anti-slavery tongue in the nation were silent---every
anti-slavery organization dissolved---every anti-slavery press
demolished---every anti-slavery periodical, paper, book, pamphlet, or
what not, were searched out, gathered together, deliberately burned to
ashes, and their ashes given to the four winds of heaven, still, still
the slaveholder could have "\emph{no peace.}" In every pulsation of his
heart, in every throb of his life, in every glance of his eye, in the
breeze that soothes, and in the thunder that startles, would be waked up
an accuser, whose cause is, ``Thou art, verily, guilty concerning thy
brother.''

{\protect\hypertarget{ux5cux7bux5cux7bux5cux7b1ux5cux7dux5cux7dux5cux7d}{}{}}

{\protect\hypertarget{457}{}{}}

{THE ANTI-SLAVERY MOVEMENT.}

{EXTRACTS FROM A LECTURE BEFORE VARIOUS ANTI-SLAVERY BODIES, IN THE
WINTER OF 1855.}

\textsc{A grand} movement on the part of mankind, in any direction, or
for any purpose, moral or political, is an interesting fact, fit and
proper to be studied. It is such, not only for those who eagerly
participate in it, but also for those who stand aloof from it---even for
those by whom it is opposed. I take the anti-slavery movement to be such
an one, and a movement as sublime and glorious in its character, as it
is holy and beneficent in the ends it aims to accomplish. At this
moment, I deem it safe to say, it is properly engrossing more minds in
this country than any other subject now before the American people. The
late John C. Calhoun---one of the mightiest men that ever stood up in
the American senate---did not deem it beneath him; and he probably
studied it as deeply, though not as honestly, as Gerrit Smith, or
William Lloyd Garrison. He evinced the greatest familiarity with the
subject; and the greatest efforts of his last years in the senate had
direct reference to this movement. His eagle eye watched every new
development connected with it; and he was ever prompt to inform the
south of every important step in its progress. He never allowed himself
to make light of it; but always spoke of it and treated it as a matter
of grave import; and in this he showed himself a master of the mental,
moral, and religious constitution of human society. Daniel Webster, too,
in the better days of his life, before he gave his assent to the
fugitive slave bill, and trampled upon all his earlier and better
convictions---when his eye was yet single---he clearly comprehended the
nature of the elements involved in this movement; and in his own
majestic eloquence, warned the south, and the country, to have a care
how they attempted to put it down. He is an illustration that it is
easier to give, than to take, good advice. To these two men---the
greatest men to whom the nation has yet given
{\protect\hypertarget{458}{}{}}birth---may be traced the two great facts
of the present---the south triumphant, and the north humbled. Their
names may stand thus,---Calhoun and domination---Webster and
degradation. Yet again. If to the enemies of liberty this subject is one
of engrossing interest, vastly more so should it be such to freedom's
friends. The latter, it leads to the gates of all valuable
knowledge---philanthropic, ethical, and religious; for it brings them to
the study of man, wonderfully and fearfully made---the proper study of
man through all time---the open book, in which are the records of time
and eternity.

Of the existence and power of the anti-slavery movement, as a fact, you
need no evidence. The nation has seen its face, and felt the controlling
pressure of its hand. You have seen it moving in all directions, and in
all weathers, and in all places, appearing most where desired least, and
pressing hardest where most resisted. No place is exempt. The quiet
prayer meeting, and the stormy halls of national debate, share its
presence alike. It is a common intruder, and of course has the name of
being ungentlemanly. Brethren who had long sung, in the most
affectionate fervor, and with the greatest sense of security,

``Together let us sweetly live---together let us die,''

have been suddenly and violently separated by it, and ranged in hostile
attitude toward each other. The Methodist, one of the most powerful
religious organizations of this country, has been rent asunder, and its
strongest bolts of denominational brotherhood started at a single surge.
It has changed the tone of the northern pulpit, and modified that of the
press. A celebrated divine, who, four years ago, was for flinging his
own mother, or brother, into the remorseless jaws of the monster
slavery, lest he should swallow up the Union, now recognizes
anti-slavery as a characteristic of future civilization. Signs and
wonders follow this movement; and the fact just stated is one of them.
Party ties are loosened by it; and men are compelled to take sides for
or against it, whether they will or not. Come from where he may, or come
for what he may, he is compelled to show his hand. What is this mighty
force? What is its history? and what is its destiny? Is it ancient or
modern, transient or permanent? Has it turned aside, like a stranger and
a sojourner, to tarry for a night? or has it come to rest with us
forever? Excellent chances are here for speculation; and some of them
are quite profound. We might, for instance, proceed to inquire not
{\protect\hypertarget{459}{}{}}only into the philosophy of the
anti-slavery movement, but into the philosophy of the law, in obedience
to which that movement started into existence. We might demand to know
what is that law or power which, at different times, disposes the minds
of men to this or that particular object---now for peace, and now for
war---now for freedom, and now for slavery; but this profound question I
leave to the abolitionists of the superior class to answer. The
speculations which must precede such answer, would afford, perhaps,
about the same satisfaction as the learned theories which have rained
down upon the world, from time to time, as to the origin of evil. I
shall, therefore, avoid water in which I cannot swim, and deal with
anti-slavery as a fact, like any other fact in the history of mankind,
capable of being described and understood, both as to its internal
forces, and its external phases and relations.

{{[}After an eloquent, a full, and highly interesting exposition of the
nature, character, and history of the anti-slavery movement, from the
insertion of which want of space precludes us, he concluded in the
following happy manner.{]}}

Present organizations may perish, but the cause will go on. That cause
has a life, distinct and independent of the organizations patched up
from time to time to carry it forward. Looked at, apart from the bones
and sinews and body, it is a thing immortal. It is the very essence of
justice, liberty, and love. The moral life of human society, it cannot
die while conscience, honor, and humanity remain. If but one be filled
with it, the cause lives. Its incarnation in any one individual man,
leaves the whole world a priesthood, occupying the highest moral
eminence---even that of disinterested benevolence. Whoso has ascended
this height, and has the grace to stand there, has the world at his
feet, and is the world's teacher, as of divine right. He may set in
judgment on the age, upon the civilization of the age, and upon the
religion of the age; for he has a test, a sure and certain test, by
which to try all institutions, and to measure all men. I say, he may do
this, but this is not the chief business for which he is qualified. The
great work to which he is called is not that of judgment. Like the
Prince of Peace, he may say, if I judge, I judge righteous judgment;
still mainly, like him, he may say, this is not his work. The man who
has thoroughly embraced the principles of justice, love, and liberty,
like the true preacher of Christianity, is less anxious to reproach the
world of its sins, than to win it to repentance. His
{\protect\hypertarget{460}{}{}}great work on earth is to exemplify, and
to illustrate, and to ingraft those principles upon the living and
practical understandings of all men within the reach of his influence.
This is his work; long or short his years, many or few his adherents,
powerful or weak his instrumentalities, through good report, or through
bad report, this is his work. It is to snatch from the bosom of nature
the latent facts of each individual man's experience, and with steady
hand to hold them up fresh and glowing, enforcing, with all his power,
their acknowledgment and practical adoption. If there be but one such
man in the land, no matter what becomes of abolition societies and
parties, there will be an anti-slavery cause, and an anti-slavery
movement. Fortunately for that cause, and fortunately for him by whom it
is espoused, it requires no extraordinary amount of talent to preach it
or to receive it when preached. The grand secret of its power is, that
each of its principles is easily rendered appreciable to the faculty of
reason in man, and that the most unenlightened conscience has no
difficulty in deciding on which side to register its testimony. It can
call its preachers from among the fishermen, and raise them to power. In
every human breast, it has an advocate which can be silent only when the
heart is dead. It comes home to every man's understanding, and appeals
directly to every man's conscience. A man that does not recognize and
approve for himself the rights and privileges contended for, in behalf
of the American slave, has not yet been found. In whatever else men may
differ, they are alike in the apprehension of their natural and personal
rights. The difference between abolitionists and those by whom they are
opposed, is not as to principles. All are agreed in respect to these.
The manner of applying them is the point of difference.

The slaveholder himself, the daily robber of his equal brother,
discourses eloquently as to the excellency of justice, and the man who
employs a brutal driver to flay the flesh of his negroes, is not
offended when kindness and humanity are commended. Every time the
abolitionist speaks of justice, the anti-abolitionist assents---says,
yes, I wish the world were filled with a disposition to render to every
man what is rightfully due him; I should then get what is due me. That's
right; let us have justice. By all means, let us have justice. Every
time the abolitionist speaks in honor of human liberty, he touches a
chord in the heart of the anti-abolitionist, which responds in
harmonious vibrations. Liberty---yes,
{\protect\hypertarget{461}{}{}}that is very evidently my right, and let
him beware who attempts to invade or abridge that right. Every time he
speaks of love, of human brotherhood, and the reciprocal duties of man
and man, the anti-abolitionist assents---says, yes, all right---all
true---we cannot have such ideas too often, or too fully expressed. So
he says, and so he feels, and only shows thereby that he is a man as
well as an anti-abolitionist. You have only to keep out of sight the
manner of applying your principles, to get them endorsed every time.
Contemplating himself, he sees truth with absolute clearness and
distinctness. He only blunders when asked to lose sight of himself. In
his own cause he can beat a Boston lawyer, but he is dumb when asked to
plead the cause of others. He knows very well whatsoever he would have
done unto himself, but is quite in doubt as to having the same thing
done unto others. It is just here, that lions spring up in the path of
duty, and the battle once fought in heaven is refought on the earth. So
it is, so hath it ever been, and so must it ever be, when the claims of
justice and mercy make their demand at the door of human selfishness.
Nevertheless, there is that within which ever pleads for the right and
the just.

In conclusion, I have taken a sober view of the present anti-slavery
movement. I am sober, but not hopeless. There is no denying, for it is
everywhere admitted, that the anti-slavery question is the great moral
and social question now before the American people. A state of things
has gradually been developed, by which that question has become the
first thing in order. It must be met. Herein is my hope. The great idea
of impartial liberty is now fairly before the American people.
Anti-slavery is no longer a thing to be prevented. The time for
prevention is past. This is great gain. When the movement was younger
and weaker---when it wrought in a Boston garret to human apprehension,
it might have been silently put out of the way. Things are different
now. It has grown too large---its friends are too numerous---its
facilities too abundant---its ramifications too extended---its power too
omnipotent, to be snuffed out by the contingencies of infancy. A
thousand strong men might be struck down, and its ranks still be
invincible. One flash from the heart-supplied intellect of Harriet
Beecher Stowe could light a million camp fires in front of the embattled
host of slavery, which not all the waters of the Mississippi, mingled as
they are with blood, could extinguish. The present will be looked to by
after coming generations, as the age of anti-slavery literature
{\protect\hypertarget{462}{}{}}---when supply on the gallop could not
keep pace with the ever-growing demand---when a picture of a negro on
the cover was a help to the sale of a book---when conservative lyceums
and other American literary associations began first to select their
orators for distinguished occasions from the ranks of the previously
despised abolitionists. If the anti-slavery movement shall fail now, it
will not be from outward opposition, but from inward decay. Its
auxiliaries are everywhere. Scholars, authors, orators, poets, and
statesmen give it their aid. The most brilliant of American poets
volunteer in its service. Whittier speaks in burning verse to more than
thirty thousand, in the National Era. Your own Longfellow whispers, in
every hour of trial and disappointment, ``labor and wait.'' James
Russell Lowell is reminding us that ``men are more than institutions.''
Pierpont cheers the heart of the pilgrim in search of liberty, by
singing the praises of ``the north star.'' Bryant, too, is with us; and
though chained to the car of party, and dragged on amidst a whirl of
political excitement, he snatches a moment for letting drop a smiling
verse of sympathy for the man in chains. The poets are with us. It would
seem almost absurd to say it, considering the use that has been made of
them, that we have allies in the Ethiopian songs; those songs that
constitute our national music, and without which we have no national
music. They are heart songs, and the finest feelings of human nature are
expressed in them. ``Lucy Neal,'' ``Old Kentucky Home,'' and ``Uncle
Ned,'' can make the heart sad as well as merry, and can call forth a
tear as well as a smile. They awaken the sympathies for the slave, in
which anti-slavery principles take root, grow, and flourish. In addition
to authors, poets, and scholars at home, the moral sense of the
civilized world is with us. England, France, and Germany, the three
great lights of modern civilization, are with us, and every American
traveler learns to regret the existence of slavery in his country. The
growth of intelligence, the influence of commerce, steam, wind, and
lightning are our allies. It would be easy to amplify this summary, and
to swell the vast conglomeration of our material forces; but there is a
deeper and truer method of measuring the power of our cause, and of
comprehending its vitality. This is to be found in its accordance with
the best elements of human nature. It is beyond the power of slavery to
annihilate affinities recognized and established by the Almighty. The
slave is bound to mankind by the powerful and inextricable net-work of
human brotherhood. His {\protect\hypertarget{463}{}{}}voice is the voice
of a man, and his cry is the cry of a man in distress, and man must
cease to be man before he can become insensible to that cry. It is the
righteousness of the cause---the humanity of the cause---which
constitutes its potency. As one genuine bank-bill is worth more than a
thousand counterfeits, so is one man, with right on his side, worth more
than a thousand in the wrong. ``One may chase a thousand, and put ten
thousand to flight.'' It is, therefore, upon the goodness of our cause,
more than upon all other auxiliaries, that we depend for its final
triumph.

Another source of congratulation is the fact that, amid all the efforts
made by the church, the government, and the people at large, to stay the
onward progress of this movement, its course has been onward, steady,
straight, unshaken, and unchecked from the beginning. Slavery has gained
victories large and numerous; but never as against this
movement---against a temporizing policy, and against northern timidity,
the slave power has been victorious; but against the spread and
prevalence in the country, of a spirit of resistance to its aggression,
and of sentiments favorable to its entire overthrow, it has yet
accomplished nothing. Every measure, yet devised and executed, having
for its object the suppression of anti-slavery, has been as idle and
fruitless as pouring oil to extinguish fire. A general rejoicing took
place on the passage of ``the compromise measures'' of 1850. Those
measures were called peace measures, and were afterward termed by both
the great parties of the country, as well as by leading statesmen, a
final settlement of the whole question of slavery; but experience has
laughed to scorn the wisdom of pro-slavery statesmen; and their final
settlement of agitation seems to be the final revival, on a broader and
grander scale than ever before, of the question which they vainly
attempted to suppress forever. The fugitive slave bill has especially
been of positive service to the anti-slavery movement. It has
illustrated before all the people the horrible character of slavery
toward the slave, in hunting him down in a free state, and tearing him
away from wife and children, thus setting its claims higher than
marriage or parental claims. It has revealed the arrogant and
overbearing spirit of the slave states toward the free states; despising
their principles---shocking their feelings of humanity, not only by
bringing before them the abominations of slavery, but by attempting to
make them parties to the crime. It has called into exercise among the
colored people, the hunted ones, a spirit of manly resistance well
calculated to surround {\protect\hypertarget{464}{}{}}them with a
bulwark of sympathy and respect hitherto unknown. For men are always
disposed to respect and defend rights, when the victims of oppression
stand up manfully for themselves.

There is another element of power added to the anti-slavery movement, of
great importance; it is the conviction, becoming every day more general
and universal, that slavery must be abolished at the south, or it will
demoralize and destroy liberty at the north. It is the nature of slavery
to beget a state of things all around it favorable to its own
continuance. This fact, connected with the system of bondage, is
beginning to be more fully realized. The slave-holder is not satisfied
to associate with men in the church or in the state, unless he can
thereby stain them with the blood of his slaves. To be a slave-holder is
to be a propagandist from necessity; for slavery can only live by
keeping down the under-growth morality which nature supplies. Every
new-born white babe comes armed from the Eternal presence, to make war
on slavery. The heart of pity, which would melt in due time over the
brutal chastisements it sees inflicted on the helpless, must be
hardened. And this work goes on every day in the year, and every hour in
the day.

What is done at home is being done also abroad here in the north. And
even now the question may be asked, have we at this moment a single free
state in the Union? The alarm at this point will become more general.
The slave power must go on in its career of exactions. Give, give, will
be its cry, till the timidity which concedes shall give place to
courage, which shall resist. Such is the voice of experience, such has
been the past, such is the present, and such will be that future, which,
so sure as man is man, will come. Here I leave the subject; and I leave
off where I began, consoling myself and congratulating the friends of
freedom upon the fact that the anti-slavery cause is not a new thing
under the sun; not some moral delusion which a few years' experience may
dispel. It has appeared among men in all ages, and summoned its
advocates from all ranks. Its foundations are laid in the deepest and
holiest convictions, and from whatever soul the demon, selfishness, is
expelled, there will this cause take up its abode. Old as the
everlasting hills; immovable as the throne of God; and certain as the
purposes of eternal power, against all hinderances, and against all
delays, and despite all the mutations of human instrumentalities, it is
the faith of my soul, that this anti-slavery cause will triumph.

\begin{center}\rule{0.5\linewidth}{\linethickness}\end{center}

\begin{enumerate}
\item
  \hypertarget{cite_note-1}{}

  {\protect\hyperlink{cite_ref-1}{↑}} {Mr. Douglass' published speeches
  alone, would fill two volumes of the size of this. Our space will only
  permit the insertion of the extracts which follow; and which, for
  originality of thought, beauty and force of expression, and for
  impassioned, indignatory eloquence, have seldom been equaled.}
\item
  \hypertarget{cite_note-2}{}

  {\protect\hyperlink{cite_ref-2}{↑}} {It is not often that chattels
  address their owners. The following letter is unique; and probably the
  only specimen of the kind extant. It was written while in England.}
\end{enumerate}
