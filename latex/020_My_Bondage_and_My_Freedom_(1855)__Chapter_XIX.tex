{}

~

{CHAPTER XIX.}

THE RUN-AWAY PLOT.

{NEW YEAR'S THOUGHTS AND MEDITATIONS---AGAIN BOUGHT BY FREELAND---NO
AMBITION TO BE A SLAVE---KINDNESS NO COMPENSATION FOR
SLAVERY---INCIPIENT STEPS TOWARD ESCAPE---CONSIDERATIONS LEADING
THERETO---IRRECONCILABLE HOSTILITY TO SLAVERY---SOLEMN VOW TAKEN---PLAN
DIVULGED TO THE SLAVES---COLUMBIAN ORATOR---SCHEME GAINS FAVOR, DESPITE
PRO-SLAVERY PREACHING---DANGER OF DISCOVERY---SKILL OF SLAVEHOLDERS IN
READING THE MINDS OF THEIR SLAVES---SUSPICION AND COERCION---HYMNS WITH
DOUBLE MEANING---VALUE, IN DOLLARS, OF OUR COMPANY---PRELIMINARY
CONSULTATION---PASS-WORD---CONFLICTS OF HOPE AND FEAR---DIFFICULTIES TO
BE OVERCOME---IGNORANCE OF GEOGRAPHY---SURVEY OF IMAGINARY
DIFFICULTIES---EFFECT ON OUR MINDS---PATRICK HENRY---SANDY BECOMES A
DREAMER---ROUTE TO THE NORTH LAID OUT---OBJECTIONS CONSIDERED---FRAUDS
PRACTICED ON FREEMEN---PASSES WRITTEN---ANXIETIES AS THE TIME DREW
NEAR---DREAD OF FAILURE---APPEALS TO COMRADES---STRANGE
PRESENTIMENT---COINCIDENCE---THE BETRAYAL DISCOVERED---THE MANNER OF
ARRESTING US---RESISTANCE MADE BY HENRY HARRIS---ITS EFFECT---THE UNIQUE
SPEECH OF MRS. FREELAND---OUR SAD PROCESSION TO PRISON---BRUTAL JEERS BY
THE MULTITUDE ALONG THE ROAD---PASSES EATEN---THE DENIAL---SANDY TOO
WELL LOVED TO BE SUSPECTED---DRAGGED BEHIND HORSES---THE JAIL A
RELIEF---A NEW SET OF TORMENTORS---SLAVE-TRADERS---JOHN, CHARLES AND
HENRY RELEASED---THE AUTHOR LEFT ALONE IN PRISON---HE IS TAKEN OUT, AND
SENT TO BALTIMORE.}

\textsc{I am} now at the beginning of the year 1836, a time favorable
for serious thoughts. The mind naturally occupies itself with the
mysteries of life in all its phases---the ideal, the real and the
actual. Sober people look both ways at the beginning of the year,
surveying the errors of the past, and providing against possible errors
of the future. I, too, was thus exercised. I had little pleasure in
retrospect, and the {}prospect was not very brilliant.
``Notwithstanding,'' thought I, ``the many resolutions and prayers I
have made, in behalf of freedom, I am, this first day of the year 1836,
still a slave, still wandering in the depths of spirit-devouring
thralldom. My faculties and powers of body and soul are not my own, but
are the property of a fellow mortal, in no sense superior to me, except
that he has the physical power to compel me to be owned and controlled
by him. By the combined physical force of the community, I am his
slave,---a slave for life.'' With thoughts like these, I was perplexed
and chafed; they rendered me gloomy and disconsolate. The anguish of my
mind may not be written.

At the close of the year 1835, Mr. Freeland, my temporary master, had
bought me of Capt. Thomas Auld, for the year 1836. His promptness in
securing my services, would have been flattering to my vanity, had I
been ambitious to win the reputation of being a valuable slave. Even as
it was, I felt a slight degree of complacency at the circumstance. It
showed he was as well pleased with me as a slave, as I was with him as a
master. I have already intimated my regard for Mr. Freeland, and I may
say here, in addressing northern readers---where there is no selfish
motive for speaking in praise of a slaveholder---that Mr. Freeland was a
man of many excellent qualities, and to me quite preferable to any
master I ever had.

But the kindness of the slavemaster only gilds the chain of slavery, and
detracts nothing from its weight or power. The thought that men are made
for other {}and better uses than slavery, thrives best under the gentle
treatment of a kind master. But the grim visage of slavery can assume no
smiles which can fascinate the partially enlightened slave, into a
forgetfulness of his bondage, nor of the desirableness of liberty.

I was not through the first month of this, my second year with the kind
and gentlemanly Mr. Freeland, before I was earnestly considering and
devising plans for gaining that freedom, which, when I was but a mere
child, I had ascertained to be the natural and inborn right of every
member of the human family. The desire for this freedom had been
benumbed, while I was under the brutalizing dominion of Covey; and it
had been postponed, and rendered inoperative, by my truly pleasant
Sunday school engagements with my friends, during the year 1835, at Mr.
Freeland's. It had, however, never entirely subsided. I hated slavery,
always, and the desire for freedom only needed a favorable breeze, to
fan it into a blaze, at any moment. The thought of only being a creature
of the \emph{present} and the \emph{past}, troubled me, and I longed to
have a \emph{future}---a future with hope in it. To be shut up entirely
to the past and present, is abhorrent to the human mind; it is to the
soul---whose life and happiness is unceasing progress---what the prison
is to the body; a blight and mildew, a hell of horrors. The dawning of
this, another year, awakened me from my temporary slumber, and roused
into life my latent, but long cherished aspirations for freedom. I was
now not only ashamed to be contented in slavery, but ashamed to
\emph{seem} to be contented, and in my present favorable condition,
under the mild rule of {}Mr. F., I am not sure that some kind reader
will not condemn me for being over ambitious, and greatly wanting in
proper humility, when I say the truth, that I now drove from me all
thoughts of making the best of my lot, and welcomed only such thoughts
as led me away from the house of bondage. The intense desire, now felt,
\emph{to be free}, quickened by my present favorable circumstances,
brought me to the determination to \emph{act}, as well as to think and
speak. Accordingly, at the beginning of this year 1836, I took upon me a
solemn vow, that the year which had now dawned upon me should not close,
without witnessing an earnest attempt, on my part, to gain my liberty.
This vow only bound me to make my escape individually; but the year
spent with Mr. Freeland had attached me, as with ``hooks of steel,'' to
my brother slaves. The most affectionate and confiding friendship
existed between us; and I felt it my duty to give them an opportunity to
share in my virtuous determination, by frankly disclosing to them my
plans and purposes. Toward Henry and John Harris, I felt a friendship as
strong as one man can feel for another; for I could have died with and
for them. To them, therefore, with a suitable degree of caution, I began
to disclose my sentiments and plans; sounding them, the while, on the
subject of running away, provided a good chance should offer. I scarcely
need tell the reader, that I did my \emph{very best} to imbue the minds
of my dear friends with my own views and feelings. Thoroughly awakened,
now, and with a definite vow upon me, all my little reading, which had
any bearing on the subject of human rights, was rendered {}available in
for communications with my friends. That (to me) gem of a book, the
Columbian Orator, with its eloquent orations and spicy dialogues,
denouncing oppression and slavery---telling of what had been dared, done
and suffered by men, to obtain the inestimable boon of liberty---was
still fresh in my memory, and whirled into the ranks of my speech with
the aptitude of well trained soldiers, going through the drill. The fact
is, I here began my public speaking. I canvassed, with Henry and John,
the subject of slavery, and dashed against it the condemning brand of
God's eternal justice, which it every hour violates. My fellow servants
were neither indifferent, dull, nor inapt. Our feelings were more alike
than our opinions. All, however, were ready to act, when a feasible plan
should be proposed. "Show us \emph{how} the thing is to be done," said
they, ``and all else is clear.''

We were all, except Sandy, quite free from slaveholding priestcraft. It
was in vain that we had been taught from the pulpit at St. Michael's,
the duty of obedience to our masters; to recognize God as the author of
our enslavement; to regard running away an offense, alike against God
and man; to deem our enslavement a merciful and beneficial arrangement;
to esteem our condition, in this country, a paradise to that from which
we had been snatched in Africa; to consider our hard hands and dark
color as God's mark of displeasure, and as pointing us out as the proper
subjects of slavery; that the relation of master and slave was one of
reciprocal benefits; that our work was not more serviceable to our
masters, than our master's thinking was serviceable to us. I {}say, it
was in vain that the pulpit of St. Michael's had constantly inculcated
these plausible doctrines. Nature laughed them to scorn. For my own
part, I had now become altogether too big for my chains. Father Lawson's
solemn words, of what I ought to be, and might be, in the providence of
God, had not fallen dead on my soul. I was fast verging toward manhood,
and the prophecies of my childhood were still unfulfilled. The thought,
that year after year had passed away, and my best resolutions to run
away had failed and faded---that I was \emph{still a slave}, and a
slave, too, with chances for gaining my freedom diminished and still
diminishing---was not a matter to be slept over easily; nor did I easily
sleep over it.

But here came a new trouble. Thoughts and purposes so incendiary as
those I now cherished, could not agitate the mind long, without danger
of making themselves manifest to scrutinizing and unfriendly beholders.
I had reason to fear that my sable face might prove altogether too
transparent for the safe concealment of my hazardous enterprise. Plans
of greater moment have leaked through stone walls, and revealed their
projectors. But, here was no stone wall to hide my purpose. I would have
given my poor, tell tale face for the immovable countenance of an
Indian, for it was far from being proof against the daily, searching
glances of those with whom I met.

It is the interest and business of slaveholders to study human nature,
with a view to practical results; and many of them attain astonishing
proficiency in discerning the thoughts and emotions of slaves. They have
to deal not with earth, wood, or stone, but with {}\emph{men;} and, by
every regard they have for their safety and prosperity, they must study
to know the material on which they are at work. So much intellect as the
slaveholder has around him, requires watching. Their safety depends upon
their vigilance. Conscious of the injustice and wrong they are every
hour perpetrating, and knowing what they themselves would do if made the
victims of such wrongs, they are looking out for the first signs of the
dread retribution of justice. They watch, therefore, with skilled and
practiced eyes, and have learned to read, with great accuracy, the state
of mind and heart of the slave, through his sable face. These uneasy
sinners are quick to inquire into the matter, where the slave is
concerned. Unusual sobriety, apparent abstraction, sullenness and
indifference---indeed, any mood out of the common way---afford ground
for suspicion and inquiry. Often relying on their superior position and
wisdom, they hector and torture the slave into a confession, by
affecting to know the truth of their accusations. ``You have got the
devil in you,'' say they, ``and we will whip him out of you.'' I have
often been put thus to the torture, on bare suspicion. This system has
its disadvantages as well as their opposite. The slave is sometimes
whipped into the confession of offenses which he never committed. The
reader will see that the good old rule---``a man is to be held innocent
until proved to be guilty''---does not hold good on the slave
plantation. Suspicion and torture are the approved methods of getting at
the truth, here. It was necessary for me, therefore, to {}keep a watch
over my deportment, lest the enemy should get the better of me.

But with all our caution and studied reserve, I am not sure that Mr.
Freeland did not suspect that all was not right with us. It \emph{did}
seem that he watched us more narrowly, after the plan of escape had been
conceived and discussed amongst us. Men seldom see themselves as others
see them; and while, to ourselves, everything connected with our
contemplated escape appeared concealed, Mr. Freeland may have, with the
peculiar prescience of a slaveholder, mastered the huge thought which
was disturbing our peace in slavery.

I am the more inclined to think that he suspected us, because, prudent
as we were, as I now look back, I can see that we did many silly things,
very well calculated to awaken suspicion. We were, at times, remarkably
buoyant, singing hymns and making joyous exclamations, almost as
triumphant in their tone as if we had reached a land of freedom and
safety. A keen observer might have detected in our repeated singing of

"O Canaan, sweet Canaan,\\
I am bound for the land of Canaan,"

something more than a hope of reaching heaven. We meant to reach the
\emph{north}---and the north was our Canaan

"I thought I heard them say,\\
There were lions in the way,\\
I don't expect to stay\\
{}Much longer here.\\
{}Run to Jesus---shun the danger---\\
I don't expect to stay\\
{}Much longer here,"\\

was a favorite air, and had a double meaning. In the lips of some, it
meant the expectation of a speedy summons to a world of spirits; but, in
the lips of \emph{our} company, it simply meant, a speedy pilgrimage
toward a free state, and deliverance from all the evils and dangers of
slavery.

I had succeeded in winning to my (what slaveholders would call wicked)
scheme, a company of five young men, the very flower of the
neighborhood, each one of whom would have commanded one thousand dollars
in the home market. At New Orleans, they would have brought fifteen
hundred dollars a piece, and, perhaps, more. The names of our party were
as follows: Henry Harris; John Harris, brother to Henry; Sandy Jenkins,
of root memory; Charles Roberts, and Henry Bailey. I was the youngest,
but one, of the party. I had, however, the advantage of them all, in
experience, and in a knowledge of letters. This gave me great influence
over them. Perhaps not one of them, left to himself, would have dreamed
of escape as a possible thing. Not one of them was self-moved in the
matter. They all wanted to be free; but the serious thought of running
away, had not entered into their minds, until I won them to the
undertaking. They all were tolerably well off---for slaves---and had dim
hopes of being set free, some day, by their masters. If any one is to
blame for disturbing the quiet of the slaves and slave-masters of the
neighborhood of St. Michael's, \emph{I am the man}. {}I claim to be the
instigator of the high crime, (as the slaveholders regard it,) and I
kept life in it, until life could be kept in it no longer.

Pending the time of our contemplated departure out of our Egypt, we met
often by night, and on every Sunday. At these meetings we talked the
matter over; told our hopes and fears, and the difficulties discovered
or imagined; and, like men of sense, we counted the cost of the
enterprise to which we were committing ourselves.

These meetings must have resembled, on a small scale, the meetings of
revolutionary conspirators, in their primary condition. We were plotting
against our (so called) lawful rulers; with this difference---that we
sought our own good, and not the harm of our enemies. We did not seek to
overthrow them, but to escape from them. As for Mr. Freeland, we all
liked him, and would have gladly remained with him, \emph{as freemen}.
\textsc{Liberty} was our aim; and we had now come to think that we had a
right to liberty, against every obstacle---even against the lives of our
enslavers.

We had several words, expressive of things, important to us, which we
understood, but which, even if distinctly heard by an outsider, would
convey no certain meaning. I have reasons for suppressing these
\emph{pass-words,} which the reader will easily divine. I hated the
secrecy; but where slavery is powerful, and liberty is weak, the latter
is driven to concealment or to destruction.

The prospect was not always a bright one. At times, we were almost
tempted to abandon the enterprise, and to get back to that comparative
peace of {}mind, which even a man under the gallows might feel, when all
hope of escape had vanished. Quiet bondage was felt to be better than
the doubts, fears and uncertainties, which now so sadly perplexed and
disturbed us.

The infirmities of humanity, generally, were represented in our little
band. We were confident, bold and determined, at times; and, again,
doubting, timid and wavering; whistling, like the boy in the graveyard,
to keep away the spirits.

To look at the map, and observe the proximity of Eastern Shore,
Maryland, to Delaware and Pennsylvania, it may seem to the reader quite
absurd, to regard the proposed escape as a formidable undertaking. But
to \emph{understand}, some one has said a man must \emph{stand under}.
The real distance was great enough, but the imagined distance was, to
our ignorance, even greater. Every slaveholder seeks to impress his
slave with a belief in the boundlessness of slave territory, and of his
own almost illimitable power. We all had vague and indistinct notions of
the geography of the country.

The distance, however, is not the chief trouble. The nearer are the
lines of a slave state and the borders of a free one, the greater the
peril. Hired kidnappers infest these borders. Then, too, we knew that
merely reaching a free state did not free us; that, wherever caught, we
could be returned to slavery. We could see no spot on this side the
ocean, where we could be free. We had heard of Canada, the real Canaan
of the American bondmen, simply as a country to which the wild goose and
the swan repaired at {}the end of winter, to escape the heat of summer,
but not as the home of man. I knew something of theology, but nothing of
geography. I really did not, at that time, know that there was a state
of New York, or a state of Massachusetts. I had heard of Pennsylvania,
Delaware and New Jersey, and all the southern states, but was ignorant
of the free states, generally. New York city was our northern limit, and
to go there, and to be forever harassed with the liability of being
hunted down and returned to slavery---with the certainty of being
treated ten times worse than we had ever been treated before---was a
prospect far from delightful, and it might well cause some hesitation
about engaging in the enterprise. The case, sometimes, to our excited
visions, stood thus: At every gate through which we had to pass, we saw
a watchman; at every ferry, a guard; on every bridge, a sentinel; and in
every wood, a patrol or slave-hunter. We were hemmed in on every side.
The good to be sought, and the evil to be shunned, were flung in the
balance, and weighed against each other. On the one hand, there stood
slavery; a stern reality, glaring frightfully upon us, with the blood of
millions in his polluted skirts---terrible to behold---greedily
devouring our hard earnings and feeding himself upon our flesh. Here was
the evil from which to escape. On the other hand, far away, back in the
hazy distance, where all forms seemed but shadows, under the flickering
light of the north star---behind some craggy hill or snow-covered
mountain---stood a doubtful freedom, half frozen, beckoning us to her
icy domain. This was, the good to be sought. The inequality was as
{}great as that between certainty and uncertainty. This, in itself, was
enough to stagger us; but when we came to survey the untrodden road, and
conjecture the many possible difficulties, we were appalled, and at
times, as I have said, were upon the point of giving over the struggle
altogether.

The reader can have little idea of the phantoms of trouble which flit,
in such circumstances, before the uneducated mind of the slave. Upon
either side, we saw grim death assuming a variety of horrid shapes. Now,
it was starvation, causing us, in a strange and friendless land, to eat
our own flesh. Now, we were contending with the waves, (for our journey
was in part by water,) and were drowned. Now, we were hunted by dogs,
and overtaken and torn to pieces by their merciless fangs. We were stung
by scorpions---chased by wild beasts---bitten by snakes; and, worst of
all, after having succeeded in swimming rivers---encountering wild
beasts---sleeping in the woods---suffering hunger, cold, heat and
nakedness---we supposed ourselves to be overtaken by hired kidnappers,
who, in the name of the law, and for their thrice accursed reward,
would, perchance, fire upon us---kill some, wound others, and capture
all. This dark picture, drawn by ignorance and fear, at times greatly
shook our determination, and not unfrequently caused us to

{"}Rather bear those ills we had\\
Than fly to others which we knew not of."

I am not disposed to magnify this circumstance in my experience, and yet
I think I shall seem to be so {}disposed, to the reader. No man can tell
the intense agony which is felt by the slave, when wavering on the point
of making his escape. All that he has is at stake; and even that which
he has not, is at stake, also. The life which he has, may be lost, and
the liberty which he seeks, may not be gained.

Patrick Henry, to a listening senate, thrilled by his magic eloquence,
and ready to stand by him in his boldest flights, could say,
"\textsc{Give me Liberty or give me Death}," and this saying was a
sublime one, even for a freeman; but, incomparably more sublime, is the
same sentiment, when \emph{practically} asserted by men accustomed to
the lash and chain---men whose sensibilities must have become more or
less deadened by their bondage. With us it was a \emph{doubtful}
liberty, at best, that we sought; and a certain, lingering death in the
rice swamps and sugar fields, if we failed. Life is not lightly regarded
by men of sane minds. It is precious, alike to the pauper and to the
prince---to the slave, and to his master; and yet, I believe there was
not one among us, who would not rather have been shot down, than pass
away life in hopeless bondage.

In the progress of our preparations, Sandy, the root man, became
troubled. He began to have dreams, and some of them were very
distressing. One of these, which happened on a Friday night, was, to
him, of great significance; and I am quite ready to confess, that I felt
somewhat damped by it myself. He said, "I dreamed, last night, that I
was roused from sleep, by strange noises, like the voices of a swarm of
angry birds, that caused a roar as they passed, which fell {}upon my ear
like a coming gale over the tops of the trees. Looking up to see what it
could mean," said Sandy, ``I saw you, Frederick, in the claws of a huge
bird, surrounded by a large number of birds, of all colors and sizes.
These were all picking at you, while you, with your arms, seemed to be
trying to protect your eyes. Passing over me, the birds flew in a
south-westerly direction, and I watched them until they were clean out
of sight. Now, I saw this as plainly as I now see you; and furder,
honey, watch de Friday night dream; dare is sumpon in it, shose you
born; dare is, indeed, honey.''

I confess I did not like this dream; but I threw off concern about it,
by attributing it to the general excitement and perturbation consequent
upon our contemplated plan of escape. I could not, however, shake off
its effect at once. I felt that it boded me no good. Sandy was unusually
emphatic and oracular, and his manner had much to do with the impression
made upon me.

The plan of escape which I recommended, and to which my comrades
assented, was to take a large canoe, owned by Mr. Hamilton, and, on the
Saturday night previous to the Easter holidays, launch out into the
Chesapeake bay, and paddle for its head,---a distance of seventy
miles---with all our might. Our course, on reaching this point, was, to
turn the canoe adrift, and bend our steps toward the north star, till we
reached a free state.

There were several objections to this plan. One was, the danger from
gales on the bay. In rough weather, the waters of the Chesapeake are
much {}agitated, and there is danger, in a canoe, of being swamped by
the waves. Another objection was, that the canoe would soon be missed;
the absent persons would, at once, be suspected of having taken it; and
we should be pursued by some of the fast sailing bay craft out of St.
Michael's. Then, again, if we reached the head of the bay, and turned
the canoe adrift, she might prove a guide to our track, and bring the
land hunters after us.

These and other objections were set aside, by the stronger ones which
could be urged against every other plan that could then be suggested. On
the water, we had a chance of being regarded as fishermen, in the
service of a master. On the other hand, by taking the land route,
through the counties adjoining Delaware, we should be subjected to all
manner of interruptions, and many very disagreeable questions, which
might give us serious trouble. Any white man is authorized to stop a man
of color, on any road, and examine him, and arrest him, if he so
desires.

By this arrangement, many abuses (considered such even by slaveholders)
occur. Cases have been known, where freemen have been called upon to
show their free papers, by a pack of ruffians---and, on the presentation
of the papers, the ruffians have torn them up, and seized their victim,
and sold him to a life of endless bondage.

The week before our intended start, I wrote a pass for each of our
party, giving them permission to visit Baltimore, during the Easter
holidays. The pass ran after this manner: {}

"This is to certify, that I, the undersigned, have given the bearer, my
servant, John, full liberty to go to Baltimore, to spend the Easter
holidays.

"W. H.

{``Near St. Michael's, Talbot county, Maryland.''}

Although we were not going to Baltimore, and were intending to land east
of North Point, in the direction where I had seen the Philadelphia
steamers go, these passes might be made useful to us in the lower part
of the bay, while steering toward Baltimore. These were not, however, to
be shown by us, until all other answers failed to satisfy the inquirer.
We were all fully alive to the importance of being calm and
self-possessed, when accosted, if accosted we should be; and we more
times than one rehearsed to each other how we should behave in the hour
of trial.

Those were long, tedious days and nights. The suspense was painful, in
the extreme. To balance probabilities, where life and liberty hang on
the result, requires steady nerves. I panted for action, and was glad
when the day, at the close of which we were to start, dawned upon us.
Sleeping, the night before, was out of the question. I probably felt
more deeply than any of my companions, because I was the instigator of
the movement. The responsibility of the whole enterprise rested on my
shoulders. The glory of success, and the shame and confusion of failure,
could not be matters of indifference to me. Our food was prepared; our
clothes were packed up; we were all ready to go, and impatient for
Saturday morning---considering that the last morning of our bondage.

{}I cannot describe the tempest and tumult of my brain, that morning.
The reader will please to bear in mind, that, in a slave state, an
unsuccessful runaway is not only subjected to cruel torture, and sold
away to the far south, but he is frequently execrated by the other
slaves. He is charged with making the condition of the other slaves
intolerable, by laying them all under the suspicion of their
masters---subjecting them to greater vigilance, and imposing greater
limitations on their privileges. I dreaded murmurs from this quarter. It
is difficult, too, for a slave-master to believe that slaves escaping
have not been aided in their flight by some one of their fellow slaves.
When, therefore, a slave is missing, every slave on the place is closely
examined as to his knowledge of the undertaking; and they are sometimes
even tortured, to make them disclose what they are suspected of knowing
of such escape.

Our anxiety grew more and more intense, as the time of our intended
departure for the north drew nigh. It was truly felt to be a matter of
life and death with us; and we fully intended to \emph{fight} as well as
\emph{run}, if necessity should occur for that extremity. But the trial
hour was not yet come. It was easy to resolve, but not so easy to act. I
expected there might be some drawing back, at the last. It was natural
that there should be; therefore, during the intervening time, I lost no
opportunity to explain away difficulties, to remove doubts, to dispel
fears, and to inspire all with firmness. It was too late to look back;
and \emph{now} was the time to go forward. Like most other men, we had
done the talking part {}of our work, long and well; and the time had
come to \emph{act} as if we were in earnest, and meant to be as true in
action as in words. I did not forget to appeal to the pride of my
comrades, by telling them that, if after having solemnly promised to go,
as they had done, they now failed to make the attempt, they would, in
effect, brand themselves with cowardice, and might as well sit down,
fold their arms, and acknowledge themselves as fit only to be
\emph{slaves}. This detestable character, all were unwilling to assume.
Every man except Sandy (he, much to our regret, withdrew) stood firm;
and at our last meeting we pledged ourselves afresh, and in the most
solemn manner, that, at the time appointed, we \emph{would} certainly
start on our long journey for a free country. This meeting was in the
middle of the week, at the end of which we were to start.

Early that morning we went, as usual, to the field, but with hearts that
beat quickly and anxiously. Any one intimately acquainted with us, might
have seen that all was not well with us, and that some monster lingered
in our thoughts. Our work that morning was the same as it had been for
several days past---drawing out and spreading manure. While thus
engaged, I had a sudden presentiment, which flashed upon me like
lightning in a dark night, revealing to the lonely traveler the gulf
before, and the enemy behind. I instantly turned to Sandy Jenkins, who
was near me, and said to him, "\emph{Sandy, we are betrayed;} something
has just told me so." I felt as sure of it, as if the officers were
there in sight. Sandy said, ``Man, dat is strange; but I feel just as
you do.'' If my mother---then long in her grave---had appeared {}before
me, and told me that we were betrayed, I could not, at that moment, have
felt more certain of the fact.

In a few minutes after this, the long, low and distant notes of the horn
summoned us from the field to breakfast. I felt as one may be supposed
to feel before being led forth to be executed for some great offense. I
wanted no breakfast; but I went with the other slaves toward the house,
for form's sake. My feelings were not disturbed as to the right of
running away; on that point I had no trouble, whatever. My anxiety arose
from a sense of the consequences of failure.

In thirty minutes after that vivid presentiment, came the apprehended
crash. On reaching the house, for breakfast, and glancing my eye toward
the lane gate, the worst was at once made known. The lane gate of Mr.
Freeland's house, is nearly a half a mile from the door, and much shaded
by the heavy wood which bordered the main road. I was, however, able to
descry four white men, and two colored men, approaching. The white men
were on horseback, and the colored men were walking behind, and seemed
to be tied. "\emph{It is all over with us,}" thought I, "\emph{we are
surely betrayed.}" I now became composed, or at least comparatively so,
and calmly awaited the result. I watched the ill-omened company, till I
saw them enter the gate. Successful flight was impossible, and I made up
my mind to stand, and meet the evil, whatever it might be; for I was now
not without a slight hope that things might turn differently from what I
at first expected. In a few moments, in {}came Mr. William Hamilton,
riding very rapidly, and evidently much excited. He was in the habit of
riding very slowly, and was seldom known to gallop his horse. This time,
his horse was nearly at full speed, causing the dust to roll thick
behind him. Mr. Hamilton, though one of the most resolute men in the
whole neighborhood, was, nevertheless, a remarkably mild spoken man;
and, even when greatly excited, his language was cool and circumspect.
He came to the door, and inquired if Mr. Freeland was in. I told him
that Mr. Freeland was at the barn. Off the old gentleman rode, toward
the barn, with unwonted speed. Mary, the cook, was at a loss to know
what was the matter, and I did not profess any skill in making her
understand. I knew she would have united, as readily as any one, in
cursing me for bringing trouble into the family; so I held my peace,
leaving matters to develop themselves, without my assistance. In a few
moments, Mr. Hamilton and Mr. Freeland came down from the barn to the
house; and, just as they made their appearance in the front yard, three
men (who proved to be constables) came dashing into the lane, on
horseback, as if summoned by a sign requiring quick work. A few seconds
brought them into the front yard, where they hastily dismounted, and
tied their horses. This done, they joined Mr. Freeland and Mr. Hamilton,
who were standing a short distance from the kitchen. A few moments were
spent, as if in consulting how to proceed, and then the whole party
walked up to the kitchen door. There was now no one in the kitchen but
myself and John Harris. Henry and Sandy were yet at the barn. {}Mr.
Freeland came inside the kitchen door, and with an agitated voice,
called me by name, and told me to come forward; that there were some
gentlemen who wished to see me. I stepped toward them, at the door, and
asked what they wanted, when the constables grabbed me, and told me that
I had better not resist; that I had been in a scrape, or was said to
have been in one; that they were merely going to take me where I could
be examined; that they were going to carry me to St. Michael's, to have
me brought before my master. They further said, that, in case the
evidence against me was not true, I should be acquitted. I was now
firmly tied, and completely at the mercy of my captors. Resistance was
idle. They were five in number, armed to the very teeth. When they had
secured me, they next turned to John Harris, and, in a few moments,
succeeded in tying him as firmly as they had already tied me. They next
turned toward Henry Harris, who had now returned from the barn. ``Cross
your hands,'' said the constables, to Henry. ``I won't'' said Henry, in
a voice so firm and clear, and in a manner so determined, as for a
moment to arrest all proceedings. ``Won't you cross your hands?'' said
Tom Graham, the constable. "\emph{No I won't,}" said Henry, with
increasing emphasis. Mr. Hamilton, Mr. Freeland, and the officers, now
came near to Henry. Two of the constables drew out their shining
pistols, and swore by the name of God, that should cross his hands, or
they would shoot him down. Each of these hired ruffians now cocked their
pistols, and, with fingers apparently on the triggers, presented their
deadly weapons to the breast of the unarmed {}slave, saying, at the same
time, if he did not cross his hands, they would ``blow his d---d heart
out of him.''

"\emph{Shoot! shoot me!}" said Henry. "\emph{You can't kill me but
once}. Shoot!---shoot! and be d---d. \emph{I won't be tied}" This, the
brave fellow said in a voice as defiant and heroic in its tone, as was
the language itself; and, at the moment of saying this, with the pistols
at his very breast, he quickly raised his arms, and dashed them from the
puny hands of his assassins, the weapons flying in opposite directions.
Now came the struggle. All hands now rushed upon the brave fellow, and,
after beating him for some time, they succeeded in overpowering and
tying him. Henry put me to shame; he fought, and fought bravely. John
and I had made no resistance. The fact is, I never see much use in
fighting, unless there is a reasonable probability of whipping somebody.
Yet there was something almost providential in the resistance made by
the gallant Henry. But for that resistance, every soul of us would have
been hurried off to the far south. Just a moment previous to the trouble
with Henry, Mr. Hamilton \emph{mildly} said and this gave me the
unmistakable clue to the cause of our arrest---``Perhaps we had now
better make a search for those protections, which we understand
Frederick has written for himself and the rest.'' Had these passes been
found, they would have been point blank proof against us, and would have
confirmed all the statements of our betrayer. Thanks to the resistance
of Henry, the excitement produced by the scuffle drew all attention in
that direction, and I succeeded in flinging my pass, unobserved, into
the fire. {}The confusion attendant upon the scuffle, and the
apprehension of further trouble, perhaps, led our captors to forego, for
the present, any search for "\emph{those protections}" \emph{which
Frederick was said to have written for his companions;} so we were not
yet convicted of the purpose to run away; and it was evident that there
was some doubt, on the part of all, whether we had been guilty of such a
purpose.

Just as we were all completely tied, and about ready to start toward St.
Michael's, and thence to jail, Mrs. Betsey Freeland (mother to William,
who was very much attached---after the southern fashion to Henry and
John, they having been reared from childhood in her house) came to the
kitchen door, with her hands full of biscuits,---for we had not had time
to take our breakfast that morning---and divided them between Henry and
John. This done, the lady made the following parting address to me,
looking and pointing her bony finger at me. "You devil! you yellow
devil! It was you that put it into the heads of Henry and John to run
away. But for \emph{you}, you \emph{long legged yellow devil}, Henry and
John would never have thought of running away." I gave the lady a look,
which called forth a scream of mingled wrath and terror, as she slammed
the kitchen door, and went in, leaving me, with the rest, in hands as
harsh as her own broken voice.

Could the kind reader have been quietly riding along the main road to or
from Easton, that morning, his eye would have met a painful sight. He
would have seen five young men, guilty of no crime, save that of
preferring \emph{liberty} to a life of \emph{bondage}, drawn {}along the
public highway---firmly bound together---tramping through dust and heat,
bare-footed and bare-headed fastened to three strong horses, whose
riders were armed to the teeth, with pistols and daggers---on their way
to prison, like felons, and suffering every possible insult from the
crowds of idle, vulgar people, who clustered around, and heartlessly
made their failure the occasion for all manner of ribaldry and sport. As
I looked upon this crowd of vile persons, and saw myself and friends
thus assailed and persecuted, I could not help seeing the fulfillment of
Sandy's dream. I was in the hands of moral vultures, and firmly held in
their sharp talons, and was being hurried away toward Easton, in a
south-easterly direction, amid the jeers of new birds of the same
feather, through every neighborhood we passed. It seemed to me, (and
this shows the good understanding between the slaveholders and their
allies,) that every body we met knew the cause of our arrest, and were
out, awaiting our passing by, to feast their vindictive eyes on our
misery and to gloat over our ruin. Some said, \emph{I ought to be
hanged,} and others, \emph{I ought to be burnt;} others, I ought to have
the "\emph{hide}" taken from my back; while no one gave us a kind word
or sympathizing look, except the poor slaves, who were lifting their
heavy hoes, and who cautiously glanced at us through the post-and-rail
fences, behind which they were at work. Our sufferings, that morning,
can be more easily imagined than described. Our hopes were all blasted,
at a blow. The cruel injustice, the victorious crime, and the
helplessness of innocence, led me to ask, in my ignorance and
weakness---"Where now is the God {}of justice and mercy? and why have
these wicked men the power thus to trample upon our rights, and to
insult our feelings?" And yet, in the next moment, came the consoling
thought, "\emph{the day of the oppressor will come at last}" Of one
thing I could be glad---not one of my dear friends, upon whom I had
brought this great calamity, either by word or look, reproached me for
having led them into it. We were a band of brothers, and never dearer to
each other than now. The thought which gave us the most pain, was the
probable separation which would now take place, in case we were sold off
to the far south, as we were likely to be. While the constables were
looking forward, Henry and I, being fastened together, could
occasionally exchange a word, without being observed by the kidnappers
who had us in charge. ``What shall I do with my pass?'' said Henry.
``Eat it with your biscuit,'' said I; ``it won't do to tear it up.'' We
were now near St. Michael's. The direction concerning the passes was
passed around, and executed. "\emph{Own nothing!}" said I. "\emph{Own
nothing!}" was passed around and enjoined, and assented to. Our
confidence in each other was unshaken; and we were quite resolved to
succeed or fail together---as much after the calamity which had befallen
us, as before.

On reaching St. Michael's, we underwent a sort of examination at my
master's store, and it was evident to my mind, that Master Thomas
suspected the truthfulness of the evidence upon which they had acted in
arresting us; and that he only affected, to some extent, the
positiveness with which he asserted our guilt. There was nothing said by
any of our company, which {}could, in any manner, prejudice our cause;
and there was hope, yet, that we should be able to return to our
homes---if for nothing else, at least to find out the guilty man or
woman who had betrayed us.

To this end, we all denied that we had been guilty of intended night.
Master Thomas said that the evidence he had of our intention to run
away, was strong enough to hang us, in a case of murder. ``But,'' said
I, ``the cases are not equal. If murder were committed, some one must
have committed it---the thing is done! In our case, nothing has been
done! We have not run away. Where is the evidence against us? We were
quietly at our work.'' I talked thus, with unusual freedom, to bring out
the evidence against us, for we all wanted, above all things, to know
the guilty wretch who had betrayed us, that we might have something
tangible upon which to pour our execrations. From something which
dropped, in the course of the talk, it appeared that there was but one
witness against us and that that witness could not be produced. Master
Thomas would not tell us who his informant was; but we suspected, and
suspected \emph{one} person \emph{only}. Several circumstances seemed to
point \textsc{Sandy} out, as our betrayer. His entire knowledge of our
plans---his participation in them---his withdrawal from us---his dream,
and his simultaneous presentiment that we were betrayed---the taking us,
and the leaving him---were calculated to turn suspicion toward him; and
yet, we could not suspect him. We all loved him too well to think it
\emph{possible} that he could have betrayed us. So we rolled the guilt
on other shoulders.

{}We were literally dragged, that morning, behind horses, a distance of
fifteen miles, and placed in the Easton jail. We were glad to reach the
end of our journey, for our pathway had been the scene of insult and
mortification. Such is the power of public opinion, that it is hard,
even for the innocent, to feel the happy consolations of innocence, when
they fall under the maledictions of this power. How could we regard
ourselves as in the right, when all about us denounced us as criminals,
and had the power and the disposition to treat us as such.

In jail, we were placed under the care of Mr. Joseph Graham, the sheriff
of the county. Henry, and John, and myself, were placed in one room, and
Henry Baily and Charles Roberts, in another, by themselves. This
separation was intended to deprive us of the advantage of concert, and
to prevent trouble in jail.

Once shut up, a new set of tormentors came upon us. A swarm of imps, in
human shape---the slave-traders, deputy slave-traders, and agents of
slave-traders---that gather in every country town of the state, watching
for chances to buy human flesh, (as buzzards to eat carrion,) flocked in
upon us, to ascertain if our masters had placed us in jail to be sold.
Such a set of debased and villainous creatures, I never saw before, and
hope never to see again. I felt myself surrounded as by a pack of
\emph{fiends}, fresh from \emph{perdition}. They laughed, leered, and
grinned at us; saying, ``Ah! boys, we've got you, havn't we? So you were
about to make your escape? Where were you going to?'' After taunting us,
and jeering at us, {}as long as they liked, they one by one subjected us
to an examination, with a view to ascertain our value; feeling our arms
and legs, and shaking us by the shoulders to see if we were sound and
healthy; impudently asking us, ``how we would like to have them for
masters?'' To such questions, we were, very much to their annoyance,
quite dumb, disdaining to answer them. For one, I detested the
whisky-bloated gamblers in human flesh; and I believe I was as much
detested by them in turn. One fellow told me, ``if he had me, he would
cut the devil out of me pretty quick.''

These negro buyers are very offensive to the genteel southron christian
public. They are looked upon, in respectable Maryland society, as
necessary, but detestable characters. As a class, they are hardened
ruffians, made such by nature and by occupation. Their ears are made
quite familiar with the agonizing cry of outraged and woe-smitten
humanity. Their eyes are forever open to human misery. They walk amid
desecrated affections, insulted virtue, and blasted hopes. They have
grown intimate with vice and blood; they gloat over the wildest
illustrations of their soul-damning and earth-polluting business, and
are moral pests. Yes; they are a legitimate fruit of slavery; and it is
a puzzle to make out a case of greater villainy for them, than for the
slaveholders, who make such a class \emph{possible}. They are mere
hucksters of the surplus slave produce of Maryland and
Virginia---coarse, cruel, and swaggering bullies, whose very breathing
is of blasphemy and blood.

Aside from these slave-buyers, who infested the {}prison, from time to
time, our quarters were much more comfortable than we had any right to
expect they would be. Our allowance of food was small and coarse, but
our room was the best in the jail---neat and spacious, and with nothing
about it necessarily reminding us of being in prison, but its heavy
locks and bolts and the black, iron lattice-work at the windows. We were
prisoners of state, compared with most slaves who are put into that
Easton jail. But the place was not one of contentment. Bolts, bars and
grated windows are not acceptable to freedom-loving people of any color.
The suspense, too, was painful. Every step on the stairway was listened
to, in the hope that the comer would cast a ray of light on our fate. We
would have given the hair off our heads for half a dozen words with one
of the waiters in Sol. Lowe's hotel. Such waiters were in the way of
hearing, at the table, the probable course of things. We could see them
flitting about in their white jackets, in front of this hotel, but could
speak to none of them.

Soon after the holidays were over, contrary to all our expectations,
Messrs. Hamilton and Freeland came up to Easton; not to make a bargain
with the ``Georgia traders,'' nor to send us up to Austin Woldfolk, as
is usual in the case of run-away slaves, but to release Charles, Henry
Harris, Henry Baily and John Harris, from prison, and this, too, without
the infliction of a single blow. I was now left entirely alone in
prison. The innocent had been taken, and the guilty left. My friends
were separated from me, and apparently forever. This circumstance caused
me {}more pain than any other incident connected with our capture and
imprisonment. Thirty-nine lashes on my naked and bleeding back, would
have been joyfully borne, in preference to this separation from these,
the friends of my youth. And yet, I could not but feel that I was the
victim of something like justice. Why should these young men, who were
led into this scheme by me, suffer as much as the instigator? I felt
glad that they were released from prison, and from the dread prospect of
a life (or death I should rather say) in the rice swamps. It is due to
the noble Henry, to say, that he seemed almost as reluctant to leave the
prison with me in it, as he was to be tied and dragged to prison. But he
and the rest knew that we should, in all the likelihoods of the case, be
separated, in the event of being sold; and since we were now completely
in the hands of our owners, we all concluded it would be best to go
peaceably home.

Not until this last separation, dear reader, had I touched those
profounder depths of desolation, which it is the lot of slaves often to
reach. I was solitary in the world, and alone within the walls of a
stone prison, left to a fate of life-long misery. I had hoped and
expected much, for months before, but my hopes and expectations were now
withered and blasted. The ever dreaded slave life in Georgia, Louisiana
and Alabama---from which escape is next to impossible---now, in my
loneliness, stared me in the face. The possibility of ever becoming
anything but an abject slave, a mere machine in the hands of an owner,
had now fled, and it seemed to me it had fled forever. A life of living
death, beset with the innumerable {}horrors of the cotton field, and the
sugar plantation, seemed to be my doom. The fiends, who rushed into the
prison when we were first put there, continued to visit me, and to ply
me with questions and with their tantalizing remarks. I was insulted,
but helpless; keenly alive to the demands of justice and liberty, but
with no means of asserting them. To talk to those imps about justice and
mercy, would have been as absurd as to reason with bears and tigers.
Lead and steel are the only arguments that they understand.

After remaining in this life of misery and despair about a week, which,
by the way, seemed a month, Master Thomas, very much to my surprise, and
greatly to my relief, came to the prison, and took me out, for the
purpose, as he said, of sending me to Alabama, with a friend of his, who
would emancipate me at the end of eight years. I was glad enough to get
out of prison; but I had no faith in the story that this friend of Capt.
Auld would emancipate me, at the end of the time indicated. Besides, I
never had heard of his having a friend in Alabama, and I took the
announcement, simply as an easy and comfortable method of shipping me
off to the far south. There was a little scandal, too, connected with
the idea of one christian selling another to the Georgia traders, while
it was deemed every way proper for them to sell to others. I thought
this friend in Alabama was an invention, to meet this difficulty, for
Master Thomas was quite jealous of his christian reputation, however
unconcerned he might be about his real christian character. In these
remarks, however, it is possible that I do {}Master Thomas Auld
injustice. He certainly did not exhaust his power upon me, in the case,
but acted, upon the whole, very generously, considering the nature of my
offense. He had the power and the provocation to send me, without
reserve, into the very everglades of Florida, beyond the remotest hope
of emancipation; and his refusal to exercise that power, must be set
down to his credit.

After lingering about St. Michael's a few days, and no friend from
Alabama making his appearance, to take me there, Master Thomas decided
to send me back again to Baltimore, to live with his brother Hugh, with
whom he was now at peace; possibly he became so by his profession of
religion, at the camp-meeting in the Bay Side. Master Thomas told me
that he wished me to go to Baltimore, and learn a trade; and that, if I
behaved myself properly, he would \emph{emancipate me at twenty-five!}
Thanks for this one beam of hope in the future. The promise had but one
fault; it seemed too good to be true.
