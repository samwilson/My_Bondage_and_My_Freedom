{\protect\hypertarget{233}{}{}}

~

{CHAPTER XVII.}

THE LAST FLOGGING.

{A SLEEPLESS NIGHT---RETURN TO COVEY'S---PURSUED BY COVEY---THE CHASE
DEFEATED---VENGEANCE POSTPONED---MUSINGS IN THE WOODS---THE
ALTERNATIVE---DEPLORABLE SPECTACLE---NIGHT IN THE WOODS---EXPECTED
ATTACK---ACCOSTED BY SANDY, A FRIEND, NOT A HUNTER---SANDY'S
HOSPITALITY---THE ``ASH CAKE'' SUPPER---THE INTERVIEW WITH SANDY---HIS
ADVICE---SANDY A CONJURER AS WELL AS A CHRISTIAN---THE MAGIC
ROOT---STRANGE MEETING WITH COVEY---HIS MANNER---COVEY'S SUNDAY
FACE---AUTHOR'S DEFENSIVE RESOLVE---THE FIGHT---THE VICTORY, AND ITS
RESULTS.}

\textsc{Sleep} itself does not always come to the relief of the weary in
body, and the broken in spirit; especially when past troubles only
foreshadow coming disasters. The last hope had been extinguished. My
master, who I did not venture to hope would protect me as \emph{a man},
had even now refused to protect me as \emph{his property;} and had cast
me back, covered with reproaches and bruises, into the hands of a
stranger to that mercy which was the soul of the religion he professed.
May the reader never spend such a night as that allotted to me, previous
to the morning which was to herald my return to the den of horrors from
which I had made a temporary escape.

I remained all night---sleep I did not---at St. Michael's; and in the
morning (Saturday) I started off, according to the order of Master
Thomas, feeling that {\protect\hypertarget{234}{}{}}I had no friend on
earth, and doubting if I had one in heaven. I reached Covey's about nine
o'clock; and just as I stepped into the field, before I had reached the
house, Covey, true to his snakish habits, darted out at me from a fence
corner, in which he had secreted himself, for the purpose of securing
me. He was amply provided with a cowskin and a rope; and he evidently
intended to \emph{tie me up}, and to wreak his vengeance on me to the
fullest extent. I should have been an easy prey, had he succeeded in
getting his hands upon me, for I had taken no refreshment since noon on
Friday; and this, together with the pelting, excitement, and the loss of
blood, had reduced my strength. I, however, darted back into the woods,
before the ferocious hound could get hold of me, and buried myself in a
thicket, where he lost sight of me. The corn-field afforded me cover, in
getting to the woods. But for the tall corn, Covey would have overtaken
me, and made me his captive. He seemed very much chagrined that he did
not catch me, and gave up the chase, very reluctantly; for I could see
his angry movements, toward the house from which he had sallied, on his
foray.

Well, now I am clear of Covey, and of his wrathful lash, for the
present. I am in the wood, buried in its somber gloom, and hushed in its
solemn silence; hid from all human eyes; shut in with nature and
nature's God, and absent from all human contrivances. Here was a good
place to pray; to pray for help for deliverance---a prayer I had often
made before. But how could I pray? Covey could pray---Capt. Auld could
pray---I would fain pray; but doubts (arising
{\protect\hypertarget{235}{}{}}partly from my own neglect of the means
of grace, and partly from the sham religion which everywhere prevailed,
cast in my mind a doubt upon all religion, and led me to the conviction
that prayers were unavailing and delusive) prevented my embracing the
opportunity, as a religious one. Life, in itself, had almost become
burdensome to me. All my outward relations were against me; I must stay
here and starve, (I was already hungry,) or go home to Covey's, and have
my flesh torn to pieces, and my spirit humbled under the cruel lash of
Covey. This was the painful alternative presented to me. The day was
long and irksome. My physical condition was deplorable. I was weak, from
the toils of the previous day, and from the want of food and rest; and
had been so little concerned about my appearance, that I had not yet
washed the blood from my garments. I was an object of horror, even to
myself. Life, in Baltimore, when most oppressive, was a paradise to
this. What had I done, what had my parents done, that such a life as
this should be mine? That day, in the woods, I would have exchanged my
manhood for the brutehood of an ox.

Night came. I was still in the woods, unresolved what to do. Hunger had
not yet pinched me to the point of going home, and I laid myself down in
the leaves to rest; for I had been watching for hunters all day, but not
being molested during the day, I expected no disturbance during the
night. I had come to the conclusion that Covey relied upon hunger to
drive me home; and in this I was quite correct---the
{\protect\hypertarget{236}{}{}}facts showed that he had made no effort
to catch me, since morning.

During the night, I heard the step of a man in the woods. He was coming
toward the place where I lay. A person lying still has the advantage
over one walking in the woods, in the day time, and this advantage is
much greater at night. I was not able to engage in a physical struggle,
and I had recourse to the common resort of the weak. I hid myself in the
leaves to prevent discovery. But, as the night rambler in the woods drew
nearer, I found him to be a \emph{friend}, not an enemy; it was a slave
of Mr. William Groomes, of Easton, a kind hearted fellow, named
``Sandy.'' Sandy lived with Mr. Kemp that year, about four miles from
St. Michael's. He, like myself, had been hired out by the year; but,
unlike myself, had not been hired out to be broken. Sandy was the
husband of a free woman, who lived in the lower part of "\emph{Pot-pie
Neck}," and he was now on his way through the woods, to see her, and to
spend the Sabbath with her.

As soon as I had ascertained that the disturber of my solitude was not
an enemy, but the good-hearted Sandy---a man as famous among the slaves
of the neighborhood for his good nature, as for his good sense---I came
out from my hiding place, and made myself known to him. I explained the
circumstances of the past two days, which had driven me to the woods,
and he deeply compassionated my distress. It was a bold thing for him to
shelter me, and I could not ask him to do so; for, had I been found in
his hut, he would have suffered the penalty of thirty-nine lashes on his
bare back, if not something worse. But,
{\protect\hypertarget{237}{}{}}Sandy was too generous to permit the fear
of punishment to prevent his relieving a brother bondman from hunger and
exposure; and, therefore, on his own motion, I accompanied him to his
home, or rather to the home of his wife---for the house and lot were
hers. His wife was called up---for it was now about midnight---a fire
was made, some Indian meal was soon mixed with salt and water, and an
ash cake was baked in a hurry to relieve my hunger. Sandy's wife was not
behind him in kindness---both seemed to esteem it a privilege to succor
me; for, although I was hated by Covey and by my master, I was loved by
the colored people, because \emph{they} thought I was hated for my
knowledge, and persecuted because I was feared. I was the \emph{only}
slave \emph{now} in that region who could read and write. There had been
one other man, belonging to Mr. Hugh Hamilton, who could read, (his name
was ``Jim,'') but he, poor fellow, had, shortly after my coming into the
neighborhood, been sold off to the far south. I saw Jim ironed, in the
cart, to be carried to Easton for sale,---pinioned like a yearling for
the slaughter. My knowledge was now the pride of my brother slaves; and,
no doubt, Sandy felt something of the general interest in me on that
account. The supper was soon ready, and though I have feasted since,
with honorables, lord mayors and aldermen, over the sea, my supper on
ash cake and cold water, with Sandy, was the meal, of all my life, most
sweet to my taste, and now most vivid in my memory.

Supper over, Sandy and I went into a discussion of what was
\emph{possible} for me, under the perils and
{\protect\hypertarget{238}{}{}}hardships which now overshadowed my path.
The question was, must I go back to Covey, or must I now attempt to run
away? Upon a careful survey, the latter was found to be impossible; for
I was on a narrow neck of land, every avenue from which would bring me
in sight of pursuers. There was the Chesapeake bay to the right, and
``Pot-pie'' river to the left, and St. Michael's and its neighborhood
occupying the only space through which there was any retreat.

I found Sandy an old adviser. He was not only a religious man, but he
professed to believe in a system for which I have no name. He was a
genuine African, and had inherited some of the so called magical powers,
said to be possessed by African and eastern nations. He told me that he
could help me; that, in those very woods, there was an herb, which in
the morning might be found, possessing all the powers required for my
protection, (I put his thoughts in my own language;) and that, if I
would take his advice, he would procure me the root of the herb of which
he spoke. He told me further, that if I would take that root and wear it
on my right side, it would be impossible for Covey to strike me a blow;
that with this root about my person, no white man could whip me. He said
he had carried it for years, and that he had fully tested its virtues.
He had never received a blow from a slaveholder since he carried it; and
he never expected to receive one, for he always meant to carry that root
as a protection. He knew Covey well, for Mrs. Covey was the daughter of
Mr. Kemp; and he (Sandy) had heard of the barbarous treatment
{\protect\hypertarget{239}{}{}}to which I was subjected, and he wanted
to do something for me.

Now all this talk about the root, was, to me, very absurd and
ridiculous, if not positively sinful. I at first rejected the idea that
the simple carrying a root on my right side, (a root, by the way, over
which I walked every time I went into the woods,) could possess any such
magic power as he ascribed to it, and I was, therefore, not disposed to
cumber my pocket with it. I had a positive aversion to all pretenders to
"\emph{divination}." It was beneath one of my intelligence to
countenance such dealings with the devil, as this power implied. But,
with all my learning---it was really precious little---Sandy was more
than a match for me. ``My book learning,'' he said, ``had not kept Covey
off me,'' (a powerful argument just then,) and he entreated me, with
flashing eyes, to try this. If it did me no good, it could do me no
harm, and it would cost me nothing, any way. Sandy was so earnest, and
so confident of the good qualities of this weed, that, to please him,
rather than from any conviction of its excellence, I was induced to take
it. He had been to me the good Samaritan, and had, almost
providentially, found me, and helped me when I could not help myself;
how did I know but that the hand of the Lord was in it? With thoughts of
this sort, I took the roots from Sandy, and put them in my right hand
pocket.

This was, of course, Sunday morning. Sandy now urged me to go home, with
all speed, and to walk up bravely to the house, as though nothing had
happened. I saw in Sandy too deep an insight into
{\protect\hypertarget{240}{}{}}human nature, with all his superstition,
not to have some respect for his advice; and perhaps, too, a slight
gleam or shadow of his superstition had fallen upon me. At any rate, I
started off toward Covey's, as directed by Sandy. Having, the previous
night, poured my griefs into Sandy's ears, and got him enlisted in my
behalf, having made his wife a sharer in my sorrows, and having, also,
become well refreshed by sleep and food, I moved off, quite
courageously, toward the much dreaded Covey's. Singularly enough, just
as I entered his yard gate, I met him and his wife, dressed in their
Sunday best---looking as smiling as angels---on their way to church. The
manner of Covey astonished me. There was something really benignant in
his countenance. He spoke to me as never before; told me that the pigs
had got into the lot, and he wished me to drive them out; inquired how I
was, and seemed an altered man. This extraordinary conduct of Covey,
really made me begin to think that Sandy's herb had more virtue in it
than I, in my pride, had been willing to allow; and, had the day been
other than Sunday, I should have attributed Covey's altered manner
solely to the magic power of the root. I suspected, however, that the
\emph{Sabbath}, and not the \emph{root}, was the real explanation of
Covey's manner. His religion hindered him from breaking the Sabbath, but
not from breaking my skin. He had more respect for the \emph{day} than
for the \emph{man}, for whom the day was mercifully given; for while he
would cut and slash my body during the week, he would not hesitate, on
Sunday, to teach me the value {\protect\hypertarget{241}{}{}}of my soul,
or the way of life and salvation by Jesus Christ.

All went well with me till Monday morning; and then, whether the root
had lost its virtue, or whether my tormentor had gone deeper into the
black art than myself, (as was sometimes said of him,) or whether he had
obtained a special indulgence, for his faithful Sabbath day's worship,
it is not necessary for me to know, or to inform the reader; but, this
much I \emph{may} say,---the pious and benignant smile which graced
Covey's face on \emph{Sunday}, wholly disappeared on \emph{Monday}. Long
before daylight, I was called up to go and feed, rub, and curry the
horses. I obeyed the call, and I would have so obeyed it, had it been
made at an earlier hour, for I had brought my mind to a firm resolve,
during that Sunday's reflection, viz: to obey every order, however
unreasonable, if it were possible, and, if Mr. Covey should then
undertake to beat me, to defend and protect myself to the best of my
ability. My religious views on the subject of resisting my master, had
suffered a serious shock, by the savage persecution to which I had been
subjected, and my hands were no longer tied by my religion. Master
Thomas's indifference had severed the last link. I had now to this
extent ``backslidden'' from this point in the slave's religious creed;
and I soon had occasion to make my fallen state known to my Sunday-pious
brother, Covey.

Whilst I was obeying his order to feed and get the horses ready for the
field, and when in the act of going up the stable loft for the purpose
of throwing down some blades, Covey sneaked into the stable, in his
{\protect\hypertarget{242}{}{}}peculiar snake-like way, and seizing me
suddenly by the leg, he brought me to the stable floor, giving my newly
mended body a fearful jar. I now forgot my \emph{roots}, and remembered
my pledge to \emph{stand up in my own defense}. The brute was
endeavoring skillfully to get a slip-knot on my legs, before I could
draw up my feet. As soon as I found what he was up to, I gave a sudden
spring, (my two day's rest had been of much service to me,) and by that
means, no doubt, he was able to bring me to the floor so heavily. He was
defeated in his plan of tying me. While down, he seemed to think he had
me very securely in his power. He little thought he was---as the rowdies
say---``in'' for a ``rough and tumble'' fight; but such was the fact.
Whence came the daring spirit necessary to grapple with a man who,
eight-and-forty hours before, could, with his slightest word have made
me tremble like a leaf in a storm, I do not know; at any rate, \emph{I
was resolved to fight}, and, what was better still, I was actually hard
at it. The fighting madness had come upon me, and I found my strong
fingers firmly attached to the throat of my cowardly tormentor; as
heedless of consequences, at the moment, as though we stood as equals
before the law. The very color of the man was forgotten. I felt as
supple as a cat, and was ready for the snakish creature at every turn.
Every blow of his was parried, though I dealt no blows in turn. I was
strictly on the \emph{defensive}, preventing him from injuring me,
rather than trying to injure him. I flung him on the ground several
times, when he meant to have hurled me there. I held him
{\protect\hypertarget{243}{}{}}so firmly by the throat, that his blood
followed my nails. He held me, and I held him.

All was fair, thus far, and the contest was about equal. My resistance
was entirely unexpected, and Covey was taken all aback by it, for he
trembled in every limb. "\emph{Are you going to resist}, you scoundrel?"
said he. To which, I returned a polite "\emph{yes {sir;}} steadily
gazing my interrogator in the eye, to meet the first approach or dawning
of the blow, which I expected my answer would call forth. But, the
conflict did not long remain thus equal. Covey soon cried out lustily
for help; not that I was obtaining any marked advantage over him, or was
injuring him, but because he was gaining none over me, and was not able,
single handed, to conquer me. He called for his cousin Hughes, to come
to his assistance, and now the scene was changed. I was compelled to
give blows, as well as to parry them; and, since I was, in any case, to
suffer for resistance, I felt (as the musty proverb goes) that ``I might
as well be hanged for an old sheep as a lamb.'' I was still
\emph{defensive} toward Covey, but \emph{aggressive} toward Hughes; and,
at the first approach of the latter, I dealt a blow, in my desperation,
which fairly sickened my youthful assailant. He went off, bending over
with pain, and manifesting no disposition to come within my reach again.
The poor fellow was in the act of trying to catch and tie my right hand,
and while flattering himself with success, I gave him the kick which
sent him staggering away in pain, at the same time that I held Covey
with a firm hand.

Taken completely by surprise, Covey seemed to
{\protect\hypertarget{244}{}{}}have lost his usual strength and
coolness. He was frightened, and stood puffing and blowing, seemingly
unable to command words or blows. When he saw that poor Hughes was
standing half bent with pain---his courage quite gone---the cowardly
tyrant asked if I ``meant to persist in my resistance.'' I told him "I
\emph{did mean to resist, come what might;}" that I had been by him
treated like a \emph{brute}, during the last six months; and that I
should stand it \emph{no longer}. With that, he gave me a shake, and
attempted to drag me toward a stick of wood, that was lying just outside
the stable door. He meant to knock me down with it; but, just as he
leaned over to get the stick, I seized him with both hands by the
collar, and, with a vigorous and sudden snatch, I brought my assailant
harmlessly, his full length, on the \emph{not over} clean ground---for
we were now in the cow yard. He had selected the place, for the fight,
and it was but right that he should have all the advantages of his own
selection.

By this time, Bill, the hired man, came home. He had been to Mr.
Hemsley's, to spend the Sunday with his nominal wife, and was coming
home on Monday morning, to go to work. Covey and I had been skirmishing
from before daybreak, till now, that the sun was almost shooting his
beams over the eastern woods, and we were still at it. I could not see
where the matter was to terminate. He evidently was afraid to let me go,
lest I should again make off to the woods; otherwise, he would probably
have obtained arms from the house, to frighten me. Holding me, Covey
called upon Bill for assistance. The scene here, had something comic
about it. ``Bill,'' who knew \emph{precisely}
{\protect\hypertarget{245}{}{}}what Covey wished him to do, affected
ignorance, and pretended he did not know what to do. ``What shall I do,
Mr. Covey,'' said Bill. ``Take hold of him---take hold of him!'' said
Covey. ``With a toss of his head, peculiar to Bill, he said, ''indeed,
Mr. Covey, I want to go to work." "\emph{This is} your work," said
Covey; ``take hold of him.'' Bill replied, with spirit, "My master hired
me here, to work, and \emph{not} to help you whip Frederick." It was now
my turn to speak. ``Bill,'' said I, ``don't put your hands on me.'' To
which he replied, "\textsc{My God}! Frederick, I aint goin' to tech ye,"
and Bill walked off, leaving Covey and myself to settle our matters as
best we might.

But, my present advantage was threatened when I saw Caroline (the
slave-woman of Covey) coming to the cow yard to milk, for she was a
powerful woman, and could have mastered me very easily, exhausted as I
now was. As soon as she came into the yard, Covey attempted to rally her
to his aid. Strangely and, I may add, fortunately Caroline was in no
humor to take a hand in any such sport. ``We were all in open rebellion,
that morning. Caroline answered the command of her master to
''\emph{take hold of me}," precisely as Bill had answered, but in
\emph{her}, it was at greater peril so to answer; she was the slave of
Covey, and he could do what he pleased with her. It was \emph{not} so
with Bill, and Bill knew it. Samuel Harris, to whom Bill belonged, did
not allow his slaves to be beaten, unless they were guilty of some crime
which the law would punish. But, poor Caroline, like myself, was at the
mercy of the merciless Covey; nor {\protect\hypertarget{246}{}{}}did she
escape the dire effects of her refusal. He gave her several sharp blows.

Covey at length (two hours had elapsed) gave up the contest. Letting me
go, he said,---puffing and blowing at a great rate---``now, you
scoundrel, go to your work; I would not have whipped you half so much as
I have had you not resisted.'' The fact was, \emph{he had not whipped me
at all}. He had not, in all the scuffle, drawn a single drop of blood
from me. I had drawn blood from him; and, even without this
satisfaction, I should have been victorious, because my aim had not been
to injure him, but to prevent his injuring me.

During the whole six months that I lived with Covey, after this
transaction, he never laid on me the weight of his finger in anger. He
would, occasionally, say he did not want to have to get hold of me
again---a declaration which I had no difficulty in believing; and I had
a secret feeling, which answered, ``you need not wish to get hold of me
again, for you will be likely to come off worse in a second fight than
you did in the first.''

Well, my dear reader, this battle with Mr. Covey,---undignified as it
was, and as I fear my narration of it is---was the turning point in my
"\emph{life as a slave.}" It rekindled in my breast the smouldering
embers of liberty; it brought up my Baltimore dreams, and revived a
sense of my own manhood. I was a changed being after that fight. I was
\emph{nothing} before; I \textsc{was a man now}. It recalled to life my
crushed self-respect and my self-confidence, and inspired me with a
renewed determination to be \textsc{a freeman}. A man,
{\protect\hypertarget{247}{}{}}without force, is without the essential
dignity of humanity. Human nature is so constituted, that it cannot
\emph{honor} a helpless man, although it can \emph{pity} him; and even
this it cannot do long, if the signs of power do not arise.

He only can understand the effect of this combat on my spirit, who has
himself incurred something, hazarded something, in repelling the unjust
and cruel aggressions of a tyrant. Covey was a tyrant, and a cowardly
one, withal. After resisting him, I felt as I had never felt before. It
was a resurrection from the dark and pestiferous tomb of slavery, to the
heaven of comparative freedom. I was no longer a servile coward,
trembling under the frown of a brother worm of the dust, but, my
long-cowed spirit was roused to an attitude of manly independence. I had
reached the point, at which I was \emph{not afraid to die}. This spirit
made me a freeman in \emph{fact}, while I remained a slave in
\emph{form}. When a slave cannot be flogged he is more than half free.
He has a domain as broad as his own manly heart to defend, and he is
really "\emph{a power on earth.}" While slaves prefer their lives, with
flogging, to instant death, they will always find christians enough,
like unto Covey, to accommodate that preference. From this time, until
that of my escape from slavery, I was never fairly whipped. Several
attempts were made to whip me, but they were always unsuccessful.
Bruises I did get, as I shall hereafter inform the reader; but the case
I have been describing, was the end of the brutification to which
slavery had subjected me.

The reader will be glad to know why, after I had
{\protect\hypertarget{248}{}{}}so grievously offended Mr. Covey, he did
not have me taken in hand by the authorities; indeed, why the law of
Maryland, which assigns hanging to the slave who resists his master, was
not put in force against me; at any rate, why I was not taken up, as is
usual in such cases, and publicly whipped, for an example to other
slaves, and as a means of deterring me from committing the same offense
again. I confess, that the easy manner in which I got off, was, for a
long time, a surprise to me, and I cannot, even now, fully explain the
cause.

The only explanation I can venture to suggest, is the fact, that Covey
was, probably, ashamed to have it known and confessed that he had been
mastered by a boy of sixteen. Mr. Covey enjoyed the unbounded and very
valuable reputation, of being a first rate overseer and \emph{negro
breaker}. By means of this reputation, he was able to procure his hands
for \emph{very trifling} compensation, and with very great ease. His
interest and his pride mutually suggested the wisdom of passing the
matter by, in silence. The story that he had undertaken to whip a lad,
and had been resisted, was, of itself, sufficient to damage him; for his
bearing should, in the estimation of slaveholders, be of that imperial
order that should make such an occurrence \emph{impossible}. I judge
from these circumstances, that Covey deemed it best to give me the
go-by. It is, perhaps, not altogether creditable to my natural temper,
that, after this conflict with Mr. Covey, I did, at times, purposely aim
to provoke him to an attack, by refusing to keep with the other hands in
the field, {\protect\hypertarget{249}{}{}}but I could never bully him to
another battle. I had made up my mind to do him serious damage, if he
ever again attempted to lay violent hands on me.

{"}Hereditary bondmen, know ye not\\
Who would be free, themselves must strike the blow?"
