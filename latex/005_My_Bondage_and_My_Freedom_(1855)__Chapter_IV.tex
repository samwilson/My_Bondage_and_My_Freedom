{\protect\hypertarget{61}{}{}}

~

{CHAPTER IV.}

A GENERAL SURVEY OF THE SLAVE PLANTATION.

{ISOLATION OF LLOYD'S PLANTATION---PUBLIC OPINION THERE NO PROTECTION TO
THE SLAVE---ABSOLUTE POWER OF THE OVERSEER---NATURAL AND ARTIFICIAL
CHARMS OF THE PLACE---ITS BUSINESS-LIKE APPEARANCE---SUPERSTITION ABOUT
THE BURIAL GROUND---GREAT IDEAS OF COL. LLOYD---ETIQUETTE AMONG
SLAVES---THE COMIC SLAVE DOCTOR---PRAYING AND FLOGGING---``OLD MASTER''
LOSING ITS TERRORS---HIS BUSINESS---CHARACTER OF ``AUNT
KATY''---SUFFERINGS FROM HUNGER---OLD MASTER'S HOME---JARGON OF THE
PLANTATION---GUINEA SLAVES---MASTER DANIEL---FAMILY OF COL.
LLOYD---FAMILY OF CAPT. ANTHONY---HIS SOCIAL POSITION---NOTIONS OF RANK
AND STATION.}

\textsc{It} is generally supposed that slavery, in the state of
Maryland, exists in its mildest form, and that it is totally divested of
those harsh and terrible peculiarities, which mark and characterize the
slave system, in the southern and south-western states of the American
union. The argument in favor of this opinion, is the contiguity of the
free states, and the exposed condition of slavery in Maryland to the
moral, religious and humane sentiment of the free states.

I am not about to refute this argument, so far as it relates to slavery
in that State, generally; on the contrary, I am willing to admit that,
to this general point, the argument is well grounded. Public opinion is,
indeed, an unfailing restraint upon the cruelty and barbarity of
masters, overseers, and slave-drivers, whenever and wherever it can
reach them; but there {\protect\hypertarget{62}{}{}}are certain secluded
and out-of-the way places, even in the state of Maryland, seldom visited
by a single ray of healthy public sentiment---where slavery, wrapt in
its own congenial, midnight darkness, \emph{can}, and \emph{does},
develop all its malign and shocking characteristics; where it can be
indecent without shame, cruel without shuddering, and murderous without
apprehension or fear of exposure.

Just such a secluded, dark, and out-of-the-way place, is the ``home
plantation'' of Col. Edward Lloyd, on the Eastern Shore, Maryland. It is
far away from all the great thoroughfares, and is proximate to no town
or village. There is neither school-house, nor town-house in its
neighborhood. The school-house is unnecessary, for there are no children
to go to school. The children and grand-children of Col. Lloyd were
taught in the house, by a private tutor---a Mr. Page---a tall, gaunt
sapling of a man, who did not speak a dozen words to a slave in a whole
year. The overseers' children go off somewhere to school; and they,
therefore, bring no foreign or dangerous influence from abroad, to
embarrass the natural operation of the slave system of the place. Not
even the mechanics---through whom there is an occasional out-burst of
honest and telling indignation, at cruelty and wrong on other
plantations---are white men, on this plantation. Its whole public is
made up of, and divided into, three classes---\textsc{slaveholders},
\textsc{slaves} and \textsc{overseers}. Its blacksmiths, wheelwrights,
shoemakers, weavers, and coopers, are slaves. Not even commerce, selfish
and iron-hearted at it is, and ready, as it ever is, to side with the
strong against the {\protect\hypertarget{63}{}{}}weak---the rich against
the poor---is trusted or permitted within its secluded precincts.
Whether with a view of guarding against the escape of its secrets, I
know not, but it is a fact, that every leaf and grain of the produce of
this plantation, and those of the neighboring farms belonging to Col.
Lloyd, are transported to Baltimore in Col. Lloyd's own vessels; every
man and boy on board of which---except the captain---are owned by him.
In return, everything brought to the plantation, comes through the same
channel. Thus, even the glimmering and unsteady light of trade, which
sometimes exerts a civilizing influence, is excluded from this
``tabooed'' spot.

Nearly all the plantations or farms in the vicinity of the ``home
plantation'' of Col. Lloyd, belong to him; and those which do not, are
owned by personal friends of his, as deeply interested in maintaining
the slave system, in all its rigor, as Col. Lloyd himself. Some of his
neighbors are said to be even more stringent than he. The Skinners, the
Peakers, the Tilgmans, the Lockermans, and the Gipsons, are in the same
boat; being slaveholding neighbors, they may have strengthened each
other in their iron rule. They are on intimate terms, and their
interests and tastes are identical.

Public opinion in such a quarter, the reader will see, is not likely to
be very efficient in protecting the slave from cruelty. On the contrary,
it must increase and intensify his wrongs. Public opinion seldom differs
very widely from public practice. To be a restraint upon cruelty and
vice, public opinion must emanate from a humane and virtuous community.
To no such {\protect\hypertarget{64}{}{}}humane and virtuous community,
is Col. Lloyd's plantation exposed. That plantation is a little nation
of its own, having its own language, its own rules, regulations and
customs. The laws and institutions of the state, apparently touch it
nowhere. The troubles arising here, are not settled by the civil power
of the state. The overseer is generally accuser, judge, jury, advocate
and executioner. The criminal is always dumb. The overseer attends to
all sides of a case.

There are no conflicting rights of property, for all the people are
owned by one man; and they can themselves own no property. Religion and
politics are alike excluded. One class of the population is too high to
be reached by the preacher; and the other class is too low to be cared
for by the preacher. The poor have the gospel preached to them, in this
neighborhood, only when they are able to pay for it. The slaves, having
no money, get no gospel. The politician keeps away, because the people
have no votes, and the preacher keeps away, because the people have no
money. The rich planter can afford to learn politics in the parlor, and
to dispense with religion altogether.

In its isolation, seclusion, and self-reliant independence, Col. Lloyd's
plantation resembles what the baronial domains were, during the middle
ages in Europe. Grim, cold, and unapproachable by all genial influences
from communities without, \emph{there it stands;} full three hundred
years behind the age, in all that relates to humanity and morals.

This, however, is not the only view that the place presents.
Civilization is shut out, but nature cannot be. Though separated from
the rest of the world; {\protect\hypertarget{65}{}{}}though public
opinion, as I have said, seldom gets a chance to penetrate its dark
domain; though the whole place is stamped with its own peculiar,
iron-like individuality; and though crimes, high-handed and atrocious,
may there be committed, with almost as much impunity as upon the deck of
a pirate ship,---it is, nevertheless, altogether, to outward seeming, a
most strikingly interesting place, full of life, activity, and spirit;
and presents a very favorable contrast to the indolent monotony and
languor of Tuckahoe. Keen as was my regret and great as was my sorrow at
leaving the latter, I was not long in adapting myself to this, my new
home. A man's troubles are always half disposed of, when he finds
endurance his only remedy. I found myself here; there was no getting
away; and what remained for me, but to make the best of it? Here were
plenty of children to play with, and plenty of places of pleasant resort
for boys of my age, and boys older. The little tendrils of affection, so
rudely and treacherously broken from around the darling objects of my
grandmother's hut, gradually began to extend, and to entwine about the
new objects by which I now found myself surrounded.

There was a windmill (always a commanding object to a child's eye) on
Long Point---a tract of land dividing Miles river from the Wye---a mile
or more from my old master's house. There was a creek to swim in, at the
bottom of an open flat space, of twenty acres or more, called ``the Long
Green''---a very beautiful play-ground for the children.

In the river, a short distance from the shore, lying quietly at anchor,
with her small boat dancing at her {\protect\hypertarget{66}{}{}}stern,
was a large sloop---the Sally Lloyd; called by that name in honor of a
favorite daughter of the colonel. The sloop and the mill were wondrous
things, full of thoughts and ideas. A child cannot well look at such
objects without \emph{thinking}.

Then here were a great many houses; human habitations, full of the
mysteries of life at every stage of it. There was the little red house,
up the road, occupied by Mr. Sevier, the overseer. A little nearer to my
old master's, stood a very long, rough, low building, literally alive
with slaves, of all ages, conditions and sizes. This was called ``the
Long Quarter.'' Perched upon a hill, across the Long Green, was a very
tall, dilapidated, old brick building---the architectural dimensions of
which proclaimed its erection for a different purpose---now occupied by
slaves, in a similar manner to the Long Quarter. Besides these, there
were numerous other slave houses and huts, scattered around in the
neighborhood, every nook and corner of which was completely occupied.
Old master's house, a long, brick building, plain, but substantial,
stood in the center of the plantation life, and constituted one
independent establishment on the premises of Col. Lloyd.

Besides these dwellings, there were barns, stables, store-houses, and
tobacco-houses; blacksmiths' shops, wheelwrights' shops, coopers'
shops---all objects of interest; but, above all, there stood the
grandest building my eyes had then ever beheld, called, by every one on
the plantation, the ``Great House.'' This was occupied by Col. Lloyd and
his family. They occupied it; \emph{I} enjoyed it. The great house
{\protect\hypertarget{67}{}{}}was surrounded by numerous and variously
shaped out-buildings. There were kitchens, wash-houses, dairies,
summer-house, green-houses, hen-houses, turkey-houses, pigeon-houses,
and arbors, of many sizes and devices, all neatly painted, and
altogether interspersed with grand old trees, ornamental and primitive,
which afforded delightful shade in summer, and imparted to the scene a
high degree of stately beauty. The great house itself was a large,
white, wooden building, with wings on three sides of it. In front, a
large portico, extending the entire length of the building, and
supported by a long range of columns, gave to the whole establishment an
air of solemn grandeur. It was a treat to my young and gradually opening
mind, to behold this elaborate exhibition of wealth, power, and vanity.
The carriage entrance to the house was a large gate, more than a quarter
of a mile distant from it; the intermediate space was a beautiful lawn,
very neatly trimmed, and watched with the greatest care. It was dotted
thickly over with delightful trees, shrubbery, and flowers. The road, or
lane, from the gate to the great house, was richly paved with white
pebbles from the beach, and, in its course, formed a complete circle
around the beautiful lawn. Carriages going in and retiring from the
great house, made the circuit of the lawn, and their passengers were
permitted to behold a scene of almost Eden-like beauty. Outside this
select inclosure, were parks, where---as about the residences of the
English nobility---rabbits, deer, and other wild game, might be seen,
peering and playing about, with none to molest them or make them afraid.
The {\protect\hypertarget{68}{}{}}tops of the stately poplars were often
covered with the red-winged black-birds, making all nature vocal with
the joyous life and beauty of their wild, warbling notes. These all
belonged to me, as well as to Col. Edward Lloyd, and for a time I
greatly enjoyed them.

A short distance from the great house, were the stately mansions of the
dead, a place of somber aspect. Vast tombs, embowered beneath the
weeping willow and the fir tree, told of the antiquities of the Lloyd
family, as well as of their wealth. Superstition was rife among the
slaves about this family burying ground. Strange sights had been seen
there by some of the older slaves. Shrouded ghosts, riding on great
black horses, had been seen to enter; balls of fire had been seen to fly
there at midnight, and horrid sounds had been repeatedly heard. Slaves
know enough of the rudiments of theology to believe that those go to
hell who die slaveholders; and they often fancy such persons wishing
themselves back again, to wield the lash. Tales of sights and sounds,
strange and terrible, connected with the huge black tombs, were a very
great security to the grounds about them, for few of the slaves felt
like approaching them even in the day time. It was a dark, gloomy and
forbidding place, and it was difficult to feel that the spirits of the
sleeping dust there deposited, reigned with the blest in the realms of
eternal peace.

The business of twenty or thirty farms was transacted at this, called,
by way of eminence, ``great house farm.'' These farms all belonged to
Col. Lloyd, as did, also, the slaves upon them. Each farm was under the
management of an overseer. As I have {\protect\hypertarget{69}{}{}}said
of the overseer of the home plantation, so I may say of the overseers on
the smaller ones; they stand between the slave and all civil
constitutions---their word is law, and is implicitly obeyed.

The colonel, at this time, was reputed to be, and he apparently was,
very rich. His slaves, alone, were an immense fortune. These small and
great, could not have been fewer than one thousand in number, and though
scarcely a month passed without the sale of one or more lots to the
Georgia traders, there was no apparent diminution in the number of his
human stock: the home plantation merely groaned at a removal of the
young increase, or human crop, then proceeded as lively as ever.
Horse-shoeing, cart-mending, plow-repairing, coopering, grinding, and
weaving, for all the neighboring farms, were performed here, and slaves
were employed in all these branches. ``Uncle Tony'' was the blacksmith;
``Uncle Harry'' was the cartwright; ``Uncle Abel'' was the shoemaker;
and all these had hands to assist them in their several departments.

These mechanics were called ``uncles'' by all the younger slaves, not
because they really sustained that relationship to any, but according to
plantation \emph{etiquette}, as a mark of respect, due from the younger
to the older slaves. Strange, and even ridiculous as it may seem, among
a people so uncultivated, and with so many stern trials to look in the
face, there is not to be found, among any people, a more rigid
enforcement of the law of respect to elders, than they maintain. I set
this down as partly constitutional with my race, and partly
conventional. There is no better {\protect\hypertarget{70}{}{}}material
in the world for making a gentleman, than is furnished in the African.
He shows to others, and exacts for himself, all the tokens of respect
which he is compelled to manifest toward his master. A young slave must
approach the company of the older with hat in hand, and woe betide him,
if he fails to acknowledge a favor, of any sort, with the accustomed
"\emph{tank'ee}," \&c. So uniformly are good manners enforced among
slaves, that I can easily detect a ``bogus'' fugitive by his manners.

Among other slave notabilities of the plantation, was one called by
everybody Uncle Isaac Copper. It is seldom that a slave gets a surname
from anybody in Maryland; and so completely has the south shaped the
manners of the north, in this respect, that even abolitionists make very
little of the surname of a negro. The only improvement on the ``Bills,''
``Jacks,'' ``Jims,'' and ``Neds'' of the south, observable here is, that
``William,'' ``John,'' ``James,'' ``Edward,'' are substituted. It goes
against the grain to treat and address a negro precisely as they would
treat and address a white man. But, once in a while, in slavery as in
the free states, by some extraordinary circumstance, the negro has a
surname fastened to him, and holds it against all conventionalties. This
was the case with Uncle Isaac Copper. When the ``uncle'' was dropped, he
generally had the prefix ``doctor,'' in its stead. He was our doctor of
medicine, and doctor of divinity as well. Where he took his degree I am
unable to say, for he was not very communicative to inferiors, and I was
emphatically such, being but a boy seven or eight years old. He
{\protect\hypertarget{71}{}{}}was too well established in his profession
to permit questions as to his native skill, or his attainments. One
qualification he undoubtedly had---he was a confirmed \emph{cripple;}
and he could neither work, nor would he bring anything if offered for
sale in the market. The old man, though lame, was no sluggard. He was a
man that made his crutches do him good service. He was always on the
alert, looking up the sick, and all such as were supposed to need his
counsel. His remedial prescriptions embraced four articles. For diseases
of the body, \emph{Epsom salts} and \emph{castor oil;} for those of the
soul, \emph{the Lord's Prayer}, and \emph{hickory switches!}

I was not long at Col. Lloyd's before I was placed under the care of
Doctor Isaac Copper. I was sent to him with twenty or thirty other
children, to learn the ``Lord's Prayer.'' I found the old gentleman
seated on a huge three-legged oaken stool, armed with several large
hickory switches; and, from his position, he could reach---lame as he
was---any boy in the room. After standing awhile to learn what was
expected of us, the old gentleman, in any other than a devotional tone,
commanded us to kneel down. This done, he commenced telling us to say
everything he said. ``Our Father''---this we repeated after him with
promptness and uniformity; ``Who art in heaven''---was less promptly and
uniformly repeated; and the old gentleman paused in the prayer, to give
us a short lecture upon the consequences of inattention, both immediate
and future, and especially those more immediate. About these he was
absolutely certain, for he held in his right hand the means of
{\protect\hypertarget{72}{}{}}bringing all his predictions and warnings
to pass. On he proceeded with the prayer; and we with our thick tongues
and unskilled ears, followed him to the best of our ability. This,
however, was not sufficient to please the old gentleman. Everybody, in
the south, wants the privilege of whipping somebody else. Uncle Isaac
shared the common passion of his country, and, therefore, seldom found
any means of keeping his disciples in order short of flogging. ``Say
everything I say;'' and bang would come the switch on some poor boy's
undevotional head. "\emph{What you looking at there}``---''\emph{Stop
that pushing}"---and down again would come the lash.

The whip is all in all. It is supposed to secure obedience to the
slaveholder, and is held as a sovereign remedy among the slaves
themselves, for every form of disobedience, temporal or spiritual.
Slaves, as well as slaveholders, use it with an unsparing hand. Our
devotions at Uncle Isaac's combined too much of the tragic and comic, to
make them very salutary in a spiritual point of view; and it is due to
truth to say, I was often a truant when the time for attending the
praying and flogging of Doctor Isaac Copper came on.

The windmill under the care of Mr. Kinney, a kind hearted old
Englishman, was to me a source of infinite interest and pleasure. The
old man always seemed pleased when he saw a troop of darkey little
urchins, with their tow-linen shirts fluttering in the breeze,
approaching to view and admire the whirling wings of his wondrous
machine. From the mill we could see other objects of deep interest.
These were, {\protect\hypertarget{73}{}{}}the vessels from St.
Michael's, on their way to Baltimore. It was a source of much amusement
to view the flowing sails and complicated rigging, as the little crafts
dashed by, and to speculate upon Baltimore, as to the kind and quality
of the place. With so many sources of interest around me, the reader may
be prepared to learn that I began to think very highly of Col. L.'s
plantation. It was just a place to my boyish taste. There were fish to
be caught in the creek, if one only had a hook and line; and crabs,
clams and oysters were to be caught by wading, digging and raking for
them. Here was a field for industry and enterprise, strongly inviting;
and the reader may be assured that I entered upon it with spirit.

Even the much dreaded old master, whose merciless fiat had brought me
from Tuckahoe, gradually, to my mind, parted with his terrors. Strange
enough, his reverence seemed to take no particular notice of me, nor of
my coming. Instead of leaping out and devouring me, he scarcely seemed
conscious of my presence. The fact is, he was occupied with matters more
weighty and important than either looking after or vexing me. He
probably thought as little of my advent, as he would have thought of the
addition of a single pig to his stock!

As the chief butler on Col. Lloyd's plantation, his duties were numerous
and perplexing. In almost all important matters he answered in Col.
Lloyd's stead. The overseers of all the farms were in some sort under
him, and received the law from his mouth. The colonel himself seldom
addressed an overseer, or {\protect\hypertarget{74}{}{}}allowed an
overseer to address him. Old master carried the keys of all the store
houses; measured out the allowance for each slave at the end of every
month; superintended the storing of all goods brought to the plantation;
dealt out the raw material to all the handicraftsmen; shipped the grain,
tobacco, and all saleable produce of the plantation to market, and had
the general oversight of the coopers' shop, wheelwrights' shop,
blacksmiths' shop, and shoemakers' shop. Besides the care of these, he
often had business for the plantation which required him to be absent
two and three days.

Thus largely employed, he had little time, and perhaps as little
disposition, to interfere with the children individually. What he was to
Col. Lloyd, he made Aunt Katy to him. When he had anything to say or do
about us, it was said or done in a wholesale manner; disposing of us in
classes or sizes, leaving all minor details to Aunt Katy, a person of
whom the reader has already received no very favorable impression. Aunt
Katy was a woman who never allowed herself to act greatly within the
margin of power granted to her, no matter how broad that authority might
be. Ambitious, ill-tempered and cruel, she found in her present position
an ample field for the exercise of her ill-omened qualities. She had a
strong hold on old master---she was considered a first rate cook, and
she really was very industrious. She was, therefore, greatly favored by
old master, and as one mark of his favor, she was the only mother who
was permitted to retain her children around her. Even to these children
she was often fiendish in her {\protect\hypertarget{75}{}{}}brutality.
She pursued her son Phil, one day, in my presence, with a huge butcher
knife, and dealt a blow with its edge which left a shocking gash on his
arm, near the wrist. For this, old master did sharply rebuke her, and
threatened that if she ever should do the like again, he would take the
skin off her back. Cruel, however, as Aunt Katy was to her own children,
at times she was not destitute of maternal feeling, as I often had
occasion to know, in the bitter pinches of hunger I had to endure.
Differing from the practice of Col. Lloyd, old master, instead of
allowing so much for each slave, committed the allowance for all to the
care of Aunt Katy, to be divided after cooking it, amongst us. The
allowance, consisting of coarse corn-meal, was not very
abundant---indeed, it was very slender; and in passing through Aunt
Katy's hands, it was made more slender still, for some of us. William,
Phil and Jerry were her children, and it is not to accuse her too
severely, to allege that she was often guilty of starving myself and the
other children, while she was literally cramming her own. Want of food
was my chief trouble the first summer at my old master's. Oysters and
clams would do very well, with an occasional supply of bread, but they
soon failed in the absence of bread. I speak but the simple truth, when
I say, I have often been so pinched with hunger, that I have fought with
the dog---``Old Nep''---for the smallest crumbs that fell from the
kitchen table, and have been glad when I won a single crumb in the
combat. Many times have I followed, with eager step, the waiting-girl
when she went out to shake the table cloth, to get
{\protect\hypertarget{76}{}{}}the crumbs and small bones flung out for
the cats. The water, in which meat had been boiled, was as eagerly
sought for by me. It was a great thing to get the privilege of dipping a
piece of bread in such water; and the skin taken from rusty bacon, was a
positive luxury. Nevertheless, I sometimes got full meals and kind words
from sympathizing old slaves, who knew my sufferings, and received the
comforting assurance that I should be a man some day. ``Never mind,
honey---better day comin','' was even then a solace, a cheering
consolation to me in my troubles. Nor were all the kind words I received
from slaves. I had a friend in the parlor, as well, and one to whom I
shall be glad to do justice, before I have finished this part of my
story.

I was not long at old master's, before I learned that his surname was
Anthony, and that he was generally called ``Captain Anthony''---a title
which he probably acquired by sailing a craft in the Chesapeake Bay.
Col. Lloyd's slaves never called Capt. Anthony ``old master,'' but
always Capt. Anthony; and \emph{me} they called ``Captain Anthony Fed.''
There is not, probably, in the whole south, a plantation where the
English language is more imperfectly spoken than on Col. Lloyd's. It is
a mixture of Guinea and everything else you please. At the time of which
I am now writing, there were slaves there who had been brought from the
coast of Africa. They never used the "\emph{s}" in indication of the
possessive case. ``Cap'n Ant'ney Tom,'' ``Lloyd Bill,'' ``Aunt Rose
Harry,'' means ``Captain Anthony's Tom,'' ``Lloyd's Bill,'' \&c.
"\emph{Oo you dem long to?}" means, "Whom do you
{\protect\hypertarget{77}{}{}}belong to?" "\emph{Oo dem got any
peachy?}" means, ``Have you got any peaches?'' I could scarcely
understand them when I first went among them, so broken was their
speech; and I am persuaded that I could not have been dropped anywhere
on the globe, where I could reap less, in the way of knowledge, from my
immediate associates, than on this plantation. Even "\textsc{Mas'
Daniel}," by his association with his father's slaves, had measurably
adopted their dialect and their ideas, so far as they had ideas to be
adopted. The equality of nature is strongly asserted in childhood, and
childhood requires children for associates. \emph{Color} makes no
difference with a child. Are you a child with wants, tastes and pursuits
common to children, not put on, but natural? then, were you black as
ebony you would be welcome to the child of alabaster whiteness. The law
of compensation holds here, as well as elsewhere. Mas' Daniel could not
associate with ignorance without sharing its shade; and he could not
give his black playmates his company, without giving them his
intelligence, as well. Without knowing this, or caring about it, at the
time, I, for some cause or other, spent much of my time with Mas'
Daniel, in preference to spending it with most of the other boys.

Mas' Daniel was the youngest son of Col. Lloyd; his older brothers were
Edward and Murray---both grown up, and fine looking men. Edward was
especially esteemed by the children, and by me among the rest; not that
he ever said anything to us or for us, which could be called especially
kind; {\protect\hypertarget{78}{}{}}it was enough for us, that he never
looked nor acted scornfully toward us. There were also three sisters,
all married; one to Edward Winder; a second to Edward Nicholson; a third
to Mr. Lownes.

The family of old master consisted of two sons, Andrew and Richard; his
daughter, Lucretia, and her newly married husband, Capt. Auld. This was
the house family. The kitchen family consisted of Aunt Katy, Aunt
Esther, and ten or a dozen children, most of them older than myself.
Capt. Anthony was not considered a rich slaveholder, but was pretty well
off in the world. He owned about thirty "\emph{head}" of slaves, and
three farms in Tuckahoe. The most valuable part of his property was his
slaves, of whom he could afford to sell one every year. This crop,
therefore, brought him seven or eight hundred dollars a year, besides
his yearly salary, and other revenue from his farms.

The idea of rank and station was rigidly maintained on Col. Lloyd's
plantation. Our family never visited the great house, and the Lloyds
never came to our home. Equal non-intercourse was observed between Capt.
Anthony's family and that of Mr. Sevier, the overseer.

Such, kind reader, was the community, and such the place, in which my
earliest and most lasting impressions of slavery, and of slave-life,
were received; of which impressions you will learn more in the coming
chapters of this book.
