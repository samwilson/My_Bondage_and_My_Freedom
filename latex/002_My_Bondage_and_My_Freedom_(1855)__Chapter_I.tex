{\protect\hypertarget{Img}{}{}}

\href{/wiki/File:Frederickdouglaspg42.png}{\includegraphics[width=4.37500in,height=6.86458in]{//upload.wikimedia.org/wikipedia/commons/thumb/7/77/Frederickdouglaspg42.png/420px-Frederickdouglaspg42.png}}

{\protect\hypertarget{ux5cux7bux5cux7bux5cux7b1ux5cux7dux5cux7dux5cux7d}{}{}}

{\protect\hypertarget{33}{}{}}

~

{LIFE AS A SLAVE.}

\begin{longtable}[]{@{}lll@{}}
\toprule
\includegraphics[width=0.41667in,height=0.01042in]{//upload.wikimedia.org/wikipedia/commons/thumb/b/b2/Rule_Segment_-_Span_-_40px.svg/40px-Rule_Segment_-_Span_-_40px.svg.png}
&
\includegraphics[width=0.06250in,height=0.07292in]{//upload.wikimedia.org/wikipedia/commons/thumb/2/28/Rule_Segment_-_Circle_-_6px.svg/6px-Rule_Segment_-_Circle_-_6px.svg.png}
&
\includegraphics[width=0.41667in,height=0.01042in]{//upload.wikimedia.org/wikipedia/commons/thumb/b/b2/Rule_Segment_-_Span_-_40px.svg/40px-Rule_Segment_-_Span_-_40px.svg.png}\tabularnewline
\bottomrule
\end{longtable}

{CHAPTER I.}

THE AUTHOR'S CHILDHOOD.

{PLACE OF BIRTH---CHARACTER OF THE DISTRICT---TUCKAHOE---ORIGIN OF THE
NAME---CHOPTANK RIVER---TIME OF BIRTH---GENEALOGICAL TREES---MODE OF
COUNTING TIME---NAMES OF GRANDPARENTS---THEIR POSITION---GRANDMOTHER
ESPECIALLY ESTEEMED---``BORN TO GOOD LUCK''---SWEET
POTATOES---SUPERSTITION---THE LOG CABIN---ITS CHARMS---SEPARATING
CHILDREN---AUTHOR'S AUNTS---THEIR NAMES---FIRST KNOWLEDGE OF BEING A
SLAVE---``OLD MASTER''---GRIEFS AND JOYS OF CHILDHOOD---COMPARATIVE
HAPPINESS OF THE SLAVE-BOY AND THE SON OF A SLAVEHOLDER.}

\textsc{In} Talbot county, Eastern Shore, Maryland, near Easton, the
county town of that county, there is a small district of country, thinly
populated, and remarkable for nothing that I know of more than for the
worn-out, sandy, desert-like appearance of its soil, the general
dilapidation of its farms and fences, the indigent and spiritless
character of its inhabitants, and the prevalence of ague and fever.

The name of this singularly unpromising and truly famine stricken
district is Tuckahoe, a name well known to all Marylanders, black and
white. It was given to this section of country probably, at the first,
merely in derision; or it may possibly have been
{\protect\hypertarget{34}{}{}}applied to it, as I have heard, because
some one of its earlier inhabitants had been guilty of the petty
meanness of stealing a hoe---or taking a hoe---that did not belong to
him. Eastern Shore men usually pronounce the word \emph{took}, as
\emph{tuck; Took-a-hoe}, therefore, is, in Maryland parlance,
\emph{Tuckahoe}. But, whatever may have been its origin---and about this
I will not be positive---that name has stuck to the district in
question; and it is seldom mentioned but with contempt and derision, on
account of the barrenness of its soil, and the ignorance, indolence, and
poverty of its people. Decay and ruin are everywhere visible, and the
thin population of the place would have quitted it long ago, but for the
Choptank river, which runs through it, from which they take abundance of
shad and herring, and plenty of ague and fever.

It was in this dull, flat, and unthrifty district, or neighborhood,
surrounded by a white population of the lowest order, indolent and
drunken to a proverb, and among slaves, who seemed to ask, "\emph{Oh!
what's the use?}" every time they lifted a hoe, that I---without any
fault of mine---was born, and spent the first years of my childhood.

The reader will pardon so much about the place of my birth, on the score
that it is always a fact of some importance to know where a man is born,
if, indeed, it be important to know anything about him. In regard to the
\emph{time} of my birth, I cannot be as definite as I have been
respecting the \emph{place}. Nor, indeed, can I impart much knowledge
concerning my parents. Genealogical trees do not flourish among slaves.
A person of some consequence here in the north,
{\protect\hypertarget{35}{}{}}sometimes designated \emph{father}, is
literally abolished in slave law and slave practice. It is only once in
a while that an exception is found to this statement. I never met with a
slave who could tell me how old he was. Few slave-mothers know anything
of the months of the year, nor of the days of the month. They keep no
family records, with marriages, births, and deaths. They measure the
ages of their children by spring time, winter time, harvest time,
planting time, and the like; but these soon become undistinguishable and
forgotten. Like other slaves, I cannot tell how old I am. This
destitution was among my earliest troubles. I learned when I grew up,
that my master---and this is the case with masters generally---allowed
no questions to be put to him, by which a slave might learn his age.
Such questions are deemed evidence of impatience, and even of impudent
curiosity. From certain events, however, the dates of which I have since
learned, I suppose myself to have been born about the year 1817.

The first experience of life with me that I now remember---and I
remember it but hazily---began in the family of my grandmother and
grandfather, Betsey and Isaac Baily. They were quite advanced in life,
and had long lived on the spot where they then resided. They were
considered old settlers in the neighborhood, and, from certain
circumstances, I infer that my grandmother, especially, was held in high
esteem, far higher than is the lot of most colored persons in the slave
states. She was a good nurse, and a capital hand at making nets for
catching shad and herring; and these nets were in great demand, not
{\protect\hypertarget{36}{}{}}only in Tuckanoe, but at Denton and
Hillsboro, neighboring villages. She was not only good at making the
nets, but was also somewhat famous for her good fortune in taking the
fishes referred to. I have known her to be in the water half the day.
Grandmother was likewise more provident than most of her neighbors in
the preservation of seedling sweet potatoes, and it happened to her---as
it will happen to any careful and thrifty person residing in an ignorant
and improvident community---to enjoy the reputation of having been born
to ``good luck.'' Her ``good luck'' was owing to the exceeding care
which she took in preventing the succulent root from getting bruised in
the digging, and in placing it beyond the reach of frost, by actually
burying it under the hearth of her cabin during the winter months. In
the time of planting sweet potatoes, ``Grandmother Betty,'' as she was
familiarly called, was sent for in all directions, simply to place the
seedling potatoes in the hills; for superstition had it, that if
``Grandmamma Betty but touches them at planting, they will be sure to
grow and flourish.'' This high reputation was full of advantage to her,
and to the children around her. Though Tuckahoe had but few of the good
things of life, yet of such as it did possess grandmother got a full
share, in the way of presents. If good potato crops came after her
planting, she was not forgotten by those for whom she planted; and as
she was remembered by others, so she remembered the hungry little ones
around her.

The dwelling of my grandmother and grandfather had few pretensions. It
was a log hut, or cabin, {\protect\hypertarget{37}{}{}}built of clay,
wood, and straw. At a distance it resembled---though it was much
smaller, less commodious and less substantial---the cabins erected in
the western states by the first settlers. To my child's eye, however, it
was a noble structure, admirably adapted to promote the comforts and
conveniences of its inmates. A few rough, Virginia fence-rails, flung
loosely over the rafters above, answered the triple purpose of floors,
ceilings, and bedsteads. To be sure, this upper apartment was reached
only by a ladder---but what in the world for climbing could be better
than a ladder? To me, this ladder was really a high invention, and
possessed a sort of charm as I played with delight upon the rounds of
it. In this little hut there was a large family of children: I dare not
say how many. My grandmother---whether because too old for field
service, or because she had so faithfully discharged the duties of her
station in early life, I know not---enjoyed the high privilege of living
in a cabin, separate from the quarter, with no other burden than her own
support, and the necessary care of the little children, imposed. She
evidently esteemed it a great fortune to live so. The children were not
her own, but her grandchildren---the children of her daughters. She took
delight in having them around her, and in attending to their few wants.
The practice of separating children from their mothers, and hiring the
latter out at distances too great to admit of their meeting, except at
long intervals, is a marked feature of the cruelty and barbarity of the
slave system. But it is in harmony with the grand aim of slavery, which,
always and everywhere, is to {\protect\hypertarget{38}{}{}}reduce man to
a level with the brute. It is a successful method of obliterating from
the mind and heart of the slave, all just ideas of the sacredness of
\emph{the family}, as an institution.

Most of the children, however, in this instance, being the children of
my grandmother's daughters, the notions of family, and the reciprocal
duties and benefits of the relation, had a better chance of being
understood than where children are placed---as they often are---in the
hands of strangers, who have no care for them, apart from the wishes of
their masters. The daughters of my grandmother were five in number.
Their names were \textsc{Jenny, Esther, Milly, Priscilla}, and
\textsc{Harriet}. The daughter last named was my mother, of whom the
reader shall learn more by-and-by.

Living here, with my dear old grandmother and grandfather, it was a long
time before I knew myself to be \emph{a slave}. I knew many other things
before I knew that. Grandmother and grandfather were the greatest people
in the world to me; and being with them so snugly in their own little
cabin---I supposed it be their own---knowing no higher authority over me
or the other children than the authority of grandmamma, for a time there
was nothing to disturb me; but, as I grew larger and older, I learned by
degrees the sad fact, that the ``little hut,'' and the lot on which it
stood, belonged not to my dear old grandparents, but to some person who
lived a great distance off, and who was called, by grandmother,
"\textsc{Old Master}." I further learned the sadder fact, that not only
the house and lot, but that grandmother herself,
{\protect\hypertarget{39}{}{}}(grandfather was free,) and all the little
children around her, belonged to this mysterious personage, called by
grandmother, with every mark of reverence, ``Old Master.'' Thus early
did clouds and shadows begin to fall upon my path. Once on the
track---troubles never come singly---I was not long in finding out
another fact, still more grievous to my childish heart. I was told that
this ``old master,'' whose name seemed ever to be mentioned with fear
and shuddering, only allowed the children to live with grandmother for a
limited time, and that in fact as soon as they were big enough, they
were promptly taken away, to live with the said ``old master.'' These
were distressing revelations indeed; and though I was quite too young to
comprehend the full import of the intelligence, and mostly spent my
childhood days in gleesome sports with the other children, a shade of
disquiet rested upon me.

The absolute power of this distant ``old master'' had touched my young
spirit with but the point of its cold, cruel iron, and left me something
to brood over after the play and in moments of repose. Grandmammy was,
indeed, at that time, all the world to me; and the thought of being
separated from her, in any considerable time, was more than an unwelcome
intruder. It was intolerable.

Children have their sorrows as well as men and women; and it would be
well to remember this in our dealings with them. \textsc{Slave}-children
\emph{are} children, and prove no exceptions to the general rule. The
liability to be separated from my grandmother, seldom or never to see
her again, haunted me. I dreaded {\protect\hypertarget{40}{}{}}the
thought of going to live with that mysterious ``old master,'' whose name
I never heard mentioned with affection, but always with fear. I look
back to this as among the heaviest of my childhood's sorrows. My
grandmother! my grandmother! and the little hut, and the joyous circle
under her care, but especially \emph{she}, who made us sorry when she
left us but for an hour, and glad on her return,---how could I leave her
and the good old home?

But the sorrows of childhood, like the pleasures of after life, are
transient. It is not even within the power of slavery to write
\emph{indelible} sorrow, at a single dash, over the heart of a child.

{"}The tear down childhood's cheek that flows,\\
Is like the dew-drop on the rose,---\\
When next the summer breeze comes by,\\
And waves the bush,---the flower is dry."

There is, after all, but little difference in the measure of contentment
felt by the slave-child neglected and the slaveholder's child cared for
and petted. The spirit of the All Just mercifully holds the balance for
the young.

The slaveholder, having nothing to fear from impotent childhood, easily
affords to refrain from cruel inflictions; and if cold and hunger do not
pierce the tender frame, the first seven or eight years of the
slave-boy's life are about as full of sweet content as those of the most
favored and petted \emph{white} children of the slaveholder. The
slave-boy escapes many troubles which befall and vex his white brother.
He seldom has to listen to lectures on propriety of
{\protect\hypertarget{41}{}{}}behavior, or on anything else. He is never
chided for handling his little knife and fork improperly or awkwardly,
for he uses none. He is never reprimanded for soiling the table-cloth,
for he takes his meals on the clay floor. He never has the misfortune,
in his games or sports, of soiling or tearing his clothes, for he has
almost none to soil or tear. He is never expected to act like a nice
little gentleman, for he is only a rude little slave. Thus, freed from
all restraint, the slave-boy can be, in his life and conduct, a genuine
boy, doing whatever his boyish nature suggests; enacting, by turns, all
the strange antics and freaks of horses, dogs, pigs, and barn-door
fowls, without in any manner compromising his dignity, or incurring
reproach of any sort. He literally runs wild; has no pretty little
verses to learn in the nursery; no nice little speeches to make for
aunts, uncles, or cousins, to show how smart he is; and, if he can only
manage to keep out of the way of the heavy feet and fists of the older
slave boys, he may trot on, in his joyous and roguish tricks, as happy
as any little heathen under the palm trees of Africa. To be sure, he is
occasionally reminded, when he stumbles in the path of his master---and
this he early learns to avoid---that he is eating his "\emph{white
bread}," and that he will be made to "\emph{see sights}" by-and-by. The
threat is soon forgotten; the shadow soon passes, and our sable boy
continues to roll in the dust, or play in the mud, as bests suits him,
and in the veriest freedom. If he feels uncomfortable, from mud or from
dust, the coast is clear; he can plunge into the river or the pond,
without the ceremony of undressing, or the fear
{\protect\hypertarget{42}{}{}}of wetting his clothes; his little
tow-linen shirt---for that is all he has on---is easily dried; and it
needed ablution as much as did his skin. His food is of the coarsest
kind, consisting for the most part of cornmeal mush, which often finds
it way from the wooden tray to his mouth in an oyster shell. His days,
when the weather is warm, are spent in the pure, open air, and in the
bright sunshine. He always sleeps in airy apartments; he seldom has to
take powders, or to be paid to swallow pretty little sugar-coated pills,
to cleanse his blood, or to quicken his appetite. He eats no candies;
gets no lumps of loaf sugar; always relishes his food; cries but little,
for nobody cares for his crying; learns to esteem his bruises but
slight, because others so esteem them. In a word, he is, for the most
part of the first eight years of his life, a spirited, joyous,
uproarious, and happy boy, upon whom troubles fall only like water on a
duck's back. And such a boy, so far as I can now remember, was the boy
whose life in slavery I am now narrating.

~
