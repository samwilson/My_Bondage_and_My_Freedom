{\protect\hypertarget{173}{}{}}

~

{CHAPTER XIII.}

THE VICISSITUDES OF SLAVE LIFE.

{DEATH OF OLD MASTER'S SON RICHARD, SPEEDILY FOLLOWED BY THAT OF OLD
MASTER---VALUATION AND DIVISION OF ALL THE PROPERTY, INCLUDING THE
SLAVES---MY PRESENCE REQUIRED AT HILLSBOROUGH TO BE APPRAISED AND
ALLOTTED TO A NEW OWNER---MY SAD PROSPECTS AND GRIEF---PARTING THE UTTER
POWERLESSNESS OF THE SLAVES TO DECIDE THEIR OWN DESTINY---A GENERAL
DREAD OF MASTER ANDREW---HIS WICKEDNESS AND CRUELTY---MISS LUCRETIA MY
NEW OWNER---MY RETURN TO BALTIMORE---JOY UNDER THE ROOF OF MASTER
HUGH---DEATH OF MRS. LUCRETIA---MY POOR OLD GRANDMOTHER---HER SAD
FATE---THE LONE COT IN THE WOODS---MASTER THOMAS AULD'S SECOND
MARRIAGE---AGAIN REMOVED FROM MASTER HUGH'S---REASONS FOR REGRETTING THE
CHANGE---A PLAN OF ESCAPE ENTERTAINED.}

\textsc{I must} now ask the reader to go with me a little back in point
of time, in my humble story, and to notice another circumstance that
entered into my slavery experience, and which, doubtless, has had a
share in deepening my horror of slavery, and increasing my hostility
toward those men and measures that practically uphold the slave system.

It has already been observed, that though I was, after my removal from
Col. Lloyd's plantation, in \emph{form} the slave of Master Hugh, I was,
in \emph{fact}, and in \emph{law}, the slave of my old master, Capt.
Anthony. Very well.

In a very short time after I went to Baltimore, my old master's youngest
son, Richard, died; and, in {\protect\hypertarget{174}{}{}}three years
and six months after his death, my old master himself died, leaving only
his son, Andrew, and his daughter, Lucretia, to share his estate. The
old man died while on a visit to his daughter, in Hillsborough, where
Capt. Auld and Mrs. Lucretia now lived. The former, having given up the
command of Col. Lloyd's sloop, was now keeping a store in that town.

Cut off, thus unexpectedly, Capt. Anthony died intestate; and his
property must now be equally divided between his two children, Andrew
and Lucretia.

The valuation and the division of slaves, among contending heirs, is an
important incident in slave life. The character and tendencies of the
heirs, are generally well understood among the slaves who are to be
divided, and all have their aversions and preferences. But, neither
their aversions nor their preferences avail them anything.

On the death of old master, I was immediately sent for, to be valued and
divided with the other property. Personally, my concern was, mainly,
about my possible removal from the home of Master Hugh, which, after
that of my grandmother, was the most endeared to me. But, the whole
thing, as a feature of slavery, shocked me. It furnished me a new
insight into the unnatural power to which I was subjected. My
detestation of slavery, already great, rose with this new conception of
its enormity.

That was a sad day for me, a sad day for little Tommy, and a sad day for
my dear Baltimore mistress and teacher, when I left for the Eastern
Shore, to be valued and divided. We, all three, wept bitterly that
{\protect\hypertarget{175}{}{}}day; for we might be parting, and we
feared we were parting, forever. No one could tell among which pile of
chattels I should be flung. Thus early, I got a foretaste of that
painful uncertainty which slavery brings to the ordinary lot of mortals.
Sickness, adversity and death may interfere with the plans and purposes
of all; but the slave has the added danger of changing homes, changing
hands, and of having separations unknown to other men. Then, too, there
was the intensified degradation of the spectacle. What an assemblage!
Men and women, young and old, married and single; moral and intellectual
beings, in open contempt of their humanity, leveled at a blow with
horses, sheep, horned cattle and swine! Horses and men---cattle and
women---pigs and children---all holding the same rank in the scale of
social existence; and all subjected to the same narrow inspection, to
ascertain their value in gold and silver---the only standard of worth
applied by slaveholders to slaves! How vividly, at that moment, did the
brutalizing power of slavery flash before me! Personality swallowed up
in the sordid idea of property! Manhood lost in chattelhood!

After the valuation, then came the division. This was an hour of high
excitement and distressing anxiety. Our destiny was now to be
\emph{fixed for life}, and we had no more voice in the decision of the
question, than the oxen and cows that stood chewing at the haymow. One
word from the appraisers, against all preferences or prayers, was enough
to sunder all the ties of friendship and affection, and even to separate
husbands and wives, parents and children. We were all
{\protect\hypertarget{176}{}{}}appalled before that power, which, to
human seeming, could bless or blast us in a moment. Added to the dread
of separation, most painful to the majority of the slaves, we all had a
decided horror of the thought of falling into the hands of Master
Andrew. He was distinguished for cruelty and intemperance.

Slaves generally dread to fall into the hands of drunken owners. Master
Andrew was almost a confirmed sot, and had already, by his reckless
mismanagement and profligate dissipation, wasted a large portion of old
master's property. To fall into his hands, was, therefore, considered
merely as the first step toward being sold away to the far south. He
would spend his fortune in a few years, and his farms and slaves would
be sold, we thought, at public outcry; and we should be hurried away to
the cotton fields, and rice swamps, of the sunny south. This was the
cause of deep consternation.

The people of the north, and free people generally, I think, have less
attachment to the places where they are born and brought up, than have
the slaves. Their freedom to go and come, to be here and there, as they
list, prevents any extravagant attachment to any one particular place,
in their case. On the other hand, the slave is a fixture; he has no
choice, no goal, no destination; but is pegged down to a single spot,
and must take root here, or nowhere. The idea of removal elsewhere,
comes, generally, in the shape of a threat, and in punishment of crime.
It is, therefore, attended with fear and dread. A slave seldom thinks of
bettering his condition by being sold, and hence he looks upon
separation from his native place, with
{\protect\hypertarget{177}{}{}}none of the enthusiasm which animates the
bosoms of young freemen, when they contemplate a life in the far west,
or in some distant country where they intend to rise to wealth and
distinction. Nor can those from whom they separate, give them up with
that cheerfulness with which friends and relations yield each other up,
when they feel that it is for the good of the departing one that he is
removed from his native place. Then, too, there is correspondence, and
there is, at least, the hope of reünion, because reünion is
\emph{possible}. But, with the slave, all these mitigating circumstances
are wanting. There is no improvement in his condition
\emph{probable},---no correspondence \emph{possible},---no reünion
attainable. His going out into the world, is like a living man going
into the tomb, who, with open eyes, sees himself buried out of sight and
hearing of wife, children and friends of kindred tie.

In contemplating the likelihoods and possibilities of our circumstances,
I probably suffered more than most of my fellow servants. I had known
what it was to experience kind, and even tender treatment; they had
known nothing of the sort. Life, to them, had been rough and thorny, as
well as dark. They had---most of them---lived on my old master's farm in
Tuckahoe, and had felt the reign of Mr. Plummer's rule. The overseer had
written his character on the living parchment of most of their backs,
and left them callous; my back (thanks to my early removal from the
plantation to Baltimore,) was yet tender. I had left a kind mistress at
Baltimore, who was almost a mother to me. She was in tears when we
parted, and the probabilities of ever seeing her again, trembling
{\protect\hypertarget{178}{}{}} in the balance as they did, could not be
viewed without alarm and agony. The thought of leaving that kind
mistress forever, and, worse still, of being the slave of Andrew
Anthony---a man who, but a few days before the division of the property,
had, in my presence, seized my brother Perry by the throat, dashed him
on the ground, and with the heel of his boot stamped him on the head,
until the blood gushed from his nose and ears---was terrible! This
fiendish proceeding had no better apology than the fact, that Perry had
gone to play, when Master Andrew wanted him for some trifling service.
This cruelty, too, was of a piece with his general character. After
inflicting his heavy blows on my brother, on observing me looking at him
with intense astonishment, he said, "\emph{That} is the way I will serve
you, one of these days;" meaning, no doubt, when I should come into his
possession. This threat, the reader may well suppose, was not very
tranquilizing to my feelings. I could see that he really thirsted to get
hold of me. But I was there only for a few days. I had not received any
orders, and had violated none, and there was, therefore, no excuse for
flogging me.

At last, the anxiety and suspense were ended; and they ended, thanks to
a kind Providence, in accordance with my wishes. I fell to the portion
of Mrs. Lucretia---the dear lady who bound up my head, when the savage
Aunt Katy was adding to my sufferings her bitterest maledictions.

Capt. Thomas Auld and Mrs. Lucretia at once decided on my return to
Baltimore. They knew how sincerely and warmly Mrs. Hugh Auld was
attached {\protect\hypertarget{179}{}{}}to me, and how delighted Mr.
Hugh's son would be to have me back; and, withal, having no immediate
use for one so young, they willingly let me off to Baltimore.

I need not stop here to narrate my joy on returning to Baltimore, nor
that of little Tommy; nor the tearful joy of his mother; nor the evident
satisfaction of Master Hugh. I was just one month absent from Baltimore,
before the matter was decided; and the time really seemed full six
months.

One trouble over, and on comes another. The slave's life is full of
uncertainty. I had returned to Baltimore but a short time, when the
tidings reached me, that my kind friend, Mrs. Lucretia, who was only
second in my regard to Mrs. Hugh Auld, was dead, leaving her husband and
only one child---a daughter, named Amanda.

Shortly after the death of Mrs. Lucretia, strange to say, Master Andrew
died, leaving his wife and one child. Thus, the whole family of Anthonys
was swept away; only two children remained. All this happened within
five years of my leaving Col. Lloyd's.

No alteration took place in the condition of the slaves, in consequence
of these deaths, yet I could not help feeling less secure, after the
death of my friend, Mrs. Lucretia, than I had done during her life.
While she lived, I felt that I had a strong friend to plead for me in
any emergency. Ten years ago, while speaking of the state of things in
our family, after the events just named, I used this language:

"Now all the property of my old master, slaves included,
{\protect\hypertarget{180}{}{}}was in the hands of strangers---strangers
who had nothing-Co do in accumulating it. Not a slave was left free. All
remained slaves, from the youngest to the oldest. If any one thing in my
experience, more than another, served to deepen my conviction of the
infernal character of slavery, and to fill me with unutterable loathing
of slaveholders, it was their base ingratitude to my poor old
grandmother. She had served my old master faithfully from youth to old
age. She had been, the source of all his wealth; she had peopled his
plantation with slaves; she had become a great-grandmother in his
service. She had rocked him in infancy, attended him in childhood,
served him through life, and at his death wiped from his icy brow the
cold death-sweat, and closed his eyes forever. She was nevertheless left
a slave---a slave for life---a slave in the hands of strangers; and in
their hands she saw her children, her grandchildren, and her
great-grandchildren, divided, like so many sheep, without being
gratified with the small privilege of a single word, as to their or her
own destiny. And, to cap the climax of their base ingratitude and
fiendish barbarity, my grandmother, who was now very old, having
outlived my old master and all his children, having seen the beginning
and end of all of them, and her present owners finding she was of but
little value, her frame already racked with the pains of old age, and
complete helplessness fast stealing over her once active limbs, they
took her to the woods, built her a little hut, put up a little
mud-chimney, and then made her welcome to the privilege of supporting
herself there in perfect loneliness; thus virtually turning her out to
die! If my poor old grandmother now lives, she lives to suffer in utter
loneliness; she lives to remember and mourn over the loss of children,
the loss of grandchildren, and the loss of great-grandchildren. They
are, in the language of the slave's poet, Whittier,---

{\protect\hypertarget{181}{}{}}

{{'}Gone, gone, sold and gone,\\
To the rice swamp dank and lone,\\
Where the slave-whip ceaseless swings,\\
Where the noisome insect stings,\\
Where the fever-demon strews\\
Poison with the falling dews,\\
Where the sickly sunbeams glare\\
Through the hot and misty air:---\\
{Gone, gone, sold and gone}\\
{To the rice swamp dank and lone,}\\
{From Virginia hills and waters---}\\
{Woe is me, my stolen daughters!'}}

``The hearth is desolate. The children, the unconscious children, who
once sang and danced in her presence, are gone. She gropes her way, in
the darkness of age, for a drink of water. Instead of the voices of her
children, she hears by day the moans of the dove, and by night the
screams of the hideous owl. All is gloom. The grave is at the door. And
now, when weighed down by the pains and aches of old age, when the head
inclines to the feet, when the beginning and ending of human existence
meet, and helpless infancy and painful old age combine together at this
time, this most needful time, the time for the exercise of that
tenderness and affection which children only can exercise toward a
declining parent my poor old grandmother, the devoted mother of twelve
children, is left all alone, in yonder little hut, before a few dim
embers.''

Two years after the death of Mrs. Lucretia, Master Thomas married his
second wife. Her name was Rowena Hamilton, the eldest daughter of Mr.
"William Hamilton, a rich slaveholder on the Eastern Shore of Maryland,
who lived about five miles from St. Michael's, the then place of my
master's residence.

Not long after his marriage, Master Thomas had a
{\protect\hypertarget{182}{}{}}misunderstanding with Master Hugh, and,
as a means of punishing his brother, he ordered him to send me home.

As the ground of misunderstanding will serve to illustrate the character
of southern chivalry, and humanity, I will relate it.

Among the children of my Aunt Milly, was a daughter, named Henny. When
quite a child, Henny had fallen into the fire, and had burnt her hands
so bad that they were of very little use to her. Her fingers were drawn
almost into the palms of her hands. She could make out to do something,
but she was considered hardly worth the having---of little more value
than a horse with a broken leg. This unprofitable piece of human
property, ill shapen, and disfigured, Capt. Auld sent off to Baltimore,
making his brother Hugh welcome to her services.

After giving poor Henny a fair trial, Master Hugh and his wife came to
the conclusion, that they had no use for the crippled servant, and they
sent her back to Master Thomas. This, the latter took as an act of
ingratitude, on the part of his brother; and, as a mark of his
displeasure, he required him to send me immediately to St. Michael's,
saying, if he cannot keep "\emph{Hen,}" he shall not have "\emph{Fred.}"

Here was another shock to my nerves, another breaking up of my plans,
and another severance of my religious and social alliances. I was now a
big boy. I had become quite useful to several young colored men, who had
made me their teacher. I had taught some of them to read, and was
accustomed to spend many of my leisure hours with them. Our
{\protect\hypertarget{183}{}{}}attachment was strong, and I greatly
dreaded the separation. But regrets, especially in a slave, are
unavailing. I was only a slave; my wishes were nothing, and my happiness
was the sport of my masters.

My regrets at now leaving Baltimore, were not for the same reasons as
when I before left that city, to be valued and handed over to my proper
owner. My home was not now the pleasant place it had formerly been. A
change had taken place, both in Master Hugh, and in his once pious and
affectionate wife. The influence of brandy and bad company on him, and
the influence of slavery and social isolation upon her, had wrought
disastrously upon the characters of both. Thomas was no longer ``little
Tommy,'' but was a big boy, and had learned to assume the airs of his
class toward me. My condition, therefore, in the house of Master Hugh,
was not, by any means, so comfortable as in former years. My attachments
were now outside of our family. They were felt to those to whom I
\emph{imparted} instruction, and to those little white boys from whom I
\emph{received} instruction. There, too, was my dear old father, the
pious Lawson, who was, in christian graces, the very counterpart of
``Uncle'' Tom. The resemblance is so perfect, that he might have been
the original of Mrs. Stowe's christian hero. The thought of leaving
these dear friends, greatly troubled me, for I was going without the
hope of ever returning to Baltimore again; the feud between Master Hugh
and his brother being bitter and irreconcilable, or, at least, supposed
to be so.

In addition to thoughts of friends from whom I was parting, as I
supposed, \emph{forever}, I had the grief of
{\protect\hypertarget{184}{}{}}neglected chances of escape to brood
over. I had put off running away, until now I was to be placed where the
opportunities for escaping were much fewer than in a large city like
Baltimore.

On my way from Baltimore to St. Michael's, down the Chesapeake bay, our
sloop---the Amanda---was passed by the steamers plying between that city
and Philadelphia, and I watched the course of those steamers, and, while
going to St. Michael's, I formed a plan to escape from slavery; of which
plan, and matters connected therewith the kind reader shall learn more
hereafter.

~
