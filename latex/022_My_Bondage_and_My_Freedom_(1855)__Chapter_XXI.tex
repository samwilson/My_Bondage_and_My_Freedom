{\protect\hypertarget{321}{}{}}

~

{CHAPTER XXI.}

MY ESCAPE FROM SLAVERY.

{CLOSING INCIDENTS OF MY ``LIFE AS A SLAVE''---REASONS WHY FULL
PARTICULARS OF THE MANNER OF MY ESCAPE WILL NOT BE GIVEN---CRAFTINESS
AND MALICE OF SLAVEHOLDERS---SUSPICION OF AIDING A SLAVE'S ESCAPE ABOUT
AS DANGEROUS AS POSITIVE EVIDENCE---WANT OF WISDOM SHOWN IN PUBLISHING
DETAILS OF THE ESCAPE OF FUGITIVES---PUBLISHED ACCOUNTS REACH THE
MASTERS, NOT THE SLAVES---SLAVEHOLDERS STIMULATED TO GREATER
WATCHFULNESS---AUTHOR'S CONDITION---DISCONTENT---SUSPICIONS IMPLIED BY
MASTER HUGH'S MANNER, WHEN RECEIVING MY WAGES---HIS OCCASIONAL
GENEROSITY!---DIFFICULTIES IN THE WAY OF ESCAPE---EVERY AVENUE
GUARDED---PLAN TO OBTAIN MONEY---AUTHOR ALLOWED TO HIRE HIS TIME---A
GLEAM OF HOPE---ATTENDS CAMP-MEETING, WITHOUT PERMISSION---ANGER OF
MASTER HUGH THEREAT---THE RESULT MY PLANS OF ESCAPE ACCELERATED
THEREBY---THE DAY FOR MY DEPARTURE FIXED---HARASSED BY DOUBTS AND FEARS
PAINFUL THOUGHTS OF SEPARATION FROM FRIENDS---THE ATTEMPT MADE---ITS
SUCCESS.}

\textsc{I will} now make the kind reader acquainted with the closing
incidents of my ``Life as a Slave,'' having already trenched upon the
limit allotted to my ``Life as a Freeman.'' Before, however, proceeding
with this narration, it is, perhaps, proper that I should frankly state,
in advance, my intention to withhold a part of the facts connected with
my escape from slavery. There are reasons for this suppression, which I
trust the reader will deem altogether valid. It may be easily conceived,
that a full and complete statement of all the facts pertaining to the
flight of a bondman, might implicate and embarrass some who may
{\protect\hypertarget{322}{}{}}have, wittingly or unwittingly, assisted
him; and no one can wish me to involve any man or woman who has
befriended me, even in the liability of embarrassment or trouble.

Keen is the scent of the slaveholder; like the fangs of the rattlesnake,
his malice retains its poison long; and, although it is now nearly
seventeen years since I made my escape, it is well to be careful, in
dealing with the circumstances relating to it. Were I to give but a
shadowy outline of the process adopted, with characteristic aptitude,
the crafty and malicious among the slaveholders might, possibly, hit
upon the track I pursued, and involve some one in suspicion, which, in a
slave state, is about as bad as positive evidence. The colored man,
there, must not only shun evil, but shun the very \emph{appearance} of
evil, or be condemned as a criminal. A slaveholding community has a
peculiar taste for ferreting out offenses against the slave system,
justice there being more sensitive in its regard for the peculiar rights
of this system, than for any other interest or institution. By stringing
together a train of events and circumstances, even if I were not very
explicit, the means of escape might be ascertained, and, possibly, those
means be rendered, thereafter, no longer available to the
liberty-seeking children of bondage I have left behind me. No
anti-slavery man can wish me to do anything favoring such results, and
no slaveholding reader has any right to expect the impartment of such
information.

While, therefore, it would afford me pleasure, and perhaps would
materially add to the interest of my
{\protect\hypertarget{323}{}{}}story, were I at liberty to gratify a
curiosity which I know to exist in the minds of many, as to the manner
of my escape, I must deprive myself of this pleasure, and the curious of
the gratification, which such a statement of facts would afford. I would
allow myself to suffer under the greatest imputations that evil minded
men might suggest, rather than exculpate myself by an explanation, and
thereby run the hazard of closing the slightest avenue by which a
brother in suffering might clear himself of the chains and fetters of
slavery.

The practice of publishing every new invention by which a slave is known
to have escaped from slavery, has neither wisdom nor necessity to
sustain it. Had not Henry Box Brown and his friends attracted
slaveholding attention to the manner of his escape, we might have had a
thousand \emph{Box Browns} per annum. The singularly original plan
adopted by William and Ellen Crafts, perished with the first using,
because every slaveholder in the land was apprised of it. The \emph{salt
water slave} who hung in the guards of a steamer, being washed three
days and three nights---like another Jonah---by the waves of the sea,
has, by the publicity given to the circumstance, set a spy on the guards
of every steamer departing from southern ports.

I have never approved of the very public manner, in which some of our
western friends have conducted what \emph{they} call the
"\emph{Under-ground Railroad}," but which, I think, by their open
declarations, has been made, most emphatically, the "\emph{Upper}-ground
Railroad." Its stations are far better known to the slaveholders than to
the slaves. I honor those good men and women for their noble daring, in
willingly {\protect\hypertarget{324}{}{}}subjecting themselves to
persecution, by openly avowing their participation in the escape of
slaves; nevertheless, the good resulting from such avowals, is of a very
questionable character. It may kindle an enthusiasm, very pleasant to
inhale; but that is of no practical benefit to themselves, nor to the
slaves escaping. Nothing is more evident, than that such disclosures are
a positive evil to the slaves remaining, and seeking to escape. In
publishing such accounts, the anti-slavery man addresses the
slaveholder, \emph{not the slave;} he stimulates the former to greater
watchfulness, and adds to his facilities for capturing his slave. We owe
something to the slaves, south of Mason and Dixon's line, as well as to
those north of it; and, in discharging the duty of aiding the latter, on
their way to freedom, we should be careful to do nothing which would be
likely to hinder the former, in making their escape from slavery. Such
is my detestation of slavery, that I would keep the merciless
slaveholder profoundly ignorant of the means of flight adopted by the
slave. He should be left to imagine himself surrounded by myriads of
invisible tormentors, ever ready to snatch, from his infernal grasp, his
trembling prey. In pursuing his victim, let him be left to feel his way
in the dark; let shades of darkness, commensurate with his crime, shut
every ray of light from his pathway; and let him be made to feel, that,
at every step he takes, with the hellish purpose of reducing a brother
man to slavery, he is running the frightful risk of having his hot
brains dashed out by an invisible hand.

But, enough of this. I will now proceed to the
{\protect\hypertarget{325}{}{}}statement of those facts, connected with
my escape, for which I am alone responsible, and for which no one can be
made to suffer but myself.

My condition in the year (1838) of my escape, was, comparatively, a free
and easy one, so far, at least, as the wants of the physical man were
concerned; but the reader will bear in mind, that my troubles from the
beginning, have been less physical than mental, and he will thus be
prepared to find, after what is narrated in the previous chapters, that
slave life was adding nothing to its charms for me, as I grew older, and
became better acquainted with it. The practice, from week to week, of
openly robbing me of all my earnings, kept the nature and character of
slavery constantly before me. I could be robbed by \emph{indirection},
but this was \emph{too} open and barefaced to be endured. I could see no
reason why I should, at the end of each week, pour the reward of my
honest toil into the purse of any man. The thought itself vexed me, and
the manner in which Master Hugh received my wages, vexed me more than
the original wrong. Carefully counting the money and rolling it out,
dollar by dollar, he would look me in the face, as if he would search my
heart as well as my pocket, and reproachfully ask me, "\emph{Is that
all?}``---implying that I had, perhaps, kept back part of my wages; or,
if not so, the demand was made, possibly, to make me feel, that, after
all, I was an ''unprofitable servant." Draining me of the last cent of
my hard earnings, he would, however, occasionally---when I brought home
an extra large sum---dole out to me a sixpence or a shilling, with a
view, perhaps, of kindling up my
{\protect\hypertarget{326}{}{}}gratitude; but this practice had the
opposite effect---it was an admission of \emph{my right to the whole
sum}. The fact, that he gave me any part of my wages, was proof that he
suspected that I had a right \emph{to the whole of them}. I always felt
uncomfortable, after having received anything in this way, for I feared
that the giving me a few cents, might, possibly, ease his conscience,
and make him feel himself a pretty honorable robber, after all!

Held to a strict account, and kept under a close watch---the old
suspicion of my running away not having been entirely removed---escape
from slavery, even in Baltimore, was very difficult. The railroad from
Baltimore to Philadelphia was under regulations so stringent, that even
\emph{free} colored travelers were almost excluded. They must have
\emph{free} papers; they must be measured and carefully examined, before
they were allowed to enter the cars; they only went in the day time,
even when so examined. The steamboats were under regulations equally
stringent. All the great turnpikes, leading northward, were beset with
kidnappers, a class of men who watched the newspapers for advertisements
for runaway slaves, making their living by the accursed reward of slave
hunting.

My discontent grew upon me, and I was on the look-out for means of
escape. With money, I could easily have managed the matter, and,
therefore, I hit upon the plan of soliciting the privilege of hiring my
time. It is quite common, in Baltimore, to allow slaves this privilege,
and it is the practice, also, in New Orleans. A slave who is considered
{\protect\hypertarget{327}{}{}}trust-worthy, can, by paying his master a
definite sum regularly, at the end of each week, dispose of his time as
he likes. It so happened that I was not in very good odor, and I was far
from being a trust-worthy slave. Nevertheless, I watched my opportunity
when Master Thomas came to Baltimore, (for I was still his property,
Hugh only acted as his agent,) in the spring of 1838, to purchase his
spring supply of goods, and applied to him, directly, for the
much-coveted privilege of hiring my time. This request Master Thomas
unhesitatingly refused to grant; and he charged me, with some sternness,
with inventing this stratagem to make my escape. He told me, "I could go
\emph{nowhere} but he could catch me; and, in the event of my running
away, I might be assured he should spare no pains in his efforts to
recapture me. He recounted, with a good deal of eloquence, the many kind
offices he had done me, and exhorted me to be contented and obedient.
``Lay out no plans for the future,'' said he. ``If you behave yourself
properly, I will take care of you.'' Now, kind and considerate as this
offer was, it failed to soothe me into repose. In spite of Master
Thomas, and, I may say, in spite of myself, also, I continued to think,
and worse still, to think almost exclusively about the injustice and
wickedness of slavery. No effort of mine or of his could silence this
trouble-giving thought, or change my purpose to run away.

About two months after applying to Master Thomas for the privilege of
hiring my time, I applied to Master Hugh for the same liberty, supposing
him to be unacquainted with the fact that I had made a
{\protect\hypertarget{328}{}{}}similar application to Master Thomas, and
had been refused. My boldness in making this request, fairly astounded
him at the first. He gazed at me in amazement. But I had many good
reasons for pressing the matter; and, after listening to them awhile, he
did not absolutely refuse, but told me he would think of it. Here, then,
was a gleam of hope. Once master of my own time, I felt sure that I
could make, over and above my obligation to him, a dollar or two every
week. Some slaves have made enough, in this way, to purchase their
freedom. It is a sharp spur to industry; and some of the most
enterprising colored men in Baltimore hire themselves in this way. After
mature reflection---as I must suppose it was---Master Hugh granted me
the privilege in question, on the following terms: I was to be allowed
all my time; to make all bargains for work; to find my own employment,
and to collect my own wages; and, in return for this liberty, I was
required, or obliged, to pay him three dollars at the end of each week,
and to board and clothe myself, and buy my own calking tools. A failure
in any of these particulars would put an end to my privilege. This was a
hard bargain. The wear and tear of clothing, the losing and breaking of
tools, and the expense of board, made it necessary for me to earn at
least six dollars per week, to keep even with the world. All who are
acquainted with calking, know how uncertain and irregular that
employment is. It can be done to advantage only in dry weather, for it
is useless to put wet oakum into a seam. Rain or shine, however, work or
no work, at the end of each week the money must be forthcoming.

{\protect\hypertarget{329}{}{}}Master Hugh seemed to be very much
pleased, for a time, with this arrangement; and well he might be, for it
was decidedly in his favor. It relieved him of all anxiety concerning
me. His money was sure. He had armed my love of liberty with a lash and
a driver, far more efficient than any I had before known; and, while he
derived all the benefits of slaveholding by the arrangement, without its
evils, I endured all the evils of being a slave, and yet suffered all
the care and anxiety of a responsible freeman. ``Nevertheless,'' thought
I, ``it is a valuable privilege---another step in my career toward
freedom.'' It was something even to be permitted to stagger under the
disadvantages of liberty, and I was determined to hold on to the newly
gained footing, by all proper industry. I was ready to work by night as
well as by day; and being in the enjoyment of excellent health, I was
able not only to meet my current expenses, but also to lay by a small
sum at the end of each week. All went on thus, from the month of May
till August; then---for reasons which will become apparent as I
proceed---my much valued liberty was wrested from me.

During the week previous to this (to me) calamitous event, I had made
arrangements with a few young friends, to accompany them, on Saturday
night, to a camp-meeting, held about twelve miles from Baltimore. On the
evening of our intended start for the camp-ground, something occurred in
the ship yard where I was at work, which detained me unusually late, and
compelled me either to disappoint my young friends, or to neglect
carrying my weekly dues to Master Hugh. Knowing that I had the money,
and {\protect\hypertarget{330}{}{}}could hand it to him on another day,
I decided to go to camp-meeting, and to pay him the three dollars, for
the past week, on my return. Once on the camp-ground, I was induced to
remain one day longer than I had intended, when I left home. But, as
soon as I returned, I went straight to his house on Fell street, to hand
him his (my) money. Unhappily, the fatal mistake had been committed. I
found him exceedingly angry. He exhibited all the signs of apprehension
and wrath, which a slaveholder may be surmised to exhibit on the
supposed escape of a favorite slave. ``You rascal! I have a great mind
to give you a severe whipping. How dare you go out of the city without
first asking and obtaining my permission?'' ``Sir,'' said I, ``I hired
my time and paid you the price you asked for it. I did not know that it
was any part of the bargain that I should ask you when or where I should
go.''

``You did not know, you rascal! You are bound to show yourself here
every Saturday night.'' After reflecting, a few moments, he became
somewhat cooled down; but, evidently greatly troubled, he said, ``Now,
you scoundrel! you have done for yourself; you shall hire your time no
longer. The next thing I shall hear of, will be your running away. Bring
home your tools and your clothes, at once. I'll teach you how to go off
in this way.''

Thus ended my partial freedom. I could hire my time no longer; and I
obeyed my master's orders at once. The little taste of liberty which I
had had---although as the reader will have seen, it was far from being
unalloyed---by no means enhanced my
{\protect\hypertarget{331}{}{}}contentment with slavery. Punished thus
by Master Hugh, it was now my turn to punish him. ``Since,'' thought I,
"you \emph{will} make a slave of me, I will await your orders in all
things;" and, instead of going to look for work on Monday morning, as I
had formerly done, I remained at home during the entire week, without
the performance of a single stroke of work. Saturday night came, and he
called upon me, as usual, for my wages. I, of course, told him I had
done no work, and had no wages. Here we were at the point of coming to
blows. His wrath had been accumulating during the whole week; for he
evidently saw that I was making no effort to get work, but was most
aggravatingly awaiting his orders, in all things. As I look back to this
behavior of mine, I scarcely know what possessed me, thus to trifle with
those who had such unlimited power to bless or to blast me. Master Hugh
raved and swore his determination to "\emph{get hold of me;}" but,
wisely for \emph{him}, and happily for \emph{me}, his wrath only
employed those very harmless, impalpable missiles, which roll from a
limber tongue. In my desperation, I had fully made up my mind to measure
strength with Master Hugh, in case he should undertake to execute his
threats. I am glad there was no necessity for this; for resistance to
him could not have ended so happily for me, as it did in the case of
Covey. He was not a man to be safely resisted by a slave; and I freely
own, that in my conduct toward him, in this instance, there was more
folly than wisdom. Master Hugh closed his reproofs, by telling me that,
hereafter, I need give myself no uneasiness about getting work; that he
"would, himself, see to {\protect\hypertarget{332}{}{}}getting work for
me, and enough of it, at that." This threat I confess had some terror in
it; and, on thinking the matter over, during the Sunday, I resolved, not
only to save him the trouble of getting me work, but that, upon the
third day of September, I would attempt to make my escape from slavery.
The refusal to allow me to hire my time, therefore, hastened the period
of my flight. I had three weeks, now, in which to prepare for my
journey.

Once resolved, I felt a certain degree of repose, and on Monday, instead
of waiting for Master Hugh to seek employment for me, I was up by break
of day, and off to the ship yard of Mr. Butler, on the City Block, near
the draw-bridge. I was a favorite with Mr. B., and, young as I was, I
had served as his foreman on the float stage, at calking. Of course, I
easily obtained work, and, at the end of the week---which by the way was
exceedingly fine---I brought Master Hugh nearly nine dollars. The effect
of this mark of returning good sense, on my part, was excellent. He was
very much pleased; he took the money, commended me, and told me I might
have done the same thing the week before. It is a blessed thing that the
tyrant may not always know the thoughts and purposes of his victim.
Master Hugh little knew what my plans were. The going to camp-meeting
without asking his permission---the insolent answers made to his
reproaches---the sulky deportment the week after being deprived of the
privilege of hiring my time---had awakened in him the suspicion that I
might be cherishing disloyal purposes. My object, therefore, in working
steadily, {\protect\hypertarget{333}{}{}}was to remove suspicion, and in
this I succeeded admirably. He probably thought I was never better
satisfied with my condition, than at the very time I was planning my
escape. The second week passed, and again I carried him my full week's
wages---\emph{nine dollars;} and so well pleased was he, that he gave me
\textsc{twenty-five cents}! and ``bade me make good use of it!'' I told
him I would, for one of the uses to which I meant to put it, was to pay
my fare on the underground railroad.

Things without went on as usual; but I was passing through the same
internal excitement and anxiety which I had experienced two years and a
half before. The failure, in that instance, was not calculated to
increase my confidence in the success of this, my second attempt; and I
knew that a second failure could not leave me where my first did---I
must either get to the \emph{far north}, or be sent to the \emph{far
south}. Besides the exercise of mind from this state of facts, I had the
painful sensation of being about to separate from a circle of honest and
warm hearted friends, in Baltimore. The thought of such a separation,
where the hope of ever meeting again is excluded, and where there can be
no correspondence, is very painful. It is my opinion, that thousands
would escape from slavery who now remain there, but for the strong cords
of affection that bind them to their families, relatives and friends.
The daughter is hindered from escaping, by the love she bears her
mother, and the father, by the love he bears his children; and so, to
the end of the chapter. I had no relations in Baltimore, and I saw no
probability of ever living in the
{\protect\hypertarget{334}{}{}}neighborhood of sisters and brothers; but
the thought of leaving my friends, was among the strongest obstacles to
my running away. The last two days of the week---Friday and
Saturday---were spent mostly in collecting my things together, for my
journey. Having worked four days that week, for my master, I handed him
six dollars, on Saturday night. I seldom spent my Sundays at home; and,
for fear that something might be discovered in my conduct, I kept up my
custom, and absented myself all day. On Monday, the third day of
September, 1838, in accordance with my resolution, I bade farewell to
the city of Baltimore, and to that slavery which had been my abhorrence
from childhood.

How I got away---in what direction I traveled---whether by land or by
water; whether with or without assistance---must, for reasons already
mentioned, remain unexplained.
