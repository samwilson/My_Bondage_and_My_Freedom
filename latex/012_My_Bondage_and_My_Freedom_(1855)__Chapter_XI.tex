{}

~

{CHAPTER XI.}

``A CHANGE CAME O'ER THE SPIRIT OF MY DREAM.''

{HOW THE AUTHOR LEARNED TO READ---MY MISTRESS---HER SLAVEHOLDING
DUTIES---THEIR DEPLORABLE EFFECTS UPON HER ORIGINALLY NOBLE NATURE---THE
CONFLICT IN HER MIND---HER FINAL OPPOSITION TO MY LEARNING TO READ---TOO
LATE---SHE HAD GIVEN ME THE ``INCH,'' I WAS RESOLVED TO TAKE THE
``ELL''---HOW I PURSUED MY EDUCATION---MY TUTORS---HOW I COMPENSATED
THEM---WHAT PROGRESS I MADE---SLAVERY---WHAT I HEARD SAID ABOUT
IT---THIRTEEN YEARS OLD---THE ``COLUMBIAN ORATOR''---A RICH SCENE---A
DIALOGUE---SPEECHES OF CHATHAM, SHERIDAN, PITT AND FOX---KNOWLEDGE EVER
INCREASING---MY EYES OPENED---LIBERTY---HOW I PINED FOR IT---MY
SADNESS---THE DISSATISFACTION OF MY POOR MISTRESS---MY HATRED OF
SLAVERY---ONE UPAS TREE OVERSHADOWED US BOTH.}

\textsc{I lived} in the family of Master Hugh, at Baltimore, seven
years, during which time---as the almanac makers say of the weather---my
condition was variable. The most interesting feature of my history here,
was my learning to read and write, under somewhat marked disadvantages.
In attaining this knowledge, I was compelled to resort to indirections
by no means congenial to my nature, and which were really humiliating to
me. My mistress---who, as the reader has already seen, had begun to
teach me---was suddenly checked in her benevolent design, by the strong
advice of her husband. In faithful compliance with this advice, the good
lady had not only ceased to instruct me, herself, but had set her face
as a flint against my learning to read by any means. It is due,
{}however, to my mistress to say, that she did not adopt this course in
all its stringency at the first. She either thought it unnecessary, or
she lacked the depravity indispensable to shutting me up in mental
darkness. It was, at least, necessary for her to have some training, and
some hardening, in the exercise of the slaveholder's prerogative, to
make her equal to forgetting my human nature and character, and to
treating me as a thing destitute of a moral or an intellectual nature.
Mrs. Auld---my mistress---was, as I have said, a most kind and
tender-hearted woman; and, in the humanity of her heart, and the
simplicity of her mind, she set out, when I first went to live with her,
to treat me as she supposed one human being ought to treat another.

It is easy to see, that, in entering upon the duties of a slaveholder,
some little experience is needed. Nature has done almost nothing to
prepare men and women to be either slaves or slaveholders. Nothing but
rigid training, long persisted in, can perfect the character of the one
or the other. One cannot easily forget to love freedom; and it is as
hard to cease to respect that natural love in our fellow creatures. On
entering upon the career of a slaveholding mistress, Mrs. Auld was
singularly deficient; nature, which fits nobody for such an office, had
done less for her than any lady I had known. It was no easy matter to
induce her to think and to feel that the curly-headed boy, who stood by
her side, and even leaned on her lap; who was loved by little Tommy, and
who loved little Tommy in turn; sustained to her only the relation of a
chattel. I was \emph{more} than that, and she felt {}me to be more than
that. I could talk and sing; I could laugh and weep; I could reason and
remember; I could love and hate. I was human, and she, dear lady, knew
and felt me to be so. How could she, then, treat me as a brute, without
a mighty struggle with all the noble powers of her own soul. That
struggle came, and the will and power of the husband was victorious. Her
noble soul was overthrown; but, he that overthrew it did not, himself,
escape the consequences. He, not less than the other parties, was
injured in his domestic peace by the fall.

When I went into their family, it was the abode of happiness and
contentment. The mistress of the house was a model of affection and
tenderness. Her fervent piety and watchful uprightness made it
impossible to see her without thinking and feeling---"\emph{that woman
is a christian.}" There was no sorrow nor suffering for which she had
not a tear, and there was no innocent joy for which she had not a smile.
She had bread for the hungry, clothes for the naked, and comfort for
every mourner that came within her reach. Slavery soon proved its
ability to divest her of these excellent qualities, and her home of its
early happiness. Conscience cannot stand much violence. Once thoroughly
broken down, \emph{who} is he that can repair the damage? It may be
broken toward the slave, on Sunday, and toward tie master on Monday. It
cannot endure such shocks. It must stand entire, or it does not stand at
all. If my condition waxed bad, that of the family waxed not better. The
first step, in the wrong direction, was the violence done to {}nature
and to conscience, in arresting the benevolence that would have
enlightened my young mind. In ceasing to instruct me, she must begin to
justify herself \emph{to} herself; and, once consenting to take sides in
such a debate, she was riveted to her position. One needs very little
knowledge of moral philosophy, to see \emph{where} my mistress now
landed. She finally became even more violent in her opposition to my
learning to read, than was her husband himself. She was not satisfied
with simply doing as \emph{well} as her husband had commanded her, but
seemed resolved to better his instruction. Nothing appeared to make my
poor mistress---after her turning toward the downward path---more angry,
than seeing me, seated in some nook or corner, quietly reading a book or
a newspaper. I have had her rush at me, with the utmost fury, and snatch
from my hand such newspaper or book, with something of the wrath and
consternation which a traitor might be supposed to feel on being
discovered in a plot by some dangerous spy.

Mrs. Auld was an apt woman, and the advice of her husband, and her own
experience, soon demonstrated, to her entire satisfaction, that
education and slavery are incompatible with each other. When this
conviction was thoroughly established, I was most narrowly watched in
all my movements. If I remained in a separate room from the family for
any considerable length of time, I was sure to be suspected of having a
book, and was at once called upon to give an account of myself. All
this, however, was entirely \emph{too late.} The first, and never to be
retraced, step had been taken. In teaching me the alphabet, in the
{}days of her simplicity and kindness, my mistress had given me the
"\emph{inch}," and now, no ordinary precaution could prevent me from
taking the "\emph{ell}."

Seized with a determination to learn to read, at any cost, I hit upon
many expedients to accomplish the desired end. The plea which I mainly
adopted, and the one by which I was most successful, was that of using
my young white playmates, with whom I met in the street, as teachers. I
used to carry, almost constantly, a copy of Webster's spelling book in
my pocket; and, when sent of errands, or when play time was allowed me,
I would step, with my young friends, aside, and take a lesson in
spelling. I generally paid my \emph{tuition fee} to the boys, with
bread, which I also carried in my pocket. For a single biscuit, any of
my hungry little comrades would give me a lesson more valuable to me
than bread. Not every one, however, demanded this consideration, for
there were those who took pleasure in teaching me, whenever I had a
chance to be taught by them. I am strongly tempted to give the names of
two or three of those little boys, as a slight testimonial of the
gratitude and affection I bear them, but prudence forbids; not that it
would injure me, but it might, possibly, embarrass them; for it is
almost an unpardonable offense to do any thing, directly or indirectly,
to promote a slave's freedom, in a slave state. It is enough to say, of
my warm-hearted little play fellows, that they lived on Philpot street,
very near Durgin \& Bailey's shipyard.

Although slavery was a delicate subject, and very cautiously talked
about among grown up people in Maryland, I frequently talked about
it---and that very {}freely---with the white boys. I would, sometimes,
say to them, while seated on a curb stone or a cellar door, ``I wish I
could be free, as you will be when you get to be men.'' ``You will be
free, you know, as soon as you are twenty-one, and can go where you
like, but I am a slave for life. Have I not as good a right to be free
as you have?'' Words like these, I observed, always troubled them; and I
had no small satisfaction in wringing from the boys, occasionally, that
fresh and bitter condemnation of slavery, that springs from nature,
unseared and unperverted. Of all consciences, let me have those to deal
with which have not been bewildered by the cares of life. I do not
remember ever to have met with a \emph{boy}, while I was in slavery, who
defended the slave system; but I have often had boys to console me, with
the hope that something would yet occur, by which I might be made free.
Over and over again, they have told me, that "they believed \emph{I} had
as good a right to be free as \emph{they} had;" and that ``they did not
believe God ever made any one to be a slave.'' The reader will easily
see, that such little conversations with my play fellows, had no
tendency to weaken my love of liberty, nor to render me contented with
my condition as a slave.

When I was about thirteen years old, and had succeeded in learning to
read, every increase of knowledge, especially respecting the
\textsc{Free States}, added something to the almost intolerable burden
of the thought---"\textsc{I Am a Slave for life}." To my bondage I saw
no end. It was a terrible reality, and I shall never be able to tell how
sadly that thought chafed my young spirit. Fortunately, or
{}unfortunately, about this time in my life, I had made enough money to
buy what was then a very popular school book, viz: the ``Columbian
Orator.'' I bought this addition to my library, of Mr. Knight, on Thames
street, Fell's Point, Baltimore, and paid him fifty cents for it. I was
first led to buy this book, by hearing some little boys say that they
were going to learn some little pieces out of it for the Exhibition.
This volume was, indeed, a rich treasure, and every opportunity afforded
me, for a time, was spent in diligently perusing it. Among much other
interesting matter, that which I had perused and reperused with
unflagging satisfaction, was a short dialogue between a master and his
slave. The slave is represented as having been recaptured, in a second
attempt to run away; and the master opens the dialogue with an
upbraiding speech, charging the slave with ingratitude, and demanding to
know what he has to say in his own defense. Thus upbraided, and thus
called upon to reply, the slave rejoins, that he knows how little
anything that he can say will avail, seeing that he is completely in the
hands of his owner; and with noble resolution, calmly says, ``I submit
to my fate.'' Touched by the slave's answer, the master insists upon his
further speaking, and recapitulates the many acts of kindness which he
has performed toward the slave, and tells him he is permitted to speak
for himself. Thus invited to the debate, the quondam slave made a
spirited defense of himself, and thereafter the whole argument, for and
against slavery, was brought out. The master was vanquished at every
turn in the argument; and seeing himself to be thus vanquished, he
{}generously and meekly emancipates the slave, with his best wishes for
his prosperity. It is scarcely neccessary to say, that a dialogue, with
such an origin, and such an ending---read when the fact of my being a
slave was a constant burden of grief---powerfully affected me; and I
could not help feeling that the day might come, when the well-directed
answers made by the slave to the master, in this instance, would find
their counterpart in myself.

This, however, was not all the fanaticism which I found in this
Columbian Orator. I met there one of Sheridan's mighty speeches, on the
subject of Catholic Emancipation, Lord Chatham's speech on the American
war, and speeches by the great William Pitt and by Fox. These were all
choice documents to me, and I read them, over and over again, with an
interest that was ever increasing, because it was ever gaining in
intelligence; for the more I read them, the better I understood them.
The reading of these speeches added much to my limited stock of
language, and enabled me to give tongue to many interesting thoughts,
which had frequently flashed through my soul, and died away for want of
utterance. The mighty power and heart-searching directness of truth,
penetrating even the heart of a slaveholder, compelling him to yield up
his earthly interests to the claims of eternal justice, were finely
illustrated in the dialogue, just referred to; and from the speeches of
Sheridan, I got a bold and powerful denunciation of oppression, and a
most brilliant vindication of the rights of man. Here was, indeed, a
noble acquisition. If I ever wavered under the consideration, that the
{}Almighty, in some way, ordained slavery, and willed my enslavement for
his own glory, I wavered no longer. I had now penetrated the secret of
all slavery and oppression, and had ascertained their true foundation to
be in the pride, the power and the avarice of man. The dialogue and the
speeches were all redolent of the principles of liberty, and poured
floods of light on the nature and character of slavery. With a book of
this kind in my hand, my own human nature, and the facts of my
experience, to help me, I was equal to a contest with the religious
advocates of slavery, whether among the whites or among the colored
people, for blindness, in this matter, is not confined to the former. I
have met many religious colored people, at the south, who are under the
delusion that God requires them to submit to slavery, and to wear their
chains with meekness and humility. I could entertain no such nonsense as
this; and I almost lost my patience when I found any colored man weak
enough to believe such stuff. Nevertheless, the increase of knowledge
was attended with bitter, as well as sweet results. The more I read, the
more I was led to abhor and detest slavery, and my enslavers.
``Slaveholders,'' thought I, ``are only a band of successful robbers,
who left their homes and went into Africa for the purpose of stealing
and reducing my people to slavery.'' I loathed them as the meanest and
the most wicked of men. As I read, behold! the very discontent so
graphically predicted by Master Hugh, had already come upon me. I was no
longer the light-hearted, gleesome boy, full of mirth and play, as when
I landed first at Baltimore. Knowledge had {}come; light had penetrated
the moral dungeon where I dwelt; and, behold! there lay the bloody whip,
for my back, and here was the iron chain; and my good, \emph{kind
master}, he was the author of my situation. The revelation haunted me,
stung me, and made me gloomy and miserable. As I writhed under the sting
and torment of this knowledge, I almost envied my fellow slaves their
stupid contentment. This knowledge opened my eyes to the horrible pit,
and revealed the teeth of the frightful dragon that was ready to pounce
upon me, but it opened no way for my escape. I have often wished myself
a beast, or a bird---anything, rather than a slave. I was wretched and
gloomy, beyond my ability to describe. I was too thoughtful to be happy.
It was this everlasting thinking which distressed and tormented me; and
yet there was no getting rid of the subject of my thoughts. All nature
was redolent of it. Once awakened by the silver trump of knowledge, my
spirit was roused to eternal wakefulness. Liberty! the inestimable
birth-right of every man, had, for me, converted every object into an
asserter of this great right. It was heard in every sound, and beheld in
every object. It was ever present, to torment me with a sense of my
wretched condition. The more beautiful and charming were the smiles of
nature, the more horrible and desolate was my condition. I saw nothing
without seeing it, and I heard nothing without hearing it. I do not
exaggerate, when I say, that it looked from every star, smiled in every
calm, breathed in every wind, and moved in every storm.

I have no doubt that my state of mind had {}thing to do with the change
in the treatment adopted, by my once kind mistress toward me. I can
easily believe, that my leaden, downcast, and discontented look, was
very offensive to her. Poor lady! She did not know my trouble, and I
dared not tell her. Could I have freely made her acquainted with the
real state of my mind, and given her the reasons therefor, it might have
been well for both of us. Her abuse of me fell upon me like the blows of
the false prophet upon his ass; she did not know that an \emph{angel}
stood in the way; and---such is the relation of master and slave---I
could not tell her. Nature had made us \emph{friends}; slavery made us
\emph{enemies}. My interests were in a direction opposite to hers, and
we both had our private thoughts and plans. She aimed to keep me
ignorant; and I resolved to know, although knowledge only increased my
discontent. My feelings were not the result of any marked cruelty in the
treatment I received; they sprung from the consideration of my being a
slave at all. It was \emph{slavery}---not its mere
\emph{incidents}---that I hated. I had been cheated. I saw through the
attempt to keep me in ignorance; I saw that slaveholders would have
gladly made me believe that they were merely acting under the authority
of God, in making a slave of me, and in making slaves of others; and I
treated them as robbers and deceivers. The feeding and clothing me well,
could not atone for taking my liberty from me. The smiles of my mistress
could not remove the deep sorrow that dwelt in my young bosom. Indeed,
these, in time, came only to deepen my sorrow. She had changed; and the
reader will see that I had changed, too. We {}were both victims to the
same overshadowing evil---\emph{she,} as mistress, \emph{I}, as slave. I
will not censure her harshly; she cannot censure me, for she knows I
speak but the truth, and have acted in my opposition to slavery, just as
she herself would have acted, in a reverse of circumstances.
