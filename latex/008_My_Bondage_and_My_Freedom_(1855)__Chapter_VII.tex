{\protect\hypertarget{107}{}{}}

~

{CHAPTER VII.}

LIFE IN THE GREAT HOUSE.

{COMFORTS AND LUXURIES---ELABORATE EXPENDITURE---HOUSE SERVANTS---MEN
SERVANTS AND MAID SERVANTS---APPEARANCES---SLAVE ARISTOCRACY---STABLE
AND CARRIAGE HOUSE---BOUNDLESS HOSPITALITY---FRAGRANCE OF RICH
DISHES---THE DECEPTIVE CHARACTER OF SLAVERY---SLAVES SEEM HAPPY---SLAVES
AND SLAVEHOLDERS ALIKE WRETCHED---FRETFUL DISCONTENT OF
SLAVEHOLDERS---FAULT-FINDING---OLD BARNEY---HIS
PROFESSION---WHIPPING---HUMILIATING SPECTACLE---CASE
EXCEPTIONAL---WILLIAM WILKS---SUPPOSED SON OF COL. LLOYD---CURIOUS
INCIDENT---SLAVES PREFER RICH MASTERS TO POOR ONES.}

\textsc{The} close-fisted stinginess that fed the poor slave on coarse
corn-meal and tainted meat; that clothed him in crashy tow-linen, and
hurried him on to toil through the field, in all weathers, with wind and
rain beating through his tattered garments; that scarcely gave even the
young slave-mother time to nurse her hungry infant in the fence corner;
wholly vanishes on approaching the sacred precincts of the great house,
the home of the Lloyds. There the scriptural phrase finds a exact
illustration; the highly favored inmates of this mansion are literally
arrayed ``in purple and fine linen,'' and fare sumptuously every day!
The table groans under the heavy and blood-bought luxuries gathered with
pains-taking care, at home and abroad. Fields, forests, rivers and seas,
are made tributary here. Immense wealth, and its lavish expenditure,
fill the great house with all that can please the
{\protect\hypertarget{108}{}{}}eye, or tempt the taste. Here, appetite,
not food, is the great \emph{desideratum}. Fish, flesh and fowl, are
here in profusion. Chickens, of all breeds; ducks, of all kinds, wild
and tame, the common, and the huge Muscovite; Guinea fowls, turkeys,
geese, and pea fowls, are in their several pens, fat and fatting for the
destined vortex. The graceful swan, the mongrels, the black-necked wild
goose; partridges, quails, pheasants and pigeons; choice water fowl,
with all their strange varieties, are caught in this huge family net.
Beef, veal, mutton and venison, of the most select kinds and quality,
roll bounteously to this grand consumer. The teeming riches of the
Chesapeake bay, its rock, perch, drums, crocus, trout, oysters, crabs,
and terrapin, are drawn hither to adorn the glittering table of the
great house. The dairy, too, probably the finest on the Eastern Shore of
Maryland---supplied by cattle of the best English stock, imported for
the purpose, pours its rich donations of fragrant cheese, golden butter,
and delicious cream, to heighten the attraction of the gorgeous,
unending round of feasting. Nor are the fruits of the earth forgotten or
neglected. The fertile garden, many acres in size, constituting a
separate establishment, distinct from the common farm---with its
scientific gardener, imported from Scotland, (a Mr. McDermott,) with
four men under his direction, was not behind, either in the abundance or
in the delicacy of its contributions to the same lull board. The tender
asparagus, the succulent celery, and the delicate cauliflower; egg
plants, beets, lettuce, parsnips, peas, and French beans, early and
late; radishes, cantelopes, melons of all kinds; the fruits and flowers
of all {\protect\hypertarget{109}{}{}}climes and of all descriptions,
from the hardy apple of the north, to the lemon and orange of the south,
culminated at this point. Baltimore gathered figs, raisins, almonds and
juicy grapes from Spain. Wines and brandies from France; teas of various
flavor, from China; and rich, aromatic coffee from Java, all conspired
to swell the tide of high life, where pride and indolence rolled and
lounged in magnificence and satiety.

Behind the tall-backed and elaborately wrought chairs, stand the
servants, men and maidens---fifteen in number---discriminately selected,
not only with a view to their industry and faithfulness, but with
special regard to their personal appearance, their graceful agility and
captivating address. Some of these are armed with fans, and are fanning
reviving breezes toward the over-heated brows of the alabaster ladies;
others watch with eager eye, and with fawn-like step anticipate and
supply, wants before they are sufficiently formed to be announced by
word or sign.

These servants constituted a sort of black aristocracy on Col. Lloyd's
plantation. They resembled the field hands in nothing, except in color,
and in this they held the advantage of a velvet-like glossiness, rich
and beautiful. The hair, too, showed the same advantage. The delicate
colored maid rustled in the scarcely worn silk of her young mistress,
while the servant men were equally well attired from the over-flowing
wardrobe of their young masters; so that, in well as in form and
feature, in manner and speech, in tastes and habits, the distance
between favored low, and the sorrow and hunger-smitten
{\protect\hypertarget{110}{}{}}multitudes of the quarter and the field,
was immense; and this is seldom passed over.

Let us now glance at the stables and the carriage house, and we shall
find the same evidences of pride and luxurious extravagance. Here are
three splendid coaches, soft within and lustrous without. Here, too, are
gigs, phætons, barouches, sulkeys and sleighs. Here are saddles and
harnesses---beautifully wrought and silver mounted---kept with every
care. In the stable you will find, kept only for pleasure, full
thirty-five horses, of the most approved blood for speed and beauty.
There are two men here constantly employed in taking care of these
horses. One of these men must be always in the stable, to answer every
call from the great house. Over the way from the stable, is a house
built expressly for the hounds---a pack of twenty-five or thirty---whose
fare would have made glad the heart of a dozen slaves. Horses and hounds
are not the only consumers of the slave's toil. There was practiced, at
the Lloyd's, a hospitality which would have astonished and charmed any
health-seeking northern divine or merchant, who might have chanced to
share it. Viewed from his own table, and \emph{not} from the field, the
colonel was a model of generous hospitality. His house was, literally, a
hotel, for weeks during the summer months. At these times, especially,
the air was freighted with the rich fumes of baking, boiling, roasting
and broiling. The odors I shared with the winds; but the meats were
under a more stringent monopoly---except that, occasionally, I got a
cake from Mas' Daniel. In Mas' Daniel I had a friend at court, from whom
I learned many things {\protect\hypertarget{111}{}{}}which my eager
curiosity was excited to know. I always knew when company was expected,
and who they were, although I was an outsider, being the property, not
of Col. Lloyd, but of a servant of the wealthy colonel. On these
occasions, all that pride, taste and money could do, to dazzle and
charm, was done.

Who could say that the servants of Col. Lloyd were not well clad and
cared for, after witnessing one of his magnificent entertainments? Who
could say that they did not seem to glory in being the slaves of such a
master? Who, but a fanatic, could get up any sympathy for persons whose
every movement was agile, easy and graceful, and who evinced a
consciousness of high superiority? And who would ever venture to suspect
that Col. Lloyd was subject to the troubles of ordinary mortals? Master
and slave seem alike in their glory here? Can it all be seeming? Alas!
it may only be a sham at last! This immense wealth; this gilded
splendor; this profusion of luxury; this exemption from toil; this life
of ease; this sea of plenty; aye, what of it all? Are the pearly gates
of happiness and sweet content flung open to such suitors? \emph{far
from it!} The poor slave, on his hard, pine plank, but scantily covered
with his thin blanket, sleeps more soundly than the feverish voluptuary
who reclines upon his feather bed and downy pillow. Food, to the
indolent lounger, is poison, not sustenance. Lurking beneath all their
dishes, are invisible spirits of evil, ready to feed the self-deluded
gormandizers with aches, pains, fierce temper, uncontrolled passions,
dyspepsia, rheumatism, lumbago and gout; and
{\protect\hypertarget{112}{}{}}of these the Lloyds got their full share.
To the pampered love of ease, there is no resting place. What is
pleasant to-day, is repulsive to-morrow; what is soft now, is hard at
another time; what is sweet in the morning, is bitter in the evening.
Neither to the wicked, nor to the idler, is there any solid peace:
"\emph{Troubled, like the restless sea.}"

I had excellent opportunities of witnessing the restless discontent and
the capricious irritation of the Lloyds. My fondness for horses---not
peculiar to me more than to other boys---attracted me, much of the time,
to the stables. This establishment was especially under the care of
``old'' and ``young'' Barney---father and son. Old Barney was a fine
looking old man, of a brownish complexion, who was quite portly, and
wore a dignified aspect for a slave. He was, evidently, much devoted to
his profession, and held his office an honorable one. He was a farrier
as well as an ostler; he could bleed, remove lampers from the mouths of
the horses, and was well instructed in horse medicines. No one on the
farm knew, so well as Old Barney, what to do with a sick horse. But his
gifts and acquirements were of little advantage to him. His office was
by no means an enviable one. He often got presents, but he got stripes
as well; for in nothing was Col. Lloyd more unreasonable and exacting,
than in respect to the management of his pleasure horses. Any supposed
inattention to these animals was sure to be visited with degrading
punishment. His horses and dogs fared better than his men. Their beds
must be softer and cleaner than those of his human cattle. No excuse
could shield {\protect\hypertarget{113}{}{}}Old Barney, if the colonel
only suspected something wrong about his horses; and, consequently, he
was often punished when faultless. It was absolutely painful to listen
to the many unreasonable and fretful scoldings, poured out at the
stable, by Col. Lloyd, his sons and sons-in-law. Of the latter, he had
three---Messrs. Nicholson, Winder and Lownes. These all lived at the
great house a portion of the year, and enjoyed the luxury of whipping
the servants when they pleased, which was by no means unfrequently. A
horse was seldom brought out of the stable to which no objection could
be raised. ``There was dust in his hair;'' ``there was a twist in his
reins;'' ``his mane did not lie straight;'' ``he had not been properly
grained;'' ``his head did not look well;'' ``his fore-top was not combed
out;'' ``his fetlocks had not been properly trimmed;'' something was
always wrong. Listening to complaints, however groundless, Barney must
stand, hat in hand, lips sealed, never answering a word. He must make no
reply, no explanation; the judgment of the master must be deemed
infallible, for his power is absolute and irresponsible. In a free
state, a master, thus complaining without cause, of his ostler, might be
told---``Sir, I am sorry I cannot please you, but, since I have done the
best I can, your remedy is to dismiss me.'' Here, however, the ostler
must stand, listen and tremble. One of the most heart-saddening and
humiliating scenes I ever witnessed, was the whipping of Old Barney, by
Col. Lloyd himself. Here were two men, both advanced in years; there
were the silvery locks of Col. L., and there was the bald and toil-worn
brow of Old Barney; {\protect\hypertarget{114}{}{}}master and slave;
superior and inferior here, but \emph{equals} at the bar of God; and, in
the common course of events, they must both soon meet in another world,
in a world where all distinctions, except those based on obedience and
disobedience, are blotted out forever. ``Uncover your head!'' said the
imperious master; he was obeyed. ``Take off your jacket, you old
rascal!'' and off came Barney's jacket. ``Down on your knees!'' down
knelt the old man, his shoulders bare, his bald head glistening in the
sun, and his aged knees on the cold, damp ground. In this humble and
debasing attitude, the master---that master to whom he had given the
best years and the best strength of his life---came forward, and laid on
thirty lashes, with his horse whip. The old man bore it patiently, to
the last, answering each blow with a slight shrug of the shoulders, and
a groan. I cannot think that Col. Lloyd succeeded in marring the flesh
of Old Barney very seriously, for the whip was a light, riding whip; but
the spectacle of an aged man---a husband and a father---humbly kneeling
before a worm of the dust, surprised and shocked me at the time; and
since I have grown old enough to think on the wickedness of slavery, few
facts have been of more value to me than this, to which I was a witness.
It reveals slavery in its true color, and in its maturity of repulsive
hatefulness. I owe it to truth, however, to say, that this was the first
and the last time I ever saw Old Barney, or any other slave, compelled
to kneel to receive a whipping.

I saw, at the stable, another incident, which I will relate, as it is
illustrative of a phase of slavery to which I have already referred in
another connection. {\protect\hypertarget{115}{}{}}Besides two other
coachmen, Col. Lloyd owned one named William, who, strangely enough, was
often called by his surname, Wilks, by white and colored people on the
home plantation. Wilks was a very fine looking man. He was about as
white as anybody on the plantation; and in manliness of form, and
comeliness of features, he bore a very striking resemblance to Mr.
Murray Lloyd. It was whispered, and pretty generally admitted as a fact,
that William Wilks was a son of Col. Lloyd, by a highly favored
slave-woman, who was still on the plantation. There were many reasons
for believing this whisper, not only in William's appearance, but in the
undeniable freedom which he enjoyed over all others, and his apparent
consciousness of being something more than a slave to his master. It was
notorious, too, that William had a deadly enemy in Murray Lloyd, whom he
so much resembled, and that the latter greatly worried his father with
importunities to sell William. Indeed, he gave his father no rest until
he did sell him, to Austin Woldfolk, the great slave-trader at that
time. Before selling him, however, Mr. L. tried what giving William a
whipping would do, toward making things smooth; but this was a failure.
It was a compromise, and defeated itself; for, immediately after the
infliction, the heart-sickened colonel atoned to William for the abuse,
by giving him a gold watch and chain. Another fact, somewhat curious,
is, that though sold to the remorseless \emph{Woldfolk}, taken in irons
to Baltimore and cast into prison, with a view to being driven to the
south, William, by \emph{some} means---always a mystery to me---outbid
all his {\protect\hypertarget{116}{}{}}purchasers, paid for himself,
\emph{and now resides in Baltimore, a} \textsc{freeman}. Is there not
room to suspect, that, as the gold watch was presented to atone for the
whipping, a purse of gold was given him by the same hand, with which to
effect his purchase, as an atonement for the indignity involved in
selling his own flesh and blood. All the circumstances of William, on
the great house farm, show him to have occupied a different position
from the other slaves, and, certainly, there is nothing in the supposed
hostility of slaveholders to amalgamation, to forbid the supposition
that William Wilks was the son of Edward Lloyd. \emph{Practical}
amalgamation is common in every neighborhood where I have been in
slavery.

Col. Lloyd was not in the way of knowing much of the real opinions and
feelings of his slaves respecting him. The distance between him and them
was far too great to admit of such knowledge. His slaves were so
numerous, that he did not know them when he saw them. Nor, indeed, did
all his slaves know him. In this respect, he was inconveniently rich. It
is reported of him, that, while riding along the road one day, he met a
colored man, and {addresssed} him in the usual way of speaking to
colored people on the public highways of the south: ``Well, boy, who do
you belong to?'' ``To Col. Lloyd,'' replied the slave. ``Well, does the
colonel treat you well?'' ``No, sir,'' was the ready reply. ``What! does
he work you too hard?'' ``Yes, sir.'' ``Well, don't he give enough to
eat?'' ``Yes, sir, he gives me enough, such as it is.'' The colonel,
after ascertaining where the slave belonged, rode on; the slave
{\protect\hypertarget{117}{}{}}also went on about his business, not
dreaming that he had been conversing with his master. He thought, said
and heard nothing more of the matter, until two or three weeks
afterwards. The poor man was then informed by his overseer, that, for
having found fault with his master, he was now to be sold to a Georgia
trader. He was immediately chained and handcuffed; and thus, without a
moment's warning he was snatched away, and forever sundered from his
family and friends, by a hand more unrelenting than that of death.
\emph{This} is the penalty of telling the simple truth, in answer to a
series of plain questions. It is partly in consequence of such facts,
that slaves, when inquired of as to their condition and the character of
their masters, almost invariably say they are contented, and that their
masters are kind. Slaveholders have been known to send spies among their
slaves, to ascertain, if possible, their views and feelings in regard to
their condition. The frequency of this has had the effect to establish
among the slaves the maxim, that a still tongue makes a wise head. They
suppress the truth rather than take the consequence of telling it, and,
in so doing, they prove themselves a part of the human family. If they
have anything to say of their master, it is, generally, something in his
favor, especially when speaking to strangers. I was frequently asked,
while a slave, if I had a kind master, and I do not remember ever to
have given a negative reply. Nor did I, when pursuing this course,
consider myself as uttering what was utterly false; for I always
measured the kindness of my master by the standard of
{\protect\hypertarget{118}{}{}}kindness set up by slaveholders around
us. However, slaves are like other people, and imbibe similar
prejudices. They are apt to think \emph{their condition} better than
that of others. Many, under the influence of this prejudice, think their
own masters are better than the masters of other slaves; and this, too,
in some cases, when the very reverse is true. Indeed, it is not uncommon
for slaves even to fall out and quarrel among themselves about the
relative kindness of their masters, each contending for the superior
goodness of his own over that of others. At the very same time, they
mutually execrate their masters, when viewed separately. It was so on
our plantation. When Col. Lloyd's slaves met those of Jacob Jepson, they
seldom parted without a quarrel about their masters; Col. Lloyd's slaves
contending that he was the richest, and Mr. Jepson's slaves that he was
the smartest, man of the two. Col. Lloyd's slaves would boast his
ability to buy and sell Jacob Jepson; Mr. Jepson's slaves would boast
his ability to whip Col. Lloyd. These quarrels would almost always end
in a fight between the parties; those that beat were supposed to have
gained the point at issue. They seemed to think that the greatness of
their masters was transferable to themselves. To be a \textsc{slave},
was thought to be bad enough; but to be a \emph{poor man's} slave, was
deemed a disgrace, indeed.
