{}

{CHAPTER XXV.}

VARIOUS INCIDENTS.

{NEWSPAPER ENTERPRISE---UNEXPECTED OPPOSITION---THE OBJECTIONS TO
IT---THEIR PLAUSIBILITY ADMITTED---MOTIVES FOR COMING TO
ROCHESTER---DISCIPLE OF MR. GARRISON---CHANGE OF OPINION---CAUSES
LEADING TO IT---THE CONSEQUENCES OF THE CHANGE---PREJUDICE AGAINST
COLOR---AMUSING CONDESCENSION---``JIM CROW CARS''---COLLISIONS WITH
CONDUCTORS AND BRAKEMEN---TRAINS ORDERED NOT TO STOP AT LYNN---AMUSING
DOMESTIC SCENE---SEPARATE TABLES FOR MASTER AND MAN---PREJUDICE
UNNATURAL---ILLUSTRATIONS---THE AUTHOR IN HIGH COMPANY---ELEVATION OF
THE FREE PEOPLE OF COLOR---PLEDGE FOR THE FUTURE.}

\textsc{I have} now given the reader an imperfect sketch of nine years'
experience in freedom---three years as a common laborer on the wharves
of New Bedford, four years as a lecturer in New England, and two years
of semi-exile in Great Britain and Ireland. A single ray of light
remains to be flung upon my life during the last eight years, and my
story will be done.

A trial awaited me on my return from England to the United States, for
which I was but very imperfectly prepared. My plans for my then future
usefulness as an anti-slavery advocate were all settled. My friends in
England had resolved to raise a given sum to purchase for me a press and
printing materials; and I already saw myself wielding my pen, as well as
my voice, in the great work of renovating the public mind, and building
up a public sentiment {}which should, at least, send slavery and
oppression to the grave, and restore to ``liberty and the pursuit of
happiness'' the people with whom I had suffered, both as a slave and as
a freeman. Intimation had reached my friends in Boston of what I
intended to do, before my arrival, and I was prepared to find them
favorably disposed toward my much cherished enterprise. In this I was
mistaken. I found them very earnestly opposed to the idea of my starting
a paper, and for several reasons. First, the paper was not needed;
secondly, it would interfere with my usefulness as a lecturer; thirdly,
I was better fitted to speak than to write; fourthly, the paper could
not succeed. This opposition, from a quarter so highly esteemed, and to
which I had been accustomed to look for advice and direction, caused me
not only to hesitate, but inclined me to abandon the enterprise. All
previous attempts to establish such a journal having failed, I felt that
probably I should but add another to the list of failures, and thus
contribute another proof of the mental and moral deficiencies of my
race. Very much that was said to me in respect to my imperfect literary
acquirements, I felt to be most painfully true. The unsuccessful
projectors of all the previous colored newspapers were my superiors in
point of education, and if they failed, how could I hope for success?
Yet I did hope for success, and persisted in the undertaking. Some of my
English friends greatly encouraged me to go forward, and I shall never
cease to be grateful for their words of cheer and generous deeds.

I can easily pardon those who have denounced me {}as ambitious and
presumptuous, in view of my persistence in this enterprise. I was but
nine years from slavery. In point of mental experience, I was but nine
years old. That one, in such circumstances, should aspire to establish a
printing press, among an educated people, might well be considered, if
not ambitious, quite silly. My American friends looked at me with
astonishment! ``A wood-sawyer'' offering himself to the public as an
editor! A slave, brought up in the very depths of ignorance, assuming to
instruct the highly civilized people of the north in the principles of
liberty, justice, and humanity! The thing looked absurd. Nevertheless, I
persevered. I felt that the want of education, great as it was, could be
overcome by study, and that knowledge would come by experience; and
further, (which was perhaps the most controlling consideration,) I
thought that an intelligent public, knowing my early history, would
easily pardon a large share of the deficiencies which I was sure that my
paper would exhibit. The most distressing thing, however, was the
offense which I was about to give my Boston friends, by what seemed to
them a reckless disregard of their sage advice. I am not sure that I was
not under the influence of something like a slavish adoration of my
Boston friends, and I labored hard to convince them of the wisdom of my
undertaking, but without success. Indeed, I never expect to succeed,
although time has answered all their original objections. The paper has
been successful. It is a large sheet, costing eighty dollars per
week---has three thousand subscribers---has been published regularly
nearly eight {}years---and bids fair to stand eight years longer. At any
rate, the eight years to come are as full of promise as were the eight
that are past.

It is not to be concealed, however, that the maintenance of such a
journal, under the circumstances, has been a work of much difficulty;
and could all the perplexity, anxiety, and trouble attending it, have
been clearly foreseen, I might have shrunk from the undertaking. As it
is, I rejoice in having engaged in the enterprise, and count it joy to
have been able to suffer, in many ways, for its success, and for the
success of the cause to which it has been faithfully devoted. I look
upon the time, money, and labor bestowed upon it, as being amply
rewarded, in the development of my own mental and moral energies, and in
the corresponding development of my deeply injured and oppressed people.

From motives of peace, instead of issuing my paper in Boston, among my
New England friends, I came to Rochester, Western New York, among
strangers, where the circulation of my paper could not interfere with
the local circulation of the Liberator and the Standard; for at that
time I was, on the anti-slavery question, a faithful disciple of William
Lloyd Garrison, and fully committed to his doctrine touching the
pro-slavery character of the constitution of the United States, and the
\emph{non-voting principle}, of which he is the known and distinguished
advocate. With Mr. Garrison, I held it to be the first duty of the
non-slaveholding states to dissolve the union with the slaveholding
states; and hence my cry, like his, was, ``No union with slaveholders.''
With these views, I {}came into Western New York; and during the first
four years of my labor here, I advocated them with pen and tongue,
according to the best of my ability.

About four years ago, upon a reconsideration of the whole subject, I
became convinced that there was no necessity for dissolving the ``union
between the northern and southern states;'' that to seek this
dissolution was no part of my duty as an abolitionist; that to abstain
from voting, was to refuse to exercise a legitimate and powerful means
for abolishing slavery; and that the constitution of the United States
not only contained no guarantees in favor of slavery, but, on the
contrary, it is, in its letter and spirit, an anti-slavery instrument,
demanding the abolition of slavery as a condition of its own existence,
as the supreme law of the land.

Here was a radical change in my opinions, and in the action logically
resulting from that change. To those with whom I had been in agreement
and in sympathy, I was now in opposition. What they held to be a great
and important truth, I now looked upon as a dangerous error. A very
painful, and yet a very natural, thing now happened. Those who could not
see any honest reasons for changing their views, as I had done, could
not easily see any such reasons for my change, and the common punishment
of apostates was mine.

The opinions first entertained were naturally derived and honestly
entertained, and I trust that my present opinions have the same claims
to respect. Brought directly, when I escaped from slavery, into contact
with a class of abolitionists regarding the {}constitution as a
slaveholding instrument, and finding their views supported by the united
and entire history of every department of the government, it is not
strange that I assumed the constitution to be just what their
interpretation made it. I was bound, not only by their superior
knowledge, to take their opinions as the true ones, in respect to the
subject, but also because I had no means of showing their unsoundness.
But for the responsibility of conducting a public journal, and the
necessity imposed upon me of meeting opposite views from abolitionists
in this state, I should in all probability have remained as firm in my
disunion views as any other disciple of William Lloyd Garrison.

My new circumstances compelled me to re-think the whole subject, and to
study, with some care, not only the just and proper rules of legal
interpretation, but the origin, design, nature, rights, powers, and
duties of civil government, and also the relations which human beings
sustain to it. By such a course of thought and reading, I was conducted
to the conclusion that the constitution of the United
States---inaugurated ``to form a more perfect union, establish justice,
insure domestic tranquillity, provide for the common defense, promote
the general welfare, and secure the blessings of liberty''---could not
well have been designed at the same time to maintain and perpetuate a
system of rapine and murder like slavery; especially, as not one word
can be found in the constitution to authorize such a belief. Then,
again, if the declared purposes of an instrument are to govern the
meaning of all its parts and details, as they clearly {}should, the
constitution of our country is our warrant for the abolition of slavery
in every state in the American Union. I mean, however, not to argue, but
simply to state my views. It would require very many pages of a volume
like this, to set forth the arguments demonstrating the
unconstitutionality and the complete illegality of slavery in our land;
and as my experience, and not my arguments, is within the scope and
contemplation of this volume, I omit the latter and proceed with the
former.

I will now ask the kind reader to go back a little in my story, while I
bring up a thread left behind for convenience sake, but which, small as
it is, cannot be properly omitted altogether; and that thread is
American prejudice against color, and its varied illustrations in my own
experience.

When I first went among the abolitionists of New England, and began to
travel, I found this prejudice very strong and very annoying. The
abolitionists themselves were not entirely free from it, and I could see
that they were nobly struggling against it. In their eagerness,
sometimes, to show their contempt for the feeling, they proved that they
had not entirely recovered from it; often illustrating the saying, in
their conduct that a man may ``stand up so straight as to lean
backward.'' When it was said to me, ``Mr. Douglass, I will walk to
meeting with you; I am not afraid of a black man,'' I could not help
thinking---seeing nothing very frightful in my appearance---``And why
should you be?'' The children at the north had all been educated to
believe that if they were bad, the old \emph{black} man---not the old
\emph{devil}--- {}would get them; and it was evidence of some courage,
for any so educated to get the better of their fears.

The custom of providing separate cars for the accommodation of colored
travelers, was established on nearly all the railroads of New England, a
dozen years ago. Regarding this custom as fostering the spirit of caste,
I made it a rule to seat myself in the cars for the accommodation of
passengers generally. Thus seated, I was sure to be called upon to
betake myself to the \emph{"Jim Crow car.}" Refusing to obey, I was
often dragged out of my seat, beaten, and severely bruised, by
conductors and brakemen. Attempting to start from Lynn, one day, for
Newburyport, on the Eastern railroad, I went, as my custom was, into one
of the best railroad carriages on the road. The seats were very
luxuriant and beautiful. I was soon waited upon by the conductor, and
ordered out; whereupon I demanded the reason for my invidious removal.
After a good deal of parleying, I was told that it was because I was
black. This I denied, and appealed to the company to sustain my denial;
but they were evidently unwilling to commit themselves, on a point so
delicate, and requiring such nice powers of discrimination, for they
remained as dumb as death. I was soon waited on by half a dozen fellows
of the baser sort, (just such as would volunteer to take a bull-dog out
of a meeting-house in time of public worship,) and told that I must move
out of that seat, and if I did not, they would drag me out. I refused to
move, and they clutched me, head, neck, and shoulders. But, in
anticipation of the {}stretching to which I was about to be subjected, I
had interwoven myself among the seats. In dragging me out, on this
occasion, it must have cost the company twenty-five or thirty dollars,
for I tore up seats and all. So great was the excitement in Lynn, on the
subject, that the superintendent, Mr. Stephen A. Chase, ordered the
trains to run through Lynn without stopping, while I remained in that
town; and this ridiculous farce was enacted. For several days the trains
went dashing through Lynn without stopping. At the same time that they
excluded a free colored man from their cars, this same company allowed
slaves, in company with their masters and mistresses, to ride
unmolested.

After many battles with the railroad conductors, and being roughly
handled in not a few instances, proscription was at last abandoned; and
the ``Jim Crow car''---set up for the degradation of colored people---is
nowhere found in New England. This result was not brought about without
the intervention of the people, and the threatened enactment of a law
compelling railroad companies to respect the rights of travelers. Hon.
Charles Francis Adams performed signal service in the Massachusetts
legislature, in bringing about this reformation; and to him the colored
citizens of that state are deeply indebted.

Although often annoyed, and sometimes outraged, by this prejudice
against color, I am indebted to it for many passages of quiet amusement.
A half-cured subject of it is sometimes driven into awkward straits,
{}especially if he happens to get a genuine specimen of the race into
his house.

In the summer of 1843, I was traveling and lecturing, in company with
William A. White, Esq., through the state of Indiana. Anti-slavery
friends were not very abundant in Indiana, at that time, and beds were
not more plentiful than friends. We often slept out, in preference to
sleeping in the houses, at some points. At the close of one of our
meetings, we were invited home with a kindly-disposed old farmer, who,
in the generous enthusiasm of the moment, seemed to have forgotten that
he had but one spare bed, and that his guests were an ill-matched pair.
All went on pretty well, till near bed time, when signs of uneasiness
began to show themselves, among the unsophisticated sons and daughters.
White is remarkably fine looking, and very evidently a born gentleman;
the idea of putting us in the same bed was hardly to be tolerated; and
yet, there we were, and but the one bed for us, and that, by the way,
was in the same room occupied by the other members of the family. White,
as well as I, perceived the difficulty, for yonder slept the old folks,
there the sons, and a little farther along slept the daughters; and but
one other bed remained. Who should have this bed, was the puzzling
question. There was some whispering between the old folks, some confused
looks among the young, as the time for going to bed approached. After
witnessing the confusion as long as I liked, I relieved the
kindly-disposed family by playfully saying, "Friend White, having got
entirely rid of my prejudice against color, I think, as a proof of it, I
{}must allow you to sleep with me to-night." White kept up the joke, by
seeming to esteem himself the favored party, and thus the difficulty was
removed. If we went to a hotel, and called for dinner, the landlord was
sure to set one table for White and another for me, always taking him to
be master, and me the servant. Large eyes were generally made when the
order was given to remove the dishes from my table to that of White's.
In those days, it was thought strange that a white man and a colored man
could dine peaceably at the same table, and in some parts the
strangeness of such a sight has not entirely subsided.

Some people will have it that there is a natural, an inherent, and an
invincible repugnance in the breast of the white race toward
dark-colored people; and some very intelligent colored men think that
their proscription is owing solely to the color which nature has given
them. They hold that they are rated according to their color, and that
it is impossible for white people ever to look upon dark races of men,
or men belonging to the African race, with other than feelings of
aversion. My experience, both serious and mirthful, combats this
conclusion. Leaving out of sight, for a moment, grave facts, to this
point, I will state one or two, which illustrate a very interesting
feature of American character as well as American prejudice. Riding from
Boston to Albany, a few years ago, I found myself in a large car, well
filled with passengers. The seat next to me was about the only vacant
one. At every stopping place we took in new passengers, all of whom, on
reaching {}the seat next to me, cast a disdainful glance upon it, and
passed to another car, leaving me in the full enjoyment of a whole form.
For a time, I did not know but that my riding there was prejudicial to
the interest of the railroad company. A circumstance occurred, however,
which gave me an elevated position at once. Among the passengers on this
train was Gov. George N. Briggs. I was not acquainted with him, and had
no idea that I was known to him. Known to him, however, I was, for upon
observing me, the governor left his place, and making his way toward me,
respectfully asked the privilege of a seat by my side; and upon
introducing himself, we entered into a conversation very pleasant and
instructive to me. The despised seat now became honored. His excellency
had removed all the prejudice against sitting by the side of a negro;
and upon his leaving it, as he did, on reaching Pittsfield, there were
at least one dozen applicants for the place. The governor had, without
changing my skin a single shade, made the place respectable which before
was despicable.

A similar incident happened to me once on the Boston and New Bedford
railroad, and the leading party to it has since been governor of the
state of Massachusetts. I allude to Col. John Henry Clifford. Lest the
reader may fancy I am aiming to elevate myself, by claiming too much
intimacy with great men, I must state that my only acquaintance with
Col. Clifford was formed while I was \emph{his hired servant} during the
first winter of my escape from slavery. I owe it him to say, that in
that relation I found him {}always kind and gentlemanly. But to the
incident. I entered a car at Boston, for New Bedford, which, with the
exception of a single seat, was full, and found I must occupy this, or
stand up, during the journey. Having no mind to do this, I stepped up to
the man having the next seat, and who had a few parcels on the seat, and
gently asked leave to take a seat by his side. My fellow-passenger gave
me a look made up of reproach and indignation, and asked me why I should
come to that particular seat. I assured him, in the gentlest manner,
that of all others this was the seat for me. Finding that I was actually
about to sit down, he sang out, ``O! stop, stop! and let me get out!''
Suiting the action to the word, up the agitated man got, and sauntered
to the other end of the car, and was compelled to stand for most of the
way thereafter. Half-way to New Bedford, or more, Col. Clifford,
recognizing me, left his seat, and not having seen me before since I had
ceased to wait on him, (in everything except hard arguments against his
pro-slavery position,) apparently forgetful of his rank, manifested, in
greeting me, something of the feeling of an old friend. This
demonstration was not lost on the gentleman whose dignity I had, an hour
before, most seriously offended. Col. Clifford was known to be about the
most aristocratic gentleman in Bristol county; and it was evidently
thought that I must be somebody, else I should not have been thus
noticed, by a person so distinguished. Sure enough, after Col. Clifford
left me, I found myself surrounded with friends; and among the number,
my offended friend stood nearest, and with an apology for his rudeness,
{}which I could not resist, although it was one of the lamest ever
offered. With such facts as these before me---and I have many of
them---I am inclined to think that pride and fashion have much to do
with the treatment commonly extended to colored people in the United
States. I once heard a very plain man say, (and he was cross-eyed, and
awkwardly flung together in other respects,) that he should be a
handsome man when public opinion shall be changed.

Since I have been editing and publishing a journal devoted to the cause
of liberty and progress, I have had my mind more directed to the
condition and circumstances of the free colored people than when I was
the agent of an abolition society. The result has been a corresponding
change in the disposition of my time and labors. I have felt it to be a
part of my mission---under a gracious Providence---to impress my sable
brothers in this country with the conviction that, notwithstanding the
ten thousand discouragements and the powerful hinderances, which beset
their existence in this country---notwithstanding the blood-written
history of Africa, and her children, from whom we have descended, or the
clouds and darkness, (whose stillness and gloom are made only more awful
by wrathful thunder and lightning,) now overshadowing---them progress is
yet possible, and bright skies shall yet shine upon their pathway; and
that ``Ethiopia shall yet reach forth her hand unto God.''

Believing that one of the best means of emancipating the slaves of the
south is to improve and elevate the character of the free colored people
of the north, {}I shall labor in the future, as I have labored in the
past, to promote the moral, social, religious, and intellectual
elevation of the free colored people; never forgetting my own humble
origin, nor refusing, while Heaven lends me ability, to use my voice, my
pen, or my vote, to advocate the great and primary work of the universal
and unconditional emancipation of my entire race.
