{}

~

{CHAPTER XVIII.}

NEW RELATIONS AND DUTIES.

{CHANGE OF MASTERS---BENEFITS DERIVED BY THE CHANGE---FAME OF THE FIGHT
WITH COVEY---RECKLESS UNCONCERN AUTHOR'S ABHORRENCE OF SLAVERY---ABILITY
TO READ A CAUSE OF PREJUDICE---THE HOLIDAYS---HOW SPENT---SHARP HIT AT
SLAVERY---EFFECTS OF HOLIDAYS---A DEVICE OF SLAVERY---DIFFERENCE BETWEEN
COVEY AND FREELAND---AN IRRELIGIOUS MASTER PREFERRED TO A RELIGIOUS
ONE---CATALOGUE OF FLOGGABLE OFFENSES---HARD LIFE AT COVEY'S USEFUL TO
THE AUTHOR---IMPROVED CONDITION NOT FOLLOWED BY CONTENTMENT---CONGENIAL
SOCIETY AT FREELAND'S---AUTHOR'S SABBATH SCHOOL INSTITUTED---SECRECY
NECESSARY---AFFECTIONATE RELATIONS OF TUTOR AND PUPILS---CONFIDENCE AND
FRIENDSHIP AMONG SLAVES---THE AUTHOR DECLINES PUBLISHING PARTICULARS OF
CONVERSATIONS WITH HIS FRIENDS---SLAVERY THE INVITER OF VENGEANCE.}

\textsc{My} term of actual service to Mr. Edward Covey ended on
Christmas day, 1834. I gladly left the snakish Covey, although he was
now as gentle as a lamb. My home for the year 1835 was already
secured---my next master was already selected. There is always more or
less excitement about the matter of changing hands, but I had become
somewhat reckless. I cared very little into whose hands I fell---I meant
to fight my way. Despite of Covey, too, the report got abroad, that I
was hard to whip; that I was guilty of kicking back; that though
generally a good tempered negro, I sometimes "\emph{got the devil in
me}." These sayings were rife in Talbot county, and they distinguished
me among my servile brethren. Slaves, {}generally, will fight each
other, and die at each other's hands; but there are few who are not held
in awe by a white man. Trained from the cradle up, to think and feel
that their masters are superior, and invested with a sort of sacredness,
there are few who can outgrow or rise above the control which that
sentiment exercises. I had now got free from it, and the thing was
known. One bad sheep will spoil a whole flock. Among the slaves, I was a
bad sheep. I hated slavery, slaveholders, and all pertaining to them;
and I did not fail to inspire others with the same feeling, wherever and
whenever opportunity was presented. This made me a marked lad among the
slaves, and a suspected one among the slaveholders. A knowledge of my
ability to read and write, got pretty widely spread, which was very much
against me.

The days between Christmas day and New Year's, are allowed the slaves as
holidays. During these days, all regular work was suspended, and there
was nothing to do but to keep fires, and look after the stock. This time
we regarded as our own, by the grace of our masters, and we, therefore
used it, or abused it, as we pleased. Those who had families at a
distance, were now expected to visit them, and to spend with them the
entire week. The younger slaves, or the unmarried ones, were expected to
see to the cattle, and attend to incidental duties at home. The holidays
were variously spent. The sober, thinking and industrious ones of our
number, would employ themselves in manufacturing corn brooms, mats,
horse collars and baskets, and some of these were very well made.
Another class spent their time in hunting {}opossums, coons, rabbits,
and other game. But the majority spent the holidays in sports, ball
playing, wrestling, boxing, running foot races, dancing, and drinking
whisky; and this latter mode of spending the time was generally most
agreeable to their masters. A slave who would work during the holidays,
was thought, by his master, undeserving of holidays. Such an one had
rejected the favor of his master. There was, in this simple act of
continued work, an accusation against slaves; and a slave could not help
thinking, that if he made three dollars during the holidays, he might
make three hundred during the year. Not to be drunk during the holidays,
was disgraceful; and he was esteemed a lazy and improvident man, who
could not afford to drink whisky during Christmas.

The fiddling, dancing and "\emph{jubilee beating,}" was going on in all
directions. This latter performance is strickly southern. It supplies
the place of a violin, or of other musical instruments, and is played so
easily, that almost every farm has its ``Juba'' beater. The performer
improvises as he beats, and sings his merry songs, so ordering the words
as to have them fall pat with the movement of his hands. Among a mass of
nonsense and wild frolic, once in a while a sharp hit is given to the
meanness of slaveholders. Take the following, for an example:

{"}We raise de wheat,\\
Dey gib us de corn;\\
We bake de bread,\\
Dey gib us de cruss;\\
We sif de meal,\\
Dey gib us de huss;\\
{}We peal de meat,\\
Dey gib us de skin,\\
And dat's de way\\
Dey takes us in.\\
We skim de pot,\\
Dey gib us the liquor,\\
And say dat's good enough for nigger.\\
{}Walk over! walk over!\\
{}Tom butter and de fat;\\
{}Poor nigger you can't get over dat;\\
{}Walk over!"\\

This is not a bad summary of the palpable injustice and fraud of
slavery, giving---as it does---to the lazy and idle, the comforts which
God designed should be given solely to the honest laborer. But to the
holiday's.

Judging from my own observation and experience, I believe these holidays
to be among the most effective means, in the hands of slaveholders, of
keeping down the spirit of insurrection among the slaves.

To enslave men, successfully and safely, it is necessary to have their
minds occupied with thoughts and aspirations short of the liberty of
which they are deprived. A certain degree of attainable good must be
kept before them. These holidays serve the purpose of keeping the minds
of the slaves occupied with prospective pleasure, within the limits of
slavery. The young man can go wooing; the married man can visit his
wife; the father and mother can see their children; the industrious and
money loving can make a few dollars; the great wrestler can win laurels;
the young people can meet, and enjoy each other's society; the drunken
man can get plenty of whisky; {}and the religious man can hold prayer
meetings, preach, pray and exhort during the holidays. Before the
holidays, these are pleasures in prospect; after the holidays, they
become pleasures of memory, and they serve to keep out thoughts and
wishes of a more dangerous character. Were slaveholders at once to
abandon the practice of allowing their slaves these liberties,
periodically, and to keep them, the year round, closely confined to the
narrow circle of their homes, I doubt not that the south would blaze
with insurrections. These holidays are conductors or safety valves to
carry off the explosive elements inseparable from the human mind, when
reduced to the condition of slavery. But for these, the rigors of
bondage would become too severe for endurance, and the slave would be
forced up to dangerous desperation. Woe to the slaveholder when he
undertakes to hinder or to prevent the operation of these electric
conductors. A succession of earthquakes would be less destructive, than
the insurrectionary fires which would be sure to burst forth in
different parts of the south, from such interference.

Thus, the holidays, become part and parcel of the gross fraud, wrongs
and inhumanity of slavery. Ostensibly, they are institutions of
benevolence, designed to mitigate the rigors of slave life, but,
practically, they are a fraud, instituted by human selfishness, the
better to secure the ends of injustice and oppression. The slave's
happiness is not the end sought, but, rather, the master's safety. It is
not from a generous unconcern for the slave's labor that this cessation
from labor is allowed, but from a prudent regard to the {}safety of the
slave system. I am strengthened in this opinion, by the fact, that most
slaveholders like to have their slaves spend the holidays in such a
manner as to be of no real benefit to the slaves. It is plain, that
everything like rational enjoyment among the slaves, is frowned upon;
and only those wild and low sports, peculiar to semi-civilized people,
are encouraged. All the license allowed, appears to have no other object
than to disgust the slaves with their temporary freedom, and to make
them as glad to return to their work, as they were to leave it. By
plunging them into exhausting depths of drunkenness and dissipation,
this effect is almost certain to follow. I have known slaveholders
resort to cunning tricks, with a view of getting their slaves deplorably
drunk. A usual plan is, to make bets on a slave, that he can drink more
whisky than any other; and so to induce a rivalry among them, for the
mastery in this degradation. The scenes, brought about in this way, were
often scandalous and loathsome in the extreme. Whole multitudes might be
found stretched out in brutal drunkenness, at once helpless and
disgusting. Thus, when the slave asks for a few hours of virtuous
freedom, his cunning master takes advantage of his ignorance, and cheers
him with a dose of vicious and revolting dissipation, artfully labeled
with the name of \textsc{Liberty}. We were induced to drink, I among the
rest, and when the holidays were over, we all staggered up from our
filth and wallowing, took a long breath, and went away to our various
fields of work; feeling, upon the whole, rather glad to go from that
which our masters artfully deceived us into the belief was {}freedom,
back again to the arms of slavery. It was not what we had taken it to
be, nor what it might have been, had it not been abused by us. It was
about as well to be a slave to \emph{master}, as to be a slave to
\emph{rum} and \emph{whisky.}

I am the more induced to take this view of the holiday system, adopted
by slaveholders, from what I know of their treatment of slaves, in
regard to other things. It is the commonest thing for them to try to
disgust their slaves with what they do not want them to have, or to
enjoy. A slave, for instance, likes molasses; he steals some; to cure
him of the taste for it, his master, in many cases, will go away to
town, and buy a large quantity of the \emph{poorest} quality, and set it
before his slave, and, with whip in hand, compel him to eat it, until
the poor fellow is made to sicken at the very thought of molasses. The
same course is often adopted to cure slaves of the disagreeable and
inconvenient practice of asking for more food, when their allowance has
failed them. The same disgusting process works well, too, in other
things, but I need not cite them. When a slave is drunk, the slaveholder
has no fear that he will plan an insurrection; no fear that he will
escape to the north. It is the sober, thinking slave who is dangerous,
and needs the vigilance of his master, to keep him a slave. But, to
proceed with my narrative.

On the first of January, 1835, I proceeded from St. Michael's to Mr.
William Freeland's, my new home. Mr. Freeland lived only three miles
from St. Michael's, on an old worn out farm, which required much labor
{}to restore it to anything like a self-supporting establishment.

I was not long in finding Mr Freeland to be a very different man from
Mr. Covey. Though not rich, Mr. Freeland was what may be called a
well-bred southern gentleman, as different from Covey, as a well-trained
and hardened negro breaker is from the best specimen of the first
families of the south. Though Freeland was a slaveholder, and shared
many of the vices of his class, he seemed alive to the sentiment of
honor. He had some sense of justice, and some feelings of humanity. He
was fretful, impulsive and passionate, but I must do him the justice to
say, he was free from the mean and selfish characteristics which
distinguished the creature from which I had now, happily, escaped. He
was open, frank, imperative, and practiced no concealments, disdaining
to play the spy. In all this, he was the opposite of the crafty Covey.

Among the many advantages gained in my change from Covey's to
Freeland's---startling as the statement may be---was the fact that the
latter gentleman made no profession of religion. I assert \emph{most
unhesitatingly}, that the religion of the south---as I have observed it
and proved it---is a mere covering for the most horrid crimes; the
justifier of the most appalling barbarity; a sanctifier of the most
hateful frauds; and a secure shelter, under which the darkest, foulest,
grossest, and most infernal abominations fester and flourish. Were I
again to be reduced to the condition of a slave, \emph{next} to that
calamity, I should regard the fact of being the slave of a religious
slaveholder, the {}greatest that could befall me. For of all
slaveholders with whom I have ever met, religious slaveholders are the
worst. I have found them, almost invariably, the vilest, meanest and
basest of their class. Exceptions there may be, but this is true of
religious slaveholders, \emph{as a class}. It is not for me to explain
the fact. Others may do that; I simply state it as a fact, and leave the
theological, and psychological inquiry, which it raises, to be decided
by others more competent than myself. Religious slaveholders, like
religious persecutors, are ever extreme in their malice and violence.
Very near my new home, on an adjoining farm, there lived the Rev. Daniel
Weeden, who was both pious and cruel after the real Covey pattern. Mr.
Weeden was a local preacher of the Protestant Methodist persuasion, and
a most zealous supporter of the ordinances of religion, generally. This
Weeden owned a woman called ``Ceal,'' who was a standing proof of his
mercilessness. Poor Ceal's back, always scantily clothed, was kept
literally raw, by the lash of this religious man and gospel minister.
The most notoriously wicked man---so called in distinction from church
members---could hire hands more easily than this brute. When sent out to
find a home, a slave would never enter the gates of the preacher Weeden,
while a sinful sinner needed a hand. Behave ill, or behave well, it was
the known maxim of Weeden, that it is the duty of a master to use the
lash. If, for no other reason, he contended that this was essential to
remind a slave of his condition, and of his master's authority. The good
slave must be whipped, to be \emph{kept} good, and the bad slave must be
{}whipped, to be \emph{made} good. Such was Weeden's theory, and such
was his practice. The back of his slave-woman will, in the judgment, be
the swiftest witness against him.

While I am stating particular cases, I might as well immortalize another
of my neighbors, by calling him by name, and putting him in print. He
did not think that a ``chiel'' was near, ``taking notes,'' and will,
doubtless, feel quite angry at having his character touched off in the
ragged style of a slave's pen. I beg to introduce the reader to
\textsc{Rev. Rigby Hopkins}. Mr. Hopkins resides between Easton and St.
Michael's, in Talbot county, Maryland. The severity of this man made him
a perfect terror to the slaves of his neighborhood. The peculiar feature
of his government, was, his system of whipping slaves, as he said,
\emph{in advance} of deserving it. He always managed to have one or two
slaves to whip on Monday morning, so as to start his hands to their
work, under the inspiration of a new assurance on Monday, that his
preaching about kindness, mercy, brotherly love, and the like, on
Sunday, did not interfere with, or prevent him from establishing his
authority, by the cowskin. He seemed to wish to assure them, that his
tears over poor, lost and ruined sinners, and his pity for them, did not
reach to the blacks who tilled his fields. This saintly Hopkins used to
boast, that he was the best hand to manage a negro in the county. He
whipped for the smallest offenses, by way of preventing the commission
of large ones.

The reader might imagine a difficulty in finding faults enough for such
frequent whipping. But, this {}is because you have no idea how easy a
matter it is to offend a man who is on the look-out for offenses. The
man, unaccustomed to slaveholding, would be astonished to observe how
many \emph{floggable} offenses there are in the slaveholder's catalogue
of crimes; and how easy it is to commit any one of them, even when the
slave least intends it. A slaveholder, bent on finding fault, will hatch
up a dozen a day, if he chooses to do so, and each one of these shall be
of a punishable description. A mere look, word, or motion, a mistake,
accident, or want of power, are all matters for which a slave may be
whipped at any time. Does a slave look dissatisfied with his condition?
It is said, that he has the devil in him, and it must be whipped out.
Does he answer \emph{loudly}, when spoken to by his master, with an air
of self-consciousness? Then, must he be taken down a button-hole lower,
by the lash, well laid on. Does he forget, and omit to pull off his hat,
when approaching a white person? Then, he must, or may be, whipped for
his bad manners. Does he ever venture to vindicate his conduct, when
harshly and unjustly accused? Then, he is guilty of impudence, one of
the greatest crimes in the social catalogue of southern society. To
allow a slave to escape punishment, who has impudently attempted to
exculpate himself from unjust charges, preferred against him by some
white person, is to be guilty of great dereliction of duty. Does a slave
ever venture to suggest a better way of doing a thing, no matter what?
he is, altogether, too officious---wise above what is written---and he
deserves, even if he does not get, a flogging for his presumption. Does
he, while {}plowing, break a plow, or while hoeing, break a hoe, or
while chopping, break an ax? no matter what were the imperfections of
the implement broken, or the natural liabilities for breaking, the slave
can be whipped for carelessness. The \emph{reverend} slaveholder could
always find something of this sort, to justify him in using the lash
several times during the week. Hopkins---like Covey and Weeden---were
shunned by slaves who had the privilege (as many had) of finding their
own masters at the end of each year; and yet, there was not a man in all
that section of country, who made a louder profession of religion, than
did Mr. \textsc{Rigby Hopkins}.

But, to continue the thread of my story, through my experience when at
Mr. William Freeland's.

My poor, weather-beaten bark now reached smoother water, and gentler
breezes. My stormy life at Covey's had been of service to me. The things
that would have seemed very hard, had I gone direct to Mr. Freeland's,
from the home of Master Thomas, were now (after the hardships at
Covey's) ``trifles light as air.'' I was still a field hand, and had
come to prefer the severe labor of the field, to the enervating duties
of a house servant. I had become large and strong; and had begun to take
pride in the fact, that I could do as much hard work as some of the
older men. There is much rivalry among slaves, at times, as to which can
do the most work, and masters generally seek to promote such rivalry.
But some of us were too wise to race with each other very long. Such
racing, we had the sagacity to see, was not likely to pay. We had our
times for measuring {}each other's strength, but we knew too much to
keep up the competition so long as to produce an extraordinary day's
work. We knew that if, by extraordinary exertion, a large quantity of
work was done in one day, the fact, becoming known to the master, might
lead him to require the same amount every day. This thought was enough
to bring us to a dead halt when ever so much excited for the race.

At Mr. Freeland's, my condition was every way improved. I was no longer
the poor scape-goat that I was when at Covey's, where every wrong thing
done was saddled upon me, and where other slaves were whipped over my
shoulders. Mr. Freeland was too just a man thus to impose upon me, or
upon any one else.

It is quite usual to make one slave the object of especial abuse, and to
beat him often, with a view to its effect upon others, rather than with
any expectation that the slave whipped will be improved by it, but the
man with whom I now was, could descend to no such meanness and
wickedness. Every man here was held individually responsible for his own
conduct.

This was a vast improvement on the rule at Covey's. There, I was the
general pack horse. Bill Smith was protected, by a positive prohibition
made by his rich master, and the command of the rich slaveholder is
\textsc{law} to the poor one; Hughes was favored, because of his
relationship to Covey; and the hands hired temporarily, escaped
flogging, except as they got it over my poor shoulders. Of course, this
comparison refers to the time when Covey \emph{could} whip me.

Mr. Freeland, like Mr. Covey, gave his hands enough to eat, but, unlike
Mr. Covey, he gave them {}time to take their meals; he worked us hard
during the day, but gave us the night for rest---another advantage to be
set to the credit of the sinner, as against that of the saint. We were
seldom in the field after dark in the evening, or before sunrise in the
morning. Our implements of husbandry were of the most improved pattern,
and much superior to those used at Covey's.

Notwithstanding the improved condition which was now mine, and the many
advantages I had gained by my new home, and my new master, I was still
restless and discontented. I was about as hard to please by a master, as
a master is by a slave. The freedom from bodily torture and unceasing
labor, had given my mind an increased sensibility, and imparted to it
greater activity. I was not yet exactly in right relations. ``How be it,
that was not first which is spiritual, but that which is natural, and
afterward that which is spiritual.'' When entombed at Covey's, shrouded
in darkness and physical wretchedness, temporal well-being was the grand
\emph{desideratum}; but, temporal wants supplied, the spirit puts in its
claims. Beat and cuff your slave, keep him hungry and spiritless, and he
will follow the chain of his master like a dog; but, feed and clothe him
well,---work him moderately---surround him with physical comfort,---and
dreams of freedom intrude. Give him a \emph{bad} master, and he aspires
to a \emph{good} master; give him a good master, and he wishes to become
his \emph{own} master. Such is human nature. You may hurl a man so low,
beneath the level of his kind, that he loses all just ideas of his
natural position; but elevate him a little, and {}the clear conception
of rights rises to life and power, and leads him onward. Thus elevated,
a little, at Freeland's, the dreams called into being by that good man,
Father Lawson, when in Baltimore, began to visit me; and shoots from the
tree of liberty began to put forth tender buds, and dim hopes of the
future began to dawn.

I found myself in congenial society, at Mr. Freeland's. There were Henry
Harris, John Harris, Handy Caldwell, and Sandy
Jenkins.\textsuperscript{\protect\hyperlink{cite_note-1}{{[}1{]}}}

Henry and John were brothers, and belonged to Mr. Freeland. They were
both remarkably bright and intelligent, though neither of them could
read. Now for mischief! I had not been long at Freeland's before I was
up to my old tricks. I early began to address my companions on the
subject of education, and the advantages of intelligence over ignorance,
and, as far as I dared, I tried to show the agency of ignorance in
keeping men in slavery. Webster's spelling book and the Columbian Orator
were looked into again. As summer came on, and the long Sabbath days
stretched themselves over our idleness, I became uneasy, and wanted a
Sabbath school, in which to exercise my gifts, and to impart the little
knowledge of letters which I possessed, to my brother slaves. A house
was hardly necessary in the summer time; I could hold my school under
the shade {}of an old oak tree, as well as any where else. The thing
was, to get the scholars, and to have them thoroughly imbued with the
desire to learn. Two such boys were quickly secured, in Henry and John,
and from them the contagion spread. I was not long in bringing around me
twenty or thirty young men, who enrolled themselves, gladly, in my
Sabbath school, and were willing to meet me regularly, under the trees
or elsewhere, for the purpose of learning to read. It was surprising
with what ease they provided themselves with spelling books. These were
mostly the cast off books of their young masters or mistresses. I
taught, at first, on our own farm. All were impressed with the necessity
of keeping the matter as private as possible, for the fate of the St.
Michael's attempt was notorious, and fresh in the minds of all. Our
pious masters, at St. Michael's, must not know that a few of their dusky
brothers were learning to read the word of God, lest they should come
down upon us with the lash and chain. We might have met to drink whisky,
to wrestle, fight, and to do other unseemly things, with no fear of
interruption from the saints or the sinners of St. Michael's.

But, to meet for the purpose of improving the mind and heart, by
learning to read the sacred scriptures, was esteemed a most dangerous
nuisance, to be instantly stopped. The slaveholders of St. Michael's,
like slaveholders elsewhere, would always prefer to see the slaves
engaged in degrading sports, rather than to see them acting like moral
and accountable beings.

Had any one asked a religious white man, in St. {}Michael's, twenty
years ago, the names of three men in that town, whose lives were most
after the pattern of our Lord and Master, Jesus Christ, the first three
would have been as follows:

\textsc{Garrison West}, \emph{Class Leader.}\\
\textsc{Wright Fairbanks}, \emph{Class Leader}\\
\textsc{Thomas Auld}, \emph{Class Leader.}

And yet, these were the men who ferociously rushed in upon my Sabbath
school, at St. Michael's, armed with mob-like missiles, and forbade our
meeting again, on pain of having our backs made bloody by the lash. This
same Garrison West was my class leader, and I must say, I thought him a
christian, until he took part in breaking up my school. He led me no
more after that. The plea for this outrage was then, as it is now and at
all times,---the danger to good order. If the slaves learnt to read,
they would learn something else, and something worse. The peace of
slavery would be disturbed; slave rule would be endangered. I leave the
reader to characterize a system which is endangered by such causes. I do
not dispute the soundness of the reasoning. It is perfectly sound; and,
if slavery be \emph{right}, Sabbath schools for teaching slaves to read
the bible are \emph{wrong}, and ought to be put down. These christian
class leaders were, to this extent, consistent. They had settled the
question, that slavery is \emph{right}, and, by that standard, they
determined that Sabbath schools are wrong. To be sure, they were
Protestant, and held to the great Protestant right of every man to
"\emph{search the scriptures}" for himself; but, then, to all general
rules, there are \emph{exceptions}. How convenient! what crimes, may not
{}be committed under the doctrine of the last remark. But, my dear,
class leading Methodist brethren, did not condescend to give give me a
reason for breaking up the Sabbath school at St. Michael's; it was
enough that they had determined upon its destruction. I am, however,
digressing.

After getting the school cleverly into operation, the second
time---holding it in the woods, behind the barn, and in the shade of
trees---I succeeded in inducing a free colored man, who lived several
miles from our house, to permit me to hold my school in a room at his
house. He, very kindly, gave me this liberty; but he incurred much peril
in doing so, for the assemblage was an unlawful one. I shall not
mention, here, the name of this man; for it might, even now, subject him
to persecution, although the offenses were committed more than twenty
years ago. I had, at one time, more than forty scholars, all of the
right sort; and many of them succeeded in learning to read. I have met
several slaves from Maryland, who were once my scholars; and who
obtained their freedom, I doubt not, partly in consequence of the ideas
imparted to them in that school. I have had various employments during
my short life; but I look back to \emph{none} with more satisfaction,
than to that afforded by my Sunday school. An attachment, deep and
lasting, sprung up between me and my persecuted pupils, which made my
parting from them intensely grievous; and, when I think that most of
these dear souls are yet shut up in this abject thralldom, I am
overwhelmed with grief.

Besides my Sunday school, I devoted three {}evenings a week to my fellow
slaves, during the winter. Let the reader reflect upon the fact, that,
in this christian country, men and women are hiding from professors of
religion, in barns, in the woods and fields, in order to learn to read
the \emph{holy bible}. Those dear souls, who came to my Sabbath school,
came \emph{not} because it was popular or reputable to attend such a
place, for they came under the liability of having forty stripes laid on
their naked backs. Every moment they spent in my school, they were under
this terrible liability; and, in this respect, I was a sharer with them.
Their minds had been cramped and starved by their cruel masters; the
light of education had been completely excluded; and their hard earnings
had been taken to educate their master's children. I felt a delight in
circumventing the tyrants, and in blessing the victims of their curses.

The year at Mr. Freeland's passed off very smoothly, to outward seeming.
Not a blow was given me during the whole year. To the credit of Mr.
Freeland,---irreligious though he was---it must be stated, that he was
the best master I ever had, until I became my own master, and assumed
for myself, as I had a right to do, the responsibility of my own
existence and the exercise of my own powers. For much of the
happiness---or absence of misery---with which I passed this year with
Mr. Freeland, I am indebted to the genial temper and ardent friendship
of my brother slaves. They were, every one of them, manly, generous and
brave, yes; I say they were brave, and I will add, fine looking. It is
seldom the lot of mortals to have truer and better friends than were the
slaves {}on this farm. It is not uncommon to charge slaves with great
treachery toward each other, and to believe them incapable of confiding
in each other; but I must say, that I never loved, esteemed, or confided
in men, more than I did in these. They were as true as steel, and no
band of brothers could have been more loving. There were no mean
advantages taken of each other, as is sometimes the case were slaves are
situated as we were; no tattling; no giving each other bad names to Mr.
Freeland; and no elevating one at the expense of the other. We never
undertook to do any thing, of any importance, which was likely to affect
each other, without mutual consultation. We were generally a unit, and
moved together. Thoughts and sentiments were exchanged between us, which
might well be called very incendiary, by oppressors and tyrants; and
perhaps the time has not even now come, when it is safe to unfold all
the flying suggestions which arise in the minds of intelligent slaves.
Several of my friends and brothers, if yet alive, are still in some part
of the house of bondage; and though twenty years have passed away, the
suspicious malice of slavery might punish them for even listening to my
thoughts.

The slaveholder, kind or cruel, is a slaveholder still---the every hour
violator of the just and inalienable rights of man; and he is,
therefore, every hour silently whetting the knife of vengeance for his
own throat. He never lisps a syllable in commendation of the fathers of
this republic, nor denounces any attempted oppression of himself,
without inviting the knife to his {}own throat, and asserting the rights
of rebellion for his own slaves.

The year is ended, and we are now in the midst of the Christmas
holidays, which are kept this year as last, according to the general
description previously given.

\begin{center}\rule{0.5\linewidth}{\linethickness}\end{center}

\begin{enumerate}
\item
  \hypertarget{cite_note-1}{}

  {\protect\hyperlink{cite_ref-1}{↑}} {This is the same man who gave me
  the roots to prevent my being whipped by Mr. Covey. He was ``a clever
  soul.'' We used frequently to talk about the fight with Covey, and as
  often as we did so, he would claim my success as the result of the
  roots which he gave me. This superstition is very common among the
  more ignorant slaves. A slave seldom dies, but that his death is
  attributed to trickery.}
\end{enumerate}
