{}

~

{CHAPTER II.}

THE AUTHOR REMOVED FROM HIS FIRST HOME.

{THE NAME ``OLD MASTER'' A TERROR---COLONEL LLOYD'S PLANTATION---WYE
RIVER---WHENCE ITS NAME---POSITION OF THE LLOYDS---HOME
ATTRACTION---MEET OFFERING---JOURNEY FROM TUCKAHOE TO WYE RIVER---SCENE
ON REACHING OLD MASTER'S---DEPARTURE OF GRANDMOTHER---STRANGE MEETING OF
SISTERS AND BROTHERS---REFUSAL TO BE COMFORTED---SWEET SLEEP.}

\textsc{That} mysterious individual referred to in the first chapter as
an object of terror among the inhabitants of our little cabin, under the
ominous title of ``old master,'' was really a man of some consequence.
He owned several farms in Tuckahoe; was the chief clerk and butler on
the home plantation of Col. Edward Lloyd; had overseers on his own
farms; and gave directions to overseers on the farms belonging to Col.
Lloyd. This plantation is situated on Wye river---the river receiving
its name, doubtless, from Wales, where the Lloyds originated. They (the
Lloyds) are an old and honored family in Maryland, exceedingly wealthy.
The home plantation, where they have resided, perhaps for a century or
more, is one of the largest, most fertile, and best appointed, in the
state.

About this plantation, and about that queer old master---who must be
something more than a man, {}and something worse than an angel---the
reader will easily imagine that I was not only curious, but eager, to
know all that could be known. Unhappily for me, however, all the
information I could get concerning him but increased my great dread of
being carried thither---of being separated from and deprived of the
protection of my grandmother and grandfather. It was, evidently, a great
thing to go to Col. Lloyd's; and I was not without a little curiosity to
see the place; but no amount of coaxing could induce in me the wish to
remain there. The fact is, such was my dread of leaving the little
cabin, that I wished to remain little forever, for I knew the taller I
grew the shorter my stay. The old cabin, with its rail floor and rail
bedsteads up stairs, and its clay floor down stairs, and its dirt
chimney, and windowless sides, and that most curious piece of
workmanship of all the rest, the ladder stairway, and the hole curiously
dug in front of the fire-place, beneath which grandmammy placed the
sweet potatoes to keep them from the frost, was \textsc{my home}---the
only home I ever had; and I loved it, and all connected with it. The old
fences around it, and the stumps in the edge of the woods near it, and
the squirrels that ran, skipped, and played upon them, were objects of
interest and affection. There, too, right at the side of the hut, stood
the old well, with its stately and skyward-pointing beam, so aptly
placed between the limbs of what had once been a tree, and so nicely
balanced that I could move it up and down with only one hand, and could
get a drink myself without calling for help. Where else in the world
could such a well be found, and where could {}such another home be met
with? Nor were these all the attractions of the place. Down in a little
valley, not far from grandmammy's cabin, stood Mr. Lee's mill, where the
people came often in large numbers to get their corn ground. It was a
water-mill; and I never shall be able to tell the many things thought
and felt, while I sat on the bank and watched that mill, and the turning
of that ponderous wheel. The mill-pond, too, had its charms; and with my
pin-hook, and thread line, I could get \emph{nibbles}, if I could catch
no fish. But, in all my sports and plays, and in spite of them, there
would, occasionally, come the painful foreboding that I was not long to
remain there, and that I must soon be called away to the home of old
master.

I was \textsc{a slave}---born a slave---and though the fact was
incomprehensible to me, it conveyed to my mind a sense of my entire
dependence on the will of \emph{somebody} I had never seen; and, from
some cause or other, I had been made to fear this somebody above all
else on earth. Born for another's benefit, as the \emph{firstling} of
the cabin flock I was soon to be selected as a meet offering to the
fearful and inexorable \emph{demigod}, whose huge image on so many
occasions haunted my childhood's imagination. When the time of my
departure was decided upon, my grandmother, knowing my fears, and in
pity for them, kindly kept me ignorant of the dreaded event about to
transpire. Up to the morning (a beautiful summer morning) when we were
to start, and, indeed, during the whole journey---a journey which, child
as I was, I remember as well as if it were yesterday---she kept the sad
fact {}hidden from me. This reserve was necessary; for, could I have
known all, I should have given grandmother some trouble in getting me
started. As it was, I was helpless, and she---dear woman!---led me along
by the hand, resisting, with the reserve and solemnity of a priestess,
all my inquiring looks to the last.

The distance from Tuckahoe to Wye river---where my old master
lived---was full twelve miles, and the walk was quite a severe test of
the endurance of my young legs. The journey would have proved too severe
for me, but that my dear old grandmother---blessings on her
memory!---afforded occasional relief by ``toting'' me (as Marylanders
have it) on her shoulder. My grandmother, though advanced in years---as
was evident from more than one gray hair, which peeped from between the
ample and graceful folds of her newly-ironed bandana turban---was yet a
woman of power and spirit. She was marvelously straight in figure,
elastic, and muscular. I seemed hardly to be a burden to her. She would
have ``toted'' me farther, but that I felt myself too much of a man to
allow it, and insisted on walking. Releasing dear grandmamma from
carrying me, did not make me {altogther} independent of her, when we
happened to pass through portions of the somber woods which lay between
Tuckahoe and Wye river. She often found me increasing the energy of my
grip, and holding her clothing, lest something should come out of the
woods and eat me up. Several old logs and stumps imposed upon me, and
got themselves taken for wild beasts. I could see their legs, eyes, and
ears, {}or I could see something like eyes, legs, and ears, till I got
close enough to them to see that the eyes were knots, washed white with
rain, and the legs were broken limbs, and the ears, only ears owing to
the point from which they were seen. Thus early I learned that the point
from which a thing is viewed is of some importance.

As the day advanced the heat increased; and it was not until the
afternoon that we reached the much dreaded end of the journey. I found
myself in the midst of a group of children of many colors; black, brown,
copper colored, and nearly white. I had not seen so many children
before. Great houses loomed up in different directions, and a great many
men and women were at work in the fields. All this hurry, noise, and
singing was very different from the stillness of Tuckahoe. As a new
comer, I was an object of special interest; and, after laughing and
yelling around me, and playing all sorts of wild tricks, they (the
children) asked me to go out and play with them. This I refused to do,
preferring to stay with grandmamma. I could not help feeling that our
being there boded no good to me. Grandmamma looked sad. She was soon to
lose another object of affection, as she had lost many before. I knew
she was unhappy, and the shadow fell from her brow on me, though I knew
not the cause.

All suspense, however, must have an end; and the end of mine, in this
instance, was at hand. Affectionately patting me on the head, and
exhorting me to be a good boy, grandmamma told me to go and play with
the little children. ``They are kin to you,'' {}said she; ``go and play
with them.'' Among a number of cousins were Phil, Tom, Steve, and Jerry,
Nance and Betty.

Grandmother pointed out my brother \textsc{Perry}, my sister
\textsc{Sarah}, and my sister \textsc{Eliza}, who stood in the group. I
had never seen my brother nor my sisters before; and, though I had
sometimes heard of them, and felt a curious interest in them, I really
did not understand what they were to me, or I to them. We were brothers
and sisters, but what of that? Why should they be attached to me, or I
to them? Brothers and sisters we were by blood; but \emph{slavery} had
made us strangers. I heard the words brother and sisters, and knew, they
must mean something; but slavery had robbed these terms of their true
meaning. The experience through which I was passing, they had passed
through before. They had already been initiated into the mysteries of
old master's domicile, and they seemed to look upon me with a certain
degree of compassion; but my heart clave to my grandmother. Think it not
strange, dear reader, that so little sympathy of feeling existed between
us. The conditions of brotherly and sisterly feeling were wanting---we
had never nestled and played together. My poor mother, like many other
slave-women, had \emph{many children}, but \textsc{no family}! The
domestic hearth, with its holy lessons and precious endearments, is
abolished in the case of a slave-mother and her children. ``Little
children, love one another,'' are words seldom heard in a slave cabin.

I really wanted to play with my brother and sisters, but they were
strangers to me, and I was full of {}fear that grandmother might leave
without taking me with her. Entreated to do so, however, and that, too,
by my dear grandmother, I went to the back part of the house, to play
with them and the other children. \emph{Play}, however, I did not, but
stood with my back against the wall, witnessing the playing of the
others. At last, while standing there, one of the children, who had been
in the kitchen, ran up to me, in a sort of roguish glee, exclaiming,
``Fed, Fed! grandmammy gone! grandmammy gone!'' I could not believe it;
yet, fearing the worst, I ran into the kitchen, to see for myself, and
found it even so. Grandmammy had indeed gone, and was now far away,
``clean'' out of sight. I need not tell all that happened now. Almost
heart-broken at the discovery, I fell upon the ground, and wept a boy's
bitter tears, refusing to be comforted. My brother and sisters came
around me, and said, ``Don't cry,'' and gave me peaches and pears, but I
flung them away, and refused all their kindly advances. I had never been
deceived before; and I felt not only grieved at parting---as I supposed
forever---with my grandmother, but indignant that a trick had been
played upon me in a matter so serious.

It was now late in the afternoon. The day had been an exciting and
wearisome one, and I knew not how or where, but I suppose I sobbed
myself to sleep. There is a healing in the angel wing of sleep, even for
the slave-boy; and its balm was never more welcome to any wounded soul
than it was to mine, the first night I spent at the domicile of old
master. The reader may be surprised that I narrate so minutely {}an
incident apparently so trivial, and which must have occurred when I was
not more than seven years old; but as I wish to give a faithful history
of my experience in slavery, I cannot withhold a circumstance which, at
the time, affected me so deeply. Besides, this was, in fact, my first
introduction to the realities of slavery.

~
