{}

~

{CHAPTER XIV.}

EXPERIENCE IN ST. MICHAEL'S.

{THE VILLAGE---ITS INHABITANTS---THEIR OCCUPATION AND LOW
PROPENSITIES---CAPTAIN THOMAS AULD---HIS CHARACTER---HIS SECOND WIFE,
ROWENA---WELL MATCHED---SUFFERINGS FROM HUNGER---OBLIGED TO TAKE
FOOD---MODE OF ARGUMENT IN VINDICATION THEREOF---NO MORAL CODE OF FREE
SOCIETY CAN APPLY TO SLAVE SOCIETY---SOUTHERN CAMP MEETING---WHAT MASTER
THOMAS DID THERE---HOPES---SUSPICIONS ABOUT HIS CONVERSION---THE
RESULT---FAITH AND WORKS ENTIRELY AT VARIANCE---HIS RISE AND PROGRESS IN
THE CHURCH---POOR COUSIN ``HENNY''---HIS TREATMENT OF HER---THE
METHODIST PREACHERS---THEIR UTTER DISREGARD OF US---ONE EXCELLENT
EXCEPTION---REV. GEORGE COOKMAN---SABBATH SCHOOL---HOW BROKEN UP AND BY
WHOM---A FUNERAL PALL CAST OVER ALL MY PROSPECTS---COVEY THE
NEGRO-BREAKER.}

\textsc{St. Michael's}, the village in which was now my new home,
compared favorably with villages in slave states, generally. There were
a few comfortable dwellings in it, but the place, as a whole, wore a
dull, slovenly, enterprise-forsaken aspect. The mass of the buildings
were of wood; they had never enjoyed the artificial adornment of paint,
and time and storms had worn off the bright color of the wood, leaving
them almost as black as buildings charred by a conflagration.

St. Michael's had, in former years, (previous to 1833, for that was the
year I went to reside there,) enjoyed some reputation as a ship building
community, but that business had almost entirely given place to oyster
fishing, for the Baltimore and Philadelphia markets---a course of life
highly unfavorable to morals, {}industry, and manners. Miles river was
broad, and its oyster fishing grounds were extensive; and the fishermen
were out, often, all day, and a part of the night, during autumn, winter
and spring. This exposure was an excuse for carrying with them, in
considerable quantities, spirituous liquors, the then supposed best
antidote for cold. Each canoe was supplied with its jug of rum; and
tippling, among this class of the citizens of St. Michael's, became
general. This drinking habit, in an ignorant population, fostered
coarseness, vulgarity and an indolent disregard for the social
improvement of the place, so that it was admitted, by the few sober,
thinking people who remained there, that St. Michael's had become a very
\emph{unsaintly}, as well as an unsightly place, before I went there to
reside.

I left Baltimore, for St. Michael's in the month of March, 1833. I know
the year, because it was the one succeeding the first cholera in
Baltimore, and was the year, also, of that strange phenomenon, when the
heavens seemed about to part with its starry train. I witnessed this
gorgeous spectacle, and was awe-struck. The air seemed filled with
bright, descending messengers from the sky. It was about daybreak when I
saw this sublime scene. I was not without the suggestion, at the moment,
that it might be the harbinger of the coming of the Son of Man; and, in
my then state of mind, I was prepared to hail Him as my friend and
deliverer. I had read, that the ``stars shall fall from heaven;'' and
they were now falling. I was suffering much in my mind. It did seem that
every time the young tendrils of my affection became attached, they were
{}rudely broken by some unnatural outside power; and I was beginning to
look away to heaven for the rest denied me on earth.

But, to my story. It was now more than seven years since I had lived
with Master Thomas Auld, in the family of my old master, on Col. Lloyd's
plantation. We were almost entire strangers to each other; for, when I
knew him at the house of my old master, it was not as a \emph{master},
but simply as ``Captain Auld,'' who had married old master's daughter.
All my lessons concerning his temper and disposition, and the best
methods of pleasing him, were yet to be learnt. Slaveholders, however,
are not very ceremonious in approaching a slave; and my ignorance of the
new material in the shape of a master was but transient. Nor was my new
mistress long in making known her animus. She was not a ``Miss
Lucretia,'' traces of whom I yet remembered, and the more especially, as
I saw them shining in the face of little Amanda, her daughter, now
living under a step-mother's government. I had not forgotten the soft
hand, guided by a tender heart, that bound up with healing balsam the
gash made in my head by Ike, the son of Abel. Thomas and Rowena, I found
to be a well-matched pair. \emph{He} was stingy, and \emph{she} was
cruel; and---what was quite natural in such cases---she possessed the
ability to make him as cruel as herself, while she could easily descend
to the level of his meanness. In the house of Master Thomas, I was
made---for the first time in seven years---to feel the pinchings of
hunger, and this was not very easy to bear.

For, in all the changes of Master Hugh's family, {}there was no change
in the bountifulness with which they supplied me with food. Not to give
a slave enough to eat, is meanness intensified, and it is so recognized
among slaveholders generally, in Maryland. The rule is, no matter how
coarse the food, only let there be enough of it. This is the theory,
and---in the part of Maryland I came from---the general practice accords
with this theory. Lloyd's plantation was an exception, as was, also, the
house of Master Thomas Auld.

All know the lightness of Indian corn-meal, as an article of food, and
can easily judge from the following facts whether the statements I have
made of the stinginess of Master Thomas, are borne out. There were four
slaves of us in the kitchen, and four whites in the great house---Thomas
Auld, Mrs. Auld, Hadaway Auld, (brother of Thomas Auld,) and little
Amanda. The names of the slaves in the kitchen, were Eliza, my sister;
Priscilla, my aunt; Henny, my cousin; and myself. There were eight
persons in the family. There was, each week, one half bushel of
corn-meal brought from the mill; and in the kitchen, corn-meal was
almost our exclusive food, for very little else was allowed us. Out of
this half bushel of corn-meal, the family in the great house had a small
loaf every morning; thus leaving us, in the kitchen, with not quite a
half a peck of meal per week, apiece. This allowance was less than half
the allowance of food on Lloyd's plantation. It was not enough to
subsist upon; and we were, therefore, reduced to the wretched necessity
of living at the expense of our neighbors. We were compelled either to
beg, or to steal, and we did both. I frankly confess, that while {}I
hated everything like stealing, \emph{as such}, I nevertheless did not
hesitate to take food, when I was hungry, wherever I could find it. Nor
was this practice the mere result of an unreasoning instinct; it was, in
my case, the result of a clear apprehension of the claims of morality. I
weighed and considered the matter closely, before I ventured to satisfy
my hunger by such means. Considering that my labor and person were the
property of Master Thomas, and that I was by him deprived of the
necessaries of life---necessaries obtained by my own labor---it was easy
to deduce the right to supply myself with what was my own. It was simply
appropriating what was my own to the use of my master, since the health
and strength derived from such food were exerted in \emph{his} service.
To be sure, this was stealing, according to the law and gospel I heard
from St. Michael's pulpit; but I had already begun to attach less
importance to what dropped from that quarter, on that point, while, as
yet, I retained my reverence for religion. It was not always convenient
to steal from master, and the same reason why I might, innocently, steal
from him, did not seem to justify me in stealing from others. In the
case of my master, it was only a question of \emph{removal}---the taking
his meat out of one tub, and putting it into another; the ownership of
the meat was not affected by the transaction. At first, he owned it in
the \emph{tub}, and last, he owned it in \emph{me}. His meat house was
not always open. There was a strict watch kept on that point, and the
key was on a large bunch in Rowena's pocket. A great many times have we,
poor creatures, been severely pinched with hunger, {}when meat and bread
have been moulding under the lock, while the key was in the pocket of
our mistress. This had been so when she \emph{knew} we were nearly half
starved; and yet, that mistress, with saintly air, would kneel with her
husband, and pray each morning that a merciful God would bless them in
basket and in store, and save them, at last, in his kingdom. But I
proceed with the argument.

It was necessary that the right to steal from \emph{others} should be
established; and this could only rest upon a wider range of
generalization than that which supposed the right to steal from my
master.

It was sometime before I arrived at this clear right. The reader will
get some idea of my train of reasoning, by a brief statement of the
case. ``I am,'' thought I, ``not only the slave of Master Thomas, but I
am the slave of society at large. Society at large has bound itself, in
form and in fact, to assist Master Thomas in robbing me of my rightful
liberty, and of the just reward of my labor; therefore, whatever rights
I have against Master Thomas, I have, equally, against those
confederated with him in robbing me of liberty. As society has marked me
out as privileged plunder, on the principle of self-preservation I am
justified in plundering in turn. Since each slave belongs to all; all
must, therefore, belong to each.''

I shall here make a profession of faith which may shock some, offend
others, and be dissented from by all. It is this: Within the bounds of
his just earnings, I hold that the slave is fully justified in helping
himself to the \emph{gold and silver, and the best apparel of}
{}\emph{his master, or that of any other slaveholder; and that such
taking is not stealing in any just sense of that word.}

The morality of \emph{free} society can have no application to
\emph{slave} society. Slaveholders have made it almost impossible for
the slave to commit any crime, known either to the laws of God or to the
laws of man. If he steals, he takes his own; if he kills his master, he
imitates only the heroes of the revolution. Slaveholders I hold to be
individually and collectively responsible for all the evils which grow
out of the horrid relation, and I believe they will be so held at the
judgment, in the sight of a just God. Make a man a slave, and you rob
him of moral responsibility. Freedom of choice is the essence of all
accountability. But my kind readers are, probably, less concerned about
my opinions, than about that which more nearly touches my personal
experience; albeit, my opinions have, in some sort, been formed by that
experience.

Bad as slaveholders are, I have seldom met with one so entirely
destitute of every element of character capable of inspiring respect, as
was my present master, Capt. Thomas Auld.

When I lived with him, I thought him incapable of a noble action. The
leading trait in his character was intense selfishness. I think he was
fully aware of this fact himself, and often tried to conceal it. Capt.
Auld was not a \emph{born} slaveholder---not a birthright member of the
slaveholding oligarchy. He was only a slaveholder by
\emph{marriage-right;} and, of all slaveholders, these latter are,
\emph{by far}, the most exacting. There was in him all the love of
domination, the pride of mastery, and the swagger of {}authority, but
his rule lacked the vital element of consistency. He could be cruel; but
his methods of showing it were cowardly, and evinced his meanness rather
than his spirit. His commands were strong, his enforcement weak.

Slaves are not insensible to the whole-souled characteristics of a
generous, dashing slaveholder, who is fearless of consequences; and they
prefer a master of this bold and daring kind---even with the risk of
being shot down for impudence---to the fretful, little soul, who never
uses the lash but at the suggestion of a love of gain.

Slaves, too, readily distinguish between the birthright bearing of the
original slaveholder and the assumed attitudes of the accidental
slaveholder; and while they cannot respect either, they certainly
despise the latter more than the former.

The luxury of having slaves wait upon him was something new to Master
Thomas; and for it he was wholly unprepared. He was a slaveholder,
without the ability to hold or manage his slaves. We seldom called him
``master,'' but generally addressed him by his ``bay craft''
title---"\emph{Capt. Auld.}" It is easy to see that such conduct might
do much to make him appear awkward, and, consequently, fretful. His wife
was especially solicitous to have us call her husband ``master.'' Is
your \emph{master} at the store?``---''Where is your
\emph{master?}``---''Go and tell your \emph{master}``---''I will make
your \emph{master} acquainted with your conduct"---she would say; but we
were inapt scholars. Especially were I and my sister Eliza inapt in this
particular. Aunt Priscilla was less {}stubborn and defiant in her spirit
than Eliza and myself; and, I think, her road was less rough than ours.

In the month of August, 1833, when I had almost become desperate under
the treatment of Master Thomas, and when I entertained more strongly
than ever the oft-repeated determination to run away, a circumstance
occurred which seemed to promise brighter and better days for us all. At
a Methodist camp-meeting, held in the Bay Side, (a famous place for
camp-meetings,) about eight miles from St. Michael's, Master Thomas came
out with a profession of religion. He had long been an object of
interest to the church, and to the ministers, as I had seen by the
repeated visits and lengthy exhortations of the latter. He was a fish
quite worth catching, for he had money and standing. In the community of
St. Michael's he was equal to the best citizen. He was strictly
temperate; \emph{perhaps,} from principle, but most likely, from
interest. There was very little to do for him, to give him the
appearance of piety, and to make him a pillar in the church. Well, the
camp-meeting continued a week; people gathered from all parts of the
county, and two steamboat loads came from Baltimore. The ground was
happily chosen; seats were arranged; a stand erected; a rude altar
fenced in, fronting the preachers' stand, with straw in it for the
accommodation of mourners. This latter would hold at least one hundred
persons. In front, and on the sides of the preachers' stand, and outside
the long rows of seats, rose the first class of stately tents, each
vieing with the other in strength, neatness, and capacity for
accommodating its inmates. Behind this {}first circle of tents was
another, less imposing, which reached round the camp-ground to the
speakers' stand. Outside this second class of tents were covered wagons,
ox carts, and vehicles of every shape and size. These served as tents to
their owners. Outside of these, huge fires were burning, in all
directions, where roasting, and boiling, and frying, were going on, for
the benefit of those who were attending to their own spiritual welfare
within the circle. \emph{Behind} the preachers' stand, a narrow space
was marked out for the use of the colored people. There were no seats
provided for this class of persons; the preachers addressed them,
"\emph{over the left,}" if they addressed them at all. After the
preaching was over, at every service, an invitation was given to
mourners to come into the pen; and, in some cases, ministers went out to
persuade men and women to come in. By one of these ministers, Master
Thomas Auld was persuaded to go inside the pen. I was deeply interested
in that matter, and followed; and, though colored people were not
allowed either in the pen or in front of the preachers' stand, I
ventured to take my stand at a sort of half-way place between the blacks
and whites, where I could distinctly see the movements of mourners, and
especially the progress of Master Thomas.

``If he has got religion,'' thought I, ``he will emancipate his slaves;
and if he should not do so much as this, he will, at any rate, behave
toward us more kindly, and feed us more generously than he has
heretofore done.'' Appealing to my own religious experience, and judging
my master by what was {}true in my own case, I could not regard him as
soundly converted, unless some such good results followed his profession
of religion.

But in my expectations I was doubly disappointed; Master Thomas was
\emph{Master Thomas} still. The fruits of his righteousness were to show
themselves in no such way as I had anticipated. His conversion was not
to change his relation toward men---at any rate not toward
\textsc{black} men---but toward God. My faith, I confess, was not great.
There was something in his appearance that, in my mind, cast a doubt
over his conversion. Standing where I did, I could see his every
movement. I watched very narrowly while he remained in the little pen;
and although I saw that his face was extremely red, and his hair
disheveled, and though I heard him groan, and saw a stray tear halting
on his cheek, as if inquiring ``which way shall I go?''---I could not
wholly confide in the genuineness of his {coversion}. The hesitating
behavior of that tear-drop, and its loneliness, distressed me, and cast
a doubt upon the whole transaction, of which it was a part. But people
said, "\emph{Capt. Auld had come through}," and it was for me to hope
for the best. I was bound to do this, in charity, for I, too, was
religious, and had been in the church full three years, although now I
was not more than sixteen years old. Slaveholders may, sometimes, have
confidence in the piety of some of their slaves; but the slaves seldom
have confidence in the piety of their masters. "\emph{He cant go to
heaven with our blood in his skirts}," a settled point in the creed of
every slave; rising superior to all teaching to the contrary, and
standing {}forever as a fixed fact. The highest evidence the slaveholder
can give the slave of his acceptance with God, is the emancipation of
his slaves. This is proof that he is willing to give up all to God, and
for the sake of God. Not to do this, was, in my estimation, and in the
opinion of all the slaves, an evidence of half-heartedness, and wholly
inconsistent with the idea of genuine conversion. I had read, also,
somewhere in the Methodist Discipline, the following question and
answer:

"\emph{Question}. What shall be done for the extirpation of slavery?

"\emph{Answer}. We declare that we are as much as ever convinced of the
great evil of slavery; therefore, no slaveholder shall be eligible to
any official station in our church."

These words sounded in my ears for a long time, and encouraged me to
hope. But, as I have before said, I was doomed to disappointment. Master
Thomas seemed to be aware of my hopes and expectations concerning him. I
have thought, before now, that he looked at me in answer to my glances,
as much as to say, ``I will teach you, young man, that, though I have
parted with my sins, I have not parted with my sense. I shall hold my
slaves, and go to heaven too.''

Possibly, to convince us that we must not presume \emph{too much} upon
his recent conversion, he became rather more rigid and stringent in his
exactions. There always was a scarcity of good nature about the man; but
now his whole countenance was \emph{soured} over with the seemings of
piety. His {}religion, therefore, neither made him emancipate his
slaves, nor caused him to treat them with greater humanity. If religion
had any effect on his character at all, it made him more cruel and
hateful in all his ways. The natural wickedness of his heart had not
been removed, but only reënforced, by the profession of religion. Do I
judge him harshly? God forbid. Facts \emph{are} facts. Capt. Auld made
the greatest profession of piety. His house was, literally, a house of
prayer. In the morning, and in the evening, loud prayers and hymns were
heard there, in which both himself and his wife joined; yet, \emph{no
more meal} was brought from the mill, \emph{no more attention} was paid
to the moral welfare of the kitchen; and nothing was done to make us
feel that the heart of Master Thomas was one whit better than it was
before he went into the little pen, opposite to the preachers' stand, on
the camp ground.

Our hopes (founded on the discipline) soon vanished; for the authorities
let him into the church \emph{at once}, and before he was out of his
term of \emph{probation}, I heard of his leading class! He distinguished
himself greatly among the brethren, and was soon an exhorter. His
progress was almost as rapid as the growth of the fabled vine of Jack's
bean. No man was more active than he, in revivals. He would go many
miles to assist in carrying them on, and in getting outsiders interested
in religion. His house being one of the holiest, if not the happiest in
St. Michael's, became the ``preachers' home.'' These preachers evidently
liked to share Master Thomas's hospitality; for while he \emph{starved}
us, he \emph{stuffed} them. Three or four of these {}ambassadors of the
gospel---according to slavery---have been there at a time; all living on
the fat of the land, while we, in the kitchen, were nearly starving. Not
often did we get a smile of recognition from these holy men. They seemed
almost as unconcerned about our getting to heaven, as they were about
our getting out of slavery. To this general charge there was one
exception---the Rev. \textsc{George Cookman}. Unlike Rev. Messrs.
Storks, Ewry, Hickey, Humphrey and Cooper, (all whom were on the St.
Michael's circuit,) he kindly took an interest in our temporal and
spiritual welfare. Our souls and our bodies were all alike sacred in his
sight; and he really had a good deal of genuine anti-slavery feeling
mingled with his colonization ideas. There was not a slave in our
neighborhood that did not love, and almost venerate, Mr. Cookman. It was
pretty generally believed that he had been chiefly instrumental in
bringing one of the largest slaveholders---Mr. Samuel Harrison---in that
neighborhood, to emancipate all his slaves, and, indeed, the general
impression was, that Mr. Cookman had labored faithfully with
slaveholders, whenever he met them, to induce them to emancipate their
bondmen, and that he did this as a religious duty. When this good man
was at our house, we were all sure to be called in to prayers in the
morning; and he was not slow in making inquiries as to the state of our
minds, nor in giving us a word of exhortation and of encouragement.
Great was the sorrow of all the slaves, when this faithful preacher of
the gospel was removed from the Talbot county circuit. He was an
eloquent preacher, and possessed what few ministers, {}south of Mason
Dixon's line, possess, or \emph{dare} to show, viz: a warm and
philanthropic heart. The Mr. Cookman, of whom I speak, was an Englishman
by birth, and perished while on his way to England, on board the
ill-fated President. Could the thousands of slaves in Maryland, know the
fate of the good man, to whose words of comfort they were so largely
indebted, they would thank me for dropping a tear on this page, in
memory of their favorite preacher, friend and benefactor.

But, let me return to Master Thomas, and to my experience, after his
conversion. In Baltimore, I could, occasionally, get into a Sabbath
school, among the free children, and receive lessons, with the rest;
but, having already learned both to read and to write, I was more of a
teacher than a pupil, even there. When, however, I went back to the
Eastern Shore, and was at the house of Master Thomas, I was neither
allowed to teach, nor to be taught. The whole community---with but a
single exception, among the whites---frowned upon everything like
imparting instruction either to slaves or to free colored persons. That
single exception, a pious young man, named Wilson, asked me, one day, if
I would like to assist him in teaching a little Sabbath school, at the
house of a free colored man in St. Michael's, named James Mitchell. The
idea was to me a delightful one, and I told him I would gladly devote as
much of my Sabbaths as I could command, to that most laudable work. Mr.
Wilson soon mustered up a dozen old spelling books, and a few
testaments; and we commenced operations, with some twenty scholars, in
our {}Sunday school. Here, thought I, is something worth living for;
here is an excellent chance for usefulness; and I shall soon have a
company of young friends, lovers of knowledge, like some of my Baltimore
friends, from whom I now felt parted forever.

Our first Sabbath passed delightfully, and I spent the week after very
joyously. I could not go to Baltimore, but I could make a little
Baltimore here. At our second meeting, I learned that there was some
objection to the existence of the Sabbath school; and, sure enough, we
had scarcely got at work---\emph{good work}, simply teaching a few
colored children how to read the gospel of the Son of God---when in
rushed a mob, headed by Mr. Wright Fairbanks and Mr. Garrison West---two
class-leaders---and Master Thomas; who, armed with sticks and other
missiles, drove us off, and commanded us never to meet for such a
purpose again. One of this pious crew told me, that as for my part, I
wanted to be another Nat Turner; and if I did not look out, I should get
as many balls into me, as Nat did into him. Thus ended the infant
Sabbath school, in the town of St. Michael's. The reader will not be
surprised when I say, that the breaking up of my Sabbath school, by
these class-leaders, and professedly holy men, did not serve to
strengthen my religious convictions. The cloud over my St. Michael's
home grew heavier and blacker than ever.

It was not merely the agency of Master Thomas, in breaking up and
destroying my Sabbath school, that shook my confidence in the power of
southern religion to make men wiser or better; but I saw in him all the
cruelty and meanness, \emph{after} his conversion, {}which he had
exhibited before he made a profession of religion. His cruelty and
meanness were especially displayed in his treatment of my unfortunate
cousin, Henny, whose lameness made her a burden to him. I have no
extraordinary personal hard usage toward myself to complain of, against
him, but I have seen him tie up the lame and maimed woman, and whip her
in a manner most brutal, and shocking; and then, with blood-chilling
blasphemy, he would quote the passage of scripture, ``That servant which
knew his lord's will, and prepared not himself, neither did according to
his will, shall be beaten with many stripes.'' Master would keep this
lacerated woman tied up by her wrists, to a bolt in the joist, three,
four and five hours at a time. He would tie her up early in the morning,
whip her with a cowskin before breakfast; leave her tied up; go to his
store, and, returning to his dinner, repeat the castigation; laying on
the rugged lash, on flesh already made raw by repeated blows. He seemed
desirous to get the poor girl out of existence, or, at any rate, off his
hands. In proof of this, he afterwards gave her away to his sister
Sarah, (Mrs. Cline;) but, as in the case of Master Hugh, Henny was soon
returned on his hands. Finally, upon a pretense that he could do nothing
with her, (I use his own words,) he ``set her adrift, to take care of
herself.'' Here was a recently converted man, holding, with tight grasp,
the well-framed, and able bodied slaves left him by old master---the
persons, who, in freedom, could have taken care of themselves; yet,
turning loose the only cripple among them, virtually to starve and die.

{}No doubt, had Master Thomas been asked, by some pious northern
brother, \emph{why} he continued to sustain the relation of a
slaveholder, to those whom he retained, his answer would have been
precisely the same as many other religious slaveholders have returned to
that inquiry, viz: ``I hold my slaves for their own good.''

Bad as my condition was when I lived with Master Thomas, I was soon to
experience a life far more goading and bitter. The many differences
springing up between myself and Master Thomas, owing to the clear
perception I had of his character, and the boldness with which I
defended myself against his capricious complaints, led him to declare
that I was unsuited to his wants; that my city life had affected me
perniciously; that, in fact, it had almost ruined me for every good
purpose, and had fitted me for everything that was bad. One of my
greatest faults, or offenses, was that of letting his horse get away,
and go down to the farm belonging to his father-in-law. The animal had a
liking for that farm, with which I fully sympathized. Whenever I let it
out, it would go dashing down the road to Mr. Hamilton's, as if going on
a grand frolic. My horse gone, of course I must go after it. The
explanation of our mutual attachment to the place is the same; the horse
found there good pasturage, and I found there plenty of bread. Mr.
Hamilton had his faults, but starving his slaves was not among them. He
gave food, in abundance, and that, too, of an excellent quality. In Mr.
Hamilton's cook---Aunt Mary---I found a most generous and considerate
friend. She never allowed me to go {}there without giving me bread
enough to make good the deficiencies of a day or two. Master Thomas at
last resolved to endure my behavior no longer; he could neither keep me,
nor his horse, we liked so well to be at his father-in-law's farm. I had
now lived with him nearly nine months, and he had given me a number of
severe whippings, without any visible improvement in my character, or my
conduct; and now he was resolved to put me out---as he said---"\emph{to
be broken.}"

There was, in the Bay Side, very near the camp ground, where my master
got his religious impressions, a man named Edward Covey, who enjoyed the
execrated reputation, of being a first rate hand at breaking young
negroes. This Covey was a poor man, a farm renter; and this reputation,
(hateful as it was to the slaves and to all good men,) was, at the same
time, of immense advantage to him. It enabled him to get his farm tilled
with very little expense, compared with what it would have cost him
without this most extraordinary reputation. Some slaveholders thought it
an advantage to let Mr. Covey have the government of their slaves a year
or two, almost free of charge, for the sake of the excellent training
such slaves got under his happy management! Like some horse breakers,
noted for their skill, who ride the best horses in the country without
expense, Mr. Covey could have under him, the most fiery bloods of the
neighborhood, for the simple reward of returning them to their owners,
\emph{well broken}. Added to the natural fitness of Mr. Covey for the
duties of his profession, he was said to ``enjoy religion,'' {}and was
as strict in the cultivation of piety, as he was in the cultivation of
his farm. I was made aware of his character by some who had been under
his hand; and while I could not look forward to going to him with any
pleasure, I was glad to get away from St. Michael's. I was sure of
getting enough to eat at Covey's, even if I suffered in other respects.
\emph{This}, to a hungry man, is not a prospect to be regarded with
indifference.
