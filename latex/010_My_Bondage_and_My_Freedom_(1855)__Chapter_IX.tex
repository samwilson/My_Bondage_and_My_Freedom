\hypertarget{headerContainer}{}
\hypertarget{navigationHeader}{}
\protect\hypertarget{headerprevious}{}{←\href{/wiki/My_Bondage_and_My_Freedom_(1855)/Chapter_VIII}{Chapter
VIII}}

\textbf{\protect\hypertarget{header_title_text}{}{\href{/wiki/My_Bondage_and_My_Freedom_(1855)}{My
Bondage and My Freedom}}} ~(1855)~ \emph{by
\href{/wiki/Author:Frederick_Douglass}{\protect\hypertarget{header_author_text}{}{{Frederick
Douglass}}}}\\
\protect\hypertarget{header_section_text}{}{Chapter IX}

\protect\hypertarget{headernext}{}{\href{/wiki/My_Bondage_and_My_Freedom_(1855)/Chapter_X}{Chapter
X}→}

\hypertarget{navigationNotes}{}

\hypertarget{ws-data}{}
\protect\hypertarget{ws-article-id}{}{2336906}\protect\hypertarget{ws-title}{}{\href{/wiki/My_Bondage_and_My_Freedom_(1855)}{My
Bondage and My Freedom} --- \emph{Chapter
IX}}\protect\hypertarget{ws-author}{}{Frederick
Douglass}\protect\hypertarget{ws-year}{}{1855}

{\protect\hypertarget{129}{}{}}

~

{CHAPTER IX.}

PERSONAL TREATMENT OF THE AUTHOR.

{MISS LUCRETIA---HER KINDNESS---HOW IT WAS MANIFESTED---``IKE''---A
BATTLE WITH HIM---THE CONSEQUENCES THEREOF---MISS LUCRETIA'S
BALSAM---BREAD---HOW I OBTAINED IT---BEAMS OF SUNLIGHT AMIDST THE
GENERAL DARKNESS---SUFFERING FROM COLD---HOW WE TOOK OUR MEALS---ORDERS
TO PREPARE FOR BALTIMORE---OVERJOYED AT THE THOUGHT OF QUITTING THE
PLANTATION---EXTRAORDINARY CLEANSING---COUSIN TOM'S VERSION OF
BALTIMORE---ARRIVAL THERE---KIND RECEPTION GIVEN ME BY MRS. SOPHIA
AULD---LITTLE TOMMY---MY NEW POSITION---MY NEW DUTIES---A TURNING POINT
IN MY HISTORY.}

\textsc{I have} nothing cruel or shocking to relate of my own personal
experience, while I remained on Col. Lloyd's plantation, at the home of
my old master. An occasional cuff from Aunt Katy, and a regular whipping
from old master, such as any heedless and mischievous boy might get from
his father, is all that I can mention of this sort. I was not old enough
to work in the field, and, there being little else than field work to
perform, I had much leisure. The most I had to do, was, to drive up the
cows in the evening, to keep the front yard clean, and to perform small
errands for my young mistress, Lucretia Auld. I have reasons for
thinking this lady was very kindly disposed toward me, and, although I
was not often the object of her attention, I constantly regarded her as
my friend, and was always glad when it was my privilege to do her a
service. In a family where there {\protect\hypertarget{130}{}{}}was so
much that was harsh, cold and indifferent, the slightest word or look of
kindness passed, with me, for its full value. Miss Lucretia---as we all
continued to call her long after her marriage---had bestowed upon me
such words and looks as taught me that she pitied me, if she did not
love me. In addition to words and looks, she sometimes gave me a piece
of bread and butter; a thing not set down in the bill of fare, and which
must have been an extra ration, planned aside from either Aunt Katy or
old master, solely out of the tender regard and friendship she had for
me. Then, too, I one day got into the wars with Uncle Abel's son,
``Ike,'' and had got sadly worsted; in fact, the little rascal had
struck me directly in the forehead with a sharp piece of cinder, fused
with iron, from the old blacksmith's forge, which made a cross in my
forehead very plainly to be seen now. The gash bled very freely, and I
roared very loudly and betook myself home. The cold-hearted Aunt Katy
paid no attention either to my wound or my roaring, except to tell me it
served me right; I had no business with Ike; it was good for me; I would
now keep away "\emph{from dem Lloyd niggers.}" Miss Lucretia, in this
state of the case, came forward; and, in quite a different spirit from
that manifested by Aunt Katy, she called me into the parlor, (an extra
privilege of itself,) and, without using toward me any of the
hard-hearted and reproachful epithets of my kitchen tormentor, she
quietly acted the good Samaritan. With her own soft hand she washed the
blood from my head and face, fetched her own balsam bottle, and with the
balsam wetted a nice piece of white
{\protect\hypertarget{131}{}{}}linen, and bound up my head. The balsam
was not more healing to the wound in my head, than her kindness was
healing to the wounds in my spirit, made by the unfeeling words of Aunt
Katy. After this, Miss Lucretia was my friend. I felt her to be such;
and I have no doubt that the simple act of binding up my head, did much
to awaken in her mind an interest in my welfare. It is quite true, that
this interest was never very marked, and it seldom showed itself in
anything more than in giving me a piece of bread when I was very hungry;
but this was a great favor on a slave plantation, and I was the only one
of the children to whom such attention was paid. When very hungry, I
would go into the back yard and play under Miss Lucretia's window. When
pretty severely pinched by hunger, I had a habit of singing, which the
good lady very soon came to understand as a petition for a piece of
bread. When I sung under Miss Lucretia's window, I was very apt to get
well paid for my music. The reader will see that I now had two friends,
both at important points---Mas' Daniel at the great house, and Miss
Lucretia at home. From Mas' Daniel I got protection from the bigger
boys; and from Miss Lucretia I got bread, by singing when I was hungry,
and sympathy when I was abused by that termagant, who had the reins of
government in the kitchen. For such friendship I felt deeply grateful,
and bitter as are my recollections of slavery, I love to recall any
instances of kindness, any sunbeams of humane treatment, which found way
to my soul through the iron grating of my house of bondage. Such beams
seem all the brighter from {\protect\hypertarget{132}{}{}}the general
darkness into which they penetrate, and the impression they make is
vividly distinct and beautiful.

As I have before intimated, I was seldom whipped---and never
severely---by my old master. I suffered little from the treatment I
received, except from hunger and cold. These were my two great physical
troubles. I could neither get a sufficiency of food nor of clothing; but
I suffered less from hunger than from cold. In hottest summer and
coldest winter, I was kept almost in a state of nudity; no shoes, no
stockings, no jacket, no trowsers; nothing but coarse sack-cloth or
tow-linen, made into a sort of shirt, reaching down to my knees. This I
wore night and day, changing it once a week. In the day time I could
protect myself pretty well, by keeping on the sunny side of the house;
and in bad weather, in the corner of the kitchen chimney. The great
difficulty was, to keep warm during the night. I had no bed. The pigs in
the pen had leaves, and the horses in the stable had straw, but the
children had no beds. They lodged anywhere in the ample kitchen. I
slept, generally, in a little closet, without even a blanket to cover
me. In very cold weather, I sometimes got down the bag in which
corn-meal was usually carried to the mill, and crawled into that.
Sleeping there, with my head in and feet out, I was partly protected,
though not comfortable. My feet have been so cracked with the frost,
that the pen with which I am writing might be laid in the gashes. The
manner of taking our meals at old master's, indicated but little
refinement. Our corn-meal mush, when sufficiently
{\protect\hypertarget{133}{}{}}cooled, was placed in a large wooden
tray, or trough, like those used in making maple sugar here in the
north. This tray was set down, either on the floor of the kitchen, or
out of doors on the ground; and the children were called, like so many
pigs; and like so many pigs they would come, and literally devour the
mush---some with oyster shells, some with pieces of shingles, and none
with spoons. He that eat fastest got most, and he that was strongest got
the best place; and few left the trough really satisfied. I was the most
unlucky of any, for Aunt Katy had no good feeling for me; and if I
pushed any of the other children, or if they told her anything
unfavorable of me, she always believed the worst, and was sure to whip
me.

As I grew older and more thoughtful, I was more and more filled with a
sense of my wretchedness. The cruelty of Aunt Katy, the hunger and cold
I suffered, and the terrible reports of wrong and outrage which came to
my ear, together with what I almost daily witnessed, led me, when yet
but eight or nine years old, to wish I had never been born. I used to
contrast my condition with the black-birds, in whose wild and sweet
songs I fancied them so happy! Their apparent joy only deepened the
shades of my sorrow. There are thoughtful days in the lives of
children---at least there were in mine---when they grapple with all the
great, primary subjects of knowledge, and reach, in a moment,
conclusions which no subsequent experience can shake. I was just as well
aware of the unjust, unnatural and murderous character of slavery, when
nine years old, as I am now. Without {\protect\hypertarget{134}{}{}}any
appeal to books, to laws, or to authorities of any kind, it was enough
to accept God as a father, to regard slavery as a crime.

I was not ten years old when I left Col. Lloyd's plantation for
Baltimore. I left that plantation with inexpressible joy. I never shall
forget the ecstacy with which I received the intelligence from my
friend, Miss Lucretia, that my old master had determined to let me go to
Baltimore to live with Mr. Hugh Auld, a brother to Mr. Thomas Auld, my
old master's son-in-law. I received this information about three days
before my departure. They were three of the happiest days of my
childhood. I spent the largest part of these three days in the creek,
washing off the plantation scurf, and preparing for my new home. Mrs.
Lucretia took a lively interest in getting me ready. She told me I must
get all the dead skin off my feet and knees, before I could go to
Baltimore, for the people there were very cleanly, and would laugh at me
if I looked dirty; and, besides, she was intending to give me a pair of
trowsers, which I should not put on unless I got all the dirt off. This
was a warning to which I was bound to take heed; for the thought of
owning a pair of trowsers, was great, indeed. It was almost a sufficient
motive, not only to induce me to scrub off the \emph{mange}, (as pig
drovers would call it,) but the skin as well. So I went at it in good
earnest, working for the first time in the hope of reward. I was greatly
excited, and could hardly consent to sleep, lest I should be left. The
ties that, ordinarily, bind children to their homes, were all severed,
or they never had any existence in {\protect\hypertarget{135}{}{}}my
case, at least so far as the home plantation of Col. L. was concerned. I
therefore found no severe trial at the moment of my departure, such as I
had experienced when separated from my home in Tuckahoe. My home at my
old master's was charmless to me; it was not home, but a prison to me;
on parting from it, I could not feel that I was leaving anything which I
could have enjoyed by staying. My mother was now long dead; my
grandmother was far away, so that I seldom saw her; Aunt Katy was my
unrelenting tormentor; and my two sisters and brothers, owing to our
early separation in life, and the family-destroying power of slavery,
were, comparatively, strangers to me. The fact of our relationship was
almost blotted out. I looked for \emph{home} elsewhere, and was
confident of finding none which I should relish less than the one I was
leaving. If, however, I found in my new home---to which I was going with
such blissful anticipations---hardship, whipping and nakedness, I had
the questionable consolation that I should not have escaped any one of
these evils by remaining under the management of Aunt Katy. Then, too, I
thought, since I had endured much in this line on Lloyd's plantation, I
could endure as much elsewhere, and especially at Baltimore; for I had
something of the feeling about that city which is expressed in the
saying, that being ``hanged in England, is better than dying a natural
death in Ireland.'' I had the strongest desire to see Baltimore. My
cousin Tom---a boy two or three years older than I---had been there, and
though not fluent (he stuttered immoderately,) in speech, he had
inspired me with that desire, by his
{\protect\hypertarget{136}{}{}}eloquent description of the place. Tom
was, sometimes, Capt. Auld's cabin boy; and when he came from Baltimore,
he was always a sort of hero amongst us, at least till his Baltimore
trip was forgotten. I could never tell him of anything, or point out
anything that struck me as beautiful or powerful, but that he had seen
something in Baltimore far surpassing it. Even the great house itself,
with all its pictures within, and pillars without, he had the hardihood
to say ``was nothing to Baltimore.'' He bought a trumpet, (worth six
pence,) and brought it home; told what he had seen in the windows of
stores; that he had heard shooting crackers, and seen soldiers; that he
had seen a steamboat; that there were ships in Baltimore that could
carry four such sloops as the ``Sally Lloyd.'' He said a great deal
about the market-house; he spoke of the bells ringing; and of many other
things which roused my curiosity very much; and, indeed, which
heightened my hopes of happiness in my new home.

We sailed out of Miles river for Baltimore early on a Saturday morning.
I remember only the day of the week; for, at that time, I had no
knowledge of the days of the month, nor, indeed, of the months of the
year. On setting sail, I walked aft, and gave to Col. Lloyd's plantation
what I hoped would be the last look I should ever give to it, or to any
place like it. My strong aversion to the great house farm, was not owing
to my own personal suffering, but the daily suffering of others, and to
the certainty, that I must, sooner or later, be placed under the
barbarous rule of an overseer, such as the accomplished Gore, or the
{\protect\hypertarget{137}{}{}}brutal and drunken Plummer. After taking
this last view, I quitted the quarter deck, made my way to the bow of
the sloop, and spent the remainder of the day in looking ahead;
interesting myself in what was in the distance, rather than what was
near by or behind. The vessels, sweeping along the bay, were very
interesting objects. The broad bay opened like a shoreless ocean on my
boyish vision, filling me with wonder and admiration.

Late in the afternoon, we reached Annapolis, the capital of the state,
stopping there not long enough to admit of my going ashore. It was the
first large town I had ever seen; and though it was inferior to many a
factory village in New England, my feelings, on seeing it, were excited
to a pitch very little below that reached by travelers at the first view
of Rome. The dome of the state house was especially imposing, and
surpassed in grandeur the appearance of the great house. The great world
was opening upon me very rapidly, and I was eagerly acquainting myself
with its multifarious lessons.

We arrived in Baltimore on Sunday morning, and landed at Smith's wharf,
not far from Bowly's wharf. We had on board the sloop a large flock of
sheep, for the Baltimore market; and, after assisting in driving them to
the slaughter house of Mr. Curtis, on Loudon Slater's Hill, I was
speedily conducted by Rich---one of the hands belonging to the
sloop---to my new home in Alliciana street, near Gardiner's ship-yard,
on Fell's Point. Mr. and Mrs. Hugh Auld, my new mistress and master,
were both at home, and met me at the door with their rosy cheeked little
son, Thomas, to take {\protect\hypertarget{138}{}{}}care of whom was to
constitute my future occupation. In fact, it was to ``little Tommy,''
rather than to his parents, that old master made a present of me; and
though there was no \emph{legal} form or arrangement entered into, I
have no doubt that Mr. and Mrs. Auld felt that, in due time, I should be
the legal property of their bright-eyed and beloved boy, Tommy. I was
struck with the appearance, especially, of my new mistress. Her face was
lighted with the kindliest emotions; and the reflex influence of her
countenance, as well as the tenderness with which she seemed to regard
me, while asking me sundry little questions, greatly delighted me, and
lit up, to my fancy, the pathway of my future. Miss Lucretia was kind;
but my new mistress, ``Miss Sophy,'' surpassed her in kindness of
manner. Little Thomas was affectionately told by his mother, that
"\emph{there was his Freddy,}" and that ``Freddy would take care of
him;'' and I was told to ``be kind to little Tommy''---an injunction I
scarcely needed, for I had already fallen in love with the dear boy; and
with these little ceremonies I was initiated into my new home, and
entered upon my peculiar duties, with not a cloud above the horizon.

I may say here, that I regard my removal from Col. Lloyd's plantation as
one of the most interesting and fortunate events of my life. Viewing it
in the light of human likelihoods, it is quite probable that, but for
the mere circumstance of being thus removed before the rigors of slavery
had fastened upon me; before my young spirit had been crushed under the
iron control of the slave-driver, instead of being,
{\protect\hypertarget{139}{}{}}day, a \textsc{freeman}, I might have
been wearing the galling chains of slavery. I have sometimes felt,
however, that there was something more intelligent than \emph{chance},
and something more certain than \emph{luck}, to be seen in the
circumstance. If I have made any progress in knowledge; if I have
cherished any honorable aspirations, or have, in any manner, worthily
discharged the duties of a member of an oppressed people; this little
circumstance must be allowed its due weight in giving my life that
direction. I have ever regarded it as the first plain manifestation of
that

"Divinity that shapes our ends,\\
Rough hew them as we will."

I was not the only boy on the plantation that might have been sent to
live in Baltimore. There was a wide margin from which to select. There
were boys younger, boys older, and boys of the same age, belonging to my
old master---some at his own house, and some at his farm---but the high
privilege fell to my lot.

I may be deemed superstitious and egotistical, in regarding this event
as a special interposition of Divine Providence in my favor; but the
thought is a part of my history, and I should be false to the earliest
and most cherished sentiments of my soul, if I suppressed, or hesitated
to avow that opinion, although it may be characterized as irrational by
the wise, and ridiculous by the scoffer. From my earliest recollections
of serious matters, I date the entertainment of something like an
ineffaceable conviction, that slavery would not always be able to
{\protect\hypertarget{140}{}{}}hold me within its foul embrace; and this
conviction, like a word of living faith, strengthened me through the
darkest trials of my lot. This good spirit was from God; and to him I
offer thanksgiving and praise.
