{\protect\hypertarget{304}{}{}}

~

{CHAPTER XX.}

APPRENTICESHIP LIFE.

{NOTHING LOST BY THE ATTEMPT TO RUN AWAY---COMRADES IN THEIR OLD
HOMES---REASONS FOR SENDING AUTHOR AWAY---RETURN TO BALTIMORE---CONTRAST
BETWEEN ``TOMMY'' AND THAT OF HIS COLORED COMPANION---TRIALS IN
GARDINER'S SHIP YARD---DESPERATE FIGHT---ITS CAUSES CONFLICT BETWEEN
WHITE AND BLACK LABOR---DESCRIPTION OF THE OUTRAGE---COLORED TESTIMONY
NOTHING---CONDUCT OF MASTER HUGH---SPIRIT OF SLAVERY IN
BALTIMORE---AUTHOR'S CONDITION IMPROVES---NEW
ASSOCIATIONS---SLAVEHOLDERS' RIGHT TO TAKE HIS WAGES---HOW TO MAKE A
CONTENTED SLAVE.}

\textsc{Well!} dear reader, I am not, as you may have already inferred,
a loser by the general upstir, described in the foregoing chapter. The
little domestic revolution, notwithstanding the sudden snub it got by
the treachery of somebody---I dare not say or think \emph{who}---did
not, after all, end so disastrously, as, when in the iron cage at
Easton, I conceived it would. The prospect, from that point, did look
about as dark as any that ever cast its gloom over the vision of the
anxious, out-looking, human spirit. ``All is well that ends well.'' My
affectionate comrades, Henry and John Harris, are still with Mr. William
Freeland. Charles Roberts and Henry Baily are safe at their homes. I
have not, therefore, any thing to regret on their account. Their masters
have mercifully forgiven them, probably on the ground suggested in the
spirited little speech of Mrs. Freeland, made tome just before
{\protect\hypertarget{305}{}{}}leaving for the jail---namely: that they
had been allured into the wicked scheme of making their escape, by me;
and that, but for me, they would never have dreamed of a thing so
shocking! My friends had nothing to regret, either; for while they were
watched more closely on account of what had happened, they were,
doubtless, treated more kindly than before, and got new assurances that
they would be legally emancipated, some day, provided their behavior
should make them deserving, from that time forward. Not a blow, as I
learned, was struck any one of them. As for Master William Freeland,
good, unsuspecting soul, he did not believe that we were intending to
run away at all. Having given---as he thought---no occasion to his boys
to leave him, he could not think it probable that they had entertained a
design so grievous. This, however, was not the view taken of the matter
by ``Mas' Billy,'' as we used to call the soft spoken, but crafty and
resolute Mr. William Hamilton. He had no doubt that the crime had been
meditated; and regarding me as the instigator of it, he frankly told
Master Thomas that he must remove me from that neighborhood, or he would
shoot me down. He would not have one so dangerous as ``Frederick''
tampering with his slaves. William Hamilton was not a man whose threat
might be safely disregarded. I have no doubt that he would have proved
as good as his word, had the warning given not been promptly taken. He
was furious at the thought of such a piece of high-handed \emph{theft},
as we were about to perpetrate---the stealing of our own bodies and
souls! The feasibility of the plan, too,
{\protect\hypertarget{306}{}{}}could the first steps have been taken,
was marvelously plain. Besides, this was a \emph{new} idea, this use of
the bay. Slaves escaping, until now, had taken to the woods; they had
never dreamed of profaning and abusing the waters of the noble
Chesapeake, by making them the highway from slavery to freedom. Here was
a broad road of destruction to slavery, which, before, had been looked
upon as a wall of security by slaveholders. But Master Billy could not
get Mr. Freeland to see matters precisely as he did; nor could he get
Master Thomas so excited as he was himself. The latter---I must say it
to his credit---showed much humane feeling in his part of the
transaction, and atoned for much that had been harsh, cruel and
unreasonable in his former treatment of me and others. His clemency was
quite unusual and unlocked for. ``Cousin Tom'' told me that while I was
in jail, Master Thomas was very unhappy; and that the night before his
going up to release me, he had walked the floor nearly all night,
evincing great distress; that very tempting offers had been made to him,
by the negro-traders, but he had rejected them all, saying that
\emph{money could not tempt him to sell me to the far south}. All this I
can easily believe, for he seemed quite reluctant to send me away, at
all. He told me that he only consented to do so, because of the very
strong prejudice against me in the neighborhood, and that he feared for
my safety if I remained there.

Thus, after three years spent in the country, roughing it in the field,
and experiencing all sorts of hardships, I was again permitted to return
to Baltimore, the very place, of all others, short of a free state,
where I {\protect\hypertarget{307}{}{}}most desired to live. The three
years spent in the country, had made some difference in me, and in the
household of Master Hugh. ``Little Tommy'' was no longer \emph{little}
Tommy; and I was not the slender lad who had left for the Eastern Shore
just three years before. The loving relations between me and Mas' Tommy
were broken up. He was no longer dependent on me for protection, but
felt himself a \emph{man}, with other and more suitable associates. In
childhood, he scarcely considered me inferior to himself---certainly, as
good as any other boy with whom he played; but the time had come when
his \emph{friend} must become his \emph{slave}. So we were cold, and we
parted. It was a sad thing to me, that, loving each other as we had
done, we must now take different roads. To him, a thousand avenues were
open. Education had made him acquainted with all the treasures of the
world, and liberty had flung open the gates thereunto; but I, who had
attended him seven years, and had watched over him with the care of a
big brother, fighting his battles in the street, and shielding him from
harm, to an extent which had induced his mother to say, ``Oh! Tommy is
always safe, when he is with Freddy,'' must be confined to a single
condition. He could grow, and become a \textsc{man}; I could grow,
though I could \emph{not} become a man, but must remain, all my life, a
minor---a mere boy. Thomas Auld, junior, obtained a situation on board
the brig Tweed, and went to sea. I know not what has become of him; he
certainly has my good wishes for his welfare and prosperity. There were
few persons to whom I was more sincerely attached
{\protect\hypertarget{308}{}{}}than to him, and there are few in the
world I would be more pleased to meet.

Very soon after I went to Baltimore to live, Master Hugh succeeded in
getting me hired to Mr. "William Gardiner, an extensive ship builder on
Fell's Point. I was placed here to learn to calk, a trade of which I
already had some knowledge, gained while in Mr. Hugh Auld's ship-yard,
when he was a master builder. Gardiner's, however, proved a very
unfavorable place for the accomplishment of that object. Mr. Gardiner
was, that season, engaged in building two large man-of-war vessels,
professedly for the Mexican government. These vessels were to be
launched in the month of July, of that year, and, in failure thereof,
Mr. G. would forfeit a very considerable sum of money. So, when I
entered the ship-yard, all was hurry and driving. There were in the yard
about one hundred men; of these about seventy or eighty were regular
carpenters---privileged men. Speaking of my condition here, I wrote,
years ago---and I have now no reason to vary the picture---as follows:

{``There was no time to learn any thing. Every man had to do that which
he knew how to do. In entering the ship-yard, my orders from Mr.
Gardiner were, to do whatever the carpenters commanded me to do. This
was placing me at the beck and call of about seventy-five men. I was to
regard all these as masters. Their word was to be my law. My situation
was a most trying one. At times I needed a dozen pair of hands. I was
called a dozen ways in the space of a single minute. Three or four
voices would strike my ear at the same moment. It was---'Fred., come
help me to cant this timber here.''---Fred., come carry this timber
yonder.'---} {\protect\hypertarget{309}{}{}}{`Fred., bring that roller
here.'---'Fred., go get a fresh can of water.`---'Fred., come help saw
off the end of this timber.'---'Fred., go quick and get the
crowbar.`---'Fred., hold on the end of this fall.'---'Fred., go the
blacksmith's shop, and get a new `punch.'---'Hurra, Fred.! run and bring
me a cold chisel.`---'I say, Fred., bear a hand, and get up a fire as
quick as lightning under that steam-box.'---'Halloo, nigger! come, turn
this grindstone.'---'Come, come! move, move! and \emph{bowse} this
timber forward.`---'I say, darkey, blast your eyes, why don't you heat
up some pitch?'---'Halloo! halloo! halloo!' (Three voices at the same
time.) `Come here!---Go there!---Hold on where you are! D---n you, if
you move, I'll knock your brains out!"'}

Such, dear reader, is a glance at the school which was mine, during the
first eight months of my stay at Baltimore. At the end of eight months,
Master Hugh refused longer to allow me to remain with Mr. Gardiner. The
circumstance which led to his taking me away, was a brutal outrage,
committed upon me by the white apprentices of the ship-yard. The fight
was a desperate one, and I came out of it most shockingly mangled. I was
cut and bruised in sundry places, and my left eye was nearly knocked out
of its socket. The facts, leading to this barbarous outrage upon me,
illustrate a phase of slavery destined to become an important element in
the overthrow of the slave system, and I may, therefore state them with
some minuteness. That phase is this: \emph{the conflict of slavery with
the interests of the white mechanics and laborers of the south}. In the
country, this conflict is not so apparent; but, in cities, such as
Baltimore, Richmond, New Orleans, Mobile, \&c., it is seen pretty
clearly. The slaveholders, with a craftiness peculiar
{\protect\hypertarget{310}{}{}}to themselves, by encouraging the enmity
of the poor, laboring white man against the blacks, succeeds in making
the said white man almost as much a slave as the black slave himself.
The difference between the white slave, and the black slave, is this:
the latter belongs to \emph{one} slaveholder, and the former belongs to
\emph{all} the slaveholders, collectively. The white slave has taken
from him, by indirection, what the black slave has taken from him,
directly, and without ceremony. Both are plundered, and by the same
plunderers. The slave is robbed, by his master, of all his earnings,
above what is required for his bare physical necessities; and the white
man is robbed by the slave system, of the just results of his labor,
because he is flung into competition with a class of laborers who work
without wages. The competition, and its injurious consequences, will,
one day, array the non-slaveholding white people of the slave states,
against the slave system, and make them the most effective workers
against the great evil. At present, the slaveholders blind them to this
competition, by keeping alive their prejudice against the slaves,
\emph{as men}---not against them \emph{as slaves}. They appeal to their
pride, often denouncing emancipation, as tending to place the white
working man, on an equality with negroes, and, by this means, they
succeed in drawing off the minds of the poor whites from the real fact,
that, by the rich slave-master, they are already regarded as but a
single remove from equality with the slave. The impression is cunningly
made, that slavery is the only power that can prevent the laboring white
man from falling to the level of the slave's poverty and
{\protect\hypertarget{311}{}{}}degradation. To make this enmity deep and
broad, between the slave and the poor white man, the latter is allowed
to abuse and whip the former, without hinderance. But---as I have
suggested---this state of facts prevails \emph{mostly} in the country.
In the city of Baltimore, there are not unfrequent murmurs, that
educating the slaves to be mechanics may, in the end, give slave-masters
power to dispense with the services of the poor white man altogether.
But, with characteristic dread of offending the slaveholders, these
poor, white mechanics in Mr. Gardiner's ship-yard---instead of applying
the natural, honest remedy for the apprehended evil, and objecting at
once to work there by the side of slaves---made a cowardly attack upon
the free colored mechanics, saying \emph{they} were eating the bread
which should be eaten by American freemen, and swearing that they would
not work with them. The feeling was, \emph{really}, against having their
labor brought into competition with that of the colored people at all;
but it was too much to strike directly at the interest of the
slaveholders; and, therefore---proving their servility and
cowardice---they dealt their blows on the poor, colored freeman, and
aimed to prevent \emph{him} from serving himself, in the evening of
life, with the trade with which he had served his master, during the
more vigorous portion of his days. Had they succeeded in driving the
black freemen out of the ship yard, they would have determined also upon
the removal of the black slaves. The feeling was very bitter toward all
colored people in Baltimore, about this time, (1836,) and they---free
and slave---suffered all manner of insult and wrong.

{\protect\hypertarget{312}{}{}}Until a very little while before I went
there, white and black ship carpenters worked side by side, in the ship
yards of Mr. Gardiner, Mr. Duncan, Mr. Walter Price, and Mr. Robb.
Nobody seemed to see any impropriety in it. To outward seeming, all
hands were well satisfied. Some of the blacks were first rate workmen,
and were given jobs requiring the highest skill. All at once, however,
the white carpenters knocked off, and swore that they would no longer
work on the same stage with free negroes. Taking advantage of the heavy
contract resting upon Mr. Gardiner, to have the war vessels for Mexico
ready to launch in July, and of the difficulty of getting other hands at
that season of the year, they swore they would not strike another blow
for him, unless he would discharge his free colored workmen.

Now, although this movement did not extend to me, \emph{in form}, it did
reach me, \emph{in fact}. The spirit which it awakened was one of malice
and bitterness, toward colored people \emph{generally}, and I suffered
with the rest, and suffered severely. My fellow apprentices very soon
began to feel it to be degrading to work with me. They began to put on
high looks, and to talk contemptuously and maliciously of "\emph{the
niggers;}" saying, that ``they would take the country,'' that ``they
ought to be killed.'' Encouraged by the cowardly workmen, who, knowing
me to be a slave, made no issue with Mr. Gardiner about my being there,
these young men did their utmost to make it impossible for me to stay.
They seldom called me to do any thing, without coupling the call with a
curse, and, Edward North, the biggest in every thing, rascality
{\protect\hypertarget{313}{}{}}included, ventured to strike me,
whereupon I picked him up, and threw him into the dock. Whenever any of
them struck me, I struck back again, regardless of consequences. I could
manage any of them \emph{singly;} and, while I could keep them from
combining, I succeeded very well. In the conflict which ended my stay at
Mr. Gardiner's, I was beset by four of them at once---Ned North, Ned
Hays, Bill Stewart, and Tom Humphreys. Two of them were as large as
myself, and they came near killing me, in broad day light. The attack
was made suddenly, and simultaneously. One came in front, armed with a
brick; there was one at each side, and one behind, and they closed up
around me. I was struck on all sides; and, while I was attending to
those in front, I received a blow on my head, from behind, dealt with a
heavy hand-spike. I was completely stunned by the blow, and fell,
heavily, on the ground, among the timbers. Taking advantage of my fall,
they rushed upon me, and began to pound me with their fists. I let them
lay on, for a while, after I came to myself, with a view of gaining
strength. They did me little damage, so far; but, finally, getting tired
of that sport, I gave a sudden surge, and, despite their weight, I rose
to my hands and knees. Just as I did this, one of their number (I know
not which) planted a blow with his boot in my left eye, which, for a
time, seemed to have burst my eyeball. When they saw my eye completely
closed, my face covered with blood, and I staggering under the stunning
blows they had given me, they left me. As soon as I gathered sufficient
strength, I picked up the hand-spike, and, madly enough,
{\protect\hypertarget{314}{}{}}attempted to pursue them; but here the
carpenters interfered, and compelled me to give up my frenzied pursuit.
It was impossible to stand against so many.

Dear reader, you can hardly believe the statement, but it is true, and,
therefore, I write it down: not fewer than fifty white men stood by, and
saw this brutal and shameless outrage committed, and not a man of them
all interposed a single word of mercy. There were four against one, and
that one's face was beaten and battered most horribly, and no one said,
``that is enough;'' but some cried out, ``kill him---kill him---kill the
d---d nigger! knock his brains out---he struck a white person.'' I
mention this inhuman outcry, to show the character of the men, and the
spirit of the times, at Gardiner's ship yard, and, indeed, in Baltimore
generally, in 1836. As I look back to this period, I am almost amazed
that I was not murdered outright, in that ship yard, so murderous was
the spirit which prevailed there. On two occasions, while there, I came
near losing my life. I was driving bolts in the hold, through the
keelson, with Hays. In its course, the bolt bent. Hays cursed me, and
said that it was my blow which bent the bolt. I denied this, and charged
it upon him. In a fit of rage he seized an adze, and darted toward me. I
met him with a maul, and parried his blow, or I should have then lost my
life. A son of old Torn Lanman, (the latter's double murder I have
elsewhere charged upon him,) in the spirit of his miserable father, made
an assault upon me, but the blow with his maul missed me. After the
united assault of North, Stewart, Hays and Humphreys, finding that the
{\protect\hypertarget{315}{}{}}carpenters were as bitter toward me as
the apprentices, and that the latter were probably set on by the former,
I found my only chance for life was in flight I succeeded in getting
away, without an additional blow. To strike a white man, was death, by
Lynch law, in Gardiner's ship yard; nor was there much of any other law
toward colored people, at that time, in any other part of Maryland. The
whole sentiment of Baltimore was murderous.

After making my escape from the ship yard, I went straight home, and
related the story of the outrage to Master Hugh Auld; and it is due to
him to say, that his conduct---though he was not a religious man---was
every way more humane than that of his brother, Thomas, when I went to
the latter in a somewhat similar plight, from the hands of
"\emph{Brother Edward Covey.}" He listened attentively to my narration
of the circumstances leading to the ruffianly outrage, and gave many
proofs of his strong indignation at what was done. Hugh was a rough, but
manly-hearted fellow, and, at this time, his best nature showed itself.

The heart of my once almost over-kind mistress, Sophia, was again melted
in pity toward me. My puffed-out eye, and my scarred and blood-covered
face, moved the dear lady to tears. She kindly drew a chair by me, and
with friendly, consoling words, she took water, and washed the blood
from my face. No mother's hand could have been more tender than hers.
She bound up my head, and covered my wounded eye with a lean piece of
fresh beef. It was almost compensation for the murderous assault, and
{\protect\hypertarget{316}{}{}}my suffering, that it furnished an
occasion for the manifestation, once more, of the originally
characteristic kindness of my mistress. Her affectionate heart was not
yet dead, though much hardened by time and by circumstances.

As for Master Hugh's part, as I have said, he was furious about it; and
he gave expression to his fury in the usual forms of speech in that
locality. He poured curses on the heads of the whole ship yard company,
and swore that he would have satisfaction for the outrage. His
indignation was really strong and healthy; but, unfortunately, it
resulted from the thought that his rights of property, in my person, had
not been respected, more than from any sense of the outrage committed on
me \emph{as a man}. I inferred as much as this, from the fact that he
could, himself, beat and mangle when it suited him to do so. Bent on
having satisfaction, as he said, just as soon as I got a little the
better of my bruises, Master Hugh took me to Esquire Watson's office, on
Bond street, Fell's Point, with a view to procuring the arrest of those
who had assaulted me. He related the outrage to the magistrate, as I had
related it to him, and seemed to expect that a warrant would, at once,
be issued for the arrest of the lawless ruffians.

Mr. Watson heard it all, and instead of drawing up his warrant, he
inquired.---

``Mr. Auld, who saw this assault of which you speak?''

``It was done, sir, in the presence of a ship yard full of hands.''

``Sir,'' said Watson, "I am sorry, but I cannot move
{\protect\hypertarget{317}{}{}}in this matter except upon the oath of
white witnesses."

``But here's the boy; look at his head and face,'' said the excited
Master Hugh; "\emph{they} show \emph{what} has been done."

But Watson insisted that he was not authorized to do anything, unless
\emph{white} witnesses of the transaction would come forward, and
testify to what had taken place. He could issue no warrant on my word,
against white persons; and, if I had been killed in the presence of a
\emph{thousand blacks}, their testimony, combined, would have been
insufficient to arrest a single murderer. Master Hugh, for once, was
compelled to say, that this state of things was \emph{too bad;} and he
left the office of the magistrate, disgusted.

Of course, it was impossible to get any white man to testify against my
assailants. The carpenters saw what was done; but the actors were but
the agents of their malice, and did only what the carpenters sanctioned.
They had cried, with one accord, "\emph{kill the nigger!" kill the
nigger!}" Even those who may have pitied me, if any such were among
them, lacked the moral courage to come and volunteer their evidence. The
slightest manifestation of sympathy or justice toward a person of color,
was denounced as abolitionism; and the name of abolitionist, subjected
its bearer to frightful liabilities. "D---n \emph{abolitionists}," and
"\emph{Kill the niggers}," were the watch-words of the foul-mouthed
ruffians of those days. Nothing was done, and probably there would not
have been anything done, had I been killed in the affray. The laws and
the morals of the christian city of
{\protect\hypertarget{318}{}{}}Baltimore, afforded no protection to the
sable denizens of that city.

Master Hugh, on finding he could get no redress for the cruel wrong,
withdrew me from the employment of Mr. Gardiner, and took me into his
own family, Mrs. Auld kindly taking care of me, and dressing my wounds,
until they were healed, and I was ready to go again to work.

While I was on the Eastern Shore, Master Hugh had met with reverses,
which overthrew his business; and he had given up ship building in his
own yard, on the City Block, and was now acting as foreman of Mr. Walter
Price. The best he could now do for me, was to take me into Mr. Price's
yard, and afford me the facilities there, for completing the trade which
I had began to learn at Gardiner's. Here I rapidly became expert in the
use of my calking tools; and, in the course of a single year, I was able
to command the highest wages paid to journeymen calkers in Baltimore.

The reader will observe that I was now of some pecuniary value to my
master. During the busy season, I was bringing six and seven dollars per
week. I have, sometimes, brought him as much as nine dollars a week, for
the wages were a dollar and a half per day.

After learning to calk, I sought my own employment, made my own
contracts, and collected my own earnings; giving Master Hugh no trouble
in any part of the transactions to which I was a party.

Here, then, were better days for the Eastern Shore \emph{slave}. I was
now free from the vexatious assaults of
{\protect\hypertarget{319}{}{}}the apprentices at Mr. Gardiner's; and
free from the perils of plantation life, and once more in a favorable
condition to increase my little stock of education, which had been at a
dead stand since my removal from Baltimore. I had, on the Eastern Shore,
been only a teacher, when in company with other slaves, but now there
were colored persons who could instruct me. Many of the young calkers
could read, write and cipher. Some of them had high notions about mental
improvement; and the free ones, on Fell's Point, organized what they
called the "\emph{East Baltimore Mental Improvement Society.}" To this
society, notwithstanding it was intended that only free persons should
attach themselves, I was admitted, and was, several times, assigned a
prominent part in its debates. I owe much to the society of these young
men.

The reader already knows enough of the \emph{ill} effects of good
treatment on a slave, to anticipate what was now the case in my improved
condition. It was not long before I began to show signs of disquiet with
slavery, and to look around for means to get out of that condition by
the shortest route. I was living among \emph{freemen;} and was, in all
respects, equal to them by nature and by attainments. \emph{Why should I
be a slave?} There was \emph{no} reason why I should be the thrall of
any man.

Besides, I was now getting---as I have said---a dollar and fifty cents
per day. I contracted for it, worked for it, earned it, collected it; it
was paid to me, and it was \emph{rightfully} my own; and yet, upon every
returning Saturday night, this money---my
{\protect\hypertarget{320}{}{}}own hard earnings, every cent of it---was
demanded of me, and taken from me by Master Hugh. He did not earn it; he
had no hand in earning it; why, then, should he have it? I owed him
nothing. He had given me no schooling, and I had received from him only
my food and raiment; and for these, my services were supposed to pay,
from the first. The right to take my earnings, was the right of the
robber. He had the power to compel me to give him the fruits of my
labor, and this power was his only right in the case. I became more and
more dissatisfied with this state of things; and, in so becoming, I only
gave proof of the same human nature which every reader of this chapter
in my life---slaveholder, or non-slaveholder---is conscious of
possessing.

To make a contented slave, you must make a thoughtless one. It is
necessary to darken his moral and mental vision, and, as far as
possible, to annihilate his power of reason. He must be able to detect
no inconsistencies in slavery. The man that takes his earnings, must be
able to convince him that he has a perfect right to do so. It must not
depend upon mere force; the slave must know no Higher Law than his
master's will. The whole relationship must not only demonstrate, to his
mind, its necessity, but its absolute rightfulness. If there be one
crevice through which a single drop can fall, it will certainly rust off
the slave's chain.
