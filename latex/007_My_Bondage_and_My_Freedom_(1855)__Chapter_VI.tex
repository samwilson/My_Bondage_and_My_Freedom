\hypertarget{headerContainer}{}
\hypertarget{navigationHeader}{}
\protect\hypertarget{headerprevious}{}{←\href{/wiki/My_Bondage_and_My_Freedom_(1855)/Chapter_V}{Chapter
V}}

\textbf{\protect\hypertarget{header_title_text}{}{\href{/wiki/My_Bondage_and_My_Freedom_(1855)}{My
Bondage and My Freedom}}} ~(1855)~ \emph{by
\href{/wiki/Author:Frederick_Douglass}{\protect\hypertarget{header_author_text}{}{{Frederick
Douglass}}}}\\
\protect\hypertarget{header_section_text}{}{Chapter VI}

\protect\hypertarget{headernext}{}{\href{/wiki/My_Bondage_and_My_Freedom_(1855)/Chapter_VII}{Chapter
VII}→}

\hypertarget{navigationNotes}{}

\hypertarget{ws-data}{}
\protect\hypertarget{ws-article-id}{}{2336902}\protect\hypertarget{ws-title}{}{\href{/wiki/My_Bondage_and_My_Freedom_(1855)}{My
Bondage and My Freedom} --- \emph{Chapter
VI}}\protect\hypertarget{ws-author}{}{Frederick
Douglass}\protect\hypertarget{ws-year}{}{1855}

{\protect\hypertarget{89}{}{}}

~

{CHAPTER VI.}

TREATMENT OF SLAVES ON LLOYD'S PLANTATION.

{THE AUTHOR'S EARLY REFLECTIONS ON SLAVERY---PRESENTIMENT OF ONE DAY
BEING A FREEMAN---COMBAT BETWEEN AN OVERSEER AND A SLAVE-WOMAN---THE
ADVANTAGES OF RESISTANCE---ALLOWANCE DAY ON THE HOME PLANTATION---THE
SINGING OF SLAVES---AN EXPLANATION---THE SLAVES' FOOD AND
CLOTHING---NAKED CHILDREN---LIFE IN THE QUARTER---DEPRIVATION OF
SLEEP---NURSING CHILDREN CARRIED TO THE FIELD---DESCRIPTION OF THE
COWSKIN---THE ASH-CAKE---MANNER OF MAKING IT---THE DINNER HOUR---THE
CONTRAST.}

\textsc{The} heart-rending incidents, related in the foregoing chapter,
led me, thus early, to inquire into the nature and history of slavery.
\emph{Why am I a slave? Why are some people slaves, and others masters?
Was there ever a time when this was not so? How did the relation
commence?} These were the perplexing questions which began now to claim
my thoughts, and to exercise the weak powers of my mind, for I was still
but a child, and knew less than children of the same age in the free
states. As my questions concerning these things were only put to
children a little older, and little better informed than myself, I was
not rapid in reaching a solid footing. By some means I learned from
these inquiries, that "\emph{God, up in the sky}," made every body; and
that he made \emph{white} people to be masters and mistresses, and
\emph{black} people to be slaves. This did not satisfy me, nor lessen my
interest in the subject. I was told, too,
{\protect\hypertarget{90}{}{}}that God was good, and that He knew what
was best for me, and best for everybody. This was less satisfactory than
the first statement; because it came, point blank, against all my
notions of goodness. It was not good to let old master cut the flesh off
Esther, and make her cry so. Besides, how did people know that God made
black people to be slaves? Did they go up in the sky and learn it? or,
did He come down and tell them so? All was dark here. It was some relief
to my hard notions of the goodness of God, that, although he made white
men to be slaveholders, he did not make them to be \emph{bad}
slaveholders, and that, in due time, he would punish the bad
slaveholders; that he would, when they died, send them to the bad place,
where they would be ``burnt up.'' Nevertheless, I could not reconcile
the relation of slavery with my crude notions of goodness.

Then, too, I found that there were puzzling exceptions to this theory of
slavery on both sides, and in the middle. I knew of blacks who were
\emph{not} slaves; I knew of whites who were \emph{not} slaveholders;
and I knew of persons who were \emph{nearly} white, who were slaves.
\emph{Color}, therefore, was a very unsatisfactory basis for slavery.

Once, however, engaged in the inquiry, I was not very long in finding
out the true solution of the matter. It was not \emph{color}, but
\emph{crime}, not \emph{God}, but \emph{man}, that afforded the true
explanation of the existence of slavery; nor was I long in finding out
another important truth, viz: what man can make, man can unmake. The
appalling darkness faded away, and I was master of the subject. There
were slaves here, {\protect\hypertarget{91}{}{}}direct from Guinea; and
there were many who could say that their fathers and mothers were stolen
from Africa---forced from their homes, and compelled to serve as slaves.
This, to me, was knowledge; but it was a kind of knowledge which filled
me with a burning hatred of slavery, increased my suffering, and left me
without the means of breaking away from my bondage. Yet it was knowledge
quite worth possessing. I could not have been more than seven or eight
years old, when I began to make this subject my study. It was with me in
the woods and fields; along the shore of the river, and wherever my
boyish wanderings led me; and though I was, at that time, quite ignorant
of the existence of the free states, I distinctly remember being,
\emph{even then}, most strongly impressed with the idea of being a
freeman some day. This cheering assurance was an inborn dream of my
human nature---a constant menace to slavery---and one which all the
powers of slavery were unable to silence or extinguish.

Up to the time of the brutal flogging of my Aunt Esther---for she was my
own aunt---and the horrid plight in which I had seen my cousin from
Tuckahoe, who had been so badly beaten by the cruel Mr. Plummer, my
attention had not been called, especially, to the gross features of
slavery. I had, of course, heard of whippings, and of savage
\emph{rencontres} between overseers and slaves, but I had always been
out of the way at the times and places of their occurrence. My plays and
sports, most of the time, took me from the corn and tobacco fields,
where the great body of the hands were at work, and where scenes of
cruelty were {\protect\hypertarget{92}{}{}}enacted and witnessed. But,
after the whipping of Aunt Esther, I saw many cases of the same shocking
nature, not only in my master's house, but on Col. Lloyd's plantation.
One of the first which I saw, and which greatly agitated me, was the
whipping of a woman belonging to Col. Lloyd, named Nelly. The offense
alleged against Nelly, was one of the commonest and most indefinite in
the whole catalogue of offenses usually laid to the charge of slaves,
viz: ``impudence.'' This may mean almost anything, or nothing at all,
just according to the caprice of the master or overseer, at the moment.
But, whatever it is, or is not, if it gets the name of ``impudence,''
the party charged with it is sure of a flogging. This offense may be
committed in various ways; in the tone of an answer; in answering at
all; in not answering; in the expression of countenance; in the motion
of the head; in the gait, manner and bearing of the slave. In the case
under consideration, I can easily believe that, according to all
slaveholding standards, here was a genuine instance of impudence. In
Nelly there were all the necessary conditions for committing the
offense. She was a bright mulatto, the recognized wife of a favorite
``hand'' on board Col. Lloyd's sloop, and the mother of five sprightly
children. She was a vigorous and spirited woman, and one of the most
likely, on the plantation, to be guilty of impudence. My attention was
called to the scene, by the noise, curses and screams that proceeded
from it; and, on going a little in that direction, I came upon the
parties engaged in the skirmish. Mr. Sevier, the overseer, had hold of
Nelly, when I caught sight of them; he {\protect\hypertarget{93}{}{}}was
endeavoring to drag her toward a tree, which endeavor Nelly was sternly
resisting; but to no purpose, except to retard the progress of the
overseer's plans. Nelly---as I have said---was the mother of five
children; three of them were present, and though quite small, (from
seven to ten years old, I should think,) they gallantly came to their
mother's defense, and gave the overseer an excellent pelting with
stones. One of the little fellows ran up, seized the overseer by the leg
and bit him; but the monster was too busily engaged with Nelly, to pay
any attention to the assaults of the children. There were numerous
bloody marks on Mr. Sevier's face, when I first saw him, and they
increased as the struggle went on. The imprints of Nelly's fingers were
visible, and I was glad to see them. Amidst the wild screams of the
children---"\emph{Let my mammy go}``---''\emph{let my mammy
go}``---there escaped, from between the teeth of the bullet-headed
overseer, a few bitter curses, mingled with threats, that ''he would
teach the d---d b---h how to give a white man impudence." There is no
doubt that Nelly felt herself superior, in some respects, to the slaves
around her. She was a wife and a mother; her husband was a valued and
favorite slave. Besides, he was one of the first hands on board of the
sloop, and the sloop hands---since they had to represent the plantation
abroad---were generally treated tenderly. The overseer never was allowed
to whip Harry; why then should he be allowed to whip Harry's wife?
Thoughts of this kind, no doubt, influenced her; but, for whatever
reason, she nobly resisted, and, unlike most of the slaves, seemed
determined to make her whipping {\protect\hypertarget{94}{}{}}cost Mr.
Sevier as much as possible. The blood on his (and her) face, attested
her skill, as well as her courage and dexterity in using her nails.
Maddened by her resistance, I expected to see Mr. Sevier level her to
the ground by a stunning blow; but no; like a savage bull-dog---which he
resembled both in temper and appearance---he maintained his grip, and
steadily dragged his victim toward the tree, disregarding alike her
blows, and the cries of the children for their mother's release. He
would, doubtless, have knocked her down with his hickory stick, but that
such act might have cost him his place. It is often deemed advisable to
knock a \emph{man} slave down, in order to tie him, but it is considered
cowardly and inexcusable, in an overseer, thus to deal with a
\emph{woman}. He is expected to tie her up, and to give her what is
called, in southern parlance, a ``genteel flogging,'' without any very
great outlay of strength or skill. I watched, with palpitating interest,
the course of the preliminary struggle, and was saddened by every new
advantage gained over her by the ruffian. There were times when she
seemed likely to get the better of the brute, but he finally overpowered
her, and succeeded in getting his rope around her arms, and in firmly
tying her to the tree, at which he had been aiming. This done, and Nelly
was at the mercy of his merciless lash; and now, what followed, I have
no heart to describe. The cowardly creature made good his every threat;
and wielded the lash with all the hot zest of furious revenge. The cries
of the woman, while undergoing the terrible infliction, were mingled
with those of the children, sounds which I hope the reader
{\protect\hypertarget{95}{}{}}may never be called upon to hear. When
Nelly was untied, her back was covered with blood. The red stripes were
all over her shoulders. She was whipped---severely whipped; but she was
not subdued, for she continued to denounce the overseer, and to call him
every vile name. He had bruised her flesh, but had left her invincible
spirit undaunted. Such floggings are seldom repeated by the same
overseer. They prefer to whip those who are most easily whipped. The old
doctrine that submission is the best cure for outrage and wrong, does
not hold good on the slave plantation. He is whipped oftenest, who is
whipped easiest; and that slave who has the courage to stand up for
himself against the overseer, although he may have many hard stripes at
the first, becomes, in the end, a freeman, even though he sustain the
formal relation of a slave. ``You can shoot me but you can't whip me,''
said a slave to Rigby Hopkins; and the result was that he was neither
whipped not shot. If the latter had been his fate, it would have been
less deplorable than the living and lingering death to which cowardly
and slavish souls are subjected. I do not know that Mr. Sevier ever
undertook to whip Nelly again. He probably never did, for it was not
long after his attempt to subdue her, that he was taken sick, and died.
The wretched man died as he had lived, unrepentant; and it was
said---with how much truth I know not---that in the very last hours of
his life, his ruling passion showed itself, and that when wrestling with
death, he was uttering horrid oaths, and flourishing the cowskin, as
though he was tearing the flesh off some helpless slave. One thing is
{\protect\hypertarget{96}{}{}}certain, that when he was in health, it
was enough to chill the blood, and to stiffen the hair of an ordinary
man, to hear Mr. Sevier talk. Nature, or his cruel habits, had given to
his face an expression of unusual savageness, even for a slave-driver.
Tobacco and rage had worn his teeth short, and nearly every sentence
that escaped their compressed grating, was commenced or concluded with
some outburst of profanity. His presence made the field alike the field
of blood, and of blasphemy. Hated for his cruelty, despised for his
cowardice, his death was deplored by no one outside his own house---if
indeed it was deplored there; it was regarded by the slaves as a
merciful interposition of Providence. Never went there a man to the
grave loaded with heavier curses. Mr. Sevier's place was promptly taken
by a Mr. Hopkins, and the change was quite a relief, he being a very
different man. He was, in all respects, a better man than his
predecessor; as good as any man can be, and yet be an overseer. His
course was characterized by no extraordinary cruelty; and when he
whipped a slave, as he sometimes did, he seemed to take no especial
pleasure in it, but, on the contrary, acted as though he felt it to be a
mean business. Mr. Hopkins stayed but a short time; his place---much to
the regret of the slaves generally---was taken by a Mr. Gore, of whom
more will be said hereafter. It is enough, for the present, to say, that
he was no improvement on Mr. Sevier, except that he was less noisy and
less profane.

I have already referred to the business-like aspect of Col. Lloyd's
plantation. This business-like appearance was much increased on the two
days at the end {\protect\hypertarget{97}{}{}}of each month, when the
slaves from the different farms came to get their monthly allowance of
meal and meat. These were gala days for the slaves, and there was much
rivalry among them as to \emph{who} should be elected to go up to the
great house farm for the allowance, and, indeed, to attend to any
business at this, (for them,) the capital. The beauty and grandeur of
the place, its numerous slave population, and the fact that Harry, Peter
and Jake---the sailors of the sloop---almost always kept, privately,
little trinkets which they bought at Baltimore, to sell, made it a
privilege to come to the great house farm. Being selected, too, for this
office, was deemed a high honor. It was taken as a proof of confidence
and favor; but, probably, the chief motive of the competitors for the
place, was, a desire to break the dull monotony of the field, and to get
beyond the overseer's eye and lash. Once on the road with an ox team,
and seated on the tongue of his cart, with no overseer to look after
him, the slave was comparatively free; and, if thoughtful, he had time
to think. Slaves are generally expected to sing as well as to work. A
silent slave is not liked by masters or overseers. "\emph{Make a
noise}," "\emph{make a noise}," and "\emph{bear a hand}," are the words
usually addressed to the slaves when there is silence amongst them. This
may account for the almost constant sinking heard in the southern
states. There was, generally, more or less singing among the teamsters,
as it was one means of letting the overseer know where they were, and
that they were moving on with the work. But, on allowance day, those who
visited the great house farm were peculiarly excited
{\protect\hypertarget{98}{}{}}and noisy. While on their way, they would
make the dense old woods, for miles around, reverberate with their wild
notes. These were not always merry because they were wild. On the
contrary, they were mostly of a plaintive cast, and told a tale of grief
and sorrow. In the most boisterous outbursts of rapturous sentiment,
there was ever a tinge of deep melancholy. I have never heard any songs
like those anywhere since I left slavery, except when in Ireland. There
I heard the same \emph{wailing notes}, and was much affected by them. It
was during the famine of 1845--6. In all the songs of the slaves, there
was ever some expression in praise of the great house farm; something
which would flatter the pride of the owner, and, possibly, draw a
favorable glance from him.

"I am going away to the great house farm,\\
{}O yea! O yea! O yea!\\
My old master is a good old master,\\
{}Oh yea! O yea! O yea!"

This they would sing, with other words of their own improvising---jargon
to others, but full of meaning to themselves. I have sometimes thought,
that the mere hearing of those songs would do more to impress truly
spiritual-minded men and women with the soul-crushing and death-dealing
character of slavery, than the reading of whole volumes of its mere
physical cruelties. They speak to the heart and to the soul of the
thoughtful. I cannot better express my sense of them now, than ten years
ago, when, in sketching my life, I thus spoke of this feature of my
plantation experience:

{\protect\hypertarget{99}{}{}}

"I did not, when a slave, understand the deep meanings of those rude,
and apparently incoherent songs. I was myself within the circle, so that
I neither saw nor heard as those without might see and hear. They told a
tale which was then altogether beyond my feeble comprehension; they were
tones, loud, long and deep, breathing the prayer and complaint of souls
boiling over with the bitterest anguish. Every tone was a testimony
against slavery, and a prayer to God for deliverance from chains. The
hearing of those wild notes always depressed my spirits, and filled my
heart with ineffable sadness. The mere recurrence, even now, afflicts my
spirit, and while I am writing these lines, my tears are falling. To
those songs I trace my first glimmering conceptions of the dehumanizing
character of slavery. I can never get rid of that conception. Those
songs still follow me, to deepen my hatred of slavery, and quicken my
sympathies for my brethren in bonds. If any one wishes to be impressed
with a sense of the soul-killing power of slavery, let him go to Col.
Lloyd's plantation, and, on allowance day, place himself in the deep,
pine woods, and there let him, in silence, thoughtfully analyze the
sounds that shall pass through the chambers of his soul, and if he is
not thus impressed, it will only be because 'there is no flesh in his
obdurate heart.{'}"

~

The remark is not unfrequently made, that slaves are the most contented
and happy laborers in the world. They dance and sing, and make all
manner of joyful noises---so they do; but it is a great mistake to
suppose them happy because they sing. The songs of the slave represent
the sorrows, rather than the joys, of his heart; and he is relieved by
them, only as an aching heart is relieved by its tears. Such is the
constitution of the human mind, that, when pressed
{\protect\hypertarget{100}{}{}}to extremes, it often avails itself of
the most opposite methods. Extremes meet in mind as in matter. When the
slaves on board of the ``Pearl'' were overtaken, arrested, and carried
to prison---their hopes for freedom blasted---as they marched in chains
they sang, and found (as Emily Edmunson tells us) a melancholy relief in
singing. The singing of a man cast away on a desolate island, might be
as appropriately considered an evidence of his contentment and
happiness, as the singing of a slave. Sorrow and desolation have their
songs, as well as joy and peace. Slaves sing more to \emph{make}
themselves happy, than to express their happiness.

It is the boast of slaveholders, that their slaves enjoy more of the
physical comforts of life than the peasantry of any country in the
world. My experience contradicts this. The men and the women slaves on
Col. Lloyd's farm, received, as their monthly allowance of food, eight
pounds of pickled pork, or their equivalent in fish. The pork was often
tainted, and the fish was of the poorest quality---herrings, which would
bring very little if offered for sale in any northern market. With their
pork or fish, they had one bushel of Indian meal---unbolted---of which
quite fifteen per cent. was fit only to feed pigs. With this, one pint
of salt was given; and this was the entire monthly allowance of a full
grown slave, working constantly in the open field, from morning until
night, every day in the month except Sunday, and living on a fraction
more than a quarter of a pound of meat per day, and less than a peck of
corn-meal per week. There is no kind of work that a man can do
{\protect\hypertarget{101}{}{}}which requires a better supply of food to
prevent physical exhaustion, than the field-work of a slave. So much for
the slave's allowance of food; now for his raiment. The yearly allowance
of clothing for the slaves on this plantation, consisted of two
tow-linen shirts---such linen as the coarsest crash towels are made of;
one pair of trowsers of the same material, for summer, and a pair of
trowsers and a jacket of woolen, most slazily put together, for winter;
one pair of yarn stockings, and one pair of shoes of the coarsest
description. The slave's entire apparel could not have cost more than
eight dollars per year. The allowance of food and clothing for the
little children, was committed to their mothers, or to the older
slave-women having the care of them. Children who were unable to work in
the field, had neither shoes, stockings, jackets nor trowsers given
them. Their clothing consisted of two coarse tow-linen shirts---already
described---per year; and when these failed them, as they often did,
they went naked until the next allowance day. Flocks of little children
from five to ten years old, might be seen on Col. Lloyd's plantation, as
destitute of clothing as any little heathen on the west coast of Africa;
and this, not merely during the summer months, but during the frosty
weather of March. The little girls were no better off than the boys; all
were nearly in a state of nudity.

As to beds to sleep on, they were known to none of the field hands;
nothing but a coarse blanket---not so good as those used in the north to
cover horses---was given them, and this only to the men and women. The
children stuck themselves in holes and corners,
{\protect\hypertarget{102}{}{}}about the quarters; often in the corner
of the huge chimneys, with their feet in the ashes to keep them warm.
The want of beds, however, was not considered a very great privation.
Time to sleep was of far greater importance, for, when the day's work is
done, most of the slaves have their washing, mending and cooking to do;
and, having few or none of the ordinary facilities for doing such
things, very many of their sleeping hours are consumed in necessary
preparations for the duties of the coming day.

The sleeping apartments---if they may be called such---have little
regard to comfort or decency. Old and young, male and female, married
and single, drop down upon the common clay floor, each covering up with
his or her blanket,---the only protection they have from cold or
exposure. The night, however, is shortened at both ends. The slaves work
often as long as they can see, and are late in cooking and mending for
the coming day; and, at the first gray streak of morning, they are
summoned to the field by the driver's horn.

More slaves are whipped for oversleeping than for any other fault.
Neither age nor sex finds any favor. The overseer stands at the quarter
door, armed with stick and cowskin, ready to whip any who may be a few
minutes behind time. When the horn is blown, there is a rush for the
door, and the hindermost one is sure to get a blow from the overseer.
Young mothers who worked in the field, were allowed an hour, about ten
o'clock in the morning, to go home to nurse their children. Sometimes
they were compelled to take their children with them, and to leave them
in the {\protect\hypertarget{103}{}{}}corner of the fences, to prevent
loss of time in nursing them. The overseer generally rides about the
field on horseback. A cowskin and a hickory stick are his constant
companions. The cowskin is a kind of whip seldom seen in the northern
states. It is made entirely of untanned, but dried, ox hide, and is
about as hard as a piece of well-seasoned live oak. It is made of
various sizes, but the usual length is about three feet. The part held
in the hand is nearly an inch in thickness; and, from the extreme end of
the butt or handle, the cowskin tapers its whole length to a point. This
makes it quite elastic and springy. A blow with it, on the hardest back,
will gash the flesh, and make the blood start. Cowskins are painted red,
blue and green, and are the favorite slave whip. I think this whip worse
than the ``cat-o'-nine-tails.'' It condenses the whole strength of the
arm to a single point, and comes with a spring that makes the air
whistle. It is a terrible instrument, and is so handy, that the overseer
can always have it on his person, and ready for use. The temptation to
use it is ever strong; and an overseer can, if disposed, always have
cause for using it. With him, it is literally a word and a blow, and, in
most cases, the blow comes first.

As a general rule, slaves do not come to the quarters for either
breakfast or dinner, but take their ``ash cake'' with them, and eat it
in the field. This was so on the home plantation; probably, because the
distance from the quarter to the field, was sometimes two, and even
three miles.

The dinner of the slaves consisted of a huge piece of ash cake, and a
small piece of pork, or two salt
{\protect\hypertarget{104}{}{}}herrings. Not having ovens, nor any
suitable cooking utensils, the slaves mixed their meal with a little
water, to such thickness that a spoon would stand erect in it; and,
after the wood had burned away to coals and ashes, they would place the
dough between oak leaves and lay it carefully in the ashes, completely
covering it; hence, the bread is called ash cake. The surface of this
peculiar bread is covered with ashes, to the depth of a sixteenth part
of an inch, and the ashes, certainly, do not make it very grateful to
the teeth, nor render it very palatable. The bran, or coarse part of the
meal, is baked with the fine, and bright scales run through the bread.
This bread, with its ashes and bran, would disgust and choke a northern
man, but it is quite liked by the slaves. They eat it with avidity, and
are more concerned about the quantity than about the quality. They are
far too scantily provided for, and are worked too steadily, to be much
concerned for the quality of their food. The few minutes allowed them at
dinner time, after partaking of their coarse repast, are variously
spent. Some lie down on the ``turning row,'' and go to sleep; others
draw together, and talk; and others are at work with needle and thread,
mending their tattered garments. Sometimes you may hear a wild, hoarse
laugh arise from a circle, and often a song. Soon, however, the overseer
comes dashing through the field. "\emph{Tumble up! Tumble up}, and to
\emph{work, work}," is the cry; and, now, from twelve o'clock (mid-day)
till dark, the human cattle are in motion, wielding their clumsy noes;
hurried on by no hope of reward, no sense of gratitude, no love of
children, no prospect of bettering {\protect\hypertarget{105}{}{}}their
condition; nothing, save the dread and terror of the slave-driver's
lash. So goes one day, and so comes and goes another.

But, let us now leave the rough usage of the field, where vulgar
coarseness and brutal cruelty spread themselves and flourish, rank as
weeds in the tropics; where a vile wretch, in the shape of a man, rides,
walks, or struts about, dealing blows, and leaving gashes on
broken-spirited men and helpless women, for thirty dollars per month---a
business so horrible, hardening, and disgraceful, that, rather than
engage in it, a decent man would blow his own brains out---and let the
reader view with me the equally wicked, but less repulsive aspects of
slave life; where pride and pomp roll luxuriously at ease; where the
toil of a thousand men supports a single family in easy idleness and
sin. This is the great house; it is the home of the \textsc{Lloyds}!
Some idea of its splendor has already been given---and, it is here that
we shall find that height of luxury which is the opposite of that depth
of poverty and physical wretchedness that we have just now been
contemplating. But, there is this difference in the two extremes; viz:
that in the case of the slave, the miseries and hardships of his lot are
imposed by others, and, in the master's case, they are imposed by
himself. The slave is a subject, subjected by others; the slaveholder is
a subject, but he is the author of his own subjection. There is more
truth in the saying, that slavery is a greater evil to the master than
to the slave, than many, who utter it, suppose. The self-executing laws
of eternal justice follow close on the heels of the evil-doer here, as
well as {\protect\hypertarget{106}{}{}}elsewhere; making escape from all
its penalties impossible. But, let others philosophize; it is my
province here to relate and describe; only allowing myself a word or
two, occasionally, to assist the reader in the proper understanding of
the facts narrated.
