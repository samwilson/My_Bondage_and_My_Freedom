{}

~

{CHAPTER VIII.}

A CHAPTER OF HORRORS.

{AUSTIN GORE---A SKETCH OF HIS CHARACTER---OVERSEERS AS A CLASS---THEIR
PECULIAR CHARACTERISTICS---THE MARKED INDIVIDUALITY OF AUSTIN GORE---HIS
SENSE OF DUTY---HOW HE WHIPPED---MURDER OF POOR DENBY---HOW IT
OCCURRED---SENSATION---HOW GORE MADE PEACE WITH COL. LLOYD---THE MURDER
UNPUNISHED---ANOTHER DREADFUL MURDER NARRATED---NO LAWS FOR THE
PROTECTION OF SLAVES CAN BE ENFORCED IN THE SOUTHERN STATES.}

\textsc{As} I have already intimated elsewhere, the slaves on Col.
Lloyd's plantation, whose hard lot, under Mr. Sevier, the reader has
already noticed and deplored, were not permitted to enjoy the
comparatively moderate rule of Mr. Hopkins. The latter was succeeded by
a very different man. The name of the new overseer was Austin Gore. Upon
this individual I would fix particular attention; for under his rule
there was more suffering from violence and bloodshed than
had---according to the older slaves---ever been experienced before on
this plantation. I confess, I hardly know how to bring this man fitly
before the reader. He was, it is true, an overseer, and possessed, to a
large extent, the peculiar characteristics of his class; yet, to call
him merely an overseer, would not give the reader a fair notion of the
man. I speak of overseers as a class. They are such. They are as
distinct from the slaveholding gentry of the south, as are the
fish-women of Paris, and the coal-heavers of {}London, distinct from
other members of society. They constitute a separate fraternity at the
south, not less marked than is the fraternity of Park lane bullies in
New York. They have been arranged and classified by that great law of
attraction, which determines the spheres and affinities of men; which
ordains, that men, whose malign and brutal propensities predominate over
their moral and intellectual endowments, shall, naturally, fall into
those employments which promise the largest gratification to those
predominating instincts or propensities. The office of overseer takes
this raw material of vulgarity and brutality, and stamps it as a
distinct class of southern society. But, in this class, as in all other
classes, there are characters of marked individuality, even while they
bear a general resemblance to the mass. Mr. Gore was one of those, to
whom a general characterization would do no manner of justice. He was an
overseer; but he was something more. With the malign and tyrannical
qualities of an overseer, he combined something of the lawful master. He
had the artfulness and the mean ambition of his class; but he was wholly
free from the disgusting swagger and noisy bravado of his fraternity.
There was an easy air of independence about him; a calm self-possession,
and a sternness of glance, which might well daunt hearts less timid than
those of poor slaves, accustomed from childhood and through life to
cower before a driver's lash. The home plantation of Col. Lloyd afforded
an ample field for the exercise of the qualifications for overseership,
which he possessed in such an eminent degree.

Mr. Gore was one of those overseers, who could {}torture the slightest
word or look into impudence; he had the nerve, not only to resent, but
to punish, promptly and severely. He never allowed himself to be
answered back, by a slave. In this, he was as lordly and as imperious as
Col. Edward Lloyd, himself; acting always up to the maxim, practically
maintained by slaveholders, that it is better that a dozen slaves suffer
under the lash, without fault, than that the master or the overseer
should \emph{seem} to have been wrong in the presence of the slave.
\emph{Everything must be absolute here.} Guilty or not guilty, it is
enough to be accused, to be sure of a flogging. The very presence of
this man Gore was painful, and I shunned him as I would have shunned a
rattlesnake. His piercing, black eyes, and sharp, shrill voice, ever
awakened sensations of terror among the slaves. For so young a man, (I
describe him as he was, twenty-five or thirty years ago,) Mr. Gore was
singularly reserved and grave in the presence of slaves. He indulged in
no jokes, said no funny things, and kept his own counsels. Other
overseers, how brutal soever they might be, were, at times, inclined to
gain favor with the slaves, by indulging a little pleasantry; but Gore
was never known to be guilty of any such weakness. He was always the
cold, distant, unapproachable \emph{overseer} of Col. Edward Lloyd's
plantation, and needed no higher pleasure than was involved in a
faithful discharge of the duties of his office. When he whipped, he
seemed to do so from a sense of duty, and feared no consequences. What
Hopkins did reluctantly, Gore did with alacrity. There was a stern will,
an iron-like reality, about this Gore, which {}would have easily made
him the chief of a band of pirates, had his environments been favorable
to such a course of life. All the coolness, savage barbarity and freedom
from moral restraint, which are necessary in the character of a
pirate-chief, centered, I think, in this man Gore. Among many other
deeds of shocking cruelty which he perpetrated, while I was at Mr.
Lloyd's, was the murder of a young colored man, named Denby. He was
sometimes called Bill Denby, or Demby; (I write from sound, and the
sounds on Lloyd's plantation are not very certain.) I knew him well. He
was a powerful young man, full of animal spirits, and, so far as I know,
he was among the most valuable of Col. Lloyd's slaves. In something---I
know not what---he offended this Mr. Austin Gore, and, in accordance
with the custom of the latter, he undertook to flog him. He gave Denby
but few stripes; the latter broke away from him and plunged into the
creek, and, standing there to the depth of his neck in water, he refused
to come out at the order of the overseer; whereupon, for this refusal,
\emph{Gore shot him dead!} It is said that Gore gave Denby three calls,
telling him that if he did not obey the last call, he would shoot him.
When the third call was given, Denby stood his ground firmly; and this
raised the question, in the minds of the by-standing slaves---``will he
dare to shoot?'' Mr. Gore, without further parley, and without making
any further effort to induce Denby to come out of the water, raised his
gun deliberately to his face, took deadly aim at his standing victim,
and, in an instant, poor Denby was numbered with the dead. His mangled
body sank out of {}sight, and only his warm, red blood marked the place
where he had stood.

This devilish outrage, this fiendish murder, produced, as it was well
calculated to do, a tremendous sensation. A thrill of horror flashed
through every soul on the plantation, if I may except the guilty wretch
who had committed the hell-black deed. While the slaves generally were
panic-struck, and howling with alarm, the murderer himself was calm and
collected, and appeared as though nothing unusual had happened. The
atrocity roused my old master, and he spoke out, in reprobation of it;
but the whole thing proved to be less than a nine days' wonder. Both
Col. Lloyd and my old master arraigned Gore for his cruelty in the
matter, but this amounted to nothing. His reply, or explanation---as I
remember to have heard it at the time---was, that the extraordinary
expedient was demanded by necessity; that Denby had become unmanageable;
that he had set a dangerous example to the other slaves; and that,
without some such prompt measure as that to which he had resorted, were
adopted, there would be an end to all rule and order on the plantation.
That very convenient covert for all manner of cruelty and outrage---that
cowardly alarm-cry, that the slaves would "\emph{take the place,}" was
pleaded, in extenuation of this revolting crime, just as it had been
cited in defense of a thousand similar ones. He argued, that if one
slave refused to be corrected, and was allowed to escape with his life,
when he had been told that he should lose it if he persisted in his
course, the other slaves would soon copy his example; the result of
which would be, the {}freedom of the slaves, and the enslavement of the
whites. I have every reason to believe that Mr. Gore's defense, or
explanation, was deemed satisfactory---at least to Col. Lloyd. He was
continued in his office on the plantation. His fame as an overseer went
abroad, and his horrid crime was not even submitted to judicial
investigation. The murder was committed in the presence of slaves, and
they, of course, could neither, institute a suit, nor testify against
the murderer. His bare word would go further in a court of law, than the
united testimony of ten thousand black witnesses.

All that Mr. Gore had to do, was to make his peace with Col. Lloyd. This
done, and the guilty perpetrator of one of the most foul murders goes
unwhipped of justice, and uncensured by the community in which he lives.
Mr. Gore lived in St. Michael's, Talbot county, when I left Maryland; if
he is still alive he probably yet resides there; and I have no reason to
doubt that he is now as highly esteemed, and as greatly respected, as
though his guilty soul had never been stained with innocent blood. I am
well aware that what I have now written will by some be branded as false
and malicious. It will be denied, not only that such a thing ever did
transpire, as I have now narrated, but that such a thing could happen in
\emph{Maryland}. I can only say---believe it or not---that I have said
nothing but the literal truth, gainsay it who may.

I speak advisedly when I say this,---that killing a slave, or any
colored person, in Talbot county, Maryland, is not treated as a crime,
either by the courts or the community. Mr Thomas Lanman, ship carpenter,
{}of St. Michael's, killed two slaves, one of whom he butchered with a
hatchet, by knocking his brains out. He used to boast of the commission
of the awful and bloody deed. I have heard him do so, laughingly,
saying, among other things, that he was the only benefactor of his
country in the company, and that when ``others would do as much as he
had done, we should be relieved of the d---d niggers.''

As an evidence of the reckless disregard of human life---where the life
is that of a slave---I may state the notorious fact, that the wife of
Mr. Giles Hicks, who lived but a short distance from Col. Lloyd's, with
her own hands murdered my wife's cousin, a young girl between fifteen
and sixteen years of age---mutilating her person in a most shocking
manner. The atrocious woman, in the paroxysm of her wrath, not content
with murdering her victim, literally mangled her face, and broke her
breast bone. Wild, however, and infuriated as she was, she took the
precaution to cause the slave-girl to be buried; but the facts of the
case coming abroad, very speedily led to the disinterment of the remains
of the murdered slave-girl. A coroner's jury was assembled, who decided
that the girl had come to her death by severe beating. It was
ascertained that the offense for which this girl was thus hurried out of
the world, was this: she had been set that night, and several preceding
nights, to mind Mrs. Hicks's baby, and having fallen into a sound sleep,
the baby cried, waking Mrs. Hicks, but not the slave-girl. Mrs. Hicks,
becoming infuriated at the girl's tardiness, after calling her several
times, jumped from her bed and seized a piece of fire-wood from the
{}fire-place; and then, as she lay fast asleep, she deliberately pounded
in her skull and breast-bone, and thus ended her life. I will not say
that this most horrid murder produced no sensation in the community. It
\emph{did} produce a sensation; but, incredible to tell, the moral sense
of the community was blunted too entirely by the ordinary nature of
slavery horrors, to bring the murderess to punishment. A warrant was
issued for her arrest, but, for some reason or other, that warrant was
never served. Thus did Mrs. Hicks not only escape condign punishment,
but even the pain and mortification of being arraigned before a court of
justice.

Whilst I am detailing the bloody deeds that took place during my stay on
Col. Lloyd's plantation, I will briefly narrate another dark
transaction, which occurred about the same time as the murder of Denby
by Mr. Gore.

On the side of the river Wye, opposite from Col. Lloyd's, there lived a
Mr. Beal Bondley, a wealthy slaveholder. In the direction of his land,
and near the shore, there was an excellent oyster fishing ground, and to
this, some of the slaves of Col Lloyd occasionally resorted in their
little canoes, at night, with a view to make up the deficiency of their
scanty allowance of food, by the oysters that they could easily get
there. This, Mr. Bondley took it into his head to regard as a trespass,
and while an old man belonging to Col. Lloyd was engaged in catching a
few of the many millions of oysters that lined the bottom of that creek,
to satisfy his hunger, the villainous Mr. Bondley, lying in ambush,
without the slightest ceremony, {}discharged the contents of his musket
into the back and shoulders of the poor old man. As good fortune would
have it, the shot did not prove mortal, and Mr. Bondley came over, the
next day, to see Col. Lloyd---whether to pay him for his property, or to
justify himself for what he had done, I know not; but this I \emph{can}
say, the cruel and dastardly transaction was speedily hushed up; there
was very little said about it at all, and nothing was publicly done
which looked like the application of the principle of justice to the man
whom \emph{chance,} only, saved from being an actual murderer. One of
the commonest sayings to which my ears early became accustomed, on Col.
Lloyd's plantation and elsewhere in Maryland, was, that it was
"\emph{worth but half a cent to kill a nigger, and a half a cent to bury
him;}" and the facts of my experience go far to justify the practical
truth of this strange proverb. Laws for the protection of the lives of
the slaves, are, as they must needs be, utterly incapable of being
enforced, where the very parties who are nominally protected, are not
permitted to give evidence, in courts of law, against the only class of
persons from whom abuse, outrage and murder might be reasonably
apprehended. While I heard of numerous murders committed by slaveholders
on the Eastern Shore of Maryland, I never knew a solitary instance in
which a slaveholder was either hung or imprisoned for having murdered a
slave. The usual pretext for killing a slave is, that the slave has
offered resistance. Should a slave, when assaulted, but raise his hand
in self-defense, the white assaulting party is fully justified by
southern, or Maryland, public opinion, in shooting the slave {}down.
Sometimes this is done, simply because it is alleged that the slave has
been saucy. But here I leave this phase of the society of my early
childhood, and will relieve the kind reader of these heart-sickening
details.
