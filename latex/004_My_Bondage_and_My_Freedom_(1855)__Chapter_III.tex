{}

~

{CHAPTER III.}

THE AUTHOR'S PARENTAGE.

{AUTHOR'S FATHER SHROUDED IN MYSTERY---AUTHOR'S MOTHER---HER PERSONAL
APPEARANCE---INTERFERENCE OF SLAVERY WITH THE NATURAL AFFECTIONS OF
MOTHER AND CHILDREN---SITUATION OF AUTHOR'S MOTHER---HER NIGHTLY VISITS
TO HER BOY---STRIKING INCIDENT---HER DEATH---HER PLACE OF BURIAL.}

\textsc{If} the reader will now be kind enough to allow me time to grow
bigger, and afford me an opportunity for my experience to become
greater, I will tell him something, by-and-by, of slave life, as I saw,
felt, and heard it, on Col. Edward Lloyd's plantation, and at the house
of old master, where I had now, despite of myself, most suddenly, but
not unexpectedly, been dropped. Meanwhile, I will redeem my promise to
say something more of my dear mother.

I say nothing of \emph{father}, for he is shrouded in a mystery I have
never been able to penetrate. Slavery does away with fathers, as it does
away with families. Slavery has no use for either fathers or families,
and its laws do not recognize their existence in the social arrangements
of the plantation. When they \emph{do} exist, they are not the
outgrowths of slavery, but are antagonistic to that system. The order of
civilization is reversed here. The name of the child is not expected to
be that of its father, and his {}condition does not necessarily affect
that of the child. He may be the slave of Mr. Tilgman; and his child,
when born, may be the slave of Mr. Gross. He may be a \emph{freeman};
and yet his child may be a \emph{chattel}. He may be white, glorying in
the purity of his Anglo-Saxon blood; and his child may be ranked with
the blackest slaves. Indeed, he \emph{may} be, and often \emph{is},
master and father to the same child. He can be father without being a
husband, and may sell his child without incurring reproach, if the child
be by a woman in whose veins courses one thirty-second part of African
blood. My father was a white man, or nearly white. It was sometimes
whispered that my master was my father.

But to return, or rather, to begin. My knowledge of my mother is very
scanty, but very distinct. Her personal appearance and bearing are
ineffaceably stamped upon my memory. She was tall, and finely
proportioned; of deep black, glossy complexion; had regular features,
and, among the other slaves, was remarkably sedate in her manners. There
is in
"\emph{\href{/w/index.php?title=The_Natural_History_of_Man_(Prichard)\&action=edit\&redlink=1}{Prichard's
Natural History of Man}}," the head of a figure---on page 157---the
features of which so resemble those of my mother, that I often recur to
it with something of the feeling which I suppose others experience when
looking upon the pictures of dear departed ones.

Yet I cannot say that I was very deeply attached to my mother; certainly
not so deeply as I should have been had our relations in childhood been
different. We were separated, according to the common custom, when I was
but an infant, and, of course, before I knew my mother from any one
else.

{}The germs of affection with which the Almighty, in his wisdom and
mercy, arms the helpless infant against the ills and vicissitudes of his
lot, had been directed in their growth toward that loving old
grandmother, whose gentle hand and kind deportment it was the first
effort of my infantile understanding to comprehend and appreciate.
Accordingly, the tenderest affection which a beneficent Father allows,
as a partial compensation to the mother for the pains and lacerations of
her heart, incident to the maternal relation, was, in my case, diverted
from its true and natural object, by the envious, greedy, and
treacherous hand of slavery. The slave-mother can be spared long enough
from the field to endure all the bitterness of a mother's anguish, when
it adds another name to a master's ledger, but \emph{not} long enough to
receive the joyous reward afforded by the intelligent smiles of her
child. I never think of this terrible interference of slavery with my
infantile affections, and its diverting them from their natural course,
without feelings to which I can give no adequate expression.

I do not remember to have seen my mother at my grandmother's at any
time. I remember her only in her visits to me at Col. Lloyd's
plantation, and in the kitchen of my old master. Her visits to me there
were few in number, brief in duration, and mostly made in the night. The
pains she took, and the toil she endured, to see me, tells me that a
true mother's heart was hers, and that slavery had difficulty in
paralyzing it with unmotherly indifference.

My mother was hired out to a Mr. Stewart, who lived about twelve miles
from old master's, and, {}being a field hand, she seldom had leisure, by
day, for the performance of the journey. The nights and the distance
were both obstacles to her visits. She was obliged to walk, unless
chance flung into her way an opportunity to ride; and the latter was
sometimes her good luck. But she always had to walk one way or the
other. It was a greater luxury than slavery could afford, to allow a
black slave-mother a horse or a mule, upon which to travel twenty-four
miles, when she could walk the distance. Besides, it is deemed a foolish
whim for a slave-mother to manifest concern to see her children, and, in
one point of view, the case is made out---she can do nothing for them.
She has no control over them; the master is even more than the mother,
in all matters touching the fate of her child. Why, then, should she
give herself any concern? She has no responsibility. Such is the
reasoning, and such the practice. The iron rule of the plantation,
always passionately and violently enforced in that neighborhood, makes
flogging the penalty of failing to be in the field before sunrise in the
morning, unless special permission be given to the absenting slave. ``I
went to see my child,'' is no excuse to the ear or heart of the
overseer.

One of the visits of my mother to me, while at Col. Lloyd's, I remember
very vividly, as affording a bright gleam of a mother's love, and the
earnestness of a mother's care.

I had on that day offended ``Aunt Katy,'' (called ``Aunt'' by way of
respect,) the cook of old master's establishment. I do not now remember
the nature of my offense in this instance, for my offenses were
{}numerous in that quarter, greatly depending, however, upon the mood of
Aunt Katy, as to their heinousness; but she had adopted, that day, her
favorite mode of punishing me, namely, making me go without food all
day---that is, from after breakfast. The first hour or two after dinner,
I succeeded pretty well in keeping up my spirits; but though I made an
excellent stand against the foe, and fought bravely during the
afternoon, I knew I must be conquered at last, unless I got the
accustomed reënforcement of a slice of corn bread, at sundown. Sundown
came, but \emph{no bread}, and, in its stead, their came the threat,
with a scowl well suited to its terrible import, that she "meant to
\emph{starve the life out of me!}" Brandishing her knife, she chopped
off the heavy slices for the other children, and put the loaf away,
muttering, all the while, her savage designs upon myself. Against this
disappointment, for I was expecting that her heart would relent at last,
I made an extra effort to maintain my dignity; but when I saw all the
other children around me with merry and satisfied faces, I could stand
it no longer. I went out behind the house, and cried like a fine fellow!
When tired of this, I returned to the kitchen, sat by the fire, and
brooded over my hard lot. I was too hungry to sleep. While I sat in the
corner, I caught sight of an ear of Indian corn on an upper shelf of the
kitchen. I watched my chance, and got it, and, shelling off a few
grains, I put it back again. The grains in my hand, I quickly put in
some ashes, and covered them with embers, to roast them. All this I did
at the risk of getting a brutal thumping, for Aunt Katy could beat, as
well as starve me. My corn was not long in {}roasting, and, with my keen
appetite, it did not matter even if the grains were not exactly done. I
eagerly pulled them out, and placed them on my stool, in a clever little
pile. Just as I began to help myself to my very dry meal, in came my
dear mother. And now, dear reader, a scene occurred which was altogether
worth beholding, and to me it was instructive as well as interesting.
The friendless and hungry boy, in his extremest need---and when he did
not dare to look for succor---found himself in the strong, protecting
arms of a mother; a mother who was, at the moment (being endowed with
high powers of manner as well as matter) more than a match for all his
enemies. I shall never forget the indescribable expression of her
countenance, when I told her that I had had no food since morning; and
that Aunt Katy said she ``meant to starve the life out of me.'' There
was pity in her glance at me, and a fiery indignation at Aunt Katy at
the same time; and, while she took the corn from me, and gave me a large
ginger cake, in its stead, she read Aunt Katy a lecture which she never
forgot. My mother threatened her with complaining to old master in my
behalf; for the latter, though harsh and cruel himself, at times, did
not sanction the meanness, injustice, partiality and oppressions enacted
by Aunt Katy in the kitchen. That night I learned the fact, that I was
not only a child, but \emph{somebody's} child. The ``sweet cake'' my
mother gave me was in the shape of a heart, with a rich, dark ring
glazed upon the edge of it. I was victorious, and well off for the
moment; prouder, on my mother's knee, than a king upon his throne. But
my triumph was short. I dropped off to {}sleep, and waked in the morning
only to find my mother gone, and myself left at the mercy of the sable
virago, dominant in my old master's kitchen, whose fiery wrath was my
constant dread.

I do not remember to have seen my mother after this occurrence. Death
soon ended the little communication that had existed between us; and
with it, I believe, a life---judging from her weary, sad, downcast
countenance and mute demeanor---full of heartfelt sorrow. I was not
allowed to visit her during any part of her long illness; nor did I see
her for a long time before she was taken ill and died. The heartless and
ghastly form of \emph{slavery} rises between mother and child, even at
the bed of death. The mother, at the verge of the grave, may not gather
her children, to impart to them her holy admonitions, and invoke for
them her dying benediction. The bondwoman lives as a slave, and is left
to die as a beast; often with fewer attentions than are paid to a
favorite horse. Scenes of sacred tenderness, around the deathbed, never
forgotten, and which often arrest the vicious and confirm the virtuous
during life, must be looked for among the free, though they sometimes
occur among the slaves. It has been a life-long, standing grief to me,
that I knew so little of my mother; and that I was so early separated
from her. The counsels of her love must have been beneficial to me. The
side view of her face is imaged on my memory, and I take few steps in
life, without feeling her presence; but the image is mute, and I have no
striking words of her's treasured up.

I learned, after my mother's death, that she could {}read, and that she
was the \emph{only} one of all the slaves and colored people in Tuckahoe
who enjoyed that advantage. How she acquired this knowledge, I know not,
for Tuckahoe is the last place in the world where she would be apt to
find facilities for learning. I can, therefore, fondly and proudly
ascribe to her an earnest love of knowledge. That a ``field hand''
should learn to read, in any slave state, is remarkable; but the
achievement of my mother, considering the place, was very extraordinary;
and, in view of that fact, I am quite willing, and even happy, to
attribute any love of letters I possess, and for which I have
got---despite of prejudices---only too much credit, \emph{not} to my
admitted Anglo-Saxon paternity, but to the native genius of my sable,
unprotected, and uncultivated \emph{mother}---a woman, who belonged to a
race whose mental endowments it is, at present, fashionable to hold in
disparagement and contempt.

Summoned away to her account, with the impassable gulf of slavery
between us during her entire illness, my mother died without leaving me
a single intimation of \emph{who} my father was. There was a whisper,
that my master was my father; yet it was only a whisper, and I cannot
say that I ever gave it credence. Indeed, I now have reason to think he
was not; nevertheless, the fact remains, in all its glaring odiousness,
that, by the laws of slavery, children, in all cases, are reduced to the
condition of their mothers. This arrangement admits of the greatest
license to brutal slaveholders, and their profligate sons, brothers,
relations and friends, and gives to the pleasure of sin, the additional
attraction of profit. A whole volume might {}be written on this single
feature of slavery, as I have observed it.

One might imagine, that the children of such connections, would fare
better, in the hands of their masters, than other slaves. The rule is
quite the other way; and a very little reflection will satisfy the
reader that such is the case. A man who will enslave his own blood, may
not be safely relied on for magnanimity. Men do not love those who
remind them of their sins---unless they have a mind to repent---and the
mulatto child's face is a standing accusation against him who is master
and father to the child. What is still worse, perhaps, such a child is a
constant offense to the wife. She hates its very presence, and when a
slaveholding woman hates, she wants not means to give that hate telling
effect. Women---white women, I mean---are \textsc{idols} at the south,
not \textsc{wives}, for the slave women are preferred in many instances;
and if these \emph{idols} but nod, or lift a finger, woe to the poor
victim: kicks, cuffs and stripes are sure to follow. Masters are
frequently compelled to sell this class of their slaves, out of
deference to the feelings of their white wives; and shocking and
scandalous as it may seem for a man to sell his own blood to the
traffickers in human flesh, it is often an act of humanity toward the
slave-child to be thus removed from his merciless tormentors.

It is not within the scope of the design of my simple story, to comment
upon every phase of slavery not within my experience as a slave.

But, I may remark, that, if the lineal descendants of Ham are only to be
enslaved, according to the {}scriptures, slavery in this country will
soon become an unscriptural institution; for thousands are ushered into
the world, annually, who---like myself---owe their existence to white
fathers, and, most frequently, to their masters, and master's sons. The
slave-woman is at the mercy of the fathers, sons or brothers of her
master. The thoughtful know the rest.

After what I have now said of the circumstances of my mother, and my
relations to her, the reader will not be surprised, nor be disposed to
censure me, when I tell but the simple truth, viz: that I received the
tidings of her death with no strong emotions of sorrow for her, and with
very little regret for myself on account of her loss. I had to learn the
value of my mother long after her death, and by witnessing the devotion
of other mothers to their children.

There is not, beneath the sky, an enemy to filial affection so
destructive as slavery. It had made my brothers and sisters strangers to
me; it converted the mother that bore me, into a myth; it shrouded my
father in mystery, and left me without an intelligible beginning in the
world.

My mother died when I could not have been more than eight or nine years
old, on one of old master's farms in Tuckahoe, in the neighborhood of
Hillsborough. Her grave is, as the grave of the dead at sea, unmarked,
and without stone or stake.
